% \iffalse meta-comment
%
% File: atbegshi.dtx
% Version: 2016/06/09 v1.18
% Info: At begin shipout hook
%
% Copyright (C) 2007-2011 by
%    Heiko Oberdiek <heiko.oberdiek at googlemail.com>
%    2016
%    https://github.com/ho-tex/oberdiek/issues
%
% This work may be distributed and/or modified under the
% conditions of the LaTeX Project Public License, either
% version 1.3c of this license or (at your option) any later
% version. This version of this license is in
%    http://www.latex-project.org/lppl/lppl-1-3c.txt
% and the latest version of this license is in
%    http://www.latex-project.org/lppl.txt
% and version 1.3 or later is part of all distributions of
% LaTeX version 2005/12/01 or later.
%
% This work has the LPPL maintenance status "maintained".
%
% This Current Maintainer of this work is Heiko Oberdiek.
%
% The Base Interpreter refers to any `TeX-Format',
% because some files are installed in TDS:tex/generic//.
%
% This work consists of the main source file atbegshi.dtx
% and the derived files
%    atbegshi.sty, atbegshi.pdf, atbegshi.ins, atbegshi.drv,
%    atbegshi-example1.tex, atbegshi-example2.tex,
%    atbegshi-test1.tex, atbegshi-test2.tex, atbegshi-test3.tex.
%
% Distribution:
%    CTAN:macros/latex/contrib/oberdiek/atbegshi.dtx
%    CTAN:macros/latex/contrib/oberdiek/atbegshi.pdf
%
% Unpacking:
%    (a) If atbegshi.ins is present:
%           tex atbegshi.ins
%    (b) Without atbegshi.ins:
%           tex atbegshi.dtx
%    (c) If you insist on using LaTeX
%           latex \let\install=y% \iffalse meta-comment
%
% File: atbegshi.dtx
% Version: 2016/06/09 v1.18
% Info: At begin shipout hook
%
% Copyright (C) 2007-2011 by
%    Heiko Oberdiek <heiko.oberdiek at googlemail.com>
%    2016
%    https://github.com/ho-tex/oberdiek/issues
%
% This work may be distributed and/or modified under the
% conditions of the LaTeX Project Public License, either
% version 1.3c of this license or (at your option) any later
% version. This version of this license is in
%    http://www.latex-project.org/lppl/lppl-1-3c.txt
% and the latest version of this license is in
%    http://www.latex-project.org/lppl.txt
% and version 1.3 or later is part of all distributions of
% LaTeX version 2005/12/01 or later.
%
% This work has the LPPL maintenance status "maintained".
%
% This Current Maintainer of this work is Heiko Oberdiek.
%
% The Base Interpreter refers to any `TeX-Format',
% because some files are installed in TDS:tex/generic//.
%
% This work consists of the main source file atbegshi.dtx
% and the derived files
%    atbegshi.sty, atbegshi.pdf, atbegshi.ins, atbegshi.drv,
%    atbegshi-example1.tex, atbegshi-example2.tex,
%    atbegshi-test1.tex, atbegshi-test2.tex, atbegshi-test3.tex.
%
% Distribution:
%    CTAN:macros/latex/contrib/oberdiek/atbegshi.dtx
%    CTAN:macros/latex/contrib/oberdiek/atbegshi.pdf
%
% Unpacking:
%    (a) If atbegshi.ins is present:
%           tex atbegshi.ins
%    (b) Without atbegshi.ins:
%           tex atbegshi.dtx
%    (c) If you insist on using LaTeX
%           latex \let\install=y% \iffalse meta-comment
%
% File: atbegshi.dtx
% Version: 2016/06/09 v1.18
% Info: At begin shipout hook
%
% Copyright (C) 2007-2011 by
%    Heiko Oberdiek <heiko.oberdiek at googlemail.com>
%    2016
%    https://github.com/ho-tex/oberdiek/issues
%
% This work may be distributed and/or modified under the
% conditions of the LaTeX Project Public License, either
% version 1.3c of this license or (at your option) any later
% version. This version of this license is in
%    http://www.latex-project.org/lppl/lppl-1-3c.txt
% and the latest version of this license is in
%    http://www.latex-project.org/lppl.txt
% and version 1.3 or later is part of all distributions of
% LaTeX version 2005/12/01 or later.
%
% This work has the LPPL maintenance status "maintained".
%
% This Current Maintainer of this work is Heiko Oberdiek.
%
% The Base Interpreter refers to any `TeX-Format',
% because some files are installed in TDS:tex/generic//.
%
% This work consists of the main source file atbegshi.dtx
% and the derived files
%    atbegshi.sty, atbegshi.pdf, atbegshi.ins, atbegshi.drv,
%    atbegshi-example1.tex, atbegshi-example2.tex,
%    atbegshi-test1.tex, atbegshi-test2.tex, atbegshi-test3.tex.
%
% Distribution:
%    CTAN:macros/latex/contrib/oberdiek/atbegshi.dtx
%    CTAN:macros/latex/contrib/oberdiek/atbegshi.pdf
%
% Unpacking:
%    (a) If atbegshi.ins is present:
%           tex atbegshi.ins
%    (b) Without atbegshi.ins:
%           tex atbegshi.dtx
%    (c) If you insist on using LaTeX
%           latex \let\install=y% \iffalse meta-comment
%
% File: atbegshi.dtx
% Version: 2016/06/09 v1.18
% Info: At begin shipout hook
%
% Copyright (C) 2007-2011 by
%    Heiko Oberdiek <heiko.oberdiek at googlemail.com>
%    2016
%    https://github.com/ho-tex/oberdiek/issues
%
% This work may be distributed and/or modified under the
% conditions of the LaTeX Project Public License, either
% version 1.3c of this license or (at your option) any later
% version. This version of this license is in
%    http://www.latex-project.org/lppl/lppl-1-3c.txt
% and the latest version of this license is in
%    http://www.latex-project.org/lppl.txt
% and version 1.3 or later is part of all distributions of
% LaTeX version 2005/12/01 or later.
%
% This work has the LPPL maintenance status "maintained".
%
% This Current Maintainer of this work is Heiko Oberdiek.
%
% The Base Interpreter refers to any `TeX-Format',
% because some files are installed in TDS:tex/generic//.
%
% This work consists of the main source file atbegshi.dtx
% and the derived files
%    atbegshi.sty, atbegshi.pdf, atbegshi.ins, atbegshi.drv,
%    atbegshi-example1.tex, atbegshi-example2.tex,
%    atbegshi-test1.tex, atbegshi-test2.tex, atbegshi-test3.tex.
%
% Distribution:
%    CTAN:macros/latex/contrib/oberdiek/atbegshi.dtx
%    CTAN:macros/latex/contrib/oberdiek/atbegshi.pdf
%
% Unpacking:
%    (a) If atbegshi.ins is present:
%           tex atbegshi.ins
%    (b) Without atbegshi.ins:
%           tex atbegshi.dtx
%    (c) If you insist on using LaTeX
%           latex \let\install=y\input{atbegshi.dtx}
%        (quote the arguments according to the demands of your shell)
%
% Documentation:
%    (a) If atbegshi.drv is present:
%           latex atbegshi.drv
%    (b) Without atbegshi.drv:
%           latex atbegshi.dtx; ...
%    The class ltxdoc loads the configuration file ltxdoc.cfg
%    if available. Here you can specify further options, e.g.
%    use A4 as paper format:
%       \PassOptionsToClass{a4paper}{article}
%
%    Programm calls to get the documentation (example):
%       pdflatex atbegshi.dtx
%       makeindex -s gind.ist atbegshi.idx
%       pdflatex atbegshi.dtx
%       makeindex -s gind.ist atbegshi.idx
%       pdflatex atbegshi.dtx
%
% Installation:
%    TDS:tex/generic/oberdiek/atbegshi.sty
%    TDS:doc/latex/oberdiek/atbegshi.pdf
%    TDS:doc/latex/oberdiek/atbegshi-example1.tex
%    TDS:doc/latex/oberdiek/atbegshi-example2.tex
%    TDS:doc/latex/oberdiek/test/atbegshi-test1.tex
%    TDS:doc/latex/oberdiek/test/atbegshi-test2.tex
%    TDS:doc/latex/oberdiek/test/atbegshi-test3.tex
%    TDS:source/latex/oberdiek/atbegshi.dtx
%
%<*ignore>
\begingroup
  \catcode123=1 %
  \catcode125=2 %
  \def\x{LaTeX2e}%
\expandafter\endgroup
\ifcase 0\ifx\install y1\fi\expandafter
         \ifx\csname processbatchFile\endcsname\relax\else1\fi
         \ifx\fmtname\x\else 1\fi\relax
\else\csname fi\endcsname
%</ignore>
%<*install>
\input docstrip.tex
\Msg{************************************************************************}
\Msg{* Installation}
\Msg{* Package: atbegshi 2016/06/09 v1.18 At begin shipout hook (HO)}
\Msg{************************************************************************}

\keepsilent
\askforoverwritefalse

\let\MetaPrefix\relax
\preamble

This is a generated file.

Project: atbegshi
Version: 2016/06/09 v1.18

Copyright (C) 2007-2011 by
   Heiko Oberdiek <heiko.oberdiek at googlemail.com>

This work may be distributed and/or modified under the
conditions of the LaTeX Project Public License, either
version 1.3c of this license or (at your option) any later
version. This version of this license is in
   http://www.latex-project.org/lppl/lppl-1-3c.txt
and the latest version of this license is in
   http://www.latex-project.org/lppl.txt
and version 1.3 or later is part of all distributions of
LaTeX version 2005/12/01 or later.

This work has the LPPL maintenance status "maintained".

This Current Maintainer of this work is Heiko Oberdiek.

The Base Interpreter refers to any `TeX-Format',
because some files are installed in TDS:tex/generic//.

This work consists of the main source file atbegshi.dtx
and the derived files
   atbegshi.sty, atbegshi.pdf, atbegshi.ins, atbegshi.drv,
   atbegshi-example1.tex, atbegshi-example2.tex,
   atbegshi-test1.tex, atbegshi-test2.tex, atbegshi-test3.tex.

\endpreamble
\let\MetaPrefix\DoubleperCent

\generate{%
  \file{atbegshi.ins}{\from{atbegshi.dtx}{install}}%
  \file{atbegshi.drv}{\from{atbegshi.dtx}{driver}}%
  \usedir{tex/generic/oberdiek}%
  \file{atbegshi.sty}{\from{atbegshi.dtx}{package}}%
  \usedir{doc/latex/oberdiek}%
  \file{atbegshi-example1.tex}{\from{atbegshi.dtx}{example1}}%
  \file{atbegshi-example2.tex}{\from{atbegshi.dtx}{example2}}%
  \usedir{doc/latex/oberdiek/test}%
  \file{atbegshi-test1.tex}{\from{atbegshi.dtx}{test1}}%
  \file{atbegshi-test2.tex}{\from{atbegshi.dtx}{test2}}%
  \file{atbegshi-test3.tex}{\from{atbegshi.dtx}{test3}}%
  \nopreamble
  \nopostamble
  \usedir{source/latex/oberdiek/catalogue}%
  \file{atbegshi.xml}{\from{atbegshi.dtx}{catalogue}}%
}

\catcode32=13\relax% active space
\let =\space%
\Msg{************************************************************************}
\Msg{*}
\Msg{* To finish the installation you have to move the following}
\Msg{* file into a directory searched by TeX:}
\Msg{*}
\Msg{*     atbegshi.sty}
\Msg{*}
\Msg{* To produce the documentation run the file `atbegshi.drv'}
\Msg{* through LaTeX.}
\Msg{*}
\Msg{* Happy TeXing!}
\Msg{*}
\Msg{************************************************************************}

\endbatchfile
%</install>
%<*ignore>
\fi
%</ignore>
%<*driver>
\NeedsTeXFormat{LaTeX2e}
\ProvidesFile{atbegshi.drv}%
  [2016/06/09 v1.18 At begin shipout hook (HO)]%
\documentclass{ltxdoc}
\usepackage{holtxdoc}[2011/11/22]
\begin{document}
  \DocInput{atbegshi.dtx}%
\end{document}
%</driver>
% \fi
%
%
% \CharacterTable
%  {Upper-case    \A\B\C\D\E\F\G\H\I\J\K\L\M\N\O\P\Q\R\S\T\U\V\W\X\Y\Z
%   Lower-case    \a\b\c\d\e\f\g\h\i\j\k\l\m\n\o\p\q\r\s\t\u\v\w\x\y\z
%   Digits        \0\1\2\3\4\5\6\7\8\9
%   Exclamation   \!     Double quote  \"     Hash (number) \#
%   Dollar        \$     Percent       \%     Ampersand     \&
%   Acute accent  \'     Left paren    \(     Right paren   \)
%   Asterisk      \*     Plus          \+     Comma         \,
%   Minus         \-     Point         \.     Solidus       \/
%   Colon         \:     Semicolon     \;     Less than     \<
%   Equals        \=     Greater than  \>     Question mark \?
%   Commercial at \@     Left bracket  \[     Backslash     \\
%   Right bracket \]     Circumflex    \^     Underscore    \_
%   Grave accent  \`     Left brace    \{     Vertical bar  \|
%   Right brace   \}     Tilde         \~}
%
% \GetFileInfo{atbegshi.drv}
%
% \title{The \xpackage{atbegshi} package}
% \date{2016/06/09 v1.18}
% \author{Heiko Oberdiek\thanks
% {Please report any issues at https://github.com/ho-tex/oberdiek/issues}\\
% \xemail{heiko.oberdiek at googlemail.com}}
%
% \maketitle
%
% \begin{abstract}
% This package is a modern reimplementation of package \xpackage{everyshi}
% without the burden of compatibility. It makes use of \eTeX's if available.
% Both \LaTeX\ and \plainTeX\ are supported.
% \end{abstract}
%
% \tableofcontents
%
% \section{Documentation}
%
% Package \xpackage{atbegshi} redefines \cs{shipout} to insert
% hooks for user code that is executed before the page is shipped out.
% The code may modify or even discard the output page.
% Three hooks are implemented:
% \begin{enumerate}
% \item A hook that is executed for every page, see\\
%       \cs{AtBeginShipout}
% \item A hook that is executed for the next page only, see\\
%       \cs{AtBeginShipoutNext}
% \item A hook that is only executed for the first page, see\\
%       \cs{AtBeginShipoutFirst}
% \end{enumerate}
% The hooks are executed in this order. The following three macros
% provide the user interface for adding code to these hooks:
%
% \begin{declcs}{AtBeginShipout} \M{code}\\
%   \cs{AtBeginShipoutBox}
% \end{declcs}
% Execute the \meta{code} for every page.
% The page contents is held in box register \cs{AtBeginShipoutBox}
% and may be modified. Use \cs{AtBeginShipoutDiscard} if you want to
% discard the page.
%
% \emph{Note:} Package \xpackage{everyshi} uses box register 255. With
% package \xpackage{atbegshi} you must use \cs{AtBeginShipoutBox} instead.
%
% If \LaTeX\ calls \cs{shipout} in \cs{@outputpage} (part of its output
% routine),
% the meaning of \cs{protect} is \cs{noexpand}. \LaTeX\ sets
% \cs{protect} to the appropriate \cs{@typeset@protect} in the box
% that is shipped out. This is too late for the hooks, they
% are called earlier in the redefined \cs{shipout}. Therefore
% package \xpackage{atbegshi} sets \cs{protect} to \cs{@typeset@protect}
% before it calls the hooks.
% (In \cs{EveryShipout} of package \xpackage{everyshi} the
% user is responsible for the correct setting of \cs{protect}.)
%
% \begin{declcs}{AtBeginShipoutNext} \M{code}
% \end{declcs}
% This reimplements package \xpackage{everyshi}'s \cs{AtNextShipout}.
% The \meta{code} is executed at shipout time of the next page only.
% It is just a convenience macro, it can be easily replaced by something
% like:
%\begin{quote}
%\begin{verbatim}
%\newcommand{\MyShipoutHook}{}%
%\AtBeginShipout{\MyShipoutHook}
%\gdef\MyShipoutHook{%
%  ... do something with next page ...
%  \gdef\MyShipoutHook{}%
% }
%\end{verbatim}
%\end{quote}
% (This can be necessary, if hook order does matter).
%
% \begin{declcs}{AtBeginShipoutFirst} \M{code}
% \end{declcs}
% This reimplements \LaTeX's \cs{AtBeginDvi}. This hook is
% usually used for \cs{special} commands that include PostScript
% header files. The \cs{code} is directly executed in a \cs{vbox}
% that is put at the beginning of the output page. Dealing with
% the output box \cs{AtBeginShipoutBox} is not necessary and
% not permitted here.
%
% \begin{declcs}{AtBeginShipoutDiscard}
% \end{declcs}
% This macro notifies package \xpackage{atbegshi} that
% the output page is discarded. The remaining hook code
% and the remaining hooks are not executed and the page is thrown away.
% Also \cs{deadcycles} is cleared to zero like an ordinary \cs{shipout}
% would do.
%
% \begin{declcs}{AtBeginShipoutInit}
% \end{declcs}
% Usually the redefinition of \cs{shipout} is delayed by
% \cs{AtBeginDocument} (if this macro exists). This can be
% too late, if other packages also redefines \cs{shipout}
% and the order does matter.
% \cs{AtBeginShipoutInit} forces the immediate redefinition
% of \cs{shipout}.
%
% \begin{declcs}{AtBeginShipoutAddToBox} \M{stuff}\\
%   \cs{AtBeginShipoutAddToBoxForeground} \M{stuff}
% \end{declcs}
% A quite common use case is the addition of \cs{special}
% or other whatsits to the page output box. Macro
% \cs{AtBeginShipoutAddToBox} puts \meta{stuff} in a
% box with zeroed dimensions. The box with the \meta{stuff}
% is put in the upper left corner of the shipout box
% \cs{AtBeginShipoutBox}. Macro \cs{AtBeginShipoutAddToBox}
% puts the \meta{stuff} in the background, the other macro
% \cs{AtBeginShipoutAddToBoxForeground} in the foreground
% after the original shipout box contents is set.
%
% A void shipout box (that means a discarded page)
% remains void that means \meta{stuff} is ignored in this case.
% The box type of \cs{AtBeginShipoutBox} is preserved.
% Also the box nesting level for the original contents
% of \cs{AtBeginShipoutBox} remains, for example, to avoid trouble
% with links across pages in case of \hologo{pdfTeX}.
%
% \begin{declcs}{AtBeginShipoutUpperLeft} \M{background material}
% \end{declcs}
% This is a macro that puts material in the background of
% box \cs{AtBeginShipoutBox}.
% The \meta{background material} is set in an \cs{hbox}, the
% reference point is the upper left corner of the output page.
% In case of \pdfTeX\ in PDF mode, the settings of
% \cs{pdfhorigin} and \cs{pdfvorigin} are respected.
%
% The macro \cs{AtBeginShipoutUpperLeft} is intended to be used
% in one of the hook setting macros, such as
% \cs{AtBeginShipout}, \cs{AtBeginShipoutFirst}, or
% \cs{AtBeginShipoutNext}.
%
% For \LaTeX\ users the \meta{background material} is set
% inside a \texttt{picture} environment:
% \begin{quote}
%   |\begin{picture}(0,0)|\\
%   \mbox{}\quad |\setlength{\unitlength}{1pt}%|\\
%   \mbox{}\quad \meta{background material}\\
%   |\end{picture}|
% \end{quote}
%
% \begin{declcs}{AtBeginShipoutUpperLeftForeground} \M{foreground material}
% \end{declcs}
% See \cs{AtBeginShipoutUpperLeft}. The difference is that the material
% is put in the foreground.
%
% \begin{declcs}{AtBeginShipoutOriginalShipout} \meta{box}
% \end{declcs}
% It stores the meaning of \cs{shipout} at the time this package
% is loaded.
%
% \begin{declcs}{AtBeginShipoutBoxWidth}\\
%   \cs{AtBeginShipoutBoxHeight}\\
%   \cs{AtBeginShipoutBoxDepth}\\
% \end{declcs}
% These macros store the dimensions of the output box \cs{AtBeginShipoutBox}
% before the original shipout is called. If \cs{shipout} is not
% redefined before the package loading or the box dimensions
% are not changed by the redefined \cs{shipout}, these macros
% contain the dimensions of the shipout box. These values can be
% remembered by \cs{label} and \cs{ref}. For example, this is done
% by the package module \xpackage{zref-pagelayout} of project \xpackage{zref}.
% The dimensions of the shipout page can be used in some \hologo{TeX}
% engines (\hologo{pdfTeX} in PDF mode, \hologo{XeTeX}) to calculate
% the media size of the shipout page if \cs{pdfpagewidth} and
% \cs{pdfpageheight} are not set.
%
% \subsection{Examples}
%
% \subsubsection{Example: circle in background}
%
% In this example we put a circle in the background in the middle
% of the paper.
%
%    \begin{macrocode}
%<*example1>
%    \end{macrocode}
%    \begin{macrocode}
\documentclass[a4paper]{article}
\usepackage{color}
\usepackage{atbegshi}
%    \end{macrocode}
%    Package \xpackage{picture} makes life a little easier, because
%    we can now also use length specifications in |picture|'s commands.
%    \begin{macrocode}
\usepackage{picture}
%    \end{macrocode}
%    Now we draw the circle in the middle of the paper.
%    \cs{put} moves downwards, because the origin is at the top
%    of the page, not at its bottom.
%    \begin{macrocode}
\AtBeginShipout{%
  \AtBeginShipoutUpperLeft{%
    \put(0.5\paperwidth,-0.5\paperheight){\circle{10}}%
  }%
}
\begin{document}
\section{Hello World}
\newpage
\AtBeginShipoutNext{%
  \AtBeginShipoutUpperLeft{%
    \color{red}%
    \put(0,-0.5\paperheight){\line(1,0){\paperwidth}}%
    \put(0.5\paperwidth, 0){\line(0,-1){\paperheight}}%
  }%
}
Only on this page we add a red cross.
\newpage
This page has the circle only.
\par
\vspace{\fill}
The next page will be discarded.
\newpage
\AtBeginShipoutNext{%
  \AtBeginShipoutDiscard
}
This page is discarded.
\newpage
The last page.
\end{document}
%    \end{macrocode}
%    \begin{macrocode}
%</example1>
%    \end{macrocode}
%
% \subsubsection{Example: adding TrimBox for dvipdfmx}
%
% Now an example from ``real life'' follows. Someone from the mailing  list
% for dvipdfmx wants to put a TrimBox on every page. If we use
% \cs{AtBeginShipout}, we have to put the \cs{special} inside the
% box \cs{AtBeginShipoutBox} that gets shipped out.
%    \begin{macrocode}
%<*example2>
\documentclass{minimal}
\usepackage{atbegshi}
\usepackage[
  dvipdfm,
  paperwidth=630bp,
  paperheight=810bp
]{geometry}
\AtBeginShipout{%
  \setbox\AtBeginShipoutBox=\hbox{%
    \special{pdf: put @thispage <</TrimBox[9 9 621 801]>>}%
    \box\AtBeginShipoutBox
  }%
}
\begin{document}
  First page
  \newpage
  Second page
\end{document}
%</example2>
%    \end{macrocode}
% Remember, in \cs{AtBeginShipoutBoxFirst} the \cs{setbox} wrapper code
% is implicitly given and the \cs{special} is used directly.
%
% \section{Method of \cs{shipout} overloading}
%
% \subsection{\cs{shipout}}
%
% The \TeX\ primitive command \cs{shipout} takes a box specification
% and puts the box as a new page in the output file.
% There are two kinds of box specifications:
% \begin{description}
% \item[Direct boxes:]
%   They are given by \cs{hbox}, \cs{vbox}, or \cs{vtop},\\
%   e.g. |\shipout\hbox{Hello World}|.
% \item[Indirect boxes:]
%   \cs{box} or \cs{copy} references a box register by number.
%   The box register contains the contents of the box.
%
%   \emph{Note:} \cs{box} also clears the box register globally.
% \end{description}
% Then we have to differentiate between void and empty boxes:
% \begin{description}
% \item[Void:]
%   Initially or after \cs{box} there is no box in the box register.
%   In this cases the box register is not empty, but \emph{void}.
% \item[Empty:]
%   A box with empty contents, such as
%   |\hbox{}| (= \cs{null}) or |\vbox{}| is an \emph{empty \cs{hbox}}
%   or \emph{empty \cs{vbox}}. If a box register holds such a
%   box, the box still exists, therefore the box register is
%   \emph{not void}.
% \end{description}
%
% \subsection{\cs{afterassignment}}
%
% We want to overload \cs{shipout} to do something with the box.
% It is quite impossible to do this reliable by catching the box
% using macro arguments. The
% variety of box specifications is too large, Examples:
% \begin{quote}
%   |\shipout\null|\\
%   |\shipout\vbox{...}|\\
%   |\shipout\vtop\bgroup ...\egroup|\\
%   |\shipout\box255|
% \end{quote}
% Even worse, the braces don't need to be balanced:
% \begin{quote}
%   |\shipout\hbox\bgroup}|\\
%   |\shipout\vbox{\egroup|
% \end{quote}
% Happily TeX provides a reliable way via \cs{afterassignment}.
% It takes a macro name and executes it just after the assignment.
%
% Now we can redefine \cs{shipout}. The box specification that
% follows \cs{shipout} is catched by \cs{setbox}. This is an
% assignment to a box register. \cs{afterassignment}
% notifies \TeX, that we want to call \cs{@test} right after
% the assignment:
%\begin{quote}
%\begin{verbatim}
%\shipout :=
%  \afterassignment\@test
%  \setbox\mybox=
%\end{verbatim}
%\end{quote}
% We have seen different box specifications. Indirect boxes are easy
% to understand:
% \begin{quote}
%   |\shipout\box0| $\Rightarrow$ |\setbox\mybox=\box0 |\fbox{\cs{@test}}
% \end{quote}
% However direct boxes can have arbitrary contents with lots of
% other assignments. It would be quite inpredictable if \TeX\ would
% put |\@test| after the first of such an assignment or after the
% box specification if the box lacks of assignments.
% Therefore \TeX\ puts |\@test| right at the beginning of the
% box specification, e.g:
% \begin{quote}
%   |\shipout\hbox{Hello World}|\\
%   $\Rightarrow$
%   |\setbox\mybox=\hbox{|\fbox{\cs{@test}}| Hello World}|
% \end{quote}
%
% \subsection{Test for direct or indirect boxes}
%
% Now we want to execute \cs{@test}, but where are we?
% We can be after the completed box assignment, if \cs{shipout}
% was called with an indirect box. Or we are right at the beginning
% of a direct box.
%
% \subsubsection{With \eTeX}
%
% With the \eTeX's extensions the answer is very easy:
% Being inside the direct box means that we are inside a new group.
% The new primitive command \cs{currentgrouplevel} tells how
% deeply the groups are currently nested. Macro \cs{@test} just
% compares the previously stored group level with the current one:
%\begin{quote}
%\begin{verbatim}
%\shipout :=
%  \edef\saved@grouplevel{\number\currentgrouplevel}
%  \afterassignment\@test
%  \setbox\mybox=
%
%\@test :=
%  \ifnum\saved@grouplevel=\currentgrouplevel
%    % case: indirect box, the assignment is completed
%    \@output
%  \else
%    % case: direct box, we are inside the box
%    \aftergroup\@outbox
%  \fi
%\end{verbatim}
%\end{quote}
%
% \subsubsection{Without \eTeX}
%
% Life becomes complicate without \eTeX. We cannot ask the
% group level. However, if we are inside a direct box,
% the box register \cs{mybox} is not yet changed by \cs{setbox}.
% Thus we need a special initial value and compare
% it in \cs{@test} with the current value of the box.
%
% What can be used as initial value? Arbitrary box contents cannot
% be compared. \TeX\ only tells us a few properties:
% \begin{itemize}
% \item Box type:\cs{ifhbox}, \cs{ifvbox}
% \item Dimensions: \cs{wd}, \cs{ht}, \cs{dp}
% \item Voidness: \cs{ifvoid}
% \end{itemize}
% Unhappily all these qualities even combined are not sufficient
% for constructing an initial box value, because \cs{shipout} can be
% called with a box that is accidently just the same as the choosen
% initial value.
%
% Nevertheless we have two alternatives for an initial value:
% \begin{itemize}
% \item A box of some type with some funny settings that are
% unlikely to occur in real life, e.g a height of |4911sp-\maxdimen|.
% \item A void box.
% \end{itemize}
% A collision between this initial value and an indirect \cs{shipout}
% box with just the same value is possible.
% Then \cs{@test} will make a wrong decision that it is executed
% inside a direct box and delays \cs{@output} by \cs{aftergroup}.
% Thus \cs{@output} is not called at the place we want.
% In contrary, the result is an uncertainty about the place:
% \begin{itemize}
% \item \cs{shipout} is used in a group that perhaps closes some
%   pages later. A bad place for \cs{@output}.
% \item Without a surrounding group \cs{aftergroup} effectively kills
%   its argument.
% \end{itemize}
% In the first case of a box with special dimensions we can even loose
% the page. However in the case of the void box, this effect is even
% desired, because the original \cs{shipout} does not output void boxes.
% All we have to do is to ensure that our box \cs{mybox} is always
% void except for the phase when the overloaded \cs{shipout} is executed.
% And secondly we must keep this semantics of \cs{shipout} for the void case
% in our macros, namely \cs{@output}.
%\begin{quote}
%\begin{verbatim}
%\shipout :=
%  % trick to get a void box \mybox
%  \begingroup
%    \setbox\mybox=\box\mybox
%  \endgroup
%  \afterassignment\@test
%  \setbox\mybox=
%
%\@test :=
%  \ifvoid\mybox
%    \aftergroup\@output
%  \else
%    \@output
%  \fi
%\end{verbatim}
%\end{quote}
% The nasty case is |\shipout\box\voidb@x| where the indirect box
% is void and that must not generate an output page.
% If a surrounding group is missing the output is ignored
% because of \cs{aftergroup}.
% Otherwise \@output is called some time
% later when the surrounding group closes.
% But \cs{mybox} is void outside the execution phase
% of the redefined \cs{shipout}. Also \cs{@output} checks for a
% void box and cancels the page output.
% The disadvantage remains that the hook in \cs{@output} is called
% for a page that will not be output.
%
% \subsubsection{\cs{lastkern} method}
%
% At the beginning of a new box, there is no \cs{kern}, the contents
% of the box is still empty and \cs{lastkern} returns 0\,pt.
% This can be used to distinguish between direct and indirect boxes:
% We execute \cs{setbox} in a box with a preceding non-zero kern.
% After an indirect box, \cs{lastkern} sees this kern, otherwise
% it returns 0\,pt.
%\begin{quote}
%\begin{verbatim}
%\shipout :=
%  \begingroup
%    \setbox\mybox=\hbox\bgroup
%      \kern1pt
%      \afterassignment\shipout@test
%      \global\setbox\mybox=
%\@test :=
%  \ifdim\lastkern=0pt
%    % direct box
%    \aftergroup\egroup
%    \aftergroup\endgroup
%    \aftergroup\@output
%  \else
%    \egroup
%    \endgroup
%    \@output
%  \fi
%\end{verbatim}
%\end{quote}
% We have two \cs{setbox} commands. The first creates a controlled
% context box where we can safely insert a \cs{kern}. We get rid of
% this temporarily used context box by putting the local \cs{setbox}
% in a group.
%
% After the group we want to have our shipout box in \cs{mybox}.
% Therefore we use a global assignment here.
%
% \subsection{Output}
%
% With or without \eTeX\ we ensure the original behaviour of
% \cs{shipout} that void boxes do not generate output pages.
%
% Now we can place the hook \cs{@hook} for the user code that
% wants to manipulate the output box.
%\begin{quote}
%\begin{verbatim}
%\@output :=
%  \ifvoid\mybox
%    % cancel output of void box
%  \else
%    \@hook
%    \ifvoid\mybox
%      % user code in \@hook could has voided the box
%    \else
%      \original@shipout\box\mybox
%    \fi
%  \fi
%\end{verbatim}
%\end{quote}
%
% \subsection{Separate box register}
%
% So far we have said nothing about the box number of \cs{mybox}.
% The following case that outputs the same page twice
% shows that we are not free in the use
% of the box register:
% \begin{quote}
%   |\shipout\copy<num> \shipout\box<num>|
% \end{quote}
% We manipulate the box by the hook and without \eTeX\ the box
% must even be voided. However, the use case above requires that
% the box contents does not change at all. Therefore we must
% reserve a separate box register to avoid collisions with
% user box registers.
%
% \emph{Note:} Box register number 255 is special for the output
% routine, because TeX complains if this box is not voided
% by the output routine.
% However, this requirement does not apply to \cs{shipout} at all.
% In fact \cs{shipout} does not change any box register. This is
% usually done by a call of \cs{box}, but the output routine
% can do it later \emph{after} invoking of \cs{shipout}.
%
% \subsection{Summary}
%
% \subsubsection{With \eTeX}
%
% Putting the pieces together we get for \eTeX:
%\begin{quote}
%\begin{verbatim}
%\newbox\mybox
%\let\original@shipout\shipout
%
%\shipout :=
%  \edef\saved@grouplevel{\number\currentgrouplevel}
%  \afterassignment\@test
%  \setbox\mybox=
%
%\@test :=
%  \ifnum\saved@grouplevel<\currentgrouplevel
%    \expandafter\aftergroup
%  \fi
%  \@output
%
%\@output :=
%  \ifvoid\mybox
%    % cancel output of void box
%  \else
%    \@hook
%    \ifvoid\mybox
%      % user code in \@hook could have voided the box
%    \else
%      \original@shipout\box\mybox
%    \fi
%  \fi
%\end{verbatim}
%\end{quote}
%
% \subsubsection{Without \eTeX, traditional way}
%
% And for \TeX\ without \eTeX:
%\begin{quote}
%\begin{verbatim}
%\newbox\mybox
%\begingroup
%   \setbox\mybox=\box\mybox % ensure \mybox is void
%\endgroup
%\let\original@shipout\shipout
%
%\shipout :=
%  % trick to get a void box \mybox
%  \begingroup
%    \setbox\mybox=\box\mybox
%  \endgroup
%  \afterassignment\@test
%  \setbox\mybox=
%
%\@test :=
%  \ifvoid\mybox
%    \expandafter\aftergroup
%  \fi
%  \@output
%
%\@output :=
%  \ifvoid\mybox
%    % cancel output of void box
%  \else
%    \@hook
%    \ifvoid\mybox
%      % user code in \@hook could have voided the box
%    \else
%      \original@shipout\box\mybox
%    \fi
%  \fi
%\end{verbatim}
%\end{quote}
%
% \subsubsection{\cs{lastkern} method}
%
% And for \TeX\ without \eTeX\ using the \cs{lastkern} method:
%\begin{quote}
%\begin{verbatim}
%\newbox\mybox
%\let\original@shipout\shipout
%
%\shipout :=
%  \begingroup
%  \setbox\mybox=\hbox\bgroup
%  \kern1pt
%  \afterassignment\@test
%  \setbox\mybox=
%
%\@test :=
%  \ifdim\lastkern=0pt
%    \expandafter\aftergroup
%  \fi
%  \@output
%
%\@output :=
%  \egroup
%  \endgroup
%  \ifvoid\mybox
%    % cancel output of void box
%  \else
%    \@hook
%    \ifvoid\mybox
%      % user code in \@hook could have voided the box
%    \else
%      \original@shipout\box\mybox
%    \fi
%  \fi
%\end{verbatim}
%\end{quote}
%
%
% \StopEventually{
% }
%
% \section{Implementation}
%
% Package \xpackage{atbegshi} uses \eTeX's \cs{currentgrouplevel},
% if it is available. Otherwise the \cs{lastkern} method is used.
%
%    \begin{macrocode}
%<*package>
%    \end{macrocode}
%
% \subsection{Reload check and package identification}
%    Reload check, especially if the package is not used with \LaTeX.
%    \begin{macrocode}
\begingroup\catcode61\catcode48\catcode32=10\relax%
  \catcode13=5 % ^^M
  \endlinechar=13 %
  \catcode35=6 % #
  \catcode39=12 % '
  \catcode44=12 % ,
  \catcode45=12 % -
  \catcode46=12 % .
  \catcode58=12 % :
  \catcode64=11 % @
  \catcode123=1 % {
  \catcode125=2 % }
  \expandafter\let\expandafter\x\csname ver@atbegshi.sty\endcsname
  \ifx\x\relax % plain-TeX, first loading
  \else
    \def\empty{}%
    \ifx\x\empty % LaTeX, first loading,
      % variable is initialized, but \ProvidesPackage not yet seen
    \else
      \expandafter\ifx\csname PackageInfo\endcsname\relax
        \def\x#1#2{%
          \immediate\write-1{Package #1 Info: #2.}%
        }%
      \else
        \def\x#1#2{\PackageInfo{#1}{#2, stopped}}%
      \fi
      \x{atbegshi}{The package is already loaded}%
      \aftergroup\endinput
    \fi
  \fi
\endgroup%
%    \end{macrocode}
%    Package identification:
%    \begin{macrocode}
\begingroup\catcode61\catcode48\catcode32=10\relax%
  \catcode13=5 % ^^M
  \endlinechar=13 %
  \catcode35=6 % #
  \catcode39=12 % '
  \catcode40=12 % (
  \catcode41=12 % )
  \catcode44=12 % ,
  \catcode45=12 % -
  \catcode46=12 % .
  \catcode47=12 % /
  \catcode58=12 % :
  \catcode64=11 % @
  \catcode91=12 % [
  \catcode93=12 % ]
  \catcode123=1 % {
  \catcode125=2 % }
  \expandafter\ifx\csname ProvidesPackage\endcsname\relax
    \def\x#1#2#3[#4]{\endgroup
      \immediate\write-1{Package: #3 #4}%
      \xdef#1{#4}%
    }%
  \else
    \def\x#1#2[#3]{\endgroup
      #2[{#3}]%
      \ifx#1\@undefined
        \xdef#1{#3}%
      \fi
      \ifx#1\relax
        \xdef#1{#3}%
      \fi
    }%
  \fi
\expandafter\x\csname ver@atbegshi.sty\endcsname
\ProvidesPackage{atbegshi}%
  [2016/06/09 v1.18 At begin shipout hook (HO)]%
%    \end{macrocode}
%
% \subsection{Catcodes}
%
%    \begin{macrocode}
\begingroup\catcode61\catcode48\catcode32=10\relax%
  \catcode13=5 % ^^M
  \endlinechar=13 %
  \catcode123=1 % {
  \catcode125=2 % }
  \catcode64=11 % @
  \def\x{\endgroup
    \expandafter\edef\csname AtBegShi@AtEnd\endcsname{%
      \endlinechar=\the\endlinechar\relax
      \catcode13=\the\catcode13\relax
      \catcode32=\the\catcode32\relax
      \catcode35=\the\catcode35\relax
      \catcode61=\the\catcode61\relax
      \catcode64=\the\catcode64\relax
      \catcode123=\the\catcode123\relax
      \catcode125=\the\catcode125\relax
    }%
  }%
\x\catcode61\catcode48\catcode32=10\relax%
\catcode13=5 % ^^M
\endlinechar=13 %
\catcode35=6 % #
\catcode64=11 % @
\catcode123=1 % {
\catcode125=2 % }
\def\TMP@EnsureCode#1#2{%
  \edef\AtBegShi@AtEnd{%
    \AtBegShi@AtEnd
    \catcode#1=\the\catcode#1\relax
  }%
  \catcode#1=#2\relax
}
\TMP@EnsureCode{40}{12}% (
\TMP@EnsureCode{41}{12}% )
\TMP@EnsureCode{44}{12}% ,
\TMP@EnsureCode{45}{12}% -
\TMP@EnsureCode{47}{12}% /
\TMP@EnsureCode{46}{12}% .
\TMP@EnsureCode{58}{12}% :
\TMP@EnsureCode{91}{12}% [
\TMP@EnsureCode{93}{12}% ]
\TMP@EnsureCode{94}{7}% ^(superscript)
\TMP@EnsureCode{96}{12}% `
\edef\AtBegShi@AtEnd{\AtBegShi@AtEnd\noexpand\endinput}
%    \end{macrocode}
%
% \subsection{Preparations}
%
%    \begin{macrocode}
\begingroup\expandafter\expandafter\expandafter\endgroup
\expandafter\ifx\csname RequirePackage\endcsname\relax
  \def\TMP@RequirePackage#1[#2]{%
    \begingroup\expandafter\expandafter\expandafter\endgroup
    \expandafter\ifx\csname ver@#1.sty\endcsname\relax
      \input #1.sty\relax
    \fi
  }%
  \TMP@RequirePackage{infwarerr}[2007/09/09]%
  \TMP@RequirePackage{ltxcmds}[2010/03/01]%
\else
  \RequirePackage{infwarerr}[2007/09/09]%
  \RequirePackage{ltxcmds}[2010/03/01]%
\fi
%    \end{macrocode}
%
%    \begin{macro}{\AtBegShi@CheckDefinable}
%    \begin{macrocode}
\begingroup\expandafter\expandafter\expandafter\endgroup
\expandafter\ifx\csname @ifdefinable\endcsname\relax
  \def\AtBegShi@CheckDefinable#1{%
    \ifcase\ifx#1\relax
             \ltx@one
           \else
             \ifx#1\@undefined
               \ltx@one
             \else
               \ltx@zero
             \fi
           \fi
      \@PackageError{atbegshi}{%
        \string#1\space is already defined%
      }\@ehd
    \fi
  }%
\else
  \def\AtBegShi@CheckDefinable#1{%
    \@ifdefinable{#1}{}%
  }%
\fi
%    \end{macrocode}
%    \end{macro}
%
%    \begin{macro}{\ifAtBegShi@Discarded}
%    \begin{macrocode}
\ltx@newif\ifAtBegShi@Discarded
%    \end{macrocode}
%    \end{macro}
%    \begin{macro}{\AtBeginShipoutDiscard}
%    \begin{macrocode}
\AtBegShi@CheckDefinable\AtBeginShipoutDiscard
\def\AtBeginShipoutDiscard{%
  \deadcycles=\ltx@zero
  \global\AtBegShi@Discardedtrue
}
%    \end{macrocode}
%    \end{macro}
%
%    \begin{macrocode}
\begingroup\expandafter\expandafter\expandafter\endgroup
\expandafter\ifx\csname currentgrouplevel\endcsname\relax
  \catcode`X=9  % ignore
  \catcode`E=14 % comment
\else
  \catcode`X=14 % comment
  \catcode`E=9  % ignore
\fi
%    \end{macrocode}
%
%    \begin{macro}{\AtBegShi@Shipout}
%    \begin{macrocode}
\def\AtBegShi@Shipout{%
X \begingroup
X \setbox\AtBeginShipoutBox=\hbox\bgroup
X \kern\p@
E \edef\AtBegShi@GroupLevel{\number\currentgrouplevel}%
  \afterassignment\AtBegShi@Test
X \global
  \setbox\AtBeginShipoutBox=%
}
%    \end{macrocode}
%    \end{macro}
%
%    \begin{macro}{\AtBegShi@Test}
%    \begin{macrocode}
\def\AtBegShi@Test{%
X \ifdim\lastkern=0pt %
E \ifnum\AtBegShi@GroupLevel<\currentgrouplevel
    \expandafter\aftergroup
  \fi
  \AtBegShi@Output
}
%    \end{macrocode}
%    \end{macro}
%
%    \begin{macro}{\AtBegShi@Output}
%    \begin{macrocode}
\def\AtBegShi@Output{%
X \egroup
X \endgroup
  \ifvoid\AtBeginShipoutBox
    \@PackageWarning{atbegshi}{Ignoring void shipout box}%
  \else
    \let\AtBegShi@OrgProtect\protect
    \csname set@typeset@protect\endcsname
    \global\AtBegShi@Discardedfalse
    \AtBegShi@Hook
    \expandafter\gdef\expandafter\AtBegShi@HookNext
    \expandafter{\expandafter}%
    \AtBegShi@HookNext
    \ifAtBegShi@Discarded
      \@PackageInfoNoLine{atbegshi}{Shipout page discarded}%
      \global\AtBegShi@Discardedfalse
      \begingroup
        \setbox\AtBeginShipoutBox\box\AtBeginShipoutBox
      \endgroup
      \let\protect\AtBegShi@OrgProtect
    \else
      \AtBegShi@First
      \let\protect\AtBegShi@OrgProtect
      \AtBegShi@GetBoxSize\AtBeginShipoutBox
      \ltx@ifundefined{AtNextShipout}{%
      }{%
        \AtNextShipout{\AtBegShi@GetBoxSize\@cclv}%
      }%
      \AtBeginShipoutOriginalShipout\box\AtBeginShipoutBox
    \fi
  \fi
}
%    \end{macrocode}
%    \end{macro}
%
%    \begin{macro}{\AtBegShi@GetBoxSize}
%    \begin{macrocode}
\def\AtBegShi@GetBoxSize#1{%
  \xdef\AtBeginShipoutBoxWidth{\the\wd#1}%
  \xdef\AtBeginShipoutBoxHeight{\the\ht#1}%
  \xdef\AtBeginShipoutBoxDepth{\the\dp#1}%
}
%    \end{macrocode}
%    \end{macro}
%    \begin{macro}{\AtBeginShipoutBoxWidth}
%    \begin{macrocode}
\def\AtBeginShipoutBoxWidth{0pt}
%    \end{macrocode}
%    \end{macro}
%    \begin{macro}{\AtBeginShipoutBoxHeight}
%    \begin{macrocode}
\def\AtBeginShipoutBoxHeight{0pt}
%    \end{macrocode}
%    \end{macro}
%    \begin{macro}{\AtBeginShipoutBoxDepth}
%    \begin{macrocode}
\def\AtBeginShipoutBoxDepth{0pt}
%    \end{macrocode}
%    \end{macro}
%
%    \begin{macrocode}
\catcode`\X=11 %
\catcode`\E=11 %
%    \end{macrocode}
%
%    \begin{macro}{\AtBegShi@First}
%    \begin{macrocode}
\def\AtBegShi@First{%
  \ifx\AtBegShi@HookFirst\ltx@empty
  \else
    \AtBeginShipoutAddToBox{\AtBegShi@HookFirst}%
  \fi
  \global\let\AtBegShi@First\ltx@empty
  \global\let\AtBeginShipoutFirst\AtBegShi@FirstDisabled
}
%    \end{macrocode}
%    \end{macro}
%
%    \begin{macro}{\AtBegShi@Hook}
%    \begin{macrocode}
\gdef\AtBegShi@Hook{}
%    \end{macrocode}
%    \end{macro}
%    \begin{macro}{\AtBegShi@HookNext}
%    \begin{macrocode}
\gdef\AtBegShi@HookNext{}
%    \end{macrocode}
%    \end{macro}
%    \begin{macro}{\AtBegShi@HookFirst}
%    \begin{macrocode}
\gdef\AtBegShi@HookFirst{}
%    \end{macrocode}
%    \end{macro}
%    \begin{macro}{\AtBeginShipout}
%    \begin{macrocode}
\AtBegShi@CheckDefinable\AtBeginShipout
\def\AtBeginShipout{%
  \AtBegShi@AddHook\AtBegShi@Hook
}
%    \end{macrocode}
%    \end{macro}
%    \begin{macro}{\AtBeginShipoutNext}
%    \begin{macrocode}
\AtBegShi@CheckDefinable\AtBeginShipoutNext
\def\AtBeginShipoutNext{%
  \AtBegShi@AddHook\AtBegShi@HookNext
}
%    \end{macrocode}
%    \end{macro}
%    \begin{macro}{\AtBeginShipoutFirst}
%    \begin{macrocode}
\AtBegShi@CheckDefinable\AtBeginShipoutFirst
\def\AtBeginShipoutFirst{%
  \AtBegShi@AddTo\AtBegShi@HookFirst
}
%    \end{macrocode}
%    \end{macro}
%    \begin{macro}{\AtBegShi@FirstDisabled}
%    \begin{macrocode}
\long\def\AtBegShi@FirstDisabled#1{%
  \@PackageWarning{atbegshi}{%
    First page is already shipped out, ignoring\MessageBreak
    \string\AtBeginShipoutFirst
  }%
}
%    \end{macrocode}
%    \end{macro}
%
%    \begin{macro}{\AtBegShi@AddTo}
%    \begin{macrocode}
\begingroup\expandafter\expandafter\expandafter\endgroup
\expandafter\ifx\csname g@addto@macro\endcsname\relax
  \long\def\AtBegShi@AddTo#1#2{%
    \begingroup
      \toks\ltx@zero\expandafter{#1#2}%
      \xdef#1{\the\toks\ltx@zero}%
    \endgroup
  }%
\else
  \let\AtBegShi@AddTo\g@addto@macro
\fi
%    \end{macrocode}
%    \end{macro}
%
%    \begin{macro}{\AtBegShi@AddHook}
%    \begin{macrocode}
\long\def\AtBegShi@AddHook#1#2{%
  \AtBegShi@AddTo#1{\AtBegShi@Item{#2}}%
}
%    \end{macrocode}
%    \end{macro}
%
%    \begin{macro}{\AtBegShi@Item}
%    \begin{macrocode}
\long\def\AtBegShi@Item#1{%
  \ifAtBegShi@Discarded
  \else
    #1%
    \ifAtBegShi@Discarded
    \else
      \ifvoid\AtBeginShipoutBox
        \@PackageWarning{atbegshi}{%
          Shipout box was voided by hook,\MessageBreak
          ignoring shipout box%
        }%
        \AtBeginShipoutDiscard
      \fi
    \fi
  \fi
}
%    \end{macrocode}
%    \end{macro}
%
%    \begin{macro}{\AtBeginShipoutInit}
%    \begin{macrocode}
\AtBegShi@CheckDefinable\AtBeginShipoutInit
\def\AtBeginShipoutInit{%
  \ltx@IfUndefined{newbox}{%
    \@PackageError{atbegshi}{%
      \string\AtBeginShipoutInit\space failed\MessageBreak
       because of missing \expandafter\string\csname newbox\endcsname
    }\@ehc
  }{%
    \csname newbox\endcsname\AtBeginShipoutBox
    \AtBegShi@CheckDefinable\AtBeginShipoutOriginalShipout
    \global\let\AtBeginShipoutOriginalShipout\shipout
    \global\let\shipout\AtBegShi@Shipout
  }%
  \gdef\AtBeginShipoutInit{}%
}
%    \end{macrocode}
%    \end{macro}
%    \begin{macrocode}
\begingroup\expandafter\expandafter\expandafter\endgroup
\expandafter\ifx\csname AtBeginDocument\endcsname\relax
  \AtBeginShipoutInit
\else
  \AtBeginDocument{\AtBeginShipoutInit}%
\fi
%    \end{macrocode}
%
% \subsection{Additions to the shipout box}
%
%    \begin{macro}{\AtBeginShipoutAddToBox}
%    \begin{macrocode}
\def\AtBeginShipoutAddToBox#1{%
  \ifhbox\AtBeginShipoutBox
    \edef\AtBegShi@restore{%
      \hfuzz=\the\hfuzz\relax
      \hbadness=\the\hbadness\relax
    }%
    \hfuzz=1073741823sp\relax
    \hbadness=2147483647\relax
    \setbox\AtBeginShipoutBox=\hbox to \wd\AtBeginShipoutBox{%
      \setbox\ltx@zero=\hbox{%
        \begingroup
          \AtBegShi@restore
          #1%
        \endgroup
      }%
      \wd\ltx@zero=0pt\relax
      \ht\ltx@zero=0pt\relax
      \dp\ltx@zero=0pt\relax
      \raise\ht\AtBeginShipoutBox\copy\ltx@zero
      \unhcopy\AtBeginShipoutBox
    }%
    \AtBegShi@restore
  \else
    \ifvbox\AtBeginShipoutBox
      \edef\AtBegShi@restore{%
        \vfuzz=\the\vfuzz\relax
        \vbadness=\the\vbadness\relax
        \dimen\ltx@zero=\the\dimen\ltx@zero\relax
      }%
      \edef\AtBegShi@restorebox{%
        \ht\AtBeginShipoutBox=\the\ht\AtBeginShipoutBox\relax
        \dp\AtBeginShipoutBox=\the\dp\AtBeginShipoutBox\relax
      }%
      \vfuzz=1073741823sp\relax
      \vbadness=2147483647\relax
      \dimen\ltx@zero=\ht\AtBeginShipoutBox
      \advance\dimen\ltx@zero by \dp\AtBeginShipoutBox
      \setbox\AtBeginShipoutBox=\vbox to \dimen\ltx@zero{%
        \setbox\ltx@zero=\hbox{%
          \begingroup
            \AtBegShi@restore
            #1%
          \endgroup
        }%
        \wd\ltx@zero=0pt\relax
        \ht\ltx@zero=0pt\relax
        \dp\ltx@zero=0pt\relax
        \baselineskip=0pt\relax
        \lineskip=0pt\relax
        \lineskiplimit=0pt\relax
        \copy\ltx@zero
        \unvbox\AtBeginShipoutBox
        \kern0pt%
      }%
      \AtBegShi@restore
      \AtBegShi@restorebox
    \fi
  \fi
}
%    \end{macrocode}
%    \end{macro}
%
%    \begin{macro}{\AtBeginShipoutAddToBoxForeground}
%    \begin{macrocode}
\def\AtBeginShipoutAddToBoxForeground#1{%
  \ifhbox\AtBeginShipoutBox
    \edef\AtBegShi@restore{%
      \hfuzz=\the\hfuzz\relax
      \hbadness=\the\hbadness\relax
    }%
    \hfuzz=1073741823sp\relax
    \hbadness=2147483647\relax
    \setbox\AtBeginShipoutBox=\hbox to \wd\AtBeginShipoutBox{%
      \unhcopy\AtBeginShipoutBox
      \kern-\wd\AtBeginShipoutBox
      \setbox\ltx@zero=\hbox{%
        \begingroup
          \AtBegShi@restore
          #1%
        \endgroup
      }%
      \wd\ltx@zero=0pt\relax
      \ht\ltx@zero=0pt\relax
      \dp\ltx@zero=0pt\relax
      \raise\ht\AtBeginShipoutBox\copy\ltx@zero
      \kern\wd\AtBeginShipoutBox
    }%
    \AtBegShi@restore
  \else
    \ifvbox\AtBeginShipoutBox
      \edef\AtBegShi@restore{%
        \vfuzz=\the\vfuzz\relax
        \vbadness=\the\vbadness\relax
        \dimen\ltx@zero=\the\dimen\ltx@zero\relax
      }%
      \edef\AtBegShi@restorebox{%
        \ht\AtBeginShipoutBox=\the\ht\AtBeginShipoutBox\relax
        \dp\AtBeginShipoutBox=\the\dp\AtBeginShipoutBox\relax
      }%
      \vfuzz=1073741823sp\relax
      \vbadness=2147483647\relax
      \dimen\ltx@zero=\ht\AtBeginShipoutBox
      \advance\dimen\ltx@zero by \dp\AtBeginShipoutBox
      \setbox\AtBeginShipoutBox=\vbox to \dimen\ltx@zero{%
        \setbox\ltx@zero=\hbox{%
          \begingroup
            \AtBegShi@restore
            #1%
          \endgroup
        }%
        \wd\ltx@zero=0pt\relax
        \ht\ltx@zero=0pt\relax
        \dp\ltx@zero=0pt\relax
        \baselineskip=0pt\relax
        \lineskip=0pt\relax
        \lineskiplimit=0pt\relax
        \unvbox\AtBeginShipoutBox
        \kern-\dimen\ltx@zero
        \copy\ltx@zero
        \kern\dimen\ltx@zero
      }%
      \AtBegShi@restore
      \AtBegShi@restorebox
    \fi
  \fi
}
%    \end{macrocode}
%    \end{macro}
%
% \subsection{Positioning}
%
%    \begin{macrocode}
\begingroup\expandafter\expandafter\expandafter\endgroup
\expandafter\ifx\csname RequirePackage\endcsname\relax
  \def\TMP@RequirePackage#1[#2]{%
    \begingroup\expandafter\expandafter\expandafter\endgroup
    \expandafter\ifx\csname ver@#1.sty\endcsname\relax
      \input #1.sty\relax
    \fi
  }%
  \TMP@RequirePackage{ifpdf}[2011/01/30]%
\else
  \RequirePackage{ifpdf}[2011/01/30]%
\fi
%    \end{macrocode}
%
%    \begin{macrocode}
\ifpdf
  \def\AtBegShi@horigin{%
    \ifx\pdfhorigin\@undefined\pdfvariable horigin\else\pdfhorigin\fi}%
  \def\AtBegShi@vorigin{%
    \ifx\pdfvorigin\@undefined\pdfvariable vorigin\else\pdfvorigin\fi}%
\else
  \def\AtBegShi@horigin{72.27pt}%
  \def\AtBegShi@vorigin{72.27pt}%
\fi
%    \end{macrocode}
%
%    \begin{macrocode}
\begingroup
\ifcase
    \expandafter\ifx\csname picture\endcsname\relax
      1%
    \else
      \expandafter\ifx\csname endpicture\endcsname\relax
        1%
      \else
        0%
      \fi
    \fi
  \endgroup
  \def\AtBegShi@BeginPicture{%
    \begingroup
    \picture(0,0)\relax
    \begingroup\expandafter\expandafter\expandafter\endgroup
    \expandafter\ifx\csname unitlength\endcsname\relax
    \else
      \unitlength=1pt\relax
    \fi
    \ignorespaces
  }%
  \def\AtBegShi@EndPicture{%
    \endpicture
    \endgroup
  }%
\else
  \endgroup
  \def\AtBegShi@BeginPicture{%
    \setbox\ltx@zero=\hbox\bgroup
    \begingroup
    \ignorespaces
  }%
  \def\AtBegShi@EndPicture{%
    \endgroup
    \egroup
    \ht\ltx@zero=0pt\relax
    \dp\ltx@zero=0pt\relax
    \copy\ltx@zero
  }%
\fi
%    \end{macrocode}
%
%    \begin{macro}{\AtBeginShipoutUpperLeft}
%    A surrounding \cs{rlap} is not necessary, because
%    the stuff is put in an \cs{hbox} with zero width.
%    \begin{macrocode}
\def\AtBeginShipoutUpperLeft#1{%
  \AtBeginShipoutAddToBox{%
    \kern-\AtBegShi@horigin\relax
    \vbox to 0pt{%
      \kern-\AtBegShi@vorigin\relax
      \AtBegShi@BeginPicture
      #1%
      \AtBegShi@EndPicture
      \vss
    }%
  }%
}
%    \end{macrocode}
%    \end{macro}
%
%    \begin{macro}{\AtBeginShipoutUpperLeftForeground}
%    \begin{macrocode}
\def\AtBeginShipoutUpperLeftForeground#1{%
  \AtBeginShipoutAddToBoxForeground{%
    \kern-\AtBegShi@horigin\relax
    \vbox to 0pt{%
      \kern-\AtBegShi@vorigin\relax
      \AtBegShi@BeginPicture
      #1%
      \AtBegShi@EndPicture
      \vss
    }%
  }%
}
%    \end{macrocode}
%    \end{macro}
%
% \subsection{Patches}
%
%    Patches for \LaTeX\ packages that redefine \cs{shipout}.
%    \LaTeX\ is now supposed to use \eTeX. Thus we do not patch,
%    without \LaTeX\ and \eTeX.
%    \begin{macrocode}
\def\AtBegShi@AbortIfUndefined#1{%
  \begingroup\expandafter\expandafter\expandafter\endgroup
  \expandafter\ifx\csname#1\endcsname\relax
    \expandafter\AtBegShi@AtEnd
  \fi
}
\AtBegShi@AbortIfUndefined{currentgrouplevel}%
\AtBegShi@AbortIfUndefined{AtBeginDocument}%
\AtBegShi@AbortIfUndefined{@ifpackageloaded}%
\AtBegShi@AbortIfUndefined{@ifclassloaded}%
%    \end{macrocode}
%
% \subsubsection{Package \xpackage{crop}}
%
% Fix of method and box.
%    \begin{macrocode}
\def\AtBegShi@PatchCrop{%
  \begingroup
    \def\AtBegShi@Crop@shipout{%
      \afterassignment\CROP@ship
      \setbox\@cclv=%
    }%
    \def\AtBegShi@Crop@ship{%
      \ifvoid\@cclv
        \expandafter\aftergroup
      \fi
      \CROP@@ship
    }%
    \def\AtBegShi@Crop@shiplist{%
      \lineskip\z@
      \lineskiplimit\z@
      \baselineskip\z@
      \CROP@kernel
      \box\@cclv
    }%
    \def\AtBegShi@Crop@@ship{%
      \CROP@shipout\vbox{%
        \CROP@shiplist
      }%
    }%
    \ifx\AtBegShi@Crop@ship\CROP@ship
      \ifx\AtBegShi@Crop@shiplist\CROP@shiplist
        \ifx\AtBegShi@Crop@@ship\CROP@@ship
          \let\AtBegShi@found\relax
          \ifx\shipout\AtBegShi@Crop@shipout
            \def\AtBegShi@found{\shipout}%
          \else\ifx\AtBeginShipoutOriginalShipout\AtBegShi@Crop@shipout
            \def\AtBegShi@found{\AtBeginShipoutOriginalShipout}%
          \else\ifx\@EveryShipout@Org@Shipout\AtBegShi@Crop@shipout
            \def\AtBegShi@found{\@EveryShipout@Org@Shipout}%
          \else\ifx\GPTorg@shipout\AtBegShi@Crop@shipout
            \def\AtBegShi@found{\GPTorg@shipout}%
          \else\ifx\THBorg@shipout\AtBegShi@Crop@shipout
            \def\AtBegShi@found{\THBorg@shipout}%
          \else\ifx\mem@oldshipout\AtBegShi@Crop@shipout
            \def\AtBegShi@found{\mem@oldshipout}%
          \fi\fi\fi\fi\fi\fi
          \ifx\AtBegShi@found\relax
          \else
            \expandafter\endgroup
            \expandafter\def\AtBegShi@found{%
              \edef\AtBegShi@GroupLevel{\number\currentgrouplevel}%
              \afterassignment\CROP@ship
              \setbox\AtBeginShipoutBox=%
            }%
            \def\CROP@ship{%
              \ifnum\AtBegShi@GroupLevel=\currentgrouplevel
              \else
                \expandafter\aftergroup
              \fi
              \CROP@@ship
            }%
            \def\CROP@shiplist{%
              \lineskip 0pt\relax
              \lineskiplimit 0pt\relax
              \baselineskip 0pt\relax
              \CROP@kernel
              \box\AtBeginShipoutBox
            }%
            \def\CROP@@ship{%
              \ifvoid\AtBeginShipoutBox
              \else
                \setbox\AtBeginShipoutBox=\vbox{%
                  \CROP@shiplist
                }%
                \AtBegShi@GetBoxSize\AtBeginShipoutBox
                \expandafter\CROP@shipout
                \expandafter\box
                \expandafter\AtBeginShipoutBox
              \fi
            }%
            \@PackageInfoNoLine{atbegshi}{Package `crop' patched}%
            \begingroup
          \fi
        \fi
      \fi
    \fi
  \endgroup
  \let\AtBegShi@PatchCrop\relax
}
\@ifpackageloaded{crop}{%
  \AtBegShi@PatchCrop
}{%
  \AtBeginDocument{\AtBegShi@PatchCrop}%
}
%    \end{macrocode}
%
% \subsubsection{Package \xpackage{everyshi}}
%
%     Fix of method. Use of box 255 is not changed.
%    \begin{macrocode}
\def\AtBegShi@PatchEveryshi{%
  \begingroup
    \long\def\AtBegShi@Everyshi@shipout{%
      \afterassignment\@EveryShipout@Test
      \global\setbox\@cclv= %
    }%
    \long\def\AtBegShi@Everyshi@Test{%
      \ifvoid\@cclv\relax
        \aftergroup\@EveryShipout@Output
      \else
        \@EveryShipout@Output
      \fi
    }%
    \ifx\AtBegShi@Everyshi@Test\@EveryShipout@Test
      \let\AtBegShi@found\relax
      \ifx\shipout\AtBegShi@Everyshi@shipout
        \def\AtBegShi@found{\shipout}%
      \else\ifx\AtBeginShipoutOriginalShipout\AtBegShi@Everyshi@shipout
        \def\AtBegShi@found{\AtBeginShipoutOriginalShipout}%
      \else\ifx\CROP@shipout\AtBegShi@Everyshi@shipout
        \def\AtBegShi@found{\CROP@shipout}%
      \else\ifx\GPTorg@shipout\AtBegShi@Everyshi@shipout
        \def\AtBegShi@found{\GPTorg@shipout}%
      \else\ifx\THBorg@shipout\AtBegShi@Everyshi@shipout
        \def\AtBegShi@found{\THBorg@shipout}%
      \else\ifx\mem@oldshipout\AtBegShi@Everyshi@shipout
        \def\AtBegShi@found{\mem@oldshipout}%
      \else
        \expandafter\ifx\csname @EveryShipout@Org@Shipout\endcsname
                        \relax
           \ifx\@EveryShipout@Shipout\AtBegShi@Everyshi@shipout
             \def\AtBegShi@found{\@EveryShipout@Shipout}%
           \fi
        \fi
      \fi\fi\fi\fi\fi\fi
      \ifx\AtBegShi@found\relax
      \else
        \expandafter\endgroup
        \expandafter\def\AtBegShi@found{%
          \edef\AtBegShi@GroupLevel{\number\currentgrouplevel}%
          \afterassignment\@EveryShipout@Test
          \setbox\AtBeginShipoutBox=%
        }%
        \def\@EveryShipout@Test{%
          \ifnum\AtBegShi@GroupLevel=\currentgrouplevel
          \else
            \expandafter\aftergroup
          \fi
          \AtBegShi@Everyshi@Output
        }%
        \def\AtBegShi@Everyshi@Output{%
          \ifvoid\AtBeginShipoutBox
          \else
            \global\setbox\ltx@cclv\box\AtBeginShipoutBox
            \expandafter\@EveryShipout@Output
          \fi
        }%
        \@PackageInfoNoLine{atbegshi}{Package `everyshi' patched}%
        \begingroup
      \fi
    \fi
  \endgroup
  \let\AtBegShi@PatchEveryshi\relax
}
\@ifpackageloaded{everyshi}{%
  \AtBegShi@PatchEveryshi
}{%
  \AtBeginDocument{\AtBegShi@PatchEveryshi}%
}
%    \end{macrocode}
%
% \subsubsection{Class \xpackage{memoir}}
%
%     Fix of method and box.
%    \begin{macrocode}
\def\AtBegShi@PatchMemoir{%
  \begingroup
    \def\AtBegShi@Memoir@shipout{%
      \afterassignment\mem@shipi
      \setbox\@cclv=%
    }%
    \def\AtBegShi@Memoir@shipi{%
      \ifvoid\@cclv
        \expandafter\aftergroup
      \fi
      \mem@shipii
    }%
    \def\AtBegShi@Memoir@shipiiA{%
      \mem@oldshipout\vbox{%
        \trimmarks
        \unvbox\@cclv
      }%
    }%
    \def\AtBegShi@Memoir@shipiiB{%
      \ifvoid\@cclv
        \mem@oldshipout\box\@cclv
      \else
        \mem@oldshipout\vbox{%
          \trimmarks
          \unvbox\@cclv
        }%
      \fi
    }%
    \def\AtBegShi@Memoir@PatchAB{%
      \ifvoid\AtBeginShipoutBox
      \else
        \setbox\AtBeginShipoutBox=\vbox{%
          \trimmarks
          \ifvbox\AtBeginShipoutBox
            \unvbox\AtBeginShipoutBox
          \else
            \box\AtBeginShipoutBox
          \fi
        }%
        \AtBegShi@GetBoxSize\AtBeginShipoutBox
        \expandafter\mem@oldshipout
        \expandafter\box
        \expandafter\AtBeginShipoutBox
      \fi
    }%
    \def\AtBegShi@Memoir@shipiiC{% 2008/08/07 v1.6180339a
      \ifvoid\@cclv
        \mem@oldshipout\box\@cclv
      \else
        \ifshowtrims
          \mem@oldshipout\vbox{\trimmarks\unvbox\@cclv}%
        \else
          \mem@oldshipout\box\@cclv
        \fi
      \fi
    }%
    \def\AtBegShi@Memoir@shipiiD{% 2011/03/06 v3.6j
      \ifvoid\@cclv
        \mem@oldshipout\box\@cclv
      \else
        \ifshowtrims
          \mem@oldshipout\vbox{%
            \trimmarks
            \nointerlineskip
            \box\@cclv
          }%
        \else
          \mem@oldshipout\box\@cclv
        \fi
      \fi
    }%
    \def\AtBegShi@Memoir@PatchCD{%
      \ifvoid\AtBeginShipoutBox
      \else
        \ifshowtrims
          \setbox\AtBeginShipoutBox=\vbox{%
            \trimmarks
            \nointerlineskip
            \box\AtBeginShipoutBox
          }%
        \fi
        \AtBegShi@GetBoxSize\AtBeginShipoutBox
        \expandafter\mem@oldshipout
        \expandafter\box
        \expandafter\AtBeginShipoutBox
      \fi
    }%
    \ifx\AtBegShi@Memoir@shipi\mem@shipi
      \let\AtBegShi@found\ltx@one
      \ifx\AtBegShi@Memoir@shipiiA\mem@shipii
        \let\AtBegShi@found\ltx@zero
        \global\let\AtBegShi@Memoir@PatchX\AtBegShi@Memoir@PatchAB
      \else\ifx\AtBegShi@Memoir@shipiiB\mem@shipii
        \let\AtBegShi@found\ltx@zero
        \global\let\AtBegShi@Memoir@PatchX\AtBegShi@Memoir@PatchAB
      \else\ifx\AtBegShi@Memoir@shipiiC\mem@shipii
        \let\AtBegShi@found\ltx@zero
        \global\let\AtBegShi@Memoir@PatchX\AtBegShi@Memoir@PatchCD
      \else\ifx\AtBegShi@Memoir@shipiiD\mem@shipii
        \let\AtBegShi@found\ltx@zero
        \global\let\AtBegShi@Memoir@PatchX\AtBegShi@Memoir@PatchCD
      \fi\fi\fi\fi
      \ifcase\AtBegShi@found
        \let\AtBegShi@found\relax
        \ifx\shipout\AtBegShi@Memoir@shipout
          \def\AtBegShi@found{\shipout}%
        \else\ifx\AtBeginShipoutOriginalShipout\AtBegShi@Memoir@shipout
          \def\AtBegShi@found{\AtBeginShipoutOriginalShipout}%
        \else\ifx\CROP@shipout\AtBegShi@Memoir@shipout
          \def\AtBegShi@found{\CROP@shipout}%
        \else\ifx\GPTorg@shipout\AtBegShi@Memoir@shipout
          \def\AtBegShi@found{\GPTorg@shipout}%
        \else\ifx\THBorg@shipout\AtBegShi@Memoir@shipout
          \def\AtBegShi@found{\THBorg@shipout}%
        \else\ifx\@EveryShipout@Org@Shipout\AtBegShi@Memoir@shipout
          \def\AtBegShi@found{\@EveryShipout@Org@Shipout}%
        \fi\fi\fi\fi\fi\fi
        \ifx\AtBegShi@found\relax
        \else
          \expandafter\endgroup
          \expandafter\def\AtBegShi@found{%
            \edef\AtBegShi@GroupLevel{\number\currentgrouplevel}%
            \afterassignment\mem@shipi
            \setbox\AtBeginShipoutBox=%
          }%
          \def\mem@shipi{%
            \ifnum\AtBegShi@GroupLevel=\currentgrouplevel
            \else
              \expandafter\aftergroup
            \fi
            \mem@shipii
          }%
          \let\mem@shipii\AtBegShi@Memoir@PatchX
          \@PackageInfoNoLine{atbegshi}{Class `memoir' patched}%
          \begingroup
        \fi
      \fi
    \fi
  \endgroup
  \let\AtBegShi@PatchMemoir\relax
}
\@ifclassloaded{memoir}{%
  \AtBegShi@PatchMemoir
}{%
  \AtBeginDocument{\AtBegShi@PatchMemoir}%
}
%    \end{macrocode}
%
%    \begin{macrocode}
\AtBegShi@AtEnd%
%</package>
%    \end{macrocode}
%
% \section{Test}
%
% \subsection{Catcode checks for loading}
%
%    \begin{macrocode}
%<*test1>
%    \end{macrocode}
%    \begin{macrocode}
\catcode`\{=1 %
\catcode`\}=2 %
\catcode`\#=6 %
\catcode`\@=11 %
\expandafter\ifx\csname count@\endcsname\relax
  \countdef\count@=255 %
\fi
\expandafter\ifx\csname @gobble\endcsname\relax
  \long\def\@gobble#1{}%
\fi
\expandafter\ifx\csname @firstofone\endcsname\relax
  \long\def\@firstofone#1{#1}%
\fi
\expandafter\ifx\csname loop\endcsname\relax
  \expandafter\@firstofone
\else
  \expandafter\@gobble
\fi
{%
  \def\loop#1\repeat{%
    \def\body{#1}%
    \iterate
  }%
  \def\iterate{%
    \body
      \let\next\iterate
    \else
      \let\next\relax
    \fi
    \next
  }%
  \let\repeat=\fi
}%
\def\RestoreCatcodes{}
\count@=0 %
\loop
  \edef\RestoreCatcodes{%
    \RestoreCatcodes
    \catcode\the\count@=\the\catcode\count@\relax
  }%
\ifnum\count@<255 %
  \advance\count@ 1 %
\repeat

\def\RangeCatcodeInvalid#1#2{%
  \count@=#1\relax
  \loop
    \catcode\count@=15 %
  \ifnum\count@<#2\relax
    \advance\count@ 1 %
  \repeat
}
\def\RangeCatcodeCheck#1#2#3{%
  \count@=#1\relax
  \loop
    \ifnum#3=\catcode\count@
    \else
      \errmessage{%
        Character \the\count@\space
        with wrong catcode \the\catcode\count@\space
        instead of \number#3%
      }%
    \fi
  \ifnum\count@<#2\relax
    \advance\count@ 1 %
  \repeat
}
\def\space{ }
\expandafter\ifx\csname LoadCommand\endcsname\relax
  \def\LoadCommand{\input atbegshi.sty\relax}%
\fi
\def\Test{%
  \RangeCatcodeInvalid{0}{47}%
  \RangeCatcodeInvalid{58}{64}%
  \RangeCatcodeInvalid{91}{96}%
  \RangeCatcodeInvalid{123}{255}%
  \catcode`\@=12 %
  \catcode`\\=0 %
  \catcode`\%=14 %
  \LoadCommand
  \RangeCatcodeCheck{0}{36}{15}%
  \RangeCatcodeCheck{37}{37}{14}%
  \RangeCatcodeCheck{38}{47}{15}%
  \RangeCatcodeCheck{48}{57}{12}%
  \RangeCatcodeCheck{58}{63}{15}%
  \RangeCatcodeCheck{64}{64}{12}%
  \RangeCatcodeCheck{65}{90}{11}%
  \RangeCatcodeCheck{91}{91}{15}%
  \RangeCatcodeCheck{92}{92}{0}%
  \RangeCatcodeCheck{93}{96}{15}%
  \RangeCatcodeCheck{97}{122}{11}%
  \RangeCatcodeCheck{123}{255}{15}%
  \RestoreCatcodes
}
\Test
\csname @@end\endcsname
\end
%    \end{macrocode}
%    \begin{macrocode}
%</test1>
%    \end{macrocode}
%
%    \begin{macrocode}
%<*test2>
\input atbegshi.sty\relax
\def\msg#{\immediate\write16}
\msg{File: atbegshi-test2.tex 2016/06/09 v1.18 Test file for plain-TeX}
\def\testmsg#1#2{%
  \msg{}%
  \msg{*** Test with box (#1), expected page output [#2]}% hash-ok
}

\newbox\voidbox
\def\void{\box\voidbox}
\begingroup
  \setbox\voidbox=\void
\endgroup

\count0=0\relax
\AtBeginShipout{%
  \global\advance\count0 by 1\relax
  \msg{* Inside \string\AtBeginShipout: [\the\count0]}%
}

\AtBeginShipoutFirst{%
  \msg{* Inside \string\AtBeginShipoutFirst}%
  Hello World%
}

\testmsg{\string\null}{1}
\shipout\null

\AtBeginShipoutFirst{%
  This is too late%
}

\testmsg{void}{}
\shipout\void

\testmsg{\string\copy255 (not void)}{2}
\setbox255\hbox{\vrule height 10bp width 10bp}
\shipout\copy255 %

\testmsg{\string\copy255 (again)}{3}
\shipout\copy255 %

\testmsg{\string\box255}{4}
\shipout\box255 %

\testmsg{\string\box255 (again)}{}
\shipout\box255 %

\testmsg{\string\hbox}{5}
\shipout\hbox{\vrule height 5bp width 20bp}

\testmsg{\string\vbox}{6}
\shipout\vbox{\hrule height 20bp width 5bp}

\testmsg{\string\null, voided by hook}{}
\def\VoidBox{%
  \begingroup
    \setbox\AtBeginShipoutBox=\box\AtBeginShipoutBox
  \endgroup
}
\AtBeginShipout{\VoidBox}
\shipout\null
\def\VoidBox{}

\msg{*** \string\begingroup}
\begingroup
  \testmsg{void}{}%
  \shipout\void
\msg{*** \string\endgroup}
\endgroup

\msg{*** \string\begingroup}
\begingroup
  \testmsg{void}{}%
  \shipout\void
  \testmsg{\string\null}{8}%
  \shipout\null
\msg{*** \string\endgroup}
\endgroup

\testmsg{output routine}{9}
Hello World
\vfill
\eject

\testmsg{\string\null\space(discarded)}{}
\AtBeginShipout{%
  \msg{* Inside \string\AtBeginShipout: DISCARD}%
  \AtBeginShipoutDiscard
}
\shipout\null

\end
%</test2>
%    \end{macrocode}
%
%    \begin{macrocode}
%<*test3>
\NeedsTeXFormat{LaTeX2e}
\ProvidesFile{atbegshi-test3.tex}[2016/06/09 v1.18 Test file for LaTeX]
\RequirePackage{color}
\pagecolor{yellow}
\documentclass[a5paper,showtrims]{memoir}
\usepackage{atbegshi}
\AtBeginShipout{%
  \setbox\AtBeginShipoutBox=\vbox{%
    \vbox to 0pt{%
      \kern-1.5in %
      \hbox to 0pt{%
        \kern-1.5in %
        \color{blue}%
        \rule{1in}{1in}%
        \hss
      }%
      \vss
    }%
    \hrule
    \hbox{\vrule\box\AtBeginShipoutBox\vrule}%
    \hrule
  }%
}
\usepackage{eso-pic}
\makeatletter
\@ifundefined{@EveryShipout@Init}{%
  \typeout{Test skipped}%
  \@@end
}{}
\@EveryShipout@Init
\let\@EveryShipout@Init\relax
\makeatother
\AddToShipoutPicture{%
  \hspace{.52\paperwidth}%
  \colorbox{cyan}{%
    \rule{0mm}{\paperheight}%
    \hspace{.48\paperwidth}%
  }%
}
%    \end{macrocode}
%    Newer versions of class \xclass{memoir} emulate package
%    \xpackage{crop} and prevents its loading. This is undone
%    in next line for this test file.
%    \begin{macrocode}
\expandafter\let\csname ver@crop.sty\endcsname\relax
\usepackage[color=red,cross,a4,center]{crop}
\begin{document}
\shipout\null
\shipout\box\csname voidb@x\endcsname
\section{Hello World}
\end{document}
%</test3>
%    \end{macrocode}
%
% \section{Installation}
%
% \subsection{Download}
%
% \paragraph{Package.} This package is available on
% CTAN\footnote{\url{http://ctan.org/pkg/atbegshi}}:
% \begin{description}
% \item[\CTAN{macros/latex/contrib/oberdiek/atbegshi.dtx}] The source file.
% \item[\CTAN{macros/latex/contrib/oberdiek/atbegshi.pdf}] Documentation.
% \end{description}
%
%
% \paragraph{Bundle.} All the packages of the bundle `oberdiek'
% are also available in a TDS compliant ZIP archive. There
% the packages are already unpacked and the documentation files
% are generated. The files and directories obey the TDS standard.
% \begin{description}
% \item[\CTAN{install/macros/latex/contrib/oberdiek.tds.zip}]
% \end{description}
% \emph{TDS} refers to the standard ``A Directory Structure
% for \TeX\ Files'' (\CTAN{tds/tds.pdf}). Directories
% with \xfile{texmf} in their name are usually organized this way.
%
% \subsection{Bundle installation}
%
% \paragraph{Unpacking.} Unpack the \xfile{oberdiek.tds.zip} in the
% TDS tree (also known as \xfile{texmf} tree) of your choice.
% Example (linux):
% \begin{quote}
%   |unzip oberdiek.tds.zip -d ~/texmf|
% \end{quote}
%
% \paragraph{Script installation.}
% Check the directory \xfile{TDS:scripts/oberdiek/} for
% scripts that need further installation steps.
% Package \xpackage{attachfile2} comes with the Perl script
% \xfile{pdfatfi.pl} that should be installed in such a way
% that it can be called as \texttt{pdfatfi}.
% Example (linux):
% \begin{quote}
%   |chmod +x scripts/oberdiek/pdfatfi.pl|\\
%   |cp scripts/oberdiek/pdfatfi.pl /usr/local/bin/|
% \end{quote}
%
% \subsection{Package installation}
%
% \paragraph{Unpacking.} The \xfile{.dtx} file is a self-extracting
% \docstrip\ archive. The files are extracted by running the
% \xfile{.dtx} through \plainTeX:
% \begin{quote}
%   \verb|tex atbegshi.dtx|
% \end{quote}
%
% \paragraph{TDS.} Now the different files must be moved into
% the different directories in your installation TDS tree
% (also known as \xfile{texmf} tree):
% \begin{quote}
% \def\t{^^A
% \begin{tabular}{@{}>{\ttfamily}l@{ $\rightarrow$ }>{\ttfamily}l@{}}
%   atbegshi.sty & tex/generic/oberdiek/atbegshi.sty\\
%   atbegshi.pdf & doc/latex/oberdiek/atbegshi.pdf\\
%   atbegshi-example1.tex & doc/latex/oberdiek/atbegshi-example1.tex\\
%   atbegshi-example2.tex & doc/latex/oberdiek/atbegshi-example2.tex\\
%   test/atbegshi-test1.tex & doc/latex/oberdiek/test/atbegshi-test1.tex\\
%   test/atbegshi-test2.tex & doc/latex/oberdiek/test/atbegshi-test2.tex\\
%   test/atbegshi-test3.tex & doc/latex/oberdiek/test/atbegshi-test3.tex\\
%   atbegshi.dtx & source/latex/oberdiek/atbegshi.dtx\\
% \end{tabular}^^A
% }^^A
% \sbox0{\t}^^A
% \ifdim\wd0>\linewidth
%   \begingroup
%     \advance\linewidth by\leftmargin
%     \advance\linewidth by\rightmargin
%   \edef\x{\endgroup
%     \def\noexpand\lw{\the\linewidth}^^A
%   }\x
%   \def\lwbox{^^A
%     \leavevmode
%     \hbox to \linewidth{^^A
%       \kern-\leftmargin\relax
%       \hss
%       \usebox0
%       \hss
%       \kern-\rightmargin\relax
%     }^^A
%   }^^A
%   \ifdim\wd0>\lw
%     \sbox0{\small\t}^^A
%     \ifdim\wd0>\linewidth
%       \ifdim\wd0>\lw
%         \sbox0{\footnotesize\t}^^A
%         \ifdim\wd0>\linewidth
%           \ifdim\wd0>\lw
%             \sbox0{\scriptsize\t}^^A
%             \ifdim\wd0>\linewidth
%               \ifdim\wd0>\lw
%                 \sbox0{\tiny\t}^^A
%                 \ifdim\wd0>\linewidth
%                   \lwbox
%                 \else
%                   \usebox0
%                 \fi
%               \else
%                 \lwbox
%               \fi
%             \else
%               \usebox0
%             \fi
%           \else
%             \lwbox
%           \fi
%         \else
%           \usebox0
%         \fi
%       \else
%         \lwbox
%       \fi
%     \else
%       \usebox0
%     \fi
%   \else
%     \lwbox
%   \fi
% \else
%   \usebox0
% \fi
% \end{quote}
% If you have a \xfile{docstrip.cfg} that configures and enables \docstrip's
% TDS installing feature, then some files can already be in the right
% place, see the documentation of \docstrip.
%
% \subsection{Refresh file name databases}
%
% If your \TeX~distribution
% (\teTeX, \mikTeX, \dots) relies on file name databases, you must refresh
% these. For example, \teTeX\ users run \verb|texhash| or
% \verb|mktexlsr|.
%
% \subsection{Some details for the interested}
%
% \paragraph{Attached source.}
%
% The PDF documentation on CTAN also includes the
% \xfile{.dtx} source file. It can be extracted by
% AcrobatReader 6 or higher. Another option is \textsf{pdftk},
% e.g. unpack the file into the current directory:
% \begin{quote}
%   \verb|pdftk atbegshi.pdf unpack_files output .|
% \end{quote}
%
% \paragraph{Unpacking with \LaTeX.}
% The \xfile{.dtx} chooses its action depending on the format:
% \begin{description}
% \item[\plainTeX:] Run \docstrip\ and extract the files.
% \item[\LaTeX:] Generate the documentation.
% \end{description}
% If you insist on using \LaTeX\ for \docstrip\ (really,
% \docstrip\ does not need \LaTeX), then inform the autodetect routine
% about your intention:
% \begin{quote}
%   \verb|latex \let\install=y\input{atbegshi.dtx}|
% \end{quote}
% Do not forget to quote the argument according to the demands
% of your shell.
%
% \paragraph{Generating the documentation.}
% You can use both the \xfile{.dtx} or the \xfile{.drv} to generate
% the documentation. The process can be configured by the
% configuration file \xfile{ltxdoc.cfg}. For instance, put this
% line into this file, if you want to have A4 as paper format:
% \begin{quote}
%   \verb|\PassOptionsToClass{a4paper}{article}|
% \end{quote}
% An example follows how to generate the
% documentation with pdf\LaTeX:
% \begin{quote}
%\begin{verbatim}
%pdflatex atbegshi.dtx
%makeindex -s gind.ist atbegshi.idx
%pdflatex atbegshi.dtx
%makeindex -s gind.ist atbegshi.idx
%pdflatex atbegshi.dtx
%\end{verbatim}
% \end{quote}
%
% \section{Catalogue}
%
% The following XML file can be used as source for the
% \href{http://mirror.ctan.org/help/Catalogue/catalogue.html}{\TeX\ Catalogue}.
% The elements \texttt{caption} and \texttt{description} are imported
% from the original XML file from the Catalogue.
% The name of the XML file in the Catalogue is \xfile{atbegshi.xml}.
%    \begin{macrocode}
%<*catalogue>
<?xml version='1.0' encoding='us-ascii'?>
<!DOCTYPE entry SYSTEM 'catalogue.dtd'>
<entry datestamp='$Date$' modifier='$Author$' id='atbegshi'>
  <name>atbegshi</name>
  <caption>Execute stuff at \shipout time.</caption>
  <authorref id='auth:oberdiek'/>
  <copyright owner='Heiko Oberdiek' year='2007-2011'/>
  <license type='lppl1.3'/>
  <version number='1.18'/>
  <description>
    This package is a modern reimplementation of package
    <xref refid='everyshi'>everyshi</xref>, providing various commands
    to be executed before a <tt>\shipout</tt> command.  It makes use of
    e-TeX&#x2019;s facilities if they are available.  The package may
    be used either with LaTeX or with plain TeX.
    <p/>
    The package is part of the <xref refid='oberdiek'>oberdiek</xref> bundle.
  </description>
  <documentation details='Package documentation'
      href='ctan:/macros/latex/contrib/oberdiek/atbegshi.pdf'/>
  <ctan file='true' path='/macros/latex/contrib/oberdiek/atbegshi.dtx'/>
  <miktex location='oberdiek'/>
  <texlive location='oberdiek'/>
  <install path='/macros/latex/contrib/oberdiek/oberdiek.tds.zip'/>
</entry>
%</catalogue>
%    \end{macrocode}
%
% \begin{History}
%   \begin{Version}{2007/04/17 v1.0}
%   \item
%     First version.
%   \end{Version}
%   \begin{Version}{2007/04/18 v1.1}
%   \item
%     New method based on \cs{lastkern} is used if \eTeX\ is missing.
%   \item
%     \cs{AtBeginShipoutDiscard} also resets \cs{deadcycles}.
%   \end{Version}
%   \begin{Version}{2007/04/19 v1.2}
%   \item
%     \cs{AtBeginShipoutEarly} removed for simplification reasons.
%   \item
%     Forgotten definition of \cs{AtBegShi@Info} added.
%   \item
%     Patches for packages \xpackage{crop} and \xpackage{everyshi}
%     and class \xclass{memoir} added.
%   \end{Version}
%   \begin{Version}{2007/04/26 v1.3}
%   \item
%     Use of package \xpackage{infwarerr}.
%   \item
%     Catcode section after generic header.
%   \end{Version}
%   \begin{Version}{2007/04/27 v1.4}
%   \item
%     Small optimizations.
%   \end{Version}
%   \begin{Version}{2007/06/06 v1.5}
%   \item
%     \cs{AtBeginShipoutUpperLeft} added.
%   \item
%     Example added.
%   \item
%     Fix in second test file for newer version of \xclass{memoir}.
%   \end{Version}
%   \begin{Version}{2007/09/09 v1.6}
%   \item
%     Catcode section rewritten.
%   \end{Version}
%   \begin{Version}{2008/07/18 v1.7}
%   \item
%     Documentation of \cs{AtBeginShipoutUpperLeft} fixed and extended.
%   \end{Version}
%   \begin{Version}{2008/07/19 v1.8}
%   \item
%     \cs{AtBeginShipoutUpperLeftForeground} added.
%   \end{Version}
%   \begin{Version}{2008/07/31 v1.9}
%   \item
%     Second example (TrimBox for dvipdfmx) added.
%   \item
%     No changes in package code.
%   \end{Version}
%   \begin{Version}{2009/12/02 v1.10}
%   \item
%     \cs{AtBeginShipoutOriginalShipout} added.
%   \item
%     Test file fixed.
%   \end{Version}
%   \begin{Version}{2010/03/01 v1.11}
%   \item
%     Compatibility with \iniTeX\ except for \cs{newbox}.
%   \end{Version}
%   \begin{Version}{2010/03/25 v1.12}
%   \item
%     \cs{AtBeginShipoutNext} can now be used inside
%     \cs{AtBeginShipoutNext}.
%   \end{Version}
%   \begin{Version}{2010/08/18 v1.13}
%   \item
%     Fixes for \cs{AtBegShi@CheckDefinable}.
%   \end{Version}
%   \begin{Version}{2010/12/02 v1.14}
%   \item
%     Remove the warning because of void box if the hook
%     calls \AtBeginShipoutDiscard.
%   \end{Version}
%   \begin{Version}{2011/01/30 v1.15}
%   \item
%     Already loaded package files are not input in \hologo{plainTeX}.
%   \end{Version}
%   \begin{Version}{2011/10/05 v1.16}
%   \item
%     \cs{AtBeginShipoutAddToBox}, \cs{AtBeginShipoutAddToBoxForeground} added.
%   \item
%     \cs{AtBeginShipoutBoxWidth}, \cs{AtBeginShipoutBoxHeight},
%     \cs{AtBeginShipoutBoxDepth} added.
%   \item
%     Updates for patches of class \xclass{memoir}.
%   \end{Version}
%   \begin{Version}{2016/05/16 v1.17}
%   \item
%     Documentation updates.
%   \end{Version}
%   \begin{Version}{2016/06/09 v1.18}
%   \item
%     Update for \cs{pdfhorign} in new LuaTeX.
%   \end{Version}
% \end{History}
%
% \PrintIndex
%
% \Finale
\endinput

%        (quote the arguments according to the demands of your shell)
%
% Documentation:
%    (a) If atbegshi.drv is present:
%           latex atbegshi.drv
%    (b) Without atbegshi.drv:
%           latex atbegshi.dtx; ...
%    The class ltxdoc loads the configuration file ltxdoc.cfg
%    if available. Here you can specify further options, e.g.
%    use A4 as paper format:
%       \PassOptionsToClass{a4paper}{article}
%
%    Programm calls to get the documentation (example):
%       pdflatex atbegshi.dtx
%       makeindex -s gind.ist atbegshi.idx
%       pdflatex atbegshi.dtx
%       makeindex -s gind.ist atbegshi.idx
%       pdflatex atbegshi.dtx
%
% Installation:
%    TDS:tex/generic/oberdiek/atbegshi.sty
%    TDS:doc/latex/oberdiek/atbegshi.pdf
%    TDS:doc/latex/oberdiek/atbegshi-example1.tex
%    TDS:doc/latex/oberdiek/atbegshi-example2.tex
%    TDS:doc/latex/oberdiek/test/atbegshi-test1.tex
%    TDS:doc/latex/oberdiek/test/atbegshi-test2.tex
%    TDS:doc/latex/oberdiek/test/atbegshi-test3.tex
%    TDS:source/latex/oberdiek/atbegshi.dtx
%
%<*ignore>
\begingroup
  \catcode123=1 %
  \catcode125=2 %
  \def\x{LaTeX2e}%
\expandafter\endgroup
\ifcase 0\ifx\install y1\fi\expandafter
         \ifx\csname processbatchFile\endcsname\relax\else1\fi
         \ifx\fmtname\x\else 1\fi\relax
\else\csname fi\endcsname
%</ignore>
%<*install>
\input docstrip.tex
\Msg{************************************************************************}
\Msg{* Installation}
\Msg{* Package: atbegshi 2016/06/09 v1.18 At begin shipout hook (HO)}
\Msg{************************************************************************}

\keepsilent
\askforoverwritefalse

\let\MetaPrefix\relax
\preamble

This is a generated file.

Project: atbegshi
Version: 2016/06/09 v1.18

Copyright (C) 2007-2011 by
   Heiko Oberdiek <heiko.oberdiek at googlemail.com>

This work may be distributed and/or modified under the
conditions of the LaTeX Project Public License, either
version 1.3c of this license or (at your option) any later
version. This version of this license is in
   http://www.latex-project.org/lppl/lppl-1-3c.txt
and the latest version of this license is in
   http://www.latex-project.org/lppl.txt
and version 1.3 or later is part of all distributions of
LaTeX version 2005/12/01 or later.

This work has the LPPL maintenance status "maintained".

This Current Maintainer of this work is Heiko Oberdiek.

The Base Interpreter refers to any `TeX-Format',
because some files are installed in TDS:tex/generic//.

This work consists of the main source file atbegshi.dtx
and the derived files
   atbegshi.sty, atbegshi.pdf, atbegshi.ins, atbegshi.drv,
   atbegshi-example1.tex, atbegshi-example2.tex,
   atbegshi-test1.tex, atbegshi-test2.tex, atbegshi-test3.tex.

\endpreamble
\let\MetaPrefix\DoubleperCent

\generate{%
  \file{atbegshi.ins}{\from{atbegshi.dtx}{install}}%
  \file{atbegshi.drv}{\from{atbegshi.dtx}{driver}}%
  \usedir{tex/generic/oberdiek}%
  \file{atbegshi.sty}{\from{atbegshi.dtx}{package}}%
  \usedir{doc/latex/oberdiek}%
  \file{atbegshi-example1.tex}{\from{atbegshi.dtx}{example1}}%
  \file{atbegshi-example2.tex}{\from{atbegshi.dtx}{example2}}%
  \usedir{doc/latex/oberdiek/test}%
  \file{atbegshi-test1.tex}{\from{atbegshi.dtx}{test1}}%
  \file{atbegshi-test2.tex}{\from{atbegshi.dtx}{test2}}%
  \file{atbegshi-test3.tex}{\from{atbegshi.dtx}{test3}}%
  \nopreamble
  \nopostamble
  \usedir{source/latex/oberdiek/catalogue}%
  \file{atbegshi.xml}{\from{atbegshi.dtx}{catalogue}}%
}

\catcode32=13\relax% active space
\let =\space%
\Msg{************************************************************************}
\Msg{*}
\Msg{* To finish the installation you have to move the following}
\Msg{* file into a directory searched by TeX:}
\Msg{*}
\Msg{*     atbegshi.sty}
\Msg{*}
\Msg{* To produce the documentation run the file `atbegshi.drv'}
\Msg{* through LaTeX.}
\Msg{*}
\Msg{* Happy TeXing!}
\Msg{*}
\Msg{************************************************************************}

\endbatchfile
%</install>
%<*ignore>
\fi
%</ignore>
%<*driver>
\NeedsTeXFormat{LaTeX2e}
\ProvidesFile{atbegshi.drv}%
  [2016/06/09 v1.18 At begin shipout hook (HO)]%
\documentclass{ltxdoc}
\usepackage{holtxdoc}[2011/11/22]
\begin{document}
  \DocInput{atbegshi.dtx}%
\end{document}
%</driver>
% \fi
%
%
% \CharacterTable
%  {Upper-case    \A\B\C\D\E\F\G\H\I\J\K\L\M\N\O\P\Q\R\S\T\U\V\W\X\Y\Z
%   Lower-case    \a\b\c\d\e\f\g\h\i\j\k\l\m\n\o\p\q\r\s\t\u\v\w\x\y\z
%   Digits        \0\1\2\3\4\5\6\7\8\9
%   Exclamation   \!     Double quote  \"     Hash (number) \#
%   Dollar        \$     Percent       \%     Ampersand     \&
%   Acute accent  \'     Left paren    \(     Right paren   \)
%   Asterisk      \*     Plus          \+     Comma         \,
%   Minus         \-     Point         \.     Solidus       \/
%   Colon         \:     Semicolon     \;     Less than     \<
%   Equals        \=     Greater than  \>     Question mark \?
%   Commercial at \@     Left bracket  \[     Backslash     \\
%   Right bracket \]     Circumflex    \^     Underscore    \_
%   Grave accent  \`     Left brace    \{     Vertical bar  \|
%   Right brace   \}     Tilde         \~}
%
% \GetFileInfo{atbegshi.drv}
%
% \title{The \xpackage{atbegshi} package}
% \date{2016/06/09 v1.18}
% \author{Heiko Oberdiek\thanks
% {Please report any issues at https://github.com/ho-tex/oberdiek/issues}\\
% \xemail{heiko.oberdiek at googlemail.com}}
%
% \maketitle
%
% \begin{abstract}
% This package is a modern reimplementation of package \xpackage{everyshi}
% without the burden of compatibility. It makes use of \eTeX's if available.
% Both \LaTeX\ and \plainTeX\ are supported.
% \end{abstract}
%
% \tableofcontents
%
% \section{Documentation}
%
% Package \xpackage{atbegshi} redefines \cs{shipout} to insert
% hooks for user code that is executed before the page is shipped out.
% The code may modify or even discard the output page.
% Three hooks are implemented:
% \begin{enumerate}
% \item A hook that is executed for every page, see\\
%       \cs{AtBeginShipout}
% \item A hook that is executed for the next page only, see\\
%       \cs{AtBeginShipoutNext}
% \item A hook that is only executed for the first page, see\\
%       \cs{AtBeginShipoutFirst}
% \end{enumerate}
% The hooks are executed in this order. The following three macros
% provide the user interface for adding code to these hooks:
%
% \begin{declcs}{AtBeginShipout} \M{code}\\
%   \cs{AtBeginShipoutBox}
% \end{declcs}
% Execute the \meta{code} for every page.
% The page contents is held in box register \cs{AtBeginShipoutBox}
% and may be modified. Use \cs{AtBeginShipoutDiscard} if you want to
% discard the page.
%
% \emph{Note:} Package \xpackage{everyshi} uses box register 255. With
% package \xpackage{atbegshi} you must use \cs{AtBeginShipoutBox} instead.
%
% If \LaTeX\ calls \cs{shipout} in \cs{@outputpage} (part of its output
% routine),
% the meaning of \cs{protect} is \cs{noexpand}. \LaTeX\ sets
% \cs{protect} to the appropriate \cs{@typeset@protect} in the box
% that is shipped out. This is too late for the hooks, they
% are called earlier in the redefined \cs{shipout}. Therefore
% package \xpackage{atbegshi} sets \cs{protect} to \cs{@typeset@protect}
% before it calls the hooks.
% (In \cs{EveryShipout} of package \xpackage{everyshi} the
% user is responsible for the correct setting of \cs{protect}.)
%
% \begin{declcs}{AtBeginShipoutNext} \M{code}
% \end{declcs}
% This reimplements package \xpackage{everyshi}'s \cs{AtNextShipout}.
% The \meta{code} is executed at shipout time of the next page only.
% It is just a convenience macro, it can be easily replaced by something
% like:
%\begin{quote}
%\begin{verbatim}
%\newcommand{\MyShipoutHook}{}%
%\AtBeginShipout{\MyShipoutHook}
%\gdef\MyShipoutHook{%
%  ... do something with next page ...
%  \gdef\MyShipoutHook{}%
% }
%\end{verbatim}
%\end{quote}
% (This can be necessary, if hook order does matter).
%
% \begin{declcs}{AtBeginShipoutFirst} \M{code}
% \end{declcs}
% This reimplements \LaTeX's \cs{AtBeginDvi}. This hook is
% usually used for \cs{special} commands that include PostScript
% header files. The \cs{code} is directly executed in a \cs{vbox}
% that is put at the beginning of the output page. Dealing with
% the output box \cs{AtBeginShipoutBox} is not necessary and
% not permitted here.
%
% \begin{declcs}{AtBeginShipoutDiscard}
% \end{declcs}
% This macro notifies package \xpackage{atbegshi} that
% the output page is discarded. The remaining hook code
% and the remaining hooks are not executed and the page is thrown away.
% Also \cs{deadcycles} is cleared to zero like an ordinary \cs{shipout}
% would do.
%
% \begin{declcs}{AtBeginShipoutInit}
% \end{declcs}
% Usually the redefinition of \cs{shipout} is delayed by
% \cs{AtBeginDocument} (if this macro exists). This can be
% too late, if other packages also redefines \cs{shipout}
% and the order does matter.
% \cs{AtBeginShipoutInit} forces the immediate redefinition
% of \cs{shipout}.
%
% \begin{declcs}{AtBeginShipoutAddToBox} \M{stuff}\\
%   \cs{AtBeginShipoutAddToBoxForeground} \M{stuff}
% \end{declcs}
% A quite common use case is the addition of \cs{special}
% or other whatsits to the page output box. Macro
% \cs{AtBeginShipoutAddToBox} puts \meta{stuff} in a
% box with zeroed dimensions. The box with the \meta{stuff}
% is put in the upper left corner of the shipout box
% \cs{AtBeginShipoutBox}. Macro \cs{AtBeginShipoutAddToBox}
% puts the \meta{stuff} in the background, the other macro
% \cs{AtBeginShipoutAddToBoxForeground} in the foreground
% after the original shipout box contents is set.
%
% A void shipout box (that means a discarded page)
% remains void that means \meta{stuff} is ignored in this case.
% The box type of \cs{AtBeginShipoutBox} is preserved.
% Also the box nesting level for the original contents
% of \cs{AtBeginShipoutBox} remains, for example, to avoid trouble
% with links across pages in case of \hologo{pdfTeX}.
%
% \begin{declcs}{AtBeginShipoutUpperLeft} \M{background material}
% \end{declcs}
% This is a macro that puts material in the background of
% box \cs{AtBeginShipoutBox}.
% The \meta{background material} is set in an \cs{hbox}, the
% reference point is the upper left corner of the output page.
% In case of \pdfTeX\ in PDF mode, the settings of
% \cs{pdfhorigin} and \cs{pdfvorigin} are respected.
%
% The macro \cs{AtBeginShipoutUpperLeft} is intended to be used
% in one of the hook setting macros, such as
% \cs{AtBeginShipout}, \cs{AtBeginShipoutFirst}, or
% \cs{AtBeginShipoutNext}.
%
% For \LaTeX\ users the \meta{background material} is set
% inside a \texttt{picture} environment:
% \begin{quote}
%   |\begin{picture}(0,0)|\\
%   \mbox{}\quad |\setlength{\unitlength}{1pt}%|\\
%   \mbox{}\quad \meta{background material}\\
%   |\end{picture}|
% \end{quote}
%
% \begin{declcs}{AtBeginShipoutUpperLeftForeground} \M{foreground material}
% \end{declcs}
% See \cs{AtBeginShipoutUpperLeft}. The difference is that the material
% is put in the foreground.
%
% \begin{declcs}{AtBeginShipoutOriginalShipout} \meta{box}
% \end{declcs}
% It stores the meaning of \cs{shipout} at the time this package
% is loaded.
%
% \begin{declcs}{AtBeginShipoutBoxWidth}\\
%   \cs{AtBeginShipoutBoxHeight}\\
%   \cs{AtBeginShipoutBoxDepth}\\
% \end{declcs}
% These macros store the dimensions of the output box \cs{AtBeginShipoutBox}
% before the original shipout is called. If \cs{shipout} is not
% redefined before the package loading or the box dimensions
% are not changed by the redefined \cs{shipout}, these macros
% contain the dimensions of the shipout box. These values can be
% remembered by \cs{label} and \cs{ref}. For example, this is done
% by the package module \xpackage{zref-pagelayout} of project \xpackage{zref}.
% The dimensions of the shipout page can be used in some \hologo{TeX}
% engines (\hologo{pdfTeX} in PDF mode, \hologo{XeTeX}) to calculate
% the media size of the shipout page if \cs{pdfpagewidth} and
% \cs{pdfpageheight} are not set.
%
% \subsection{Examples}
%
% \subsubsection{Example: circle in background}
%
% In this example we put a circle in the background in the middle
% of the paper.
%
%    \begin{macrocode}
%<*example1>
%    \end{macrocode}
%    \begin{macrocode}
\documentclass[a4paper]{article}
\usepackage{color}
\usepackage{atbegshi}
%    \end{macrocode}
%    Package \xpackage{picture} makes life a little easier, because
%    we can now also use length specifications in |picture|'s commands.
%    \begin{macrocode}
\usepackage{picture}
%    \end{macrocode}
%    Now we draw the circle in the middle of the paper.
%    \cs{put} moves downwards, because the origin is at the top
%    of the page, not at its bottom.
%    \begin{macrocode}
\AtBeginShipout{%
  \AtBeginShipoutUpperLeft{%
    \put(0.5\paperwidth,-0.5\paperheight){\circle{10}}%
  }%
}
\begin{document}
\section{Hello World}
\newpage
\AtBeginShipoutNext{%
  \AtBeginShipoutUpperLeft{%
    \color{red}%
    \put(0,-0.5\paperheight){\line(1,0){\paperwidth}}%
    \put(0.5\paperwidth, 0){\line(0,-1){\paperheight}}%
  }%
}
Only on this page we add a red cross.
\newpage
This page has the circle only.
\par
\vspace{\fill}
The next page will be discarded.
\newpage
\AtBeginShipoutNext{%
  \AtBeginShipoutDiscard
}
This page is discarded.
\newpage
The last page.
\end{document}
%    \end{macrocode}
%    \begin{macrocode}
%</example1>
%    \end{macrocode}
%
% \subsubsection{Example: adding TrimBox for dvipdfmx}
%
% Now an example from ``real life'' follows. Someone from the mailing  list
% for dvipdfmx wants to put a TrimBox on every page. If we use
% \cs{AtBeginShipout}, we have to put the \cs{special} inside the
% box \cs{AtBeginShipoutBox} that gets shipped out.
%    \begin{macrocode}
%<*example2>
\documentclass{minimal}
\usepackage{atbegshi}
\usepackage[
  dvipdfm,
  paperwidth=630bp,
  paperheight=810bp
]{geometry}
\AtBeginShipout{%
  \setbox\AtBeginShipoutBox=\hbox{%
    \special{pdf: put @thispage <</TrimBox[9 9 621 801]>>}%
    \box\AtBeginShipoutBox
  }%
}
\begin{document}
  First page
  \newpage
  Second page
\end{document}
%</example2>
%    \end{macrocode}
% Remember, in \cs{AtBeginShipoutBoxFirst} the \cs{setbox} wrapper code
% is implicitly given and the \cs{special} is used directly.
%
% \section{Method of \cs{shipout} overloading}
%
% \subsection{\cs{shipout}}
%
% The \TeX\ primitive command \cs{shipout} takes a box specification
% and puts the box as a new page in the output file.
% There are two kinds of box specifications:
% \begin{description}
% \item[Direct boxes:]
%   They are given by \cs{hbox}, \cs{vbox}, or \cs{vtop},\\
%   e.g. |\shipout\hbox{Hello World}|.
% \item[Indirect boxes:]
%   \cs{box} or \cs{copy} references a box register by number.
%   The box register contains the contents of the box.
%
%   \emph{Note:} \cs{box} also clears the box register globally.
% \end{description}
% Then we have to differentiate between void and empty boxes:
% \begin{description}
% \item[Void:]
%   Initially or after \cs{box} there is no box in the box register.
%   In this cases the box register is not empty, but \emph{void}.
% \item[Empty:]
%   A box with empty contents, such as
%   |\hbox{}| (= \cs{null}) or |\vbox{}| is an \emph{empty \cs{hbox}}
%   or \emph{empty \cs{vbox}}. If a box register holds such a
%   box, the box still exists, therefore the box register is
%   \emph{not void}.
% \end{description}
%
% \subsection{\cs{afterassignment}}
%
% We want to overload \cs{shipout} to do something with the box.
% It is quite impossible to do this reliable by catching the box
% using macro arguments. The
% variety of box specifications is too large, Examples:
% \begin{quote}
%   |\shipout\null|\\
%   |\shipout\vbox{...}|\\
%   |\shipout\vtop\bgroup ...\egroup|\\
%   |\shipout\box255|
% \end{quote}
% Even worse, the braces don't need to be balanced:
% \begin{quote}
%   |\shipout\hbox\bgroup}|\\
%   |\shipout\vbox{\egroup|
% \end{quote}
% Happily TeX provides a reliable way via \cs{afterassignment}.
% It takes a macro name and executes it just after the assignment.
%
% Now we can redefine \cs{shipout}. The box specification that
% follows \cs{shipout} is catched by \cs{setbox}. This is an
% assignment to a box register. \cs{afterassignment}
% notifies \TeX, that we want to call \cs{@test} right after
% the assignment:
%\begin{quote}
%\begin{verbatim}
%\shipout :=
%  \afterassignment\@test
%  \setbox\mybox=
%\end{verbatim}
%\end{quote}
% We have seen different box specifications. Indirect boxes are easy
% to understand:
% \begin{quote}
%   |\shipout\box0| $\Rightarrow$ |\setbox\mybox=\box0 |\fbox{\cs{@test}}
% \end{quote}
% However direct boxes can have arbitrary contents with lots of
% other assignments. It would be quite inpredictable if \TeX\ would
% put |\@test| after the first of such an assignment or after the
% box specification if the box lacks of assignments.
% Therefore \TeX\ puts |\@test| right at the beginning of the
% box specification, e.g:
% \begin{quote}
%   |\shipout\hbox{Hello World}|\\
%   $\Rightarrow$
%   |\setbox\mybox=\hbox{|\fbox{\cs{@test}}| Hello World}|
% \end{quote}
%
% \subsection{Test for direct or indirect boxes}
%
% Now we want to execute \cs{@test}, but where are we?
% We can be after the completed box assignment, if \cs{shipout}
% was called with an indirect box. Or we are right at the beginning
% of a direct box.
%
% \subsubsection{With \eTeX}
%
% With the \eTeX's extensions the answer is very easy:
% Being inside the direct box means that we are inside a new group.
% The new primitive command \cs{currentgrouplevel} tells how
% deeply the groups are currently nested. Macro \cs{@test} just
% compares the previously stored group level with the current one:
%\begin{quote}
%\begin{verbatim}
%\shipout :=
%  \edef\saved@grouplevel{\number\currentgrouplevel}
%  \afterassignment\@test
%  \setbox\mybox=
%
%\@test :=
%  \ifnum\saved@grouplevel=\currentgrouplevel
%    % case: indirect box, the assignment is completed
%    \@output
%  \else
%    % case: direct box, we are inside the box
%    \aftergroup\@outbox
%  \fi
%\end{verbatim}
%\end{quote}
%
% \subsubsection{Without \eTeX}
%
% Life becomes complicate without \eTeX. We cannot ask the
% group level. However, if we are inside a direct box,
% the box register \cs{mybox} is not yet changed by \cs{setbox}.
% Thus we need a special initial value and compare
% it in \cs{@test} with the current value of the box.
%
% What can be used as initial value? Arbitrary box contents cannot
% be compared. \TeX\ only tells us a few properties:
% \begin{itemize}
% \item Box type:\cs{ifhbox}, \cs{ifvbox}
% \item Dimensions: \cs{wd}, \cs{ht}, \cs{dp}
% \item Voidness: \cs{ifvoid}
% \end{itemize}
% Unhappily all these qualities even combined are not sufficient
% for constructing an initial box value, because \cs{shipout} can be
% called with a box that is accidently just the same as the choosen
% initial value.
%
% Nevertheless we have two alternatives for an initial value:
% \begin{itemize}
% \item A box of some type with some funny settings that are
% unlikely to occur in real life, e.g a height of |4911sp-\maxdimen|.
% \item A void box.
% \end{itemize}
% A collision between this initial value and an indirect \cs{shipout}
% box with just the same value is possible.
% Then \cs{@test} will make a wrong decision that it is executed
% inside a direct box and delays \cs{@output} by \cs{aftergroup}.
% Thus \cs{@output} is not called at the place we want.
% In contrary, the result is an uncertainty about the place:
% \begin{itemize}
% \item \cs{shipout} is used in a group that perhaps closes some
%   pages later. A bad place for \cs{@output}.
% \item Without a surrounding group \cs{aftergroup} effectively kills
%   its argument.
% \end{itemize}
% In the first case of a box with special dimensions we can even loose
% the page. However in the case of the void box, this effect is even
% desired, because the original \cs{shipout} does not output void boxes.
% All we have to do is to ensure that our box \cs{mybox} is always
% void except for the phase when the overloaded \cs{shipout} is executed.
% And secondly we must keep this semantics of \cs{shipout} for the void case
% in our macros, namely \cs{@output}.
%\begin{quote}
%\begin{verbatim}
%\shipout :=
%  % trick to get a void box \mybox
%  \begingroup
%    \setbox\mybox=\box\mybox
%  \endgroup
%  \afterassignment\@test
%  \setbox\mybox=
%
%\@test :=
%  \ifvoid\mybox
%    \aftergroup\@output
%  \else
%    \@output
%  \fi
%\end{verbatim}
%\end{quote}
% The nasty case is |\shipout\box\voidb@x| where the indirect box
% is void and that must not generate an output page.
% If a surrounding group is missing the output is ignored
% because of \cs{aftergroup}.
% Otherwise \@output is called some time
% later when the surrounding group closes.
% But \cs{mybox} is void outside the execution phase
% of the redefined \cs{shipout}. Also \cs{@output} checks for a
% void box and cancels the page output.
% The disadvantage remains that the hook in \cs{@output} is called
% for a page that will not be output.
%
% \subsubsection{\cs{lastkern} method}
%
% At the beginning of a new box, there is no \cs{kern}, the contents
% of the box is still empty and \cs{lastkern} returns 0\,pt.
% This can be used to distinguish between direct and indirect boxes:
% We execute \cs{setbox} in a box with a preceding non-zero kern.
% After an indirect box, \cs{lastkern} sees this kern, otherwise
% it returns 0\,pt.
%\begin{quote}
%\begin{verbatim}
%\shipout :=
%  \begingroup
%    \setbox\mybox=\hbox\bgroup
%      \kern1pt
%      \afterassignment\shipout@test
%      \global\setbox\mybox=
%\@test :=
%  \ifdim\lastkern=0pt
%    % direct box
%    \aftergroup\egroup
%    \aftergroup\endgroup
%    \aftergroup\@output
%  \else
%    \egroup
%    \endgroup
%    \@output
%  \fi
%\end{verbatim}
%\end{quote}
% We have two \cs{setbox} commands. The first creates a controlled
% context box where we can safely insert a \cs{kern}. We get rid of
% this temporarily used context box by putting the local \cs{setbox}
% in a group.
%
% After the group we want to have our shipout box in \cs{mybox}.
% Therefore we use a global assignment here.
%
% \subsection{Output}
%
% With or without \eTeX\ we ensure the original behaviour of
% \cs{shipout} that void boxes do not generate output pages.
%
% Now we can place the hook \cs{@hook} for the user code that
% wants to manipulate the output box.
%\begin{quote}
%\begin{verbatim}
%\@output :=
%  \ifvoid\mybox
%    % cancel output of void box
%  \else
%    \@hook
%    \ifvoid\mybox
%      % user code in \@hook could has voided the box
%    \else
%      \original@shipout\box\mybox
%    \fi
%  \fi
%\end{verbatim}
%\end{quote}
%
% \subsection{Separate box register}
%
% So far we have said nothing about the box number of \cs{mybox}.
% The following case that outputs the same page twice
% shows that we are not free in the use
% of the box register:
% \begin{quote}
%   |\shipout\copy<num> \shipout\box<num>|
% \end{quote}
% We manipulate the box by the hook and without \eTeX\ the box
% must even be voided. However, the use case above requires that
% the box contents does not change at all. Therefore we must
% reserve a separate box register to avoid collisions with
% user box registers.
%
% \emph{Note:} Box register number 255 is special for the output
% routine, because TeX complains if this box is not voided
% by the output routine.
% However, this requirement does not apply to \cs{shipout} at all.
% In fact \cs{shipout} does not change any box register. This is
% usually done by a call of \cs{box}, but the output routine
% can do it later \emph{after} invoking of \cs{shipout}.
%
% \subsection{Summary}
%
% \subsubsection{With \eTeX}
%
% Putting the pieces together we get for \eTeX:
%\begin{quote}
%\begin{verbatim}
%\newbox\mybox
%\let\original@shipout\shipout
%
%\shipout :=
%  \edef\saved@grouplevel{\number\currentgrouplevel}
%  \afterassignment\@test
%  \setbox\mybox=
%
%\@test :=
%  \ifnum\saved@grouplevel<\currentgrouplevel
%    \expandafter\aftergroup
%  \fi
%  \@output
%
%\@output :=
%  \ifvoid\mybox
%    % cancel output of void box
%  \else
%    \@hook
%    \ifvoid\mybox
%      % user code in \@hook could have voided the box
%    \else
%      \original@shipout\box\mybox
%    \fi
%  \fi
%\end{verbatim}
%\end{quote}
%
% \subsubsection{Without \eTeX, traditional way}
%
% And for \TeX\ without \eTeX:
%\begin{quote}
%\begin{verbatim}
%\newbox\mybox
%\begingroup
%   \setbox\mybox=\box\mybox % ensure \mybox is void
%\endgroup
%\let\original@shipout\shipout
%
%\shipout :=
%  % trick to get a void box \mybox
%  \begingroup
%    \setbox\mybox=\box\mybox
%  \endgroup
%  \afterassignment\@test
%  \setbox\mybox=
%
%\@test :=
%  \ifvoid\mybox
%    \expandafter\aftergroup
%  \fi
%  \@output
%
%\@output :=
%  \ifvoid\mybox
%    % cancel output of void box
%  \else
%    \@hook
%    \ifvoid\mybox
%      % user code in \@hook could have voided the box
%    \else
%      \original@shipout\box\mybox
%    \fi
%  \fi
%\end{verbatim}
%\end{quote}
%
% \subsubsection{\cs{lastkern} method}
%
% And for \TeX\ without \eTeX\ using the \cs{lastkern} method:
%\begin{quote}
%\begin{verbatim}
%\newbox\mybox
%\let\original@shipout\shipout
%
%\shipout :=
%  \begingroup
%  \setbox\mybox=\hbox\bgroup
%  \kern1pt
%  \afterassignment\@test
%  \setbox\mybox=
%
%\@test :=
%  \ifdim\lastkern=0pt
%    \expandafter\aftergroup
%  \fi
%  \@output
%
%\@output :=
%  \egroup
%  \endgroup
%  \ifvoid\mybox
%    % cancel output of void box
%  \else
%    \@hook
%    \ifvoid\mybox
%      % user code in \@hook could have voided the box
%    \else
%      \original@shipout\box\mybox
%    \fi
%  \fi
%\end{verbatim}
%\end{quote}
%
%
% \StopEventually{
% }
%
% \section{Implementation}
%
% Package \xpackage{atbegshi} uses \eTeX's \cs{currentgrouplevel},
% if it is available. Otherwise the \cs{lastkern} method is used.
%
%    \begin{macrocode}
%<*package>
%    \end{macrocode}
%
% \subsection{Reload check and package identification}
%    Reload check, especially if the package is not used with \LaTeX.
%    \begin{macrocode}
\begingroup\catcode61\catcode48\catcode32=10\relax%
  \catcode13=5 % ^^M
  \endlinechar=13 %
  \catcode35=6 % #
  \catcode39=12 % '
  \catcode44=12 % ,
  \catcode45=12 % -
  \catcode46=12 % .
  \catcode58=12 % :
  \catcode64=11 % @
  \catcode123=1 % {
  \catcode125=2 % }
  \expandafter\let\expandafter\x\csname ver@atbegshi.sty\endcsname
  \ifx\x\relax % plain-TeX, first loading
  \else
    \def\empty{}%
    \ifx\x\empty % LaTeX, first loading,
      % variable is initialized, but \ProvidesPackage not yet seen
    \else
      \expandafter\ifx\csname PackageInfo\endcsname\relax
        \def\x#1#2{%
          \immediate\write-1{Package #1 Info: #2.}%
        }%
      \else
        \def\x#1#2{\PackageInfo{#1}{#2, stopped}}%
      \fi
      \x{atbegshi}{The package is already loaded}%
      \aftergroup\endinput
    \fi
  \fi
\endgroup%
%    \end{macrocode}
%    Package identification:
%    \begin{macrocode}
\begingroup\catcode61\catcode48\catcode32=10\relax%
  \catcode13=5 % ^^M
  \endlinechar=13 %
  \catcode35=6 % #
  \catcode39=12 % '
  \catcode40=12 % (
  \catcode41=12 % )
  \catcode44=12 % ,
  \catcode45=12 % -
  \catcode46=12 % .
  \catcode47=12 % /
  \catcode58=12 % :
  \catcode64=11 % @
  \catcode91=12 % [
  \catcode93=12 % ]
  \catcode123=1 % {
  \catcode125=2 % }
  \expandafter\ifx\csname ProvidesPackage\endcsname\relax
    \def\x#1#2#3[#4]{\endgroup
      \immediate\write-1{Package: #3 #4}%
      \xdef#1{#4}%
    }%
  \else
    \def\x#1#2[#3]{\endgroup
      #2[{#3}]%
      \ifx#1\@undefined
        \xdef#1{#3}%
      \fi
      \ifx#1\relax
        \xdef#1{#3}%
      \fi
    }%
  \fi
\expandafter\x\csname ver@atbegshi.sty\endcsname
\ProvidesPackage{atbegshi}%
  [2016/06/09 v1.18 At begin shipout hook (HO)]%
%    \end{macrocode}
%
% \subsection{Catcodes}
%
%    \begin{macrocode}
\begingroup\catcode61\catcode48\catcode32=10\relax%
  \catcode13=5 % ^^M
  \endlinechar=13 %
  \catcode123=1 % {
  \catcode125=2 % }
  \catcode64=11 % @
  \def\x{\endgroup
    \expandafter\edef\csname AtBegShi@AtEnd\endcsname{%
      \endlinechar=\the\endlinechar\relax
      \catcode13=\the\catcode13\relax
      \catcode32=\the\catcode32\relax
      \catcode35=\the\catcode35\relax
      \catcode61=\the\catcode61\relax
      \catcode64=\the\catcode64\relax
      \catcode123=\the\catcode123\relax
      \catcode125=\the\catcode125\relax
    }%
  }%
\x\catcode61\catcode48\catcode32=10\relax%
\catcode13=5 % ^^M
\endlinechar=13 %
\catcode35=6 % #
\catcode64=11 % @
\catcode123=1 % {
\catcode125=2 % }
\def\TMP@EnsureCode#1#2{%
  \edef\AtBegShi@AtEnd{%
    \AtBegShi@AtEnd
    \catcode#1=\the\catcode#1\relax
  }%
  \catcode#1=#2\relax
}
\TMP@EnsureCode{40}{12}% (
\TMP@EnsureCode{41}{12}% )
\TMP@EnsureCode{44}{12}% ,
\TMP@EnsureCode{45}{12}% -
\TMP@EnsureCode{47}{12}% /
\TMP@EnsureCode{46}{12}% .
\TMP@EnsureCode{58}{12}% :
\TMP@EnsureCode{91}{12}% [
\TMP@EnsureCode{93}{12}% ]
\TMP@EnsureCode{94}{7}% ^(superscript)
\TMP@EnsureCode{96}{12}% `
\edef\AtBegShi@AtEnd{\AtBegShi@AtEnd\noexpand\endinput}
%    \end{macrocode}
%
% \subsection{Preparations}
%
%    \begin{macrocode}
\begingroup\expandafter\expandafter\expandafter\endgroup
\expandafter\ifx\csname RequirePackage\endcsname\relax
  \def\TMP@RequirePackage#1[#2]{%
    \begingroup\expandafter\expandafter\expandafter\endgroup
    \expandafter\ifx\csname ver@#1.sty\endcsname\relax
      \input #1.sty\relax
    \fi
  }%
  \TMP@RequirePackage{infwarerr}[2007/09/09]%
  \TMP@RequirePackage{ltxcmds}[2010/03/01]%
\else
  \RequirePackage{infwarerr}[2007/09/09]%
  \RequirePackage{ltxcmds}[2010/03/01]%
\fi
%    \end{macrocode}
%
%    \begin{macro}{\AtBegShi@CheckDefinable}
%    \begin{macrocode}
\begingroup\expandafter\expandafter\expandafter\endgroup
\expandafter\ifx\csname @ifdefinable\endcsname\relax
  \def\AtBegShi@CheckDefinable#1{%
    \ifcase\ifx#1\relax
             \ltx@one
           \else
             \ifx#1\@undefined
               \ltx@one
             \else
               \ltx@zero
             \fi
           \fi
      \@PackageError{atbegshi}{%
        \string#1\space is already defined%
      }\@ehd
    \fi
  }%
\else
  \def\AtBegShi@CheckDefinable#1{%
    \@ifdefinable{#1}{}%
  }%
\fi
%    \end{macrocode}
%    \end{macro}
%
%    \begin{macro}{\ifAtBegShi@Discarded}
%    \begin{macrocode}
\ltx@newif\ifAtBegShi@Discarded
%    \end{macrocode}
%    \end{macro}
%    \begin{macro}{\AtBeginShipoutDiscard}
%    \begin{macrocode}
\AtBegShi@CheckDefinable\AtBeginShipoutDiscard
\def\AtBeginShipoutDiscard{%
  \deadcycles=\ltx@zero
  \global\AtBegShi@Discardedtrue
}
%    \end{macrocode}
%    \end{macro}
%
%    \begin{macrocode}
\begingroup\expandafter\expandafter\expandafter\endgroup
\expandafter\ifx\csname currentgrouplevel\endcsname\relax
  \catcode`X=9  % ignore
  \catcode`E=14 % comment
\else
  \catcode`X=14 % comment
  \catcode`E=9  % ignore
\fi
%    \end{macrocode}
%
%    \begin{macro}{\AtBegShi@Shipout}
%    \begin{macrocode}
\def\AtBegShi@Shipout{%
X \begingroup
X \setbox\AtBeginShipoutBox=\hbox\bgroup
X \kern\p@
E \edef\AtBegShi@GroupLevel{\number\currentgrouplevel}%
  \afterassignment\AtBegShi@Test
X \global
  \setbox\AtBeginShipoutBox=%
}
%    \end{macrocode}
%    \end{macro}
%
%    \begin{macro}{\AtBegShi@Test}
%    \begin{macrocode}
\def\AtBegShi@Test{%
X \ifdim\lastkern=0pt %
E \ifnum\AtBegShi@GroupLevel<\currentgrouplevel
    \expandafter\aftergroup
  \fi
  \AtBegShi@Output
}
%    \end{macrocode}
%    \end{macro}
%
%    \begin{macro}{\AtBegShi@Output}
%    \begin{macrocode}
\def\AtBegShi@Output{%
X \egroup
X \endgroup
  \ifvoid\AtBeginShipoutBox
    \@PackageWarning{atbegshi}{Ignoring void shipout box}%
  \else
    \let\AtBegShi@OrgProtect\protect
    \csname set@typeset@protect\endcsname
    \global\AtBegShi@Discardedfalse
    \AtBegShi@Hook
    \expandafter\gdef\expandafter\AtBegShi@HookNext
    \expandafter{\expandafter}%
    \AtBegShi@HookNext
    \ifAtBegShi@Discarded
      \@PackageInfoNoLine{atbegshi}{Shipout page discarded}%
      \global\AtBegShi@Discardedfalse
      \begingroup
        \setbox\AtBeginShipoutBox\box\AtBeginShipoutBox
      \endgroup
      \let\protect\AtBegShi@OrgProtect
    \else
      \AtBegShi@First
      \let\protect\AtBegShi@OrgProtect
      \AtBegShi@GetBoxSize\AtBeginShipoutBox
      \ltx@ifundefined{AtNextShipout}{%
      }{%
        \AtNextShipout{\AtBegShi@GetBoxSize\@cclv}%
      }%
      \AtBeginShipoutOriginalShipout\box\AtBeginShipoutBox
    \fi
  \fi
}
%    \end{macrocode}
%    \end{macro}
%
%    \begin{macro}{\AtBegShi@GetBoxSize}
%    \begin{macrocode}
\def\AtBegShi@GetBoxSize#1{%
  \xdef\AtBeginShipoutBoxWidth{\the\wd#1}%
  \xdef\AtBeginShipoutBoxHeight{\the\ht#1}%
  \xdef\AtBeginShipoutBoxDepth{\the\dp#1}%
}
%    \end{macrocode}
%    \end{macro}
%    \begin{macro}{\AtBeginShipoutBoxWidth}
%    \begin{macrocode}
\def\AtBeginShipoutBoxWidth{0pt}
%    \end{macrocode}
%    \end{macro}
%    \begin{macro}{\AtBeginShipoutBoxHeight}
%    \begin{macrocode}
\def\AtBeginShipoutBoxHeight{0pt}
%    \end{macrocode}
%    \end{macro}
%    \begin{macro}{\AtBeginShipoutBoxDepth}
%    \begin{macrocode}
\def\AtBeginShipoutBoxDepth{0pt}
%    \end{macrocode}
%    \end{macro}
%
%    \begin{macrocode}
\catcode`\X=11 %
\catcode`\E=11 %
%    \end{macrocode}
%
%    \begin{macro}{\AtBegShi@First}
%    \begin{macrocode}
\def\AtBegShi@First{%
  \ifx\AtBegShi@HookFirst\ltx@empty
  \else
    \AtBeginShipoutAddToBox{\AtBegShi@HookFirst}%
  \fi
  \global\let\AtBegShi@First\ltx@empty
  \global\let\AtBeginShipoutFirst\AtBegShi@FirstDisabled
}
%    \end{macrocode}
%    \end{macro}
%
%    \begin{macro}{\AtBegShi@Hook}
%    \begin{macrocode}
\gdef\AtBegShi@Hook{}
%    \end{macrocode}
%    \end{macro}
%    \begin{macro}{\AtBegShi@HookNext}
%    \begin{macrocode}
\gdef\AtBegShi@HookNext{}
%    \end{macrocode}
%    \end{macro}
%    \begin{macro}{\AtBegShi@HookFirst}
%    \begin{macrocode}
\gdef\AtBegShi@HookFirst{}
%    \end{macrocode}
%    \end{macro}
%    \begin{macro}{\AtBeginShipout}
%    \begin{macrocode}
\AtBegShi@CheckDefinable\AtBeginShipout
\def\AtBeginShipout{%
  \AtBegShi@AddHook\AtBegShi@Hook
}
%    \end{macrocode}
%    \end{macro}
%    \begin{macro}{\AtBeginShipoutNext}
%    \begin{macrocode}
\AtBegShi@CheckDefinable\AtBeginShipoutNext
\def\AtBeginShipoutNext{%
  \AtBegShi@AddHook\AtBegShi@HookNext
}
%    \end{macrocode}
%    \end{macro}
%    \begin{macro}{\AtBeginShipoutFirst}
%    \begin{macrocode}
\AtBegShi@CheckDefinable\AtBeginShipoutFirst
\def\AtBeginShipoutFirst{%
  \AtBegShi@AddTo\AtBegShi@HookFirst
}
%    \end{macrocode}
%    \end{macro}
%    \begin{macro}{\AtBegShi@FirstDisabled}
%    \begin{macrocode}
\long\def\AtBegShi@FirstDisabled#1{%
  \@PackageWarning{atbegshi}{%
    First page is already shipped out, ignoring\MessageBreak
    \string\AtBeginShipoutFirst
  }%
}
%    \end{macrocode}
%    \end{macro}
%
%    \begin{macro}{\AtBegShi@AddTo}
%    \begin{macrocode}
\begingroup\expandafter\expandafter\expandafter\endgroup
\expandafter\ifx\csname g@addto@macro\endcsname\relax
  \long\def\AtBegShi@AddTo#1#2{%
    \begingroup
      \toks\ltx@zero\expandafter{#1#2}%
      \xdef#1{\the\toks\ltx@zero}%
    \endgroup
  }%
\else
  \let\AtBegShi@AddTo\g@addto@macro
\fi
%    \end{macrocode}
%    \end{macro}
%
%    \begin{macro}{\AtBegShi@AddHook}
%    \begin{macrocode}
\long\def\AtBegShi@AddHook#1#2{%
  \AtBegShi@AddTo#1{\AtBegShi@Item{#2}}%
}
%    \end{macrocode}
%    \end{macro}
%
%    \begin{macro}{\AtBegShi@Item}
%    \begin{macrocode}
\long\def\AtBegShi@Item#1{%
  \ifAtBegShi@Discarded
  \else
    #1%
    \ifAtBegShi@Discarded
    \else
      \ifvoid\AtBeginShipoutBox
        \@PackageWarning{atbegshi}{%
          Shipout box was voided by hook,\MessageBreak
          ignoring shipout box%
        }%
        \AtBeginShipoutDiscard
      \fi
    \fi
  \fi
}
%    \end{macrocode}
%    \end{macro}
%
%    \begin{macro}{\AtBeginShipoutInit}
%    \begin{macrocode}
\AtBegShi@CheckDefinable\AtBeginShipoutInit
\def\AtBeginShipoutInit{%
  \ltx@IfUndefined{newbox}{%
    \@PackageError{atbegshi}{%
      \string\AtBeginShipoutInit\space failed\MessageBreak
       because of missing \expandafter\string\csname newbox\endcsname
    }\@ehc
  }{%
    \csname newbox\endcsname\AtBeginShipoutBox
    \AtBegShi@CheckDefinable\AtBeginShipoutOriginalShipout
    \global\let\AtBeginShipoutOriginalShipout\shipout
    \global\let\shipout\AtBegShi@Shipout
  }%
  \gdef\AtBeginShipoutInit{}%
}
%    \end{macrocode}
%    \end{macro}
%    \begin{macrocode}
\begingroup\expandafter\expandafter\expandafter\endgroup
\expandafter\ifx\csname AtBeginDocument\endcsname\relax
  \AtBeginShipoutInit
\else
  \AtBeginDocument{\AtBeginShipoutInit}%
\fi
%    \end{macrocode}
%
% \subsection{Additions to the shipout box}
%
%    \begin{macro}{\AtBeginShipoutAddToBox}
%    \begin{macrocode}
\def\AtBeginShipoutAddToBox#1{%
  \ifhbox\AtBeginShipoutBox
    \edef\AtBegShi@restore{%
      \hfuzz=\the\hfuzz\relax
      \hbadness=\the\hbadness\relax
    }%
    \hfuzz=1073741823sp\relax
    \hbadness=2147483647\relax
    \setbox\AtBeginShipoutBox=\hbox to \wd\AtBeginShipoutBox{%
      \setbox\ltx@zero=\hbox{%
        \begingroup
          \AtBegShi@restore
          #1%
        \endgroup
      }%
      \wd\ltx@zero=0pt\relax
      \ht\ltx@zero=0pt\relax
      \dp\ltx@zero=0pt\relax
      \raise\ht\AtBeginShipoutBox\copy\ltx@zero
      \unhcopy\AtBeginShipoutBox
    }%
    \AtBegShi@restore
  \else
    \ifvbox\AtBeginShipoutBox
      \edef\AtBegShi@restore{%
        \vfuzz=\the\vfuzz\relax
        \vbadness=\the\vbadness\relax
        \dimen\ltx@zero=\the\dimen\ltx@zero\relax
      }%
      \edef\AtBegShi@restorebox{%
        \ht\AtBeginShipoutBox=\the\ht\AtBeginShipoutBox\relax
        \dp\AtBeginShipoutBox=\the\dp\AtBeginShipoutBox\relax
      }%
      \vfuzz=1073741823sp\relax
      \vbadness=2147483647\relax
      \dimen\ltx@zero=\ht\AtBeginShipoutBox
      \advance\dimen\ltx@zero by \dp\AtBeginShipoutBox
      \setbox\AtBeginShipoutBox=\vbox to \dimen\ltx@zero{%
        \setbox\ltx@zero=\hbox{%
          \begingroup
            \AtBegShi@restore
            #1%
          \endgroup
        }%
        \wd\ltx@zero=0pt\relax
        \ht\ltx@zero=0pt\relax
        \dp\ltx@zero=0pt\relax
        \baselineskip=0pt\relax
        \lineskip=0pt\relax
        \lineskiplimit=0pt\relax
        \copy\ltx@zero
        \unvbox\AtBeginShipoutBox
        \kern0pt%
      }%
      \AtBegShi@restore
      \AtBegShi@restorebox
    \fi
  \fi
}
%    \end{macrocode}
%    \end{macro}
%
%    \begin{macro}{\AtBeginShipoutAddToBoxForeground}
%    \begin{macrocode}
\def\AtBeginShipoutAddToBoxForeground#1{%
  \ifhbox\AtBeginShipoutBox
    \edef\AtBegShi@restore{%
      \hfuzz=\the\hfuzz\relax
      \hbadness=\the\hbadness\relax
    }%
    \hfuzz=1073741823sp\relax
    \hbadness=2147483647\relax
    \setbox\AtBeginShipoutBox=\hbox to \wd\AtBeginShipoutBox{%
      \unhcopy\AtBeginShipoutBox
      \kern-\wd\AtBeginShipoutBox
      \setbox\ltx@zero=\hbox{%
        \begingroup
          \AtBegShi@restore
          #1%
        \endgroup
      }%
      \wd\ltx@zero=0pt\relax
      \ht\ltx@zero=0pt\relax
      \dp\ltx@zero=0pt\relax
      \raise\ht\AtBeginShipoutBox\copy\ltx@zero
      \kern\wd\AtBeginShipoutBox
    }%
    \AtBegShi@restore
  \else
    \ifvbox\AtBeginShipoutBox
      \edef\AtBegShi@restore{%
        \vfuzz=\the\vfuzz\relax
        \vbadness=\the\vbadness\relax
        \dimen\ltx@zero=\the\dimen\ltx@zero\relax
      }%
      \edef\AtBegShi@restorebox{%
        \ht\AtBeginShipoutBox=\the\ht\AtBeginShipoutBox\relax
        \dp\AtBeginShipoutBox=\the\dp\AtBeginShipoutBox\relax
      }%
      \vfuzz=1073741823sp\relax
      \vbadness=2147483647\relax
      \dimen\ltx@zero=\ht\AtBeginShipoutBox
      \advance\dimen\ltx@zero by \dp\AtBeginShipoutBox
      \setbox\AtBeginShipoutBox=\vbox to \dimen\ltx@zero{%
        \setbox\ltx@zero=\hbox{%
          \begingroup
            \AtBegShi@restore
            #1%
          \endgroup
        }%
        \wd\ltx@zero=0pt\relax
        \ht\ltx@zero=0pt\relax
        \dp\ltx@zero=0pt\relax
        \baselineskip=0pt\relax
        \lineskip=0pt\relax
        \lineskiplimit=0pt\relax
        \unvbox\AtBeginShipoutBox
        \kern-\dimen\ltx@zero
        \copy\ltx@zero
        \kern\dimen\ltx@zero
      }%
      \AtBegShi@restore
      \AtBegShi@restorebox
    \fi
  \fi
}
%    \end{macrocode}
%    \end{macro}
%
% \subsection{Positioning}
%
%    \begin{macrocode}
\begingroup\expandafter\expandafter\expandafter\endgroup
\expandafter\ifx\csname RequirePackage\endcsname\relax
  \def\TMP@RequirePackage#1[#2]{%
    \begingroup\expandafter\expandafter\expandafter\endgroup
    \expandafter\ifx\csname ver@#1.sty\endcsname\relax
      \input #1.sty\relax
    \fi
  }%
  \TMP@RequirePackage{ifpdf}[2011/01/30]%
\else
  \RequirePackage{ifpdf}[2011/01/30]%
\fi
%    \end{macrocode}
%
%    \begin{macrocode}
\ifpdf
  \def\AtBegShi@horigin{%
    \ifx\pdfhorigin\@undefined\pdfvariable horigin\else\pdfhorigin\fi}%
  \def\AtBegShi@vorigin{%
    \ifx\pdfvorigin\@undefined\pdfvariable vorigin\else\pdfvorigin\fi}%
\else
  \def\AtBegShi@horigin{72.27pt}%
  \def\AtBegShi@vorigin{72.27pt}%
\fi
%    \end{macrocode}
%
%    \begin{macrocode}
\begingroup
\ifcase
    \expandafter\ifx\csname picture\endcsname\relax
      1%
    \else
      \expandafter\ifx\csname endpicture\endcsname\relax
        1%
      \else
        0%
      \fi
    \fi
  \endgroup
  \def\AtBegShi@BeginPicture{%
    \begingroup
    \picture(0,0)\relax
    \begingroup\expandafter\expandafter\expandafter\endgroup
    \expandafter\ifx\csname unitlength\endcsname\relax
    \else
      \unitlength=1pt\relax
    \fi
    \ignorespaces
  }%
  \def\AtBegShi@EndPicture{%
    \endpicture
    \endgroup
  }%
\else
  \endgroup
  \def\AtBegShi@BeginPicture{%
    \setbox\ltx@zero=\hbox\bgroup
    \begingroup
    \ignorespaces
  }%
  \def\AtBegShi@EndPicture{%
    \endgroup
    \egroup
    \ht\ltx@zero=0pt\relax
    \dp\ltx@zero=0pt\relax
    \copy\ltx@zero
  }%
\fi
%    \end{macrocode}
%
%    \begin{macro}{\AtBeginShipoutUpperLeft}
%    A surrounding \cs{rlap} is not necessary, because
%    the stuff is put in an \cs{hbox} with zero width.
%    \begin{macrocode}
\def\AtBeginShipoutUpperLeft#1{%
  \AtBeginShipoutAddToBox{%
    \kern-\AtBegShi@horigin\relax
    \vbox to 0pt{%
      \kern-\AtBegShi@vorigin\relax
      \AtBegShi@BeginPicture
      #1%
      \AtBegShi@EndPicture
      \vss
    }%
  }%
}
%    \end{macrocode}
%    \end{macro}
%
%    \begin{macro}{\AtBeginShipoutUpperLeftForeground}
%    \begin{macrocode}
\def\AtBeginShipoutUpperLeftForeground#1{%
  \AtBeginShipoutAddToBoxForeground{%
    \kern-\AtBegShi@horigin\relax
    \vbox to 0pt{%
      \kern-\AtBegShi@vorigin\relax
      \AtBegShi@BeginPicture
      #1%
      \AtBegShi@EndPicture
      \vss
    }%
  }%
}
%    \end{macrocode}
%    \end{macro}
%
% \subsection{Patches}
%
%    Patches for \LaTeX\ packages that redefine \cs{shipout}.
%    \LaTeX\ is now supposed to use \eTeX. Thus we do not patch,
%    without \LaTeX\ and \eTeX.
%    \begin{macrocode}
\def\AtBegShi@AbortIfUndefined#1{%
  \begingroup\expandafter\expandafter\expandafter\endgroup
  \expandafter\ifx\csname#1\endcsname\relax
    \expandafter\AtBegShi@AtEnd
  \fi
}
\AtBegShi@AbortIfUndefined{currentgrouplevel}%
\AtBegShi@AbortIfUndefined{AtBeginDocument}%
\AtBegShi@AbortIfUndefined{@ifpackageloaded}%
\AtBegShi@AbortIfUndefined{@ifclassloaded}%
%    \end{macrocode}
%
% \subsubsection{Package \xpackage{crop}}
%
% Fix of method and box.
%    \begin{macrocode}
\def\AtBegShi@PatchCrop{%
  \begingroup
    \def\AtBegShi@Crop@shipout{%
      \afterassignment\CROP@ship
      \setbox\@cclv=%
    }%
    \def\AtBegShi@Crop@ship{%
      \ifvoid\@cclv
        \expandafter\aftergroup
      \fi
      \CROP@@ship
    }%
    \def\AtBegShi@Crop@shiplist{%
      \lineskip\z@
      \lineskiplimit\z@
      \baselineskip\z@
      \CROP@kernel
      \box\@cclv
    }%
    \def\AtBegShi@Crop@@ship{%
      \CROP@shipout\vbox{%
        \CROP@shiplist
      }%
    }%
    \ifx\AtBegShi@Crop@ship\CROP@ship
      \ifx\AtBegShi@Crop@shiplist\CROP@shiplist
        \ifx\AtBegShi@Crop@@ship\CROP@@ship
          \let\AtBegShi@found\relax
          \ifx\shipout\AtBegShi@Crop@shipout
            \def\AtBegShi@found{\shipout}%
          \else\ifx\AtBeginShipoutOriginalShipout\AtBegShi@Crop@shipout
            \def\AtBegShi@found{\AtBeginShipoutOriginalShipout}%
          \else\ifx\@EveryShipout@Org@Shipout\AtBegShi@Crop@shipout
            \def\AtBegShi@found{\@EveryShipout@Org@Shipout}%
          \else\ifx\GPTorg@shipout\AtBegShi@Crop@shipout
            \def\AtBegShi@found{\GPTorg@shipout}%
          \else\ifx\THBorg@shipout\AtBegShi@Crop@shipout
            \def\AtBegShi@found{\THBorg@shipout}%
          \else\ifx\mem@oldshipout\AtBegShi@Crop@shipout
            \def\AtBegShi@found{\mem@oldshipout}%
          \fi\fi\fi\fi\fi\fi
          \ifx\AtBegShi@found\relax
          \else
            \expandafter\endgroup
            \expandafter\def\AtBegShi@found{%
              \edef\AtBegShi@GroupLevel{\number\currentgrouplevel}%
              \afterassignment\CROP@ship
              \setbox\AtBeginShipoutBox=%
            }%
            \def\CROP@ship{%
              \ifnum\AtBegShi@GroupLevel=\currentgrouplevel
              \else
                \expandafter\aftergroup
              \fi
              \CROP@@ship
            }%
            \def\CROP@shiplist{%
              \lineskip 0pt\relax
              \lineskiplimit 0pt\relax
              \baselineskip 0pt\relax
              \CROP@kernel
              \box\AtBeginShipoutBox
            }%
            \def\CROP@@ship{%
              \ifvoid\AtBeginShipoutBox
              \else
                \setbox\AtBeginShipoutBox=\vbox{%
                  \CROP@shiplist
                }%
                \AtBegShi@GetBoxSize\AtBeginShipoutBox
                \expandafter\CROP@shipout
                \expandafter\box
                \expandafter\AtBeginShipoutBox
              \fi
            }%
            \@PackageInfoNoLine{atbegshi}{Package `crop' patched}%
            \begingroup
          \fi
        \fi
      \fi
    \fi
  \endgroup
  \let\AtBegShi@PatchCrop\relax
}
\@ifpackageloaded{crop}{%
  \AtBegShi@PatchCrop
}{%
  \AtBeginDocument{\AtBegShi@PatchCrop}%
}
%    \end{macrocode}
%
% \subsubsection{Package \xpackage{everyshi}}
%
%     Fix of method. Use of box 255 is not changed.
%    \begin{macrocode}
\def\AtBegShi@PatchEveryshi{%
  \begingroup
    \long\def\AtBegShi@Everyshi@shipout{%
      \afterassignment\@EveryShipout@Test
      \global\setbox\@cclv= %
    }%
    \long\def\AtBegShi@Everyshi@Test{%
      \ifvoid\@cclv\relax
        \aftergroup\@EveryShipout@Output
      \else
        \@EveryShipout@Output
      \fi
    }%
    \ifx\AtBegShi@Everyshi@Test\@EveryShipout@Test
      \let\AtBegShi@found\relax
      \ifx\shipout\AtBegShi@Everyshi@shipout
        \def\AtBegShi@found{\shipout}%
      \else\ifx\AtBeginShipoutOriginalShipout\AtBegShi@Everyshi@shipout
        \def\AtBegShi@found{\AtBeginShipoutOriginalShipout}%
      \else\ifx\CROP@shipout\AtBegShi@Everyshi@shipout
        \def\AtBegShi@found{\CROP@shipout}%
      \else\ifx\GPTorg@shipout\AtBegShi@Everyshi@shipout
        \def\AtBegShi@found{\GPTorg@shipout}%
      \else\ifx\THBorg@shipout\AtBegShi@Everyshi@shipout
        \def\AtBegShi@found{\THBorg@shipout}%
      \else\ifx\mem@oldshipout\AtBegShi@Everyshi@shipout
        \def\AtBegShi@found{\mem@oldshipout}%
      \else
        \expandafter\ifx\csname @EveryShipout@Org@Shipout\endcsname
                        \relax
           \ifx\@EveryShipout@Shipout\AtBegShi@Everyshi@shipout
             \def\AtBegShi@found{\@EveryShipout@Shipout}%
           \fi
        \fi
      \fi\fi\fi\fi\fi\fi
      \ifx\AtBegShi@found\relax
      \else
        \expandafter\endgroup
        \expandafter\def\AtBegShi@found{%
          \edef\AtBegShi@GroupLevel{\number\currentgrouplevel}%
          \afterassignment\@EveryShipout@Test
          \setbox\AtBeginShipoutBox=%
        }%
        \def\@EveryShipout@Test{%
          \ifnum\AtBegShi@GroupLevel=\currentgrouplevel
          \else
            \expandafter\aftergroup
          \fi
          \AtBegShi@Everyshi@Output
        }%
        \def\AtBegShi@Everyshi@Output{%
          \ifvoid\AtBeginShipoutBox
          \else
            \global\setbox\ltx@cclv\box\AtBeginShipoutBox
            \expandafter\@EveryShipout@Output
          \fi
        }%
        \@PackageInfoNoLine{atbegshi}{Package `everyshi' patched}%
        \begingroup
      \fi
    \fi
  \endgroup
  \let\AtBegShi@PatchEveryshi\relax
}
\@ifpackageloaded{everyshi}{%
  \AtBegShi@PatchEveryshi
}{%
  \AtBeginDocument{\AtBegShi@PatchEveryshi}%
}
%    \end{macrocode}
%
% \subsubsection{Class \xpackage{memoir}}
%
%     Fix of method and box.
%    \begin{macrocode}
\def\AtBegShi@PatchMemoir{%
  \begingroup
    \def\AtBegShi@Memoir@shipout{%
      \afterassignment\mem@shipi
      \setbox\@cclv=%
    }%
    \def\AtBegShi@Memoir@shipi{%
      \ifvoid\@cclv
        \expandafter\aftergroup
      \fi
      \mem@shipii
    }%
    \def\AtBegShi@Memoir@shipiiA{%
      \mem@oldshipout\vbox{%
        \trimmarks
        \unvbox\@cclv
      }%
    }%
    \def\AtBegShi@Memoir@shipiiB{%
      \ifvoid\@cclv
        \mem@oldshipout\box\@cclv
      \else
        \mem@oldshipout\vbox{%
          \trimmarks
          \unvbox\@cclv
        }%
      \fi
    }%
    \def\AtBegShi@Memoir@PatchAB{%
      \ifvoid\AtBeginShipoutBox
      \else
        \setbox\AtBeginShipoutBox=\vbox{%
          \trimmarks
          \ifvbox\AtBeginShipoutBox
            \unvbox\AtBeginShipoutBox
          \else
            \box\AtBeginShipoutBox
          \fi
        }%
        \AtBegShi@GetBoxSize\AtBeginShipoutBox
        \expandafter\mem@oldshipout
        \expandafter\box
        \expandafter\AtBeginShipoutBox
      \fi
    }%
    \def\AtBegShi@Memoir@shipiiC{% 2008/08/07 v1.6180339a
      \ifvoid\@cclv
        \mem@oldshipout\box\@cclv
      \else
        \ifshowtrims
          \mem@oldshipout\vbox{\trimmarks\unvbox\@cclv}%
        \else
          \mem@oldshipout\box\@cclv
        \fi
      \fi
    }%
    \def\AtBegShi@Memoir@shipiiD{% 2011/03/06 v3.6j
      \ifvoid\@cclv
        \mem@oldshipout\box\@cclv
      \else
        \ifshowtrims
          \mem@oldshipout\vbox{%
            \trimmarks
            \nointerlineskip
            \box\@cclv
          }%
        \else
          \mem@oldshipout\box\@cclv
        \fi
      \fi
    }%
    \def\AtBegShi@Memoir@PatchCD{%
      \ifvoid\AtBeginShipoutBox
      \else
        \ifshowtrims
          \setbox\AtBeginShipoutBox=\vbox{%
            \trimmarks
            \nointerlineskip
            \box\AtBeginShipoutBox
          }%
        \fi
        \AtBegShi@GetBoxSize\AtBeginShipoutBox
        \expandafter\mem@oldshipout
        \expandafter\box
        \expandafter\AtBeginShipoutBox
      \fi
    }%
    \ifx\AtBegShi@Memoir@shipi\mem@shipi
      \let\AtBegShi@found\ltx@one
      \ifx\AtBegShi@Memoir@shipiiA\mem@shipii
        \let\AtBegShi@found\ltx@zero
        \global\let\AtBegShi@Memoir@PatchX\AtBegShi@Memoir@PatchAB
      \else\ifx\AtBegShi@Memoir@shipiiB\mem@shipii
        \let\AtBegShi@found\ltx@zero
        \global\let\AtBegShi@Memoir@PatchX\AtBegShi@Memoir@PatchAB
      \else\ifx\AtBegShi@Memoir@shipiiC\mem@shipii
        \let\AtBegShi@found\ltx@zero
        \global\let\AtBegShi@Memoir@PatchX\AtBegShi@Memoir@PatchCD
      \else\ifx\AtBegShi@Memoir@shipiiD\mem@shipii
        \let\AtBegShi@found\ltx@zero
        \global\let\AtBegShi@Memoir@PatchX\AtBegShi@Memoir@PatchCD
      \fi\fi\fi\fi
      \ifcase\AtBegShi@found
        \let\AtBegShi@found\relax
        \ifx\shipout\AtBegShi@Memoir@shipout
          \def\AtBegShi@found{\shipout}%
        \else\ifx\AtBeginShipoutOriginalShipout\AtBegShi@Memoir@shipout
          \def\AtBegShi@found{\AtBeginShipoutOriginalShipout}%
        \else\ifx\CROP@shipout\AtBegShi@Memoir@shipout
          \def\AtBegShi@found{\CROP@shipout}%
        \else\ifx\GPTorg@shipout\AtBegShi@Memoir@shipout
          \def\AtBegShi@found{\GPTorg@shipout}%
        \else\ifx\THBorg@shipout\AtBegShi@Memoir@shipout
          \def\AtBegShi@found{\THBorg@shipout}%
        \else\ifx\@EveryShipout@Org@Shipout\AtBegShi@Memoir@shipout
          \def\AtBegShi@found{\@EveryShipout@Org@Shipout}%
        \fi\fi\fi\fi\fi\fi
        \ifx\AtBegShi@found\relax
        \else
          \expandafter\endgroup
          \expandafter\def\AtBegShi@found{%
            \edef\AtBegShi@GroupLevel{\number\currentgrouplevel}%
            \afterassignment\mem@shipi
            \setbox\AtBeginShipoutBox=%
          }%
          \def\mem@shipi{%
            \ifnum\AtBegShi@GroupLevel=\currentgrouplevel
            \else
              \expandafter\aftergroup
            \fi
            \mem@shipii
          }%
          \let\mem@shipii\AtBegShi@Memoir@PatchX
          \@PackageInfoNoLine{atbegshi}{Class `memoir' patched}%
          \begingroup
        \fi
      \fi
    \fi
  \endgroup
  \let\AtBegShi@PatchMemoir\relax
}
\@ifclassloaded{memoir}{%
  \AtBegShi@PatchMemoir
}{%
  \AtBeginDocument{\AtBegShi@PatchMemoir}%
}
%    \end{macrocode}
%
%    \begin{macrocode}
\AtBegShi@AtEnd%
%</package>
%    \end{macrocode}
%
% \section{Test}
%
% \subsection{Catcode checks for loading}
%
%    \begin{macrocode}
%<*test1>
%    \end{macrocode}
%    \begin{macrocode}
\catcode`\{=1 %
\catcode`\}=2 %
\catcode`\#=6 %
\catcode`\@=11 %
\expandafter\ifx\csname count@\endcsname\relax
  \countdef\count@=255 %
\fi
\expandafter\ifx\csname @gobble\endcsname\relax
  \long\def\@gobble#1{}%
\fi
\expandafter\ifx\csname @firstofone\endcsname\relax
  \long\def\@firstofone#1{#1}%
\fi
\expandafter\ifx\csname loop\endcsname\relax
  \expandafter\@firstofone
\else
  \expandafter\@gobble
\fi
{%
  \def\loop#1\repeat{%
    \def\body{#1}%
    \iterate
  }%
  \def\iterate{%
    \body
      \let\next\iterate
    \else
      \let\next\relax
    \fi
    \next
  }%
  \let\repeat=\fi
}%
\def\RestoreCatcodes{}
\count@=0 %
\loop
  \edef\RestoreCatcodes{%
    \RestoreCatcodes
    \catcode\the\count@=\the\catcode\count@\relax
  }%
\ifnum\count@<255 %
  \advance\count@ 1 %
\repeat

\def\RangeCatcodeInvalid#1#2{%
  \count@=#1\relax
  \loop
    \catcode\count@=15 %
  \ifnum\count@<#2\relax
    \advance\count@ 1 %
  \repeat
}
\def\RangeCatcodeCheck#1#2#3{%
  \count@=#1\relax
  \loop
    \ifnum#3=\catcode\count@
    \else
      \errmessage{%
        Character \the\count@\space
        with wrong catcode \the\catcode\count@\space
        instead of \number#3%
      }%
    \fi
  \ifnum\count@<#2\relax
    \advance\count@ 1 %
  \repeat
}
\def\space{ }
\expandafter\ifx\csname LoadCommand\endcsname\relax
  \def\LoadCommand{\input atbegshi.sty\relax}%
\fi
\def\Test{%
  \RangeCatcodeInvalid{0}{47}%
  \RangeCatcodeInvalid{58}{64}%
  \RangeCatcodeInvalid{91}{96}%
  \RangeCatcodeInvalid{123}{255}%
  \catcode`\@=12 %
  \catcode`\\=0 %
  \catcode`\%=14 %
  \LoadCommand
  \RangeCatcodeCheck{0}{36}{15}%
  \RangeCatcodeCheck{37}{37}{14}%
  \RangeCatcodeCheck{38}{47}{15}%
  \RangeCatcodeCheck{48}{57}{12}%
  \RangeCatcodeCheck{58}{63}{15}%
  \RangeCatcodeCheck{64}{64}{12}%
  \RangeCatcodeCheck{65}{90}{11}%
  \RangeCatcodeCheck{91}{91}{15}%
  \RangeCatcodeCheck{92}{92}{0}%
  \RangeCatcodeCheck{93}{96}{15}%
  \RangeCatcodeCheck{97}{122}{11}%
  \RangeCatcodeCheck{123}{255}{15}%
  \RestoreCatcodes
}
\Test
\csname @@end\endcsname
\end
%    \end{macrocode}
%    \begin{macrocode}
%</test1>
%    \end{macrocode}
%
%    \begin{macrocode}
%<*test2>
\input atbegshi.sty\relax
\def\msg#{\immediate\write16}
\msg{File: atbegshi-test2.tex 2016/06/09 v1.18 Test file for plain-TeX}
\def\testmsg#1#2{%
  \msg{}%
  \msg{*** Test with box (#1), expected page output [#2]}% hash-ok
}

\newbox\voidbox
\def\void{\box\voidbox}
\begingroup
  \setbox\voidbox=\void
\endgroup

\count0=0\relax
\AtBeginShipout{%
  \global\advance\count0 by 1\relax
  \msg{* Inside \string\AtBeginShipout: [\the\count0]}%
}

\AtBeginShipoutFirst{%
  \msg{* Inside \string\AtBeginShipoutFirst}%
  Hello World%
}

\testmsg{\string\null}{1}
\shipout\null

\AtBeginShipoutFirst{%
  This is too late%
}

\testmsg{void}{}
\shipout\void

\testmsg{\string\copy255 (not void)}{2}
\setbox255\hbox{\vrule height 10bp width 10bp}
\shipout\copy255 %

\testmsg{\string\copy255 (again)}{3}
\shipout\copy255 %

\testmsg{\string\box255}{4}
\shipout\box255 %

\testmsg{\string\box255 (again)}{}
\shipout\box255 %

\testmsg{\string\hbox}{5}
\shipout\hbox{\vrule height 5bp width 20bp}

\testmsg{\string\vbox}{6}
\shipout\vbox{\hrule height 20bp width 5bp}

\testmsg{\string\null, voided by hook}{}
\def\VoidBox{%
  \begingroup
    \setbox\AtBeginShipoutBox=\box\AtBeginShipoutBox
  \endgroup
}
\AtBeginShipout{\VoidBox}
\shipout\null
\def\VoidBox{}

\msg{*** \string\begingroup}
\begingroup
  \testmsg{void}{}%
  \shipout\void
\msg{*** \string\endgroup}
\endgroup

\msg{*** \string\begingroup}
\begingroup
  \testmsg{void}{}%
  \shipout\void
  \testmsg{\string\null}{8}%
  \shipout\null
\msg{*** \string\endgroup}
\endgroup

\testmsg{output routine}{9}
Hello World
\vfill
\eject

\testmsg{\string\null\space(discarded)}{}
\AtBeginShipout{%
  \msg{* Inside \string\AtBeginShipout: DISCARD}%
  \AtBeginShipoutDiscard
}
\shipout\null

\end
%</test2>
%    \end{macrocode}
%
%    \begin{macrocode}
%<*test3>
\NeedsTeXFormat{LaTeX2e}
\ProvidesFile{atbegshi-test3.tex}[2016/06/09 v1.18 Test file for LaTeX]
\RequirePackage{color}
\pagecolor{yellow}
\documentclass[a5paper,showtrims]{memoir}
\usepackage{atbegshi}
\AtBeginShipout{%
  \setbox\AtBeginShipoutBox=\vbox{%
    \vbox to 0pt{%
      \kern-1.5in %
      \hbox to 0pt{%
        \kern-1.5in %
        \color{blue}%
        \rule{1in}{1in}%
        \hss
      }%
      \vss
    }%
    \hrule
    \hbox{\vrule\box\AtBeginShipoutBox\vrule}%
    \hrule
  }%
}
\usepackage{eso-pic}
\makeatletter
\@ifundefined{@EveryShipout@Init}{%
  \typeout{Test skipped}%
  \@@end
}{}
\@EveryShipout@Init
\let\@EveryShipout@Init\relax
\makeatother
\AddToShipoutPicture{%
  \hspace{.52\paperwidth}%
  \colorbox{cyan}{%
    \rule{0mm}{\paperheight}%
    \hspace{.48\paperwidth}%
  }%
}
%    \end{macrocode}
%    Newer versions of class \xclass{memoir} emulate package
%    \xpackage{crop} and prevents its loading. This is undone
%    in next line for this test file.
%    \begin{macrocode}
\expandafter\let\csname ver@crop.sty\endcsname\relax
\usepackage[color=red,cross,a4,center]{crop}
\begin{document}
\shipout\null
\shipout\box\csname voidb@x\endcsname
\section{Hello World}
\end{document}
%</test3>
%    \end{macrocode}
%
% \section{Installation}
%
% \subsection{Download}
%
% \paragraph{Package.} This package is available on
% CTAN\footnote{\url{http://ctan.org/pkg/atbegshi}}:
% \begin{description}
% \item[\CTAN{macros/latex/contrib/oberdiek/atbegshi.dtx}] The source file.
% \item[\CTAN{macros/latex/contrib/oberdiek/atbegshi.pdf}] Documentation.
% \end{description}
%
%
% \paragraph{Bundle.} All the packages of the bundle `oberdiek'
% are also available in a TDS compliant ZIP archive. There
% the packages are already unpacked and the documentation files
% are generated. The files and directories obey the TDS standard.
% \begin{description}
% \item[\CTAN{install/macros/latex/contrib/oberdiek.tds.zip}]
% \end{description}
% \emph{TDS} refers to the standard ``A Directory Structure
% for \TeX\ Files'' (\CTAN{tds/tds.pdf}). Directories
% with \xfile{texmf} in their name are usually organized this way.
%
% \subsection{Bundle installation}
%
% \paragraph{Unpacking.} Unpack the \xfile{oberdiek.tds.zip} in the
% TDS tree (also known as \xfile{texmf} tree) of your choice.
% Example (linux):
% \begin{quote}
%   |unzip oberdiek.tds.zip -d ~/texmf|
% \end{quote}
%
% \paragraph{Script installation.}
% Check the directory \xfile{TDS:scripts/oberdiek/} for
% scripts that need further installation steps.
% Package \xpackage{attachfile2} comes with the Perl script
% \xfile{pdfatfi.pl} that should be installed in such a way
% that it can be called as \texttt{pdfatfi}.
% Example (linux):
% \begin{quote}
%   |chmod +x scripts/oberdiek/pdfatfi.pl|\\
%   |cp scripts/oberdiek/pdfatfi.pl /usr/local/bin/|
% \end{quote}
%
% \subsection{Package installation}
%
% \paragraph{Unpacking.} The \xfile{.dtx} file is a self-extracting
% \docstrip\ archive. The files are extracted by running the
% \xfile{.dtx} through \plainTeX:
% \begin{quote}
%   \verb|tex atbegshi.dtx|
% \end{quote}
%
% \paragraph{TDS.} Now the different files must be moved into
% the different directories in your installation TDS tree
% (also known as \xfile{texmf} tree):
% \begin{quote}
% \def\t{^^A
% \begin{tabular}{@{}>{\ttfamily}l@{ $\rightarrow$ }>{\ttfamily}l@{}}
%   atbegshi.sty & tex/generic/oberdiek/atbegshi.sty\\
%   atbegshi.pdf & doc/latex/oberdiek/atbegshi.pdf\\
%   atbegshi-example1.tex & doc/latex/oberdiek/atbegshi-example1.tex\\
%   atbegshi-example2.tex & doc/latex/oberdiek/atbegshi-example2.tex\\
%   test/atbegshi-test1.tex & doc/latex/oberdiek/test/atbegshi-test1.tex\\
%   test/atbegshi-test2.tex & doc/latex/oberdiek/test/atbegshi-test2.tex\\
%   test/atbegshi-test3.tex & doc/latex/oberdiek/test/atbegshi-test3.tex\\
%   atbegshi.dtx & source/latex/oberdiek/atbegshi.dtx\\
% \end{tabular}^^A
% }^^A
% \sbox0{\t}^^A
% \ifdim\wd0>\linewidth
%   \begingroup
%     \advance\linewidth by\leftmargin
%     \advance\linewidth by\rightmargin
%   \edef\x{\endgroup
%     \def\noexpand\lw{\the\linewidth}^^A
%   }\x
%   \def\lwbox{^^A
%     \leavevmode
%     \hbox to \linewidth{^^A
%       \kern-\leftmargin\relax
%       \hss
%       \usebox0
%       \hss
%       \kern-\rightmargin\relax
%     }^^A
%   }^^A
%   \ifdim\wd0>\lw
%     \sbox0{\small\t}^^A
%     \ifdim\wd0>\linewidth
%       \ifdim\wd0>\lw
%         \sbox0{\footnotesize\t}^^A
%         \ifdim\wd0>\linewidth
%           \ifdim\wd0>\lw
%             \sbox0{\scriptsize\t}^^A
%             \ifdim\wd0>\linewidth
%               \ifdim\wd0>\lw
%                 \sbox0{\tiny\t}^^A
%                 \ifdim\wd0>\linewidth
%                   \lwbox
%                 \else
%                   \usebox0
%                 \fi
%               \else
%                 \lwbox
%               \fi
%             \else
%               \usebox0
%             \fi
%           \else
%             \lwbox
%           \fi
%         \else
%           \usebox0
%         \fi
%       \else
%         \lwbox
%       \fi
%     \else
%       \usebox0
%     \fi
%   \else
%     \lwbox
%   \fi
% \else
%   \usebox0
% \fi
% \end{quote}
% If you have a \xfile{docstrip.cfg} that configures and enables \docstrip's
% TDS installing feature, then some files can already be in the right
% place, see the documentation of \docstrip.
%
% \subsection{Refresh file name databases}
%
% If your \TeX~distribution
% (\teTeX, \mikTeX, \dots) relies on file name databases, you must refresh
% these. For example, \teTeX\ users run \verb|texhash| or
% \verb|mktexlsr|.
%
% \subsection{Some details for the interested}
%
% \paragraph{Attached source.}
%
% The PDF documentation on CTAN also includes the
% \xfile{.dtx} source file. It can be extracted by
% AcrobatReader 6 or higher. Another option is \textsf{pdftk},
% e.g. unpack the file into the current directory:
% \begin{quote}
%   \verb|pdftk atbegshi.pdf unpack_files output .|
% \end{quote}
%
% \paragraph{Unpacking with \LaTeX.}
% The \xfile{.dtx} chooses its action depending on the format:
% \begin{description}
% \item[\plainTeX:] Run \docstrip\ and extract the files.
% \item[\LaTeX:] Generate the documentation.
% \end{description}
% If you insist on using \LaTeX\ for \docstrip\ (really,
% \docstrip\ does not need \LaTeX), then inform the autodetect routine
% about your intention:
% \begin{quote}
%   \verb|latex \let\install=y% \iffalse meta-comment
%
% File: atbegshi.dtx
% Version: 2016/06/09 v1.18
% Info: At begin shipout hook
%
% Copyright (C) 2007-2011 by
%    Heiko Oberdiek <heiko.oberdiek at googlemail.com>
%    2016
%    https://github.com/ho-tex/oberdiek/issues
%
% This work may be distributed and/or modified under the
% conditions of the LaTeX Project Public License, either
% version 1.3c of this license or (at your option) any later
% version. This version of this license is in
%    http://www.latex-project.org/lppl/lppl-1-3c.txt
% and the latest version of this license is in
%    http://www.latex-project.org/lppl.txt
% and version 1.3 or later is part of all distributions of
% LaTeX version 2005/12/01 or later.
%
% This work has the LPPL maintenance status "maintained".
%
% This Current Maintainer of this work is Heiko Oberdiek.
%
% The Base Interpreter refers to any `TeX-Format',
% because some files are installed in TDS:tex/generic//.
%
% This work consists of the main source file atbegshi.dtx
% and the derived files
%    atbegshi.sty, atbegshi.pdf, atbegshi.ins, atbegshi.drv,
%    atbegshi-example1.tex, atbegshi-example2.tex,
%    atbegshi-test1.tex, atbegshi-test2.tex, atbegshi-test3.tex.
%
% Distribution:
%    CTAN:macros/latex/contrib/oberdiek/atbegshi.dtx
%    CTAN:macros/latex/contrib/oberdiek/atbegshi.pdf
%
% Unpacking:
%    (a) If atbegshi.ins is present:
%           tex atbegshi.ins
%    (b) Without atbegshi.ins:
%           tex atbegshi.dtx
%    (c) If you insist on using LaTeX
%           latex \let\install=y\input{atbegshi.dtx}
%        (quote the arguments according to the demands of your shell)
%
% Documentation:
%    (a) If atbegshi.drv is present:
%           latex atbegshi.drv
%    (b) Without atbegshi.drv:
%           latex atbegshi.dtx; ...
%    The class ltxdoc loads the configuration file ltxdoc.cfg
%    if available. Here you can specify further options, e.g.
%    use A4 as paper format:
%       \PassOptionsToClass{a4paper}{article}
%
%    Programm calls to get the documentation (example):
%       pdflatex atbegshi.dtx
%       makeindex -s gind.ist atbegshi.idx
%       pdflatex atbegshi.dtx
%       makeindex -s gind.ist atbegshi.idx
%       pdflatex atbegshi.dtx
%
% Installation:
%    TDS:tex/generic/oberdiek/atbegshi.sty
%    TDS:doc/latex/oberdiek/atbegshi.pdf
%    TDS:doc/latex/oberdiek/atbegshi-example1.tex
%    TDS:doc/latex/oberdiek/atbegshi-example2.tex
%    TDS:doc/latex/oberdiek/test/atbegshi-test1.tex
%    TDS:doc/latex/oberdiek/test/atbegshi-test2.tex
%    TDS:doc/latex/oberdiek/test/atbegshi-test3.tex
%    TDS:source/latex/oberdiek/atbegshi.dtx
%
%<*ignore>
\begingroup
  \catcode123=1 %
  \catcode125=2 %
  \def\x{LaTeX2e}%
\expandafter\endgroup
\ifcase 0\ifx\install y1\fi\expandafter
         \ifx\csname processbatchFile\endcsname\relax\else1\fi
         \ifx\fmtname\x\else 1\fi\relax
\else\csname fi\endcsname
%</ignore>
%<*install>
\input docstrip.tex
\Msg{************************************************************************}
\Msg{* Installation}
\Msg{* Package: atbegshi 2016/06/09 v1.18 At begin shipout hook (HO)}
\Msg{************************************************************************}

\keepsilent
\askforoverwritefalse

\let\MetaPrefix\relax
\preamble

This is a generated file.

Project: atbegshi
Version: 2016/06/09 v1.18

Copyright (C) 2007-2011 by
   Heiko Oberdiek <heiko.oberdiek at googlemail.com>

This work may be distributed and/or modified under the
conditions of the LaTeX Project Public License, either
version 1.3c of this license or (at your option) any later
version. This version of this license is in
   http://www.latex-project.org/lppl/lppl-1-3c.txt
and the latest version of this license is in
   http://www.latex-project.org/lppl.txt
and version 1.3 or later is part of all distributions of
LaTeX version 2005/12/01 or later.

This work has the LPPL maintenance status "maintained".

This Current Maintainer of this work is Heiko Oberdiek.

The Base Interpreter refers to any `TeX-Format',
because some files are installed in TDS:tex/generic//.

This work consists of the main source file atbegshi.dtx
and the derived files
   atbegshi.sty, atbegshi.pdf, atbegshi.ins, atbegshi.drv,
   atbegshi-example1.tex, atbegshi-example2.tex,
   atbegshi-test1.tex, atbegshi-test2.tex, atbegshi-test3.tex.

\endpreamble
\let\MetaPrefix\DoubleperCent

\generate{%
  \file{atbegshi.ins}{\from{atbegshi.dtx}{install}}%
  \file{atbegshi.drv}{\from{atbegshi.dtx}{driver}}%
  \usedir{tex/generic/oberdiek}%
  \file{atbegshi.sty}{\from{atbegshi.dtx}{package}}%
  \usedir{doc/latex/oberdiek}%
  \file{atbegshi-example1.tex}{\from{atbegshi.dtx}{example1}}%
  \file{atbegshi-example2.tex}{\from{atbegshi.dtx}{example2}}%
  \usedir{doc/latex/oberdiek/test}%
  \file{atbegshi-test1.tex}{\from{atbegshi.dtx}{test1}}%
  \file{atbegshi-test2.tex}{\from{atbegshi.dtx}{test2}}%
  \file{atbegshi-test3.tex}{\from{atbegshi.dtx}{test3}}%
  \nopreamble
  \nopostamble
  \usedir{source/latex/oberdiek/catalogue}%
  \file{atbegshi.xml}{\from{atbegshi.dtx}{catalogue}}%
}

\catcode32=13\relax% active space
\let =\space%
\Msg{************************************************************************}
\Msg{*}
\Msg{* To finish the installation you have to move the following}
\Msg{* file into a directory searched by TeX:}
\Msg{*}
\Msg{*     atbegshi.sty}
\Msg{*}
\Msg{* To produce the documentation run the file `atbegshi.drv'}
\Msg{* through LaTeX.}
\Msg{*}
\Msg{* Happy TeXing!}
\Msg{*}
\Msg{************************************************************************}

\endbatchfile
%</install>
%<*ignore>
\fi
%</ignore>
%<*driver>
\NeedsTeXFormat{LaTeX2e}
\ProvidesFile{atbegshi.drv}%
  [2016/06/09 v1.18 At begin shipout hook (HO)]%
\documentclass{ltxdoc}
\usepackage{holtxdoc}[2011/11/22]
\begin{document}
  \DocInput{atbegshi.dtx}%
\end{document}
%</driver>
% \fi
%
%
% \CharacterTable
%  {Upper-case    \A\B\C\D\E\F\G\H\I\J\K\L\M\N\O\P\Q\R\S\T\U\V\W\X\Y\Z
%   Lower-case    \a\b\c\d\e\f\g\h\i\j\k\l\m\n\o\p\q\r\s\t\u\v\w\x\y\z
%   Digits        \0\1\2\3\4\5\6\7\8\9
%   Exclamation   \!     Double quote  \"     Hash (number) \#
%   Dollar        \$     Percent       \%     Ampersand     \&
%   Acute accent  \'     Left paren    \(     Right paren   \)
%   Asterisk      \*     Plus          \+     Comma         \,
%   Minus         \-     Point         \.     Solidus       \/
%   Colon         \:     Semicolon     \;     Less than     \<
%   Equals        \=     Greater than  \>     Question mark \?
%   Commercial at \@     Left bracket  \[     Backslash     \\
%   Right bracket \]     Circumflex    \^     Underscore    \_
%   Grave accent  \`     Left brace    \{     Vertical bar  \|
%   Right brace   \}     Tilde         \~}
%
% \GetFileInfo{atbegshi.drv}
%
% \title{The \xpackage{atbegshi} package}
% \date{2016/06/09 v1.18}
% \author{Heiko Oberdiek\thanks
% {Please report any issues at https://github.com/ho-tex/oberdiek/issues}\\
% \xemail{heiko.oberdiek at googlemail.com}}
%
% \maketitle
%
% \begin{abstract}
% This package is a modern reimplementation of package \xpackage{everyshi}
% without the burden of compatibility. It makes use of \eTeX's if available.
% Both \LaTeX\ and \plainTeX\ are supported.
% \end{abstract}
%
% \tableofcontents
%
% \section{Documentation}
%
% Package \xpackage{atbegshi} redefines \cs{shipout} to insert
% hooks for user code that is executed before the page is shipped out.
% The code may modify or even discard the output page.
% Three hooks are implemented:
% \begin{enumerate}
% \item A hook that is executed for every page, see\\
%       \cs{AtBeginShipout}
% \item A hook that is executed for the next page only, see\\
%       \cs{AtBeginShipoutNext}
% \item A hook that is only executed for the first page, see\\
%       \cs{AtBeginShipoutFirst}
% \end{enumerate}
% The hooks are executed in this order. The following three macros
% provide the user interface for adding code to these hooks:
%
% \begin{declcs}{AtBeginShipout} \M{code}\\
%   \cs{AtBeginShipoutBox}
% \end{declcs}
% Execute the \meta{code} for every page.
% The page contents is held in box register \cs{AtBeginShipoutBox}
% and may be modified. Use \cs{AtBeginShipoutDiscard} if you want to
% discard the page.
%
% \emph{Note:} Package \xpackage{everyshi} uses box register 255. With
% package \xpackage{atbegshi} you must use \cs{AtBeginShipoutBox} instead.
%
% If \LaTeX\ calls \cs{shipout} in \cs{@outputpage} (part of its output
% routine),
% the meaning of \cs{protect} is \cs{noexpand}. \LaTeX\ sets
% \cs{protect} to the appropriate \cs{@typeset@protect} in the box
% that is shipped out. This is too late for the hooks, they
% are called earlier in the redefined \cs{shipout}. Therefore
% package \xpackage{atbegshi} sets \cs{protect} to \cs{@typeset@protect}
% before it calls the hooks.
% (In \cs{EveryShipout} of package \xpackage{everyshi} the
% user is responsible for the correct setting of \cs{protect}.)
%
% \begin{declcs}{AtBeginShipoutNext} \M{code}
% \end{declcs}
% This reimplements package \xpackage{everyshi}'s \cs{AtNextShipout}.
% The \meta{code} is executed at shipout time of the next page only.
% It is just a convenience macro, it can be easily replaced by something
% like:
%\begin{quote}
%\begin{verbatim}
%\newcommand{\MyShipoutHook}{}%
%\AtBeginShipout{\MyShipoutHook}
%\gdef\MyShipoutHook{%
%  ... do something with next page ...
%  \gdef\MyShipoutHook{}%
% }
%\end{verbatim}
%\end{quote}
% (This can be necessary, if hook order does matter).
%
% \begin{declcs}{AtBeginShipoutFirst} \M{code}
% \end{declcs}
% This reimplements \LaTeX's \cs{AtBeginDvi}. This hook is
% usually used for \cs{special} commands that include PostScript
% header files. The \cs{code} is directly executed in a \cs{vbox}
% that is put at the beginning of the output page. Dealing with
% the output box \cs{AtBeginShipoutBox} is not necessary and
% not permitted here.
%
% \begin{declcs}{AtBeginShipoutDiscard}
% \end{declcs}
% This macro notifies package \xpackage{atbegshi} that
% the output page is discarded. The remaining hook code
% and the remaining hooks are not executed and the page is thrown away.
% Also \cs{deadcycles} is cleared to zero like an ordinary \cs{shipout}
% would do.
%
% \begin{declcs}{AtBeginShipoutInit}
% \end{declcs}
% Usually the redefinition of \cs{shipout} is delayed by
% \cs{AtBeginDocument} (if this macro exists). This can be
% too late, if other packages also redefines \cs{shipout}
% and the order does matter.
% \cs{AtBeginShipoutInit} forces the immediate redefinition
% of \cs{shipout}.
%
% \begin{declcs}{AtBeginShipoutAddToBox} \M{stuff}\\
%   \cs{AtBeginShipoutAddToBoxForeground} \M{stuff}
% \end{declcs}
% A quite common use case is the addition of \cs{special}
% or other whatsits to the page output box. Macro
% \cs{AtBeginShipoutAddToBox} puts \meta{stuff} in a
% box with zeroed dimensions. The box with the \meta{stuff}
% is put in the upper left corner of the shipout box
% \cs{AtBeginShipoutBox}. Macro \cs{AtBeginShipoutAddToBox}
% puts the \meta{stuff} in the background, the other macro
% \cs{AtBeginShipoutAddToBoxForeground} in the foreground
% after the original shipout box contents is set.
%
% A void shipout box (that means a discarded page)
% remains void that means \meta{stuff} is ignored in this case.
% The box type of \cs{AtBeginShipoutBox} is preserved.
% Also the box nesting level for the original contents
% of \cs{AtBeginShipoutBox} remains, for example, to avoid trouble
% with links across pages in case of \hologo{pdfTeX}.
%
% \begin{declcs}{AtBeginShipoutUpperLeft} \M{background material}
% \end{declcs}
% This is a macro that puts material in the background of
% box \cs{AtBeginShipoutBox}.
% The \meta{background material} is set in an \cs{hbox}, the
% reference point is the upper left corner of the output page.
% In case of \pdfTeX\ in PDF mode, the settings of
% \cs{pdfhorigin} and \cs{pdfvorigin} are respected.
%
% The macro \cs{AtBeginShipoutUpperLeft} is intended to be used
% in one of the hook setting macros, such as
% \cs{AtBeginShipout}, \cs{AtBeginShipoutFirst}, or
% \cs{AtBeginShipoutNext}.
%
% For \LaTeX\ users the \meta{background material} is set
% inside a \texttt{picture} environment:
% \begin{quote}
%   |\begin{picture}(0,0)|\\
%   \mbox{}\quad |\setlength{\unitlength}{1pt}%|\\
%   \mbox{}\quad \meta{background material}\\
%   |\end{picture}|
% \end{quote}
%
% \begin{declcs}{AtBeginShipoutUpperLeftForeground} \M{foreground material}
% \end{declcs}
% See \cs{AtBeginShipoutUpperLeft}. The difference is that the material
% is put in the foreground.
%
% \begin{declcs}{AtBeginShipoutOriginalShipout} \meta{box}
% \end{declcs}
% It stores the meaning of \cs{shipout} at the time this package
% is loaded.
%
% \begin{declcs}{AtBeginShipoutBoxWidth}\\
%   \cs{AtBeginShipoutBoxHeight}\\
%   \cs{AtBeginShipoutBoxDepth}\\
% \end{declcs}
% These macros store the dimensions of the output box \cs{AtBeginShipoutBox}
% before the original shipout is called. If \cs{shipout} is not
% redefined before the package loading or the box dimensions
% are not changed by the redefined \cs{shipout}, these macros
% contain the dimensions of the shipout box. These values can be
% remembered by \cs{label} and \cs{ref}. For example, this is done
% by the package module \xpackage{zref-pagelayout} of project \xpackage{zref}.
% The dimensions of the shipout page can be used in some \hologo{TeX}
% engines (\hologo{pdfTeX} in PDF mode, \hologo{XeTeX}) to calculate
% the media size of the shipout page if \cs{pdfpagewidth} and
% \cs{pdfpageheight} are not set.
%
% \subsection{Examples}
%
% \subsubsection{Example: circle in background}
%
% In this example we put a circle in the background in the middle
% of the paper.
%
%    \begin{macrocode}
%<*example1>
%    \end{macrocode}
%    \begin{macrocode}
\documentclass[a4paper]{article}
\usepackage{color}
\usepackage{atbegshi}
%    \end{macrocode}
%    Package \xpackage{picture} makes life a little easier, because
%    we can now also use length specifications in |picture|'s commands.
%    \begin{macrocode}
\usepackage{picture}
%    \end{macrocode}
%    Now we draw the circle in the middle of the paper.
%    \cs{put} moves downwards, because the origin is at the top
%    of the page, not at its bottom.
%    \begin{macrocode}
\AtBeginShipout{%
  \AtBeginShipoutUpperLeft{%
    \put(0.5\paperwidth,-0.5\paperheight){\circle{10}}%
  }%
}
\begin{document}
\section{Hello World}
\newpage
\AtBeginShipoutNext{%
  \AtBeginShipoutUpperLeft{%
    \color{red}%
    \put(0,-0.5\paperheight){\line(1,0){\paperwidth}}%
    \put(0.5\paperwidth, 0){\line(0,-1){\paperheight}}%
  }%
}
Only on this page we add a red cross.
\newpage
This page has the circle only.
\par
\vspace{\fill}
The next page will be discarded.
\newpage
\AtBeginShipoutNext{%
  \AtBeginShipoutDiscard
}
This page is discarded.
\newpage
The last page.
\end{document}
%    \end{macrocode}
%    \begin{macrocode}
%</example1>
%    \end{macrocode}
%
% \subsubsection{Example: adding TrimBox for dvipdfmx}
%
% Now an example from ``real life'' follows. Someone from the mailing  list
% for dvipdfmx wants to put a TrimBox on every page. If we use
% \cs{AtBeginShipout}, we have to put the \cs{special} inside the
% box \cs{AtBeginShipoutBox} that gets shipped out.
%    \begin{macrocode}
%<*example2>
\documentclass{minimal}
\usepackage{atbegshi}
\usepackage[
  dvipdfm,
  paperwidth=630bp,
  paperheight=810bp
]{geometry}
\AtBeginShipout{%
  \setbox\AtBeginShipoutBox=\hbox{%
    \special{pdf: put @thispage <</TrimBox[9 9 621 801]>>}%
    \box\AtBeginShipoutBox
  }%
}
\begin{document}
  First page
  \newpage
  Second page
\end{document}
%</example2>
%    \end{macrocode}
% Remember, in \cs{AtBeginShipoutBoxFirst} the \cs{setbox} wrapper code
% is implicitly given and the \cs{special} is used directly.
%
% \section{Method of \cs{shipout} overloading}
%
% \subsection{\cs{shipout}}
%
% The \TeX\ primitive command \cs{shipout} takes a box specification
% and puts the box as a new page in the output file.
% There are two kinds of box specifications:
% \begin{description}
% \item[Direct boxes:]
%   They are given by \cs{hbox}, \cs{vbox}, or \cs{vtop},\\
%   e.g. |\shipout\hbox{Hello World}|.
% \item[Indirect boxes:]
%   \cs{box} or \cs{copy} references a box register by number.
%   The box register contains the contents of the box.
%
%   \emph{Note:} \cs{box} also clears the box register globally.
% \end{description}
% Then we have to differentiate between void and empty boxes:
% \begin{description}
% \item[Void:]
%   Initially or after \cs{box} there is no box in the box register.
%   In this cases the box register is not empty, but \emph{void}.
% \item[Empty:]
%   A box with empty contents, such as
%   |\hbox{}| (= \cs{null}) or |\vbox{}| is an \emph{empty \cs{hbox}}
%   or \emph{empty \cs{vbox}}. If a box register holds such a
%   box, the box still exists, therefore the box register is
%   \emph{not void}.
% \end{description}
%
% \subsection{\cs{afterassignment}}
%
% We want to overload \cs{shipout} to do something with the box.
% It is quite impossible to do this reliable by catching the box
% using macro arguments. The
% variety of box specifications is too large, Examples:
% \begin{quote}
%   |\shipout\null|\\
%   |\shipout\vbox{...}|\\
%   |\shipout\vtop\bgroup ...\egroup|\\
%   |\shipout\box255|
% \end{quote}
% Even worse, the braces don't need to be balanced:
% \begin{quote}
%   |\shipout\hbox\bgroup}|\\
%   |\shipout\vbox{\egroup|
% \end{quote}
% Happily TeX provides a reliable way via \cs{afterassignment}.
% It takes a macro name and executes it just after the assignment.
%
% Now we can redefine \cs{shipout}. The box specification that
% follows \cs{shipout} is catched by \cs{setbox}. This is an
% assignment to a box register. \cs{afterassignment}
% notifies \TeX, that we want to call \cs{@test} right after
% the assignment:
%\begin{quote}
%\begin{verbatim}
%\shipout :=
%  \afterassignment\@test
%  \setbox\mybox=
%\end{verbatim}
%\end{quote}
% We have seen different box specifications. Indirect boxes are easy
% to understand:
% \begin{quote}
%   |\shipout\box0| $\Rightarrow$ |\setbox\mybox=\box0 |\fbox{\cs{@test}}
% \end{quote}
% However direct boxes can have arbitrary contents with lots of
% other assignments. It would be quite inpredictable if \TeX\ would
% put |\@test| after the first of such an assignment or after the
% box specification if the box lacks of assignments.
% Therefore \TeX\ puts |\@test| right at the beginning of the
% box specification, e.g:
% \begin{quote}
%   |\shipout\hbox{Hello World}|\\
%   $\Rightarrow$
%   |\setbox\mybox=\hbox{|\fbox{\cs{@test}}| Hello World}|
% \end{quote}
%
% \subsection{Test for direct or indirect boxes}
%
% Now we want to execute \cs{@test}, but where are we?
% We can be after the completed box assignment, if \cs{shipout}
% was called with an indirect box. Or we are right at the beginning
% of a direct box.
%
% \subsubsection{With \eTeX}
%
% With the \eTeX's extensions the answer is very easy:
% Being inside the direct box means that we are inside a new group.
% The new primitive command \cs{currentgrouplevel} tells how
% deeply the groups are currently nested. Macro \cs{@test} just
% compares the previously stored group level with the current one:
%\begin{quote}
%\begin{verbatim}
%\shipout :=
%  \edef\saved@grouplevel{\number\currentgrouplevel}
%  \afterassignment\@test
%  \setbox\mybox=
%
%\@test :=
%  \ifnum\saved@grouplevel=\currentgrouplevel
%    % case: indirect box, the assignment is completed
%    \@output
%  \else
%    % case: direct box, we are inside the box
%    \aftergroup\@outbox
%  \fi
%\end{verbatim}
%\end{quote}
%
% \subsubsection{Without \eTeX}
%
% Life becomes complicate without \eTeX. We cannot ask the
% group level. However, if we are inside a direct box,
% the box register \cs{mybox} is not yet changed by \cs{setbox}.
% Thus we need a special initial value and compare
% it in \cs{@test} with the current value of the box.
%
% What can be used as initial value? Arbitrary box contents cannot
% be compared. \TeX\ only tells us a few properties:
% \begin{itemize}
% \item Box type:\cs{ifhbox}, \cs{ifvbox}
% \item Dimensions: \cs{wd}, \cs{ht}, \cs{dp}
% \item Voidness: \cs{ifvoid}
% \end{itemize}
% Unhappily all these qualities even combined are not sufficient
% for constructing an initial box value, because \cs{shipout} can be
% called with a box that is accidently just the same as the choosen
% initial value.
%
% Nevertheless we have two alternatives for an initial value:
% \begin{itemize}
% \item A box of some type with some funny settings that are
% unlikely to occur in real life, e.g a height of |4911sp-\maxdimen|.
% \item A void box.
% \end{itemize}
% A collision between this initial value and an indirect \cs{shipout}
% box with just the same value is possible.
% Then \cs{@test} will make a wrong decision that it is executed
% inside a direct box and delays \cs{@output} by \cs{aftergroup}.
% Thus \cs{@output} is not called at the place we want.
% In contrary, the result is an uncertainty about the place:
% \begin{itemize}
% \item \cs{shipout} is used in a group that perhaps closes some
%   pages later. A bad place for \cs{@output}.
% \item Without a surrounding group \cs{aftergroup} effectively kills
%   its argument.
% \end{itemize}
% In the first case of a box with special dimensions we can even loose
% the page. However in the case of the void box, this effect is even
% desired, because the original \cs{shipout} does not output void boxes.
% All we have to do is to ensure that our box \cs{mybox} is always
% void except for the phase when the overloaded \cs{shipout} is executed.
% And secondly we must keep this semantics of \cs{shipout} for the void case
% in our macros, namely \cs{@output}.
%\begin{quote}
%\begin{verbatim}
%\shipout :=
%  % trick to get a void box \mybox
%  \begingroup
%    \setbox\mybox=\box\mybox
%  \endgroup
%  \afterassignment\@test
%  \setbox\mybox=
%
%\@test :=
%  \ifvoid\mybox
%    \aftergroup\@output
%  \else
%    \@output
%  \fi
%\end{verbatim}
%\end{quote}
% The nasty case is |\shipout\box\voidb@x| where the indirect box
% is void and that must not generate an output page.
% If a surrounding group is missing the output is ignored
% because of \cs{aftergroup}.
% Otherwise \@output is called some time
% later when the surrounding group closes.
% But \cs{mybox} is void outside the execution phase
% of the redefined \cs{shipout}. Also \cs{@output} checks for a
% void box and cancels the page output.
% The disadvantage remains that the hook in \cs{@output} is called
% for a page that will not be output.
%
% \subsubsection{\cs{lastkern} method}
%
% At the beginning of a new box, there is no \cs{kern}, the contents
% of the box is still empty and \cs{lastkern} returns 0\,pt.
% This can be used to distinguish between direct and indirect boxes:
% We execute \cs{setbox} in a box with a preceding non-zero kern.
% After an indirect box, \cs{lastkern} sees this kern, otherwise
% it returns 0\,pt.
%\begin{quote}
%\begin{verbatim}
%\shipout :=
%  \begingroup
%    \setbox\mybox=\hbox\bgroup
%      \kern1pt
%      \afterassignment\shipout@test
%      \global\setbox\mybox=
%\@test :=
%  \ifdim\lastkern=0pt
%    % direct box
%    \aftergroup\egroup
%    \aftergroup\endgroup
%    \aftergroup\@output
%  \else
%    \egroup
%    \endgroup
%    \@output
%  \fi
%\end{verbatim}
%\end{quote}
% We have two \cs{setbox} commands. The first creates a controlled
% context box where we can safely insert a \cs{kern}. We get rid of
% this temporarily used context box by putting the local \cs{setbox}
% in a group.
%
% After the group we want to have our shipout box in \cs{mybox}.
% Therefore we use a global assignment here.
%
% \subsection{Output}
%
% With or without \eTeX\ we ensure the original behaviour of
% \cs{shipout} that void boxes do not generate output pages.
%
% Now we can place the hook \cs{@hook} for the user code that
% wants to manipulate the output box.
%\begin{quote}
%\begin{verbatim}
%\@output :=
%  \ifvoid\mybox
%    % cancel output of void box
%  \else
%    \@hook
%    \ifvoid\mybox
%      % user code in \@hook could has voided the box
%    \else
%      \original@shipout\box\mybox
%    \fi
%  \fi
%\end{verbatim}
%\end{quote}
%
% \subsection{Separate box register}
%
% So far we have said nothing about the box number of \cs{mybox}.
% The following case that outputs the same page twice
% shows that we are not free in the use
% of the box register:
% \begin{quote}
%   |\shipout\copy<num> \shipout\box<num>|
% \end{quote}
% We manipulate the box by the hook and without \eTeX\ the box
% must even be voided. However, the use case above requires that
% the box contents does not change at all. Therefore we must
% reserve a separate box register to avoid collisions with
% user box registers.
%
% \emph{Note:} Box register number 255 is special for the output
% routine, because TeX complains if this box is not voided
% by the output routine.
% However, this requirement does not apply to \cs{shipout} at all.
% In fact \cs{shipout} does not change any box register. This is
% usually done by a call of \cs{box}, but the output routine
% can do it later \emph{after} invoking of \cs{shipout}.
%
% \subsection{Summary}
%
% \subsubsection{With \eTeX}
%
% Putting the pieces together we get for \eTeX:
%\begin{quote}
%\begin{verbatim}
%\newbox\mybox
%\let\original@shipout\shipout
%
%\shipout :=
%  \edef\saved@grouplevel{\number\currentgrouplevel}
%  \afterassignment\@test
%  \setbox\mybox=
%
%\@test :=
%  \ifnum\saved@grouplevel<\currentgrouplevel
%    \expandafter\aftergroup
%  \fi
%  \@output
%
%\@output :=
%  \ifvoid\mybox
%    % cancel output of void box
%  \else
%    \@hook
%    \ifvoid\mybox
%      % user code in \@hook could have voided the box
%    \else
%      \original@shipout\box\mybox
%    \fi
%  \fi
%\end{verbatim}
%\end{quote}
%
% \subsubsection{Without \eTeX, traditional way}
%
% And for \TeX\ without \eTeX:
%\begin{quote}
%\begin{verbatim}
%\newbox\mybox
%\begingroup
%   \setbox\mybox=\box\mybox % ensure \mybox is void
%\endgroup
%\let\original@shipout\shipout
%
%\shipout :=
%  % trick to get a void box \mybox
%  \begingroup
%    \setbox\mybox=\box\mybox
%  \endgroup
%  \afterassignment\@test
%  \setbox\mybox=
%
%\@test :=
%  \ifvoid\mybox
%    \expandafter\aftergroup
%  \fi
%  \@output
%
%\@output :=
%  \ifvoid\mybox
%    % cancel output of void box
%  \else
%    \@hook
%    \ifvoid\mybox
%      % user code in \@hook could have voided the box
%    \else
%      \original@shipout\box\mybox
%    \fi
%  \fi
%\end{verbatim}
%\end{quote}
%
% \subsubsection{\cs{lastkern} method}
%
% And for \TeX\ without \eTeX\ using the \cs{lastkern} method:
%\begin{quote}
%\begin{verbatim}
%\newbox\mybox
%\let\original@shipout\shipout
%
%\shipout :=
%  \begingroup
%  \setbox\mybox=\hbox\bgroup
%  \kern1pt
%  \afterassignment\@test
%  \setbox\mybox=
%
%\@test :=
%  \ifdim\lastkern=0pt
%    \expandafter\aftergroup
%  \fi
%  \@output
%
%\@output :=
%  \egroup
%  \endgroup
%  \ifvoid\mybox
%    % cancel output of void box
%  \else
%    \@hook
%    \ifvoid\mybox
%      % user code in \@hook could have voided the box
%    \else
%      \original@shipout\box\mybox
%    \fi
%  \fi
%\end{verbatim}
%\end{quote}
%
%
% \StopEventually{
% }
%
% \section{Implementation}
%
% Package \xpackage{atbegshi} uses \eTeX's \cs{currentgrouplevel},
% if it is available. Otherwise the \cs{lastkern} method is used.
%
%    \begin{macrocode}
%<*package>
%    \end{macrocode}
%
% \subsection{Reload check and package identification}
%    Reload check, especially if the package is not used with \LaTeX.
%    \begin{macrocode}
\begingroup\catcode61\catcode48\catcode32=10\relax%
  \catcode13=5 % ^^M
  \endlinechar=13 %
  \catcode35=6 % #
  \catcode39=12 % '
  \catcode44=12 % ,
  \catcode45=12 % -
  \catcode46=12 % .
  \catcode58=12 % :
  \catcode64=11 % @
  \catcode123=1 % {
  \catcode125=2 % }
  \expandafter\let\expandafter\x\csname ver@atbegshi.sty\endcsname
  \ifx\x\relax % plain-TeX, first loading
  \else
    \def\empty{}%
    \ifx\x\empty % LaTeX, first loading,
      % variable is initialized, but \ProvidesPackage not yet seen
    \else
      \expandafter\ifx\csname PackageInfo\endcsname\relax
        \def\x#1#2{%
          \immediate\write-1{Package #1 Info: #2.}%
        }%
      \else
        \def\x#1#2{\PackageInfo{#1}{#2, stopped}}%
      \fi
      \x{atbegshi}{The package is already loaded}%
      \aftergroup\endinput
    \fi
  \fi
\endgroup%
%    \end{macrocode}
%    Package identification:
%    \begin{macrocode}
\begingroup\catcode61\catcode48\catcode32=10\relax%
  \catcode13=5 % ^^M
  \endlinechar=13 %
  \catcode35=6 % #
  \catcode39=12 % '
  \catcode40=12 % (
  \catcode41=12 % )
  \catcode44=12 % ,
  \catcode45=12 % -
  \catcode46=12 % .
  \catcode47=12 % /
  \catcode58=12 % :
  \catcode64=11 % @
  \catcode91=12 % [
  \catcode93=12 % ]
  \catcode123=1 % {
  \catcode125=2 % }
  \expandafter\ifx\csname ProvidesPackage\endcsname\relax
    \def\x#1#2#3[#4]{\endgroup
      \immediate\write-1{Package: #3 #4}%
      \xdef#1{#4}%
    }%
  \else
    \def\x#1#2[#3]{\endgroup
      #2[{#3}]%
      \ifx#1\@undefined
        \xdef#1{#3}%
      \fi
      \ifx#1\relax
        \xdef#1{#3}%
      \fi
    }%
  \fi
\expandafter\x\csname ver@atbegshi.sty\endcsname
\ProvidesPackage{atbegshi}%
  [2016/06/09 v1.18 At begin shipout hook (HO)]%
%    \end{macrocode}
%
% \subsection{Catcodes}
%
%    \begin{macrocode}
\begingroup\catcode61\catcode48\catcode32=10\relax%
  \catcode13=5 % ^^M
  \endlinechar=13 %
  \catcode123=1 % {
  \catcode125=2 % }
  \catcode64=11 % @
  \def\x{\endgroup
    \expandafter\edef\csname AtBegShi@AtEnd\endcsname{%
      \endlinechar=\the\endlinechar\relax
      \catcode13=\the\catcode13\relax
      \catcode32=\the\catcode32\relax
      \catcode35=\the\catcode35\relax
      \catcode61=\the\catcode61\relax
      \catcode64=\the\catcode64\relax
      \catcode123=\the\catcode123\relax
      \catcode125=\the\catcode125\relax
    }%
  }%
\x\catcode61\catcode48\catcode32=10\relax%
\catcode13=5 % ^^M
\endlinechar=13 %
\catcode35=6 % #
\catcode64=11 % @
\catcode123=1 % {
\catcode125=2 % }
\def\TMP@EnsureCode#1#2{%
  \edef\AtBegShi@AtEnd{%
    \AtBegShi@AtEnd
    \catcode#1=\the\catcode#1\relax
  }%
  \catcode#1=#2\relax
}
\TMP@EnsureCode{40}{12}% (
\TMP@EnsureCode{41}{12}% )
\TMP@EnsureCode{44}{12}% ,
\TMP@EnsureCode{45}{12}% -
\TMP@EnsureCode{47}{12}% /
\TMP@EnsureCode{46}{12}% .
\TMP@EnsureCode{58}{12}% :
\TMP@EnsureCode{91}{12}% [
\TMP@EnsureCode{93}{12}% ]
\TMP@EnsureCode{94}{7}% ^(superscript)
\TMP@EnsureCode{96}{12}% `
\edef\AtBegShi@AtEnd{\AtBegShi@AtEnd\noexpand\endinput}
%    \end{macrocode}
%
% \subsection{Preparations}
%
%    \begin{macrocode}
\begingroup\expandafter\expandafter\expandafter\endgroup
\expandafter\ifx\csname RequirePackage\endcsname\relax
  \def\TMP@RequirePackage#1[#2]{%
    \begingroup\expandafter\expandafter\expandafter\endgroup
    \expandafter\ifx\csname ver@#1.sty\endcsname\relax
      \input #1.sty\relax
    \fi
  }%
  \TMP@RequirePackage{infwarerr}[2007/09/09]%
  \TMP@RequirePackage{ltxcmds}[2010/03/01]%
\else
  \RequirePackage{infwarerr}[2007/09/09]%
  \RequirePackage{ltxcmds}[2010/03/01]%
\fi
%    \end{macrocode}
%
%    \begin{macro}{\AtBegShi@CheckDefinable}
%    \begin{macrocode}
\begingroup\expandafter\expandafter\expandafter\endgroup
\expandafter\ifx\csname @ifdefinable\endcsname\relax
  \def\AtBegShi@CheckDefinable#1{%
    \ifcase\ifx#1\relax
             \ltx@one
           \else
             \ifx#1\@undefined
               \ltx@one
             \else
               \ltx@zero
             \fi
           \fi
      \@PackageError{atbegshi}{%
        \string#1\space is already defined%
      }\@ehd
    \fi
  }%
\else
  \def\AtBegShi@CheckDefinable#1{%
    \@ifdefinable{#1}{}%
  }%
\fi
%    \end{macrocode}
%    \end{macro}
%
%    \begin{macro}{\ifAtBegShi@Discarded}
%    \begin{macrocode}
\ltx@newif\ifAtBegShi@Discarded
%    \end{macrocode}
%    \end{macro}
%    \begin{macro}{\AtBeginShipoutDiscard}
%    \begin{macrocode}
\AtBegShi@CheckDefinable\AtBeginShipoutDiscard
\def\AtBeginShipoutDiscard{%
  \deadcycles=\ltx@zero
  \global\AtBegShi@Discardedtrue
}
%    \end{macrocode}
%    \end{macro}
%
%    \begin{macrocode}
\begingroup\expandafter\expandafter\expandafter\endgroup
\expandafter\ifx\csname currentgrouplevel\endcsname\relax
  \catcode`X=9  % ignore
  \catcode`E=14 % comment
\else
  \catcode`X=14 % comment
  \catcode`E=9  % ignore
\fi
%    \end{macrocode}
%
%    \begin{macro}{\AtBegShi@Shipout}
%    \begin{macrocode}
\def\AtBegShi@Shipout{%
X \begingroup
X \setbox\AtBeginShipoutBox=\hbox\bgroup
X \kern\p@
E \edef\AtBegShi@GroupLevel{\number\currentgrouplevel}%
  \afterassignment\AtBegShi@Test
X \global
  \setbox\AtBeginShipoutBox=%
}
%    \end{macrocode}
%    \end{macro}
%
%    \begin{macro}{\AtBegShi@Test}
%    \begin{macrocode}
\def\AtBegShi@Test{%
X \ifdim\lastkern=0pt %
E \ifnum\AtBegShi@GroupLevel<\currentgrouplevel
    \expandafter\aftergroup
  \fi
  \AtBegShi@Output
}
%    \end{macrocode}
%    \end{macro}
%
%    \begin{macro}{\AtBegShi@Output}
%    \begin{macrocode}
\def\AtBegShi@Output{%
X \egroup
X \endgroup
  \ifvoid\AtBeginShipoutBox
    \@PackageWarning{atbegshi}{Ignoring void shipout box}%
  \else
    \let\AtBegShi@OrgProtect\protect
    \csname set@typeset@protect\endcsname
    \global\AtBegShi@Discardedfalse
    \AtBegShi@Hook
    \expandafter\gdef\expandafter\AtBegShi@HookNext
    \expandafter{\expandafter}%
    \AtBegShi@HookNext
    \ifAtBegShi@Discarded
      \@PackageInfoNoLine{atbegshi}{Shipout page discarded}%
      \global\AtBegShi@Discardedfalse
      \begingroup
        \setbox\AtBeginShipoutBox\box\AtBeginShipoutBox
      \endgroup
      \let\protect\AtBegShi@OrgProtect
    \else
      \AtBegShi@First
      \let\protect\AtBegShi@OrgProtect
      \AtBegShi@GetBoxSize\AtBeginShipoutBox
      \ltx@ifundefined{AtNextShipout}{%
      }{%
        \AtNextShipout{\AtBegShi@GetBoxSize\@cclv}%
      }%
      \AtBeginShipoutOriginalShipout\box\AtBeginShipoutBox
    \fi
  \fi
}
%    \end{macrocode}
%    \end{macro}
%
%    \begin{macro}{\AtBegShi@GetBoxSize}
%    \begin{macrocode}
\def\AtBegShi@GetBoxSize#1{%
  \xdef\AtBeginShipoutBoxWidth{\the\wd#1}%
  \xdef\AtBeginShipoutBoxHeight{\the\ht#1}%
  \xdef\AtBeginShipoutBoxDepth{\the\dp#1}%
}
%    \end{macrocode}
%    \end{macro}
%    \begin{macro}{\AtBeginShipoutBoxWidth}
%    \begin{macrocode}
\def\AtBeginShipoutBoxWidth{0pt}
%    \end{macrocode}
%    \end{macro}
%    \begin{macro}{\AtBeginShipoutBoxHeight}
%    \begin{macrocode}
\def\AtBeginShipoutBoxHeight{0pt}
%    \end{macrocode}
%    \end{macro}
%    \begin{macro}{\AtBeginShipoutBoxDepth}
%    \begin{macrocode}
\def\AtBeginShipoutBoxDepth{0pt}
%    \end{macrocode}
%    \end{macro}
%
%    \begin{macrocode}
\catcode`\X=11 %
\catcode`\E=11 %
%    \end{macrocode}
%
%    \begin{macro}{\AtBegShi@First}
%    \begin{macrocode}
\def\AtBegShi@First{%
  \ifx\AtBegShi@HookFirst\ltx@empty
  \else
    \AtBeginShipoutAddToBox{\AtBegShi@HookFirst}%
  \fi
  \global\let\AtBegShi@First\ltx@empty
  \global\let\AtBeginShipoutFirst\AtBegShi@FirstDisabled
}
%    \end{macrocode}
%    \end{macro}
%
%    \begin{macro}{\AtBegShi@Hook}
%    \begin{macrocode}
\gdef\AtBegShi@Hook{}
%    \end{macrocode}
%    \end{macro}
%    \begin{macro}{\AtBegShi@HookNext}
%    \begin{macrocode}
\gdef\AtBegShi@HookNext{}
%    \end{macrocode}
%    \end{macro}
%    \begin{macro}{\AtBegShi@HookFirst}
%    \begin{macrocode}
\gdef\AtBegShi@HookFirst{}
%    \end{macrocode}
%    \end{macro}
%    \begin{macro}{\AtBeginShipout}
%    \begin{macrocode}
\AtBegShi@CheckDefinable\AtBeginShipout
\def\AtBeginShipout{%
  \AtBegShi@AddHook\AtBegShi@Hook
}
%    \end{macrocode}
%    \end{macro}
%    \begin{macro}{\AtBeginShipoutNext}
%    \begin{macrocode}
\AtBegShi@CheckDefinable\AtBeginShipoutNext
\def\AtBeginShipoutNext{%
  \AtBegShi@AddHook\AtBegShi@HookNext
}
%    \end{macrocode}
%    \end{macro}
%    \begin{macro}{\AtBeginShipoutFirst}
%    \begin{macrocode}
\AtBegShi@CheckDefinable\AtBeginShipoutFirst
\def\AtBeginShipoutFirst{%
  \AtBegShi@AddTo\AtBegShi@HookFirst
}
%    \end{macrocode}
%    \end{macro}
%    \begin{macro}{\AtBegShi@FirstDisabled}
%    \begin{macrocode}
\long\def\AtBegShi@FirstDisabled#1{%
  \@PackageWarning{atbegshi}{%
    First page is already shipped out, ignoring\MessageBreak
    \string\AtBeginShipoutFirst
  }%
}
%    \end{macrocode}
%    \end{macro}
%
%    \begin{macro}{\AtBegShi@AddTo}
%    \begin{macrocode}
\begingroup\expandafter\expandafter\expandafter\endgroup
\expandafter\ifx\csname g@addto@macro\endcsname\relax
  \long\def\AtBegShi@AddTo#1#2{%
    \begingroup
      \toks\ltx@zero\expandafter{#1#2}%
      \xdef#1{\the\toks\ltx@zero}%
    \endgroup
  }%
\else
  \let\AtBegShi@AddTo\g@addto@macro
\fi
%    \end{macrocode}
%    \end{macro}
%
%    \begin{macro}{\AtBegShi@AddHook}
%    \begin{macrocode}
\long\def\AtBegShi@AddHook#1#2{%
  \AtBegShi@AddTo#1{\AtBegShi@Item{#2}}%
}
%    \end{macrocode}
%    \end{macro}
%
%    \begin{macro}{\AtBegShi@Item}
%    \begin{macrocode}
\long\def\AtBegShi@Item#1{%
  \ifAtBegShi@Discarded
  \else
    #1%
    \ifAtBegShi@Discarded
    \else
      \ifvoid\AtBeginShipoutBox
        \@PackageWarning{atbegshi}{%
          Shipout box was voided by hook,\MessageBreak
          ignoring shipout box%
        }%
        \AtBeginShipoutDiscard
      \fi
    \fi
  \fi
}
%    \end{macrocode}
%    \end{macro}
%
%    \begin{macro}{\AtBeginShipoutInit}
%    \begin{macrocode}
\AtBegShi@CheckDefinable\AtBeginShipoutInit
\def\AtBeginShipoutInit{%
  \ltx@IfUndefined{newbox}{%
    \@PackageError{atbegshi}{%
      \string\AtBeginShipoutInit\space failed\MessageBreak
       because of missing \expandafter\string\csname newbox\endcsname
    }\@ehc
  }{%
    \csname newbox\endcsname\AtBeginShipoutBox
    \AtBegShi@CheckDefinable\AtBeginShipoutOriginalShipout
    \global\let\AtBeginShipoutOriginalShipout\shipout
    \global\let\shipout\AtBegShi@Shipout
  }%
  \gdef\AtBeginShipoutInit{}%
}
%    \end{macrocode}
%    \end{macro}
%    \begin{macrocode}
\begingroup\expandafter\expandafter\expandafter\endgroup
\expandafter\ifx\csname AtBeginDocument\endcsname\relax
  \AtBeginShipoutInit
\else
  \AtBeginDocument{\AtBeginShipoutInit}%
\fi
%    \end{macrocode}
%
% \subsection{Additions to the shipout box}
%
%    \begin{macro}{\AtBeginShipoutAddToBox}
%    \begin{macrocode}
\def\AtBeginShipoutAddToBox#1{%
  \ifhbox\AtBeginShipoutBox
    \edef\AtBegShi@restore{%
      \hfuzz=\the\hfuzz\relax
      \hbadness=\the\hbadness\relax
    }%
    \hfuzz=1073741823sp\relax
    \hbadness=2147483647\relax
    \setbox\AtBeginShipoutBox=\hbox to \wd\AtBeginShipoutBox{%
      \setbox\ltx@zero=\hbox{%
        \begingroup
          \AtBegShi@restore
          #1%
        \endgroup
      }%
      \wd\ltx@zero=0pt\relax
      \ht\ltx@zero=0pt\relax
      \dp\ltx@zero=0pt\relax
      \raise\ht\AtBeginShipoutBox\copy\ltx@zero
      \unhcopy\AtBeginShipoutBox
    }%
    \AtBegShi@restore
  \else
    \ifvbox\AtBeginShipoutBox
      \edef\AtBegShi@restore{%
        \vfuzz=\the\vfuzz\relax
        \vbadness=\the\vbadness\relax
        \dimen\ltx@zero=\the\dimen\ltx@zero\relax
      }%
      \edef\AtBegShi@restorebox{%
        \ht\AtBeginShipoutBox=\the\ht\AtBeginShipoutBox\relax
        \dp\AtBeginShipoutBox=\the\dp\AtBeginShipoutBox\relax
      }%
      \vfuzz=1073741823sp\relax
      \vbadness=2147483647\relax
      \dimen\ltx@zero=\ht\AtBeginShipoutBox
      \advance\dimen\ltx@zero by \dp\AtBeginShipoutBox
      \setbox\AtBeginShipoutBox=\vbox to \dimen\ltx@zero{%
        \setbox\ltx@zero=\hbox{%
          \begingroup
            \AtBegShi@restore
            #1%
          \endgroup
        }%
        \wd\ltx@zero=0pt\relax
        \ht\ltx@zero=0pt\relax
        \dp\ltx@zero=0pt\relax
        \baselineskip=0pt\relax
        \lineskip=0pt\relax
        \lineskiplimit=0pt\relax
        \copy\ltx@zero
        \unvbox\AtBeginShipoutBox
        \kern0pt%
      }%
      \AtBegShi@restore
      \AtBegShi@restorebox
    \fi
  \fi
}
%    \end{macrocode}
%    \end{macro}
%
%    \begin{macro}{\AtBeginShipoutAddToBoxForeground}
%    \begin{macrocode}
\def\AtBeginShipoutAddToBoxForeground#1{%
  \ifhbox\AtBeginShipoutBox
    \edef\AtBegShi@restore{%
      \hfuzz=\the\hfuzz\relax
      \hbadness=\the\hbadness\relax
    }%
    \hfuzz=1073741823sp\relax
    \hbadness=2147483647\relax
    \setbox\AtBeginShipoutBox=\hbox to \wd\AtBeginShipoutBox{%
      \unhcopy\AtBeginShipoutBox
      \kern-\wd\AtBeginShipoutBox
      \setbox\ltx@zero=\hbox{%
        \begingroup
          \AtBegShi@restore
          #1%
        \endgroup
      }%
      \wd\ltx@zero=0pt\relax
      \ht\ltx@zero=0pt\relax
      \dp\ltx@zero=0pt\relax
      \raise\ht\AtBeginShipoutBox\copy\ltx@zero
      \kern\wd\AtBeginShipoutBox
    }%
    \AtBegShi@restore
  \else
    \ifvbox\AtBeginShipoutBox
      \edef\AtBegShi@restore{%
        \vfuzz=\the\vfuzz\relax
        \vbadness=\the\vbadness\relax
        \dimen\ltx@zero=\the\dimen\ltx@zero\relax
      }%
      \edef\AtBegShi@restorebox{%
        \ht\AtBeginShipoutBox=\the\ht\AtBeginShipoutBox\relax
        \dp\AtBeginShipoutBox=\the\dp\AtBeginShipoutBox\relax
      }%
      \vfuzz=1073741823sp\relax
      \vbadness=2147483647\relax
      \dimen\ltx@zero=\ht\AtBeginShipoutBox
      \advance\dimen\ltx@zero by \dp\AtBeginShipoutBox
      \setbox\AtBeginShipoutBox=\vbox to \dimen\ltx@zero{%
        \setbox\ltx@zero=\hbox{%
          \begingroup
            \AtBegShi@restore
            #1%
          \endgroup
        }%
        \wd\ltx@zero=0pt\relax
        \ht\ltx@zero=0pt\relax
        \dp\ltx@zero=0pt\relax
        \baselineskip=0pt\relax
        \lineskip=0pt\relax
        \lineskiplimit=0pt\relax
        \unvbox\AtBeginShipoutBox
        \kern-\dimen\ltx@zero
        \copy\ltx@zero
        \kern\dimen\ltx@zero
      }%
      \AtBegShi@restore
      \AtBegShi@restorebox
    \fi
  \fi
}
%    \end{macrocode}
%    \end{macro}
%
% \subsection{Positioning}
%
%    \begin{macrocode}
\begingroup\expandafter\expandafter\expandafter\endgroup
\expandafter\ifx\csname RequirePackage\endcsname\relax
  \def\TMP@RequirePackage#1[#2]{%
    \begingroup\expandafter\expandafter\expandafter\endgroup
    \expandafter\ifx\csname ver@#1.sty\endcsname\relax
      \input #1.sty\relax
    \fi
  }%
  \TMP@RequirePackage{ifpdf}[2011/01/30]%
\else
  \RequirePackage{ifpdf}[2011/01/30]%
\fi
%    \end{macrocode}
%
%    \begin{macrocode}
\ifpdf
  \def\AtBegShi@horigin{%
    \ifx\pdfhorigin\@undefined\pdfvariable horigin\else\pdfhorigin\fi}%
  \def\AtBegShi@vorigin{%
    \ifx\pdfvorigin\@undefined\pdfvariable vorigin\else\pdfvorigin\fi}%
\else
  \def\AtBegShi@horigin{72.27pt}%
  \def\AtBegShi@vorigin{72.27pt}%
\fi
%    \end{macrocode}
%
%    \begin{macrocode}
\begingroup
\ifcase
    \expandafter\ifx\csname picture\endcsname\relax
      1%
    \else
      \expandafter\ifx\csname endpicture\endcsname\relax
        1%
      \else
        0%
      \fi
    \fi
  \endgroup
  \def\AtBegShi@BeginPicture{%
    \begingroup
    \picture(0,0)\relax
    \begingroup\expandafter\expandafter\expandafter\endgroup
    \expandafter\ifx\csname unitlength\endcsname\relax
    \else
      \unitlength=1pt\relax
    \fi
    \ignorespaces
  }%
  \def\AtBegShi@EndPicture{%
    \endpicture
    \endgroup
  }%
\else
  \endgroup
  \def\AtBegShi@BeginPicture{%
    \setbox\ltx@zero=\hbox\bgroup
    \begingroup
    \ignorespaces
  }%
  \def\AtBegShi@EndPicture{%
    \endgroup
    \egroup
    \ht\ltx@zero=0pt\relax
    \dp\ltx@zero=0pt\relax
    \copy\ltx@zero
  }%
\fi
%    \end{macrocode}
%
%    \begin{macro}{\AtBeginShipoutUpperLeft}
%    A surrounding \cs{rlap} is not necessary, because
%    the stuff is put in an \cs{hbox} with zero width.
%    \begin{macrocode}
\def\AtBeginShipoutUpperLeft#1{%
  \AtBeginShipoutAddToBox{%
    \kern-\AtBegShi@horigin\relax
    \vbox to 0pt{%
      \kern-\AtBegShi@vorigin\relax
      \AtBegShi@BeginPicture
      #1%
      \AtBegShi@EndPicture
      \vss
    }%
  }%
}
%    \end{macrocode}
%    \end{macro}
%
%    \begin{macro}{\AtBeginShipoutUpperLeftForeground}
%    \begin{macrocode}
\def\AtBeginShipoutUpperLeftForeground#1{%
  \AtBeginShipoutAddToBoxForeground{%
    \kern-\AtBegShi@horigin\relax
    \vbox to 0pt{%
      \kern-\AtBegShi@vorigin\relax
      \AtBegShi@BeginPicture
      #1%
      \AtBegShi@EndPicture
      \vss
    }%
  }%
}
%    \end{macrocode}
%    \end{macro}
%
% \subsection{Patches}
%
%    Patches for \LaTeX\ packages that redefine \cs{shipout}.
%    \LaTeX\ is now supposed to use \eTeX. Thus we do not patch,
%    without \LaTeX\ and \eTeX.
%    \begin{macrocode}
\def\AtBegShi@AbortIfUndefined#1{%
  \begingroup\expandafter\expandafter\expandafter\endgroup
  \expandafter\ifx\csname#1\endcsname\relax
    \expandafter\AtBegShi@AtEnd
  \fi
}
\AtBegShi@AbortIfUndefined{currentgrouplevel}%
\AtBegShi@AbortIfUndefined{AtBeginDocument}%
\AtBegShi@AbortIfUndefined{@ifpackageloaded}%
\AtBegShi@AbortIfUndefined{@ifclassloaded}%
%    \end{macrocode}
%
% \subsubsection{Package \xpackage{crop}}
%
% Fix of method and box.
%    \begin{macrocode}
\def\AtBegShi@PatchCrop{%
  \begingroup
    \def\AtBegShi@Crop@shipout{%
      \afterassignment\CROP@ship
      \setbox\@cclv=%
    }%
    \def\AtBegShi@Crop@ship{%
      \ifvoid\@cclv
        \expandafter\aftergroup
      \fi
      \CROP@@ship
    }%
    \def\AtBegShi@Crop@shiplist{%
      \lineskip\z@
      \lineskiplimit\z@
      \baselineskip\z@
      \CROP@kernel
      \box\@cclv
    }%
    \def\AtBegShi@Crop@@ship{%
      \CROP@shipout\vbox{%
        \CROP@shiplist
      }%
    }%
    \ifx\AtBegShi@Crop@ship\CROP@ship
      \ifx\AtBegShi@Crop@shiplist\CROP@shiplist
        \ifx\AtBegShi@Crop@@ship\CROP@@ship
          \let\AtBegShi@found\relax
          \ifx\shipout\AtBegShi@Crop@shipout
            \def\AtBegShi@found{\shipout}%
          \else\ifx\AtBeginShipoutOriginalShipout\AtBegShi@Crop@shipout
            \def\AtBegShi@found{\AtBeginShipoutOriginalShipout}%
          \else\ifx\@EveryShipout@Org@Shipout\AtBegShi@Crop@shipout
            \def\AtBegShi@found{\@EveryShipout@Org@Shipout}%
          \else\ifx\GPTorg@shipout\AtBegShi@Crop@shipout
            \def\AtBegShi@found{\GPTorg@shipout}%
          \else\ifx\THBorg@shipout\AtBegShi@Crop@shipout
            \def\AtBegShi@found{\THBorg@shipout}%
          \else\ifx\mem@oldshipout\AtBegShi@Crop@shipout
            \def\AtBegShi@found{\mem@oldshipout}%
          \fi\fi\fi\fi\fi\fi
          \ifx\AtBegShi@found\relax
          \else
            \expandafter\endgroup
            \expandafter\def\AtBegShi@found{%
              \edef\AtBegShi@GroupLevel{\number\currentgrouplevel}%
              \afterassignment\CROP@ship
              \setbox\AtBeginShipoutBox=%
            }%
            \def\CROP@ship{%
              \ifnum\AtBegShi@GroupLevel=\currentgrouplevel
              \else
                \expandafter\aftergroup
              \fi
              \CROP@@ship
            }%
            \def\CROP@shiplist{%
              \lineskip 0pt\relax
              \lineskiplimit 0pt\relax
              \baselineskip 0pt\relax
              \CROP@kernel
              \box\AtBeginShipoutBox
            }%
            \def\CROP@@ship{%
              \ifvoid\AtBeginShipoutBox
              \else
                \setbox\AtBeginShipoutBox=\vbox{%
                  \CROP@shiplist
                }%
                \AtBegShi@GetBoxSize\AtBeginShipoutBox
                \expandafter\CROP@shipout
                \expandafter\box
                \expandafter\AtBeginShipoutBox
              \fi
            }%
            \@PackageInfoNoLine{atbegshi}{Package `crop' patched}%
            \begingroup
          \fi
        \fi
      \fi
    \fi
  \endgroup
  \let\AtBegShi@PatchCrop\relax
}
\@ifpackageloaded{crop}{%
  \AtBegShi@PatchCrop
}{%
  \AtBeginDocument{\AtBegShi@PatchCrop}%
}
%    \end{macrocode}
%
% \subsubsection{Package \xpackage{everyshi}}
%
%     Fix of method. Use of box 255 is not changed.
%    \begin{macrocode}
\def\AtBegShi@PatchEveryshi{%
  \begingroup
    \long\def\AtBegShi@Everyshi@shipout{%
      \afterassignment\@EveryShipout@Test
      \global\setbox\@cclv= %
    }%
    \long\def\AtBegShi@Everyshi@Test{%
      \ifvoid\@cclv\relax
        \aftergroup\@EveryShipout@Output
      \else
        \@EveryShipout@Output
      \fi
    }%
    \ifx\AtBegShi@Everyshi@Test\@EveryShipout@Test
      \let\AtBegShi@found\relax
      \ifx\shipout\AtBegShi@Everyshi@shipout
        \def\AtBegShi@found{\shipout}%
      \else\ifx\AtBeginShipoutOriginalShipout\AtBegShi@Everyshi@shipout
        \def\AtBegShi@found{\AtBeginShipoutOriginalShipout}%
      \else\ifx\CROP@shipout\AtBegShi@Everyshi@shipout
        \def\AtBegShi@found{\CROP@shipout}%
      \else\ifx\GPTorg@shipout\AtBegShi@Everyshi@shipout
        \def\AtBegShi@found{\GPTorg@shipout}%
      \else\ifx\THBorg@shipout\AtBegShi@Everyshi@shipout
        \def\AtBegShi@found{\THBorg@shipout}%
      \else\ifx\mem@oldshipout\AtBegShi@Everyshi@shipout
        \def\AtBegShi@found{\mem@oldshipout}%
      \else
        \expandafter\ifx\csname @EveryShipout@Org@Shipout\endcsname
                        \relax
           \ifx\@EveryShipout@Shipout\AtBegShi@Everyshi@shipout
             \def\AtBegShi@found{\@EveryShipout@Shipout}%
           \fi
        \fi
      \fi\fi\fi\fi\fi\fi
      \ifx\AtBegShi@found\relax
      \else
        \expandafter\endgroup
        \expandafter\def\AtBegShi@found{%
          \edef\AtBegShi@GroupLevel{\number\currentgrouplevel}%
          \afterassignment\@EveryShipout@Test
          \setbox\AtBeginShipoutBox=%
        }%
        \def\@EveryShipout@Test{%
          \ifnum\AtBegShi@GroupLevel=\currentgrouplevel
          \else
            \expandafter\aftergroup
          \fi
          \AtBegShi@Everyshi@Output
        }%
        \def\AtBegShi@Everyshi@Output{%
          \ifvoid\AtBeginShipoutBox
          \else
            \global\setbox\ltx@cclv\box\AtBeginShipoutBox
            \expandafter\@EveryShipout@Output
          \fi
        }%
        \@PackageInfoNoLine{atbegshi}{Package `everyshi' patched}%
        \begingroup
      \fi
    \fi
  \endgroup
  \let\AtBegShi@PatchEveryshi\relax
}
\@ifpackageloaded{everyshi}{%
  \AtBegShi@PatchEveryshi
}{%
  \AtBeginDocument{\AtBegShi@PatchEveryshi}%
}
%    \end{macrocode}
%
% \subsubsection{Class \xpackage{memoir}}
%
%     Fix of method and box.
%    \begin{macrocode}
\def\AtBegShi@PatchMemoir{%
  \begingroup
    \def\AtBegShi@Memoir@shipout{%
      \afterassignment\mem@shipi
      \setbox\@cclv=%
    }%
    \def\AtBegShi@Memoir@shipi{%
      \ifvoid\@cclv
        \expandafter\aftergroup
      \fi
      \mem@shipii
    }%
    \def\AtBegShi@Memoir@shipiiA{%
      \mem@oldshipout\vbox{%
        \trimmarks
        \unvbox\@cclv
      }%
    }%
    \def\AtBegShi@Memoir@shipiiB{%
      \ifvoid\@cclv
        \mem@oldshipout\box\@cclv
      \else
        \mem@oldshipout\vbox{%
          \trimmarks
          \unvbox\@cclv
        }%
      \fi
    }%
    \def\AtBegShi@Memoir@PatchAB{%
      \ifvoid\AtBeginShipoutBox
      \else
        \setbox\AtBeginShipoutBox=\vbox{%
          \trimmarks
          \ifvbox\AtBeginShipoutBox
            \unvbox\AtBeginShipoutBox
          \else
            \box\AtBeginShipoutBox
          \fi
        }%
        \AtBegShi@GetBoxSize\AtBeginShipoutBox
        \expandafter\mem@oldshipout
        \expandafter\box
        \expandafter\AtBeginShipoutBox
      \fi
    }%
    \def\AtBegShi@Memoir@shipiiC{% 2008/08/07 v1.6180339a
      \ifvoid\@cclv
        \mem@oldshipout\box\@cclv
      \else
        \ifshowtrims
          \mem@oldshipout\vbox{\trimmarks\unvbox\@cclv}%
        \else
          \mem@oldshipout\box\@cclv
        \fi
      \fi
    }%
    \def\AtBegShi@Memoir@shipiiD{% 2011/03/06 v3.6j
      \ifvoid\@cclv
        \mem@oldshipout\box\@cclv
      \else
        \ifshowtrims
          \mem@oldshipout\vbox{%
            \trimmarks
            \nointerlineskip
            \box\@cclv
          }%
        \else
          \mem@oldshipout\box\@cclv
        \fi
      \fi
    }%
    \def\AtBegShi@Memoir@PatchCD{%
      \ifvoid\AtBeginShipoutBox
      \else
        \ifshowtrims
          \setbox\AtBeginShipoutBox=\vbox{%
            \trimmarks
            \nointerlineskip
            \box\AtBeginShipoutBox
          }%
        \fi
        \AtBegShi@GetBoxSize\AtBeginShipoutBox
        \expandafter\mem@oldshipout
        \expandafter\box
        \expandafter\AtBeginShipoutBox
      \fi
    }%
    \ifx\AtBegShi@Memoir@shipi\mem@shipi
      \let\AtBegShi@found\ltx@one
      \ifx\AtBegShi@Memoir@shipiiA\mem@shipii
        \let\AtBegShi@found\ltx@zero
        \global\let\AtBegShi@Memoir@PatchX\AtBegShi@Memoir@PatchAB
      \else\ifx\AtBegShi@Memoir@shipiiB\mem@shipii
        \let\AtBegShi@found\ltx@zero
        \global\let\AtBegShi@Memoir@PatchX\AtBegShi@Memoir@PatchAB
      \else\ifx\AtBegShi@Memoir@shipiiC\mem@shipii
        \let\AtBegShi@found\ltx@zero
        \global\let\AtBegShi@Memoir@PatchX\AtBegShi@Memoir@PatchCD
      \else\ifx\AtBegShi@Memoir@shipiiD\mem@shipii
        \let\AtBegShi@found\ltx@zero
        \global\let\AtBegShi@Memoir@PatchX\AtBegShi@Memoir@PatchCD
      \fi\fi\fi\fi
      \ifcase\AtBegShi@found
        \let\AtBegShi@found\relax
        \ifx\shipout\AtBegShi@Memoir@shipout
          \def\AtBegShi@found{\shipout}%
        \else\ifx\AtBeginShipoutOriginalShipout\AtBegShi@Memoir@shipout
          \def\AtBegShi@found{\AtBeginShipoutOriginalShipout}%
        \else\ifx\CROP@shipout\AtBegShi@Memoir@shipout
          \def\AtBegShi@found{\CROP@shipout}%
        \else\ifx\GPTorg@shipout\AtBegShi@Memoir@shipout
          \def\AtBegShi@found{\GPTorg@shipout}%
        \else\ifx\THBorg@shipout\AtBegShi@Memoir@shipout
          \def\AtBegShi@found{\THBorg@shipout}%
        \else\ifx\@EveryShipout@Org@Shipout\AtBegShi@Memoir@shipout
          \def\AtBegShi@found{\@EveryShipout@Org@Shipout}%
        \fi\fi\fi\fi\fi\fi
        \ifx\AtBegShi@found\relax
        \else
          \expandafter\endgroup
          \expandafter\def\AtBegShi@found{%
            \edef\AtBegShi@GroupLevel{\number\currentgrouplevel}%
            \afterassignment\mem@shipi
            \setbox\AtBeginShipoutBox=%
          }%
          \def\mem@shipi{%
            \ifnum\AtBegShi@GroupLevel=\currentgrouplevel
            \else
              \expandafter\aftergroup
            \fi
            \mem@shipii
          }%
          \let\mem@shipii\AtBegShi@Memoir@PatchX
          \@PackageInfoNoLine{atbegshi}{Class `memoir' patched}%
          \begingroup
        \fi
      \fi
    \fi
  \endgroup
  \let\AtBegShi@PatchMemoir\relax
}
\@ifclassloaded{memoir}{%
  \AtBegShi@PatchMemoir
}{%
  \AtBeginDocument{\AtBegShi@PatchMemoir}%
}
%    \end{macrocode}
%
%    \begin{macrocode}
\AtBegShi@AtEnd%
%</package>
%    \end{macrocode}
%
% \section{Test}
%
% \subsection{Catcode checks for loading}
%
%    \begin{macrocode}
%<*test1>
%    \end{macrocode}
%    \begin{macrocode}
\catcode`\{=1 %
\catcode`\}=2 %
\catcode`\#=6 %
\catcode`\@=11 %
\expandafter\ifx\csname count@\endcsname\relax
  \countdef\count@=255 %
\fi
\expandafter\ifx\csname @gobble\endcsname\relax
  \long\def\@gobble#1{}%
\fi
\expandafter\ifx\csname @firstofone\endcsname\relax
  \long\def\@firstofone#1{#1}%
\fi
\expandafter\ifx\csname loop\endcsname\relax
  \expandafter\@firstofone
\else
  \expandafter\@gobble
\fi
{%
  \def\loop#1\repeat{%
    \def\body{#1}%
    \iterate
  }%
  \def\iterate{%
    \body
      \let\next\iterate
    \else
      \let\next\relax
    \fi
    \next
  }%
  \let\repeat=\fi
}%
\def\RestoreCatcodes{}
\count@=0 %
\loop
  \edef\RestoreCatcodes{%
    \RestoreCatcodes
    \catcode\the\count@=\the\catcode\count@\relax
  }%
\ifnum\count@<255 %
  \advance\count@ 1 %
\repeat

\def\RangeCatcodeInvalid#1#2{%
  \count@=#1\relax
  \loop
    \catcode\count@=15 %
  \ifnum\count@<#2\relax
    \advance\count@ 1 %
  \repeat
}
\def\RangeCatcodeCheck#1#2#3{%
  \count@=#1\relax
  \loop
    \ifnum#3=\catcode\count@
    \else
      \errmessage{%
        Character \the\count@\space
        with wrong catcode \the\catcode\count@\space
        instead of \number#3%
      }%
    \fi
  \ifnum\count@<#2\relax
    \advance\count@ 1 %
  \repeat
}
\def\space{ }
\expandafter\ifx\csname LoadCommand\endcsname\relax
  \def\LoadCommand{\input atbegshi.sty\relax}%
\fi
\def\Test{%
  \RangeCatcodeInvalid{0}{47}%
  \RangeCatcodeInvalid{58}{64}%
  \RangeCatcodeInvalid{91}{96}%
  \RangeCatcodeInvalid{123}{255}%
  \catcode`\@=12 %
  \catcode`\\=0 %
  \catcode`\%=14 %
  \LoadCommand
  \RangeCatcodeCheck{0}{36}{15}%
  \RangeCatcodeCheck{37}{37}{14}%
  \RangeCatcodeCheck{38}{47}{15}%
  \RangeCatcodeCheck{48}{57}{12}%
  \RangeCatcodeCheck{58}{63}{15}%
  \RangeCatcodeCheck{64}{64}{12}%
  \RangeCatcodeCheck{65}{90}{11}%
  \RangeCatcodeCheck{91}{91}{15}%
  \RangeCatcodeCheck{92}{92}{0}%
  \RangeCatcodeCheck{93}{96}{15}%
  \RangeCatcodeCheck{97}{122}{11}%
  \RangeCatcodeCheck{123}{255}{15}%
  \RestoreCatcodes
}
\Test
\csname @@end\endcsname
\end
%    \end{macrocode}
%    \begin{macrocode}
%</test1>
%    \end{macrocode}
%
%    \begin{macrocode}
%<*test2>
\input atbegshi.sty\relax
\def\msg#{\immediate\write16}
\msg{File: atbegshi-test2.tex 2016/06/09 v1.18 Test file for plain-TeX}
\def\testmsg#1#2{%
  \msg{}%
  \msg{*** Test with box (#1), expected page output [#2]}% hash-ok
}

\newbox\voidbox
\def\void{\box\voidbox}
\begingroup
  \setbox\voidbox=\void
\endgroup

\count0=0\relax
\AtBeginShipout{%
  \global\advance\count0 by 1\relax
  \msg{* Inside \string\AtBeginShipout: [\the\count0]}%
}

\AtBeginShipoutFirst{%
  \msg{* Inside \string\AtBeginShipoutFirst}%
  Hello World%
}

\testmsg{\string\null}{1}
\shipout\null

\AtBeginShipoutFirst{%
  This is too late%
}

\testmsg{void}{}
\shipout\void

\testmsg{\string\copy255 (not void)}{2}
\setbox255\hbox{\vrule height 10bp width 10bp}
\shipout\copy255 %

\testmsg{\string\copy255 (again)}{3}
\shipout\copy255 %

\testmsg{\string\box255}{4}
\shipout\box255 %

\testmsg{\string\box255 (again)}{}
\shipout\box255 %

\testmsg{\string\hbox}{5}
\shipout\hbox{\vrule height 5bp width 20bp}

\testmsg{\string\vbox}{6}
\shipout\vbox{\hrule height 20bp width 5bp}

\testmsg{\string\null, voided by hook}{}
\def\VoidBox{%
  \begingroup
    \setbox\AtBeginShipoutBox=\box\AtBeginShipoutBox
  \endgroup
}
\AtBeginShipout{\VoidBox}
\shipout\null
\def\VoidBox{}

\msg{*** \string\begingroup}
\begingroup
  \testmsg{void}{}%
  \shipout\void
\msg{*** \string\endgroup}
\endgroup

\msg{*** \string\begingroup}
\begingroup
  \testmsg{void}{}%
  \shipout\void
  \testmsg{\string\null}{8}%
  \shipout\null
\msg{*** \string\endgroup}
\endgroup

\testmsg{output routine}{9}
Hello World
\vfill
\eject

\testmsg{\string\null\space(discarded)}{}
\AtBeginShipout{%
  \msg{* Inside \string\AtBeginShipout: DISCARD}%
  \AtBeginShipoutDiscard
}
\shipout\null

\end
%</test2>
%    \end{macrocode}
%
%    \begin{macrocode}
%<*test3>
\NeedsTeXFormat{LaTeX2e}
\ProvidesFile{atbegshi-test3.tex}[2016/06/09 v1.18 Test file for LaTeX]
\RequirePackage{color}
\pagecolor{yellow}
\documentclass[a5paper,showtrims]{memoir}
\usepackage{atbegshi}
\AtBeginShipout{%
  \setbox\AtBeginShipoutBox=\vbox{%
    \vbox to 0pt{%
      \kern-1.5in %
      \hbox to 0pt{%
        \kern-1.5in %
        \color{blue}%
        \rule{1in}{1in}%
        \hss
      }%
      \vss
    }%
    \hrule
    \hbox{\vrule\box\AtBeginShipoutBox\vrule}%
    \hrule
  }%
}
\usepackage{eso-pic}
\makeatletter
\@ifundefined{@EveryShipout@Init}{%
  \typeout{Test skipped}%
  \@@end
}{}
\@EveryShipout@Init
\let\@EveryShipout@Init\relax
\makeatother
\AddToShipoutPicture{%
  \hspace{.52\paperwidth}%
  \colorbox{cyan}{%
    \rule{0mm}{\paperheight}%
    \hspace{.48\paperwidth}%
  }%
}
%    \end{macrocode}
%    Newer versions of class \xclass{memoir} emulate package
%    \xpackage{crop} and prevents its loading. This is undone
%    in next line for this test file.
%    \begin{macrocode}
\expandafter\let\csname ver@crop.sty\endcsname\relax
\usepackage[color=red,cross,a4,center]{crop}
\begin{document}
\shipout\null
\shipout\box\csname voidb@x\endcsname
\section{Hello World}
\end{document}
%</test3>
%    \end{macrocode}
%
% \section{Installation}
%
% \subsection{Download}
%
% \paragraph{Package.} This package is available on
% CTAN\footnote{\url{http://ctan.org/pkg/atbegshi}}:
% \begin{description}
% \item[\CTAN{macros/latex/contrib/oberdiek/atbegshi.dtx}] The source file.
% \item[\CTAN{macros/latex/contrib/oberdiek/atbegshi.pdf}] Documentation.
% \end{description}
%
%
% \paragraph{Bundle.} All the packages of the bundle `oberdiek'
% are also available in a TDS compliant ZIP archive. There
% the packages are already unpacked and the documentation files
% are generated. The files and directories obey the TDS standard.
% \begin{description}
% \item[\CTAN{install/macros/latex/contrib/oberdiek.tds.zip}]
% \end{description}
% \emph{TDS} refers to the standard ``A Directory Structure
% for \TeX\ Files'' (\CTAN{tds/tds.pdf}). Directories
% with \xfile{texmf} in their name are usually organized this way.
%
% \subsection{Bundle installation}
%
% \paragraph{Unpacking.} Unpack the \xfile{oberdiek.tds.zip} in the
% TDS tree (also known as \xfile{texmf} tree) of your choice.
% Example (linux):
% \begin{quote}
%   |unzip oberdiek.tds.zip -d ~/texmf|
% \end{quote}
%
% \paragraph{Script installation.}
% Check the directory \xfile{TDS:scripts/oberdiek/} for
% scripts that need further installation steps.
% Package \xpackage{attachfile2} comes with the Perl script
% \xfile{pdfatfi.pl} that should be installed in such a way
% that it can be called as \texttt{pdfatfi}.
% Example (linux):
% \begin{quote}
%   |chmod +x scripts/oberdiek/pdfatfi.pl|\\
%   |cp scripts/oberdiek/pdfatfi.pl /usr/local/bin/|
% \end{quote}
%
% \subsection{Package installation}
%
% \paragraph{Unpacking.} The \xfile{.dtx} file is a self-extracting
% \docstrip\ archive. The files are extracted by running the
% \xfile{.dtx} through \plainTeX:
% \begin{quote}
%   \verb|tex atbegshi.dtx|
% \end{quote}
%
% \paragraph{TDS.} Now the different files must be moved into
% the different directories in your installation TDS tree
% (also known as \xfile{texmf} tree):
% \begin{quote}
% \def\t{^^A
% \begin{tabular}{@{}>{\ttfamily}l@{ $\rightarrow$ }>{\ttfamily}l@{}}
%   atbegshi.sty & tex/generic/oberdiek/atbegshi.sty\\
%   atbegshi.pdf & doc/latex/oberdiek/atbegshi.pdf\\
%   atbegshi-example1.tex & doc/latex/oberdiek/atbegshi-example1.tex\\
%   atbegshi-example2.tex & doc/latex/oberdiek/atbegshi-example2.tex\\
%   test/atbegshi-test1.tex & doc/latex/oberdiek/test/atbegshi-test1.tex\\
%   test/atbegshi-test2.tex & doc/latex/oberdiek/test/atbegshi-test2.tex\\
%   test/atbegshi-test3.tex & doc/latex/oberdiek/test/atbegshi-test3.tex\\
%   atbegshi.dtx & source/latex/oberdiek/atbegshi.dtx\\
% \end{tabular}^^A
% }^^A
% \sbox0{\t}^^A
% \ifdim\wd0>\linewidth
%   \begingroup
%     \advance\linewidth by\leftmargin
%     \advance\linewidth by\rightmargin
%   \edef\x{\endgroup
%     \def\noexpand\lw{\the\linewidth}^^A
%   }\x
%   \def\lwbox{^^A
%     \leavevmode
%     \hbox to \linewidth{^^A
%       \kern-\leftmargin\relax
%       \hss
%       \usebox0
%       \hss
%       \kern-\rightmargin\relax
%     }^^A
%   }^^A
%   \ifdim\wd0>\lw
%     \sbox0{\small\t}^^A
%     \ifdim\wd0>\linewidth
%       \ifdim\wd0>\lw
%         \sbox0{\footnotesize\t}^^A
%         \ifdim\wd0>\linewidth
%           \ifdim\wd0>\lw
%             \sbox0{\scriptsize\t}^^A
%             \ifdim\wd0>\linewidth
%               \ifdim\wd0>\lw
%                 \sbox0{\tiny\t}^^A
%                 \ifdim\wd0>\linewidth
%                   \lwbox
%                 \else
%                   \usebox0
%                 \fi
%               \else
%                 \lwbox
%               \fi
%             \else
%               \usebox0
%             \fi
%           \else
%             \lwbox
%           \fi
%         \else
%           \usebox0
%         \fi
%       \else
%         \lwbox
%       \fi
%     \else
%       \usebox0
%     \fi
%   \else
%     \lwbox
%   \fi
% \else
%   \usebox0
% \fi
% \end{quote}
% If you have a \xfile{docstrip.cfg} that configures and enables \docstrip's
% TDS installing feature, then some files can already be in the right
% place, see the documentation of \docstrip.
%
% \subsection{Refresh file name databases}
%
% If your \TeX~distribution
% (\teTeX, \mikTeX, \dots) relies on file name databases, you must refresh
% these. For example, \teTeX\ users run \verb|texhash| or
% \verb|mktexlsr|.
%
% \subsection{Some details for the interested}
%
% \paragraph{Attached source.}
%
% The PDF documentation on CTAN also includes the
% \xfile{.dtx} source file. It can be extracted by
% AcrobatReader 6 or higher. Another option is \textsf{pdftk},
% e.g. unpack the file into the current directory:
% \begin{quote}
%   \verb|pdftk atbegshi.pdf unpack_files output .|
% \end{quote}
%
% \paragraph{Unpacking with \LaTeX.}
% The \xfile{.dtx} chooses its action depending on the format:
% \begin{description}
% \item[\plainTeX:] Run \docstrip\ and extract the files.
% \item[\LaTeX:] Generate the documentation.
% \end{description}
% If you insist on using \LaTeX\ for \docstrip\ (really,
% \docstrip\ does not need \LaTeX), then inform the autodetect routine
% about your intention:
% \begin{quote}
%   \verb|latex \let\install=y\input{atbegshi.dtx}|
% \end{quote}
% Do not forget to quote the argument according to the demands
% of your shell.
%
% \paragraph{Generating the documentation.}
% You can use both the \xfile{.dtx} or the \xfile{.drv} to generate
% the documentation. The process can be configured by the
% configuration file \xfile{ltxdoc.cfg}. For instance, put this
% line into this file, if you want to have A4 as paper format:
% \begin{quote}
%   \verb|\PassOptionsToClass{a4paper}{article}|
% \end{quote}
% An example follows how to generate the
% documentation with pdf\LaTeX:
% \begin{quote}
%\begin{verbatim}
%pdflatex atbegshi.dtx
%makeindex -s gind.ist atbegshi.idx
%pdflatex atbegshi.dtx
%makeindex -s gind.ist atbegshi.idx
%pdflatex atbegshi.dtx
%\end{verbatim}
% \end{quote}
%
% \section{Catalogue}
%
% The following XML file can be used as source for the
% \href{http://mirror.ctan.org/help/Catalogue/catalogue.html}{\TeX\ Catalogue}.
% The elements \texttt{caption} and \texttt{description} are imported
% from the original XML file from the Catalogue.
% The name of the XML file in the Catalogue is \xfile{atbegshi.xml}.
%    \begin{macrocode}
%<*catalogue>
<?xml version='1.0' encoding='us-ascii'?>
<!DOCTYPE entry SYSTEM 'catalogue.dtd'>
<entry datestamp='$Date$' modifier='$Author$' id='atbegshi'>
  <name>atbegshi</name>
  <caption>Execute stuff at \shipout time.</caption>
  <authorref id='auth:oberdiek'/>
  <copyright owner='Heiko Oberdiek' year='2007-2011'/>
  <license type='lppl1.3'/>
  <version number='1.18'/>
  <description>
    This package is a modern reimplementation of package
    <xref refid='everyshi'>everyshi</xref>, providing various commands
    to be executed before a <tt>\shipout</tt> command.  It makes use of
    e-TeX&#x2019;s facilities if they are available.  The package may
    be used either with LaTeX or with plain TeX.
    <p/>
    The package is part of the <xref refid='oberdiek'>oberdiek</xref> bundle.
  </description>
  <documentation details='Package documentation'
      href='ctan:/macros/latex/contrib/oberdiek/atbegshi.pdf'/>
  <ctan file='true' path='/macros/latex/contrib/oberdiek/atbegshi.dtx'/>
  <miktex location='oberdiek'/>
  <texlive location='oberdiek'/>
  <install path='/macros/latex/contrib/oberdiek/oberdiek.tds.zip'/>
</entry>
%</catalogue>
%    \end{macrocode}
%
% \begin{History}
%   \begin{Version}{2007/04/17 v1.0}
%   \item
%     First version.
%   \end{Version}
%   \begin{Version}{2007/04/18 v1.1}
%   \item
%     New method based on \cs{lastkern} is used if \eTeX\ is missing.
%   \item
%     \cs{AtBeginShipoutDiscard} also resets \cs{deadcycles}.
%   \end{Version}
%   \begin{Version}{2007/04/19 v1.2}
%   \item
%     \cs{AtBeginShipoutEarly} removed for simplification reasons.
%   \item
%     Forgotten definition of \cs{AtBegShi@Info} added.
%   \item
%     Patches for packages \xpackage{crop} and \xpackage{everyshi}
%     and class \xclass{memoir} added.
%   \end{Version}
%   \begin{Version}{2007/04/26 v1.3}
%   \item
%     Use of package \xpackage{infwarerr}.
%   \item
%     Catcode section after generic header.
%   \end{Version}
%   \begin{Version}{2007/04/27 v1.4}
%   \item
%     Small optimizations.
%   \end{Version}
%   \begin{Version}{2007/06/06 v1.5}
%   \item
%     \cs{AtBeginShipoutUpperLeft} added.
%   \item
%     Example added.
%   \item
%     Fix in second test file for newer version of \xclass{memoir}.
%   \end{Version}
%   \begin{Version}{2007/09/09 v1.6}
%   \item
%     Catcode section rewritten.
%   \end{Version}
%   \begin{Version}{2008/07/18 v1.7}
%   \item
%     Documentation of \cs{AtBeginShipoutUpperLeft} fixed and extended.
%   \end{Version}
%   \begin{Version}{2008/07/19 v1.8}
%   \item
%     \cs{AtBeginShipoutUpperLeftForeground} added.
%   \end{Version}
%   \begin{Version}{2008/07/31 v1.9}
%   \item
%     Second example (TrimBox for dvipdfmx) added.
%   \item
%     No changes in package code.
%   \end{Version}
%   \begin{Version}{2009/12/02 v1.10}
%   \item
%     \cs{AtBeginShipoutOriginalShipout} added.
%   \item
%     Test file fixed.
%   \end{Version}
%   \begin{Version}{2010/03/01 v1.11}
%   \item
%     Compatibility with \iniTeX\ except for \cs{newbox}.
%   \end{Version}
%   \begin{Version}{2010/03/25 v1.12}
%   \item
%     \cs{AtBeginShipoutNext} can now be used inside
%     \cs{AtBeginShipoutNext}.
%   \end{Version}
%   \begin{Version}{2010/08/18 v1.13}
%   \item
%     Fixes for \cs{AtBegShi@CheckDefinable}.
%   \end{Version}
%   \begin{Version}{2010/12/02 v1.14}
%   \item
%     Remove the warning because of void box if the hook
%     calls \AtBeginShipoutDiscard.
%   \end{Version}
%   \begin{Version}{2011/01/30 v1.15}
%   \item
%     Already loaded package files are not input in \hologo{plainTeX}.
%   \end{Version}
%   \begin{Version}{2011/10/05 v1.16}
%   \item
%     \cs{AtBeginShipoutAddToBox}, \cs{AtBeginShipoutAddToBoxForeground} added.
%   \item
%     \cs{AtBeginShipoutBoxWidth}, \cs{AtBeginShipoutBoxHeight},
%     \cs{AtBeginShipoutBoxDepth} added.
%   \item
%     Updates for patches of class \xclass{memoir}.
%   \end{Version}
%   \begin{Version}{2016/05/16 v1.17}
%   \item
%     Documentation updates.
%   \end{Version}
%   \begin{Version}{2016/06/09 v1.18}
%   \item
%     Update for \cs{pdfhorign} in new LuaTeX.
%   \end{Version}
% \end{History}
%
% \PrintIndex
%
% \Finale
\endinput
|
% \end{quote}
% Do not forget to quote the argument according to the demands
% of your shell.
%
% \paragraph{Generating the documentation.}
% You can use both the \xfile{.dtx} or the \xfile{.drv} to generate
% the documentation. The process can be configured by the
% configuration file \xfile{ltxdoc.cfg}. For instance, put this
% line into this file, if you want to have A4 as paper format:
% \begin{quote}
%   \verb|\PassOptionsToClass{a4paper}{article}|
% \end{quote}
% An example follows how to generate the
% documentation with pdf\LaTeX:
% \begin{quote}
%\begin{verbatim}
%pdflatex atbegshi.dtx
%makeindex -s gind.ist atbegshi.idx
%pdflatex atbegshi.dtx
%makeindex -s gind.ist atbegshi.idx
%pdflatex atbegshi.dtx
%\end{verbatim}
% \end{quote}
%
% \section{Catalogue}
%
% The following XML file can be used as source for the
% \href{http://mirror.ctan.org/help/Catalogue/catalogue.html}{\TeX\ Catalogue}.
% The elements \texttt{caption} and \texttt{description} are imported
% from the original XML file from the Catalogue.
% The name of the XML file in the Catalogue is \xfile{atbegshi.xml}.
%    \begin{macrocode}
%<*catalogue>
<?xml version='1.0' encoding='us-ascii'?>
<!DOCTYPE entry SYSTEM 'catalogue.dtd'>
<entry datestamp='$Date$' modifier='$Author$' id='atbegshi'>
  <name>atbegshi</name>
  <caption>Execute stuff at \shipout time.</caption>
  <authorref id='auth:oberdiek'/>
  <copyright owner='Heiko Oberdiek' year='2007-2011'/>
  <license type='lppl1.3'/>
  <version number='1.18'/>
  <description>
    This package is a modern reimplementation of package
    <xref refid='everyshi'>everyshi</xref>, providing various commands
    to be executed before a <tt>\shipout</tt> command.  It makes use of
    e-TeX&#x2019;s facilities if they are available.  The package may
    be used either with LaTeX or with plain TeX.
    <p/>
    The package is part of the <xref refid='oberdiek'>oberdiek</xref> bundle.
  </description>
  <documentation details='Package documentation'
      href='ctan:/macros/latex/contrib/oberdiek/atbegshi.pdf'/>
  <ctan file='true' path='/macros/latex/contrib/oberdiek/atbegshi.dtx'/>
  <miktex location='oberdiek'/>
  <texlive location='oberdiek'/>
  <install path='/macros/latex/contrib/oberdiek/oberdiek.tds.zip'/>
</entry>
%</catalogue>
%    \end{macrocode}
%
% \begin{History}
%   \begin{Version}{2007/04/17 v1.0}
%   \item
%     First version.
%   \end{Version}
%   \begin{Version}{2007/04/18 v1.1}
%   \item
%     New method based on \cs{lastkern} is used if \eTeX\ is missing.
%   \item
%     \cs{AtBeginShipoutDiscard} also resets \cs{deadcycles}.
%   \end{Version}
%   \begin{Version}{2007/04/19 v1.2}
%   \item
%     \cs{AtBeginShipoutEarly} removed for simplification reasons.
%   \item
%     Forgotten definition of \cs{AtBegShi@Info} added.
%   \item
%     Patches for packages \xpackage{crop} and \xpackage{everyshi}
%     and class \xclass{memoir} added.
%   \end{Version}
%   \begin{Version}{2007/04/26 v1.3}
%   \item
%     Use of package \xpackage{infwarerr}.
%   \item
%     Catcode section after generic header.
%   \end{Version}
%   \begin{Version}{2007/04/27 v1.4}
%   \item
%     Small optimizations.
%   \end{Version}
%   \begin{Version}{2007/06/06 v1.5}
%   \item
%     \cs{AtBeginShipoutUpperLeft} added.
%   \item
%     Example added.
%   \item
%     Fix in second test file for newer version of \xclass{memoir}.
%   \end{Version}
%   \begin{Version}{2007/09/09 v1.6}
%   \item
%     Catcode section rewritten.
%   \end{Version}
%   \begin{Version}{2008/07/18 v1.7}
%   \item
%     Documentation of \cs{AtBeginShipoutUpperLeft} fixed and extended.
%   \end{Version}
%   \begin{Version}{2008/07/19 v1.8}
%   \item
%     \cs{AtBeginShipoutUpperLeftForeground} added.
%   \end{Version}
%   \begin{Version}{2008/07/31 v1.9}
%   \item
%     Second example (TrimBox for dvipdfmx) added.
%   \item
%     No changes in package code.
%   \end{Version}
%   \begin{Version}{2009/12/02 v1.10}
%   \item
%     \cs{AtBeginShipoutOriginalShipout} added.
%   \item
%     Test file fixed.
%   \end{Version}
%   \begin{Version}{2010/03/01 v1.11}
%   \item
%     Compatibility with \iniTeX\ except for \cs{newbox}.
%   \end{Version}
%   \begin{Version}{2010/03/25 v1.12}
%   \item
%     \cs{AtBeginShipoutNext} can now be used inside
%     \cs{AtBeginShipoutNext}.
%   \end{Version}
%   \begin{Version}{2010/08/18 v1.13}
%   \item
%     Fixes for \cs{AtBegShi@CheckDefinable}.
%   \end{Version}
%   \begin{Version}{2010/12/02 v1.14}
%   \item
%     Remove the warning because of void box if the hook
%     calls \AtBeginShipoutDiscard.
%   \end{Version}
%   \begin{Version}{2011/01/30 v1.15}
%   \item
%     Already loaded package files are not input in \hologo{plainTeX}.
%   \end{Version}
%   \begin{Version}{2011/10/05 v1.16}
%   \item
%     \cs{AtBeginShipoutAddToBox}, \cs{AtBeginShipoutAddToBoxForeground} added.
%   \item
%     \cs{AtBeginShipoutBoxWidth}, \cs{AtBeginShipoutBoxHeight},
%     \cs{AtBeginShipoutBoxDepth} added.
%   \item
%     Updates for patches of class \xclass{memoir}.
%   \end{Version}
%   \begin{Version}{2016/05/16 v1.17}
%   \item
%     Documentation updates.
%   \end{Version}
%   \begin{Version}{2016/06/09 v1.18}
%   \item
%     Update for \cs{pdfhorign} in new LuaTeX.
%   \end{Version}
% \end{History}
%
% \PrintIndex
%
% \Finale
\endinput

%        (quote the arguments according to the demands of your shell)
%
% Documentation:
%    (a) If atbegshi.drv is present:
%           latex atbegshi.drv
%    (b) Without atbegshi.drv:
%           latex atbegshi.dtx; ...
%    The class ltxdoc loads the configuration file ltxdoc.cfg
%    if available. Here you can specify further options, e.g.
%    use A4 as paper format:
%       \PassOptionsToClass{a4paper}{article}
%
%    Programm calls to get the documentation (example):
%       pdflatex atbegshi.dtx
%       makeindex -s gind.ist atbegshi.idx
%       pdflatex atbegshi.dtx
%       makeindex -s gind.ist atbegshi.idx
%       pdflatex atbegshi.dtx
%
% Installation:
%    TDS:tex/generic/oberdiek/atbegshi.sty
%    TDS:doc/latex/oberdiek/atbegshi.pdf
%    TDS:doc/latex/oberdiek/atbegshi-example1.tex
%    TDS:doc/latex/oberdiek/atbegshi-example2.tex
%    TDS:doc/latex/oberdiek/test/atbegshi-test1.tex
%    TDS:doc/latex/oberdiek/test/atbegshi-test2.tex
%    TDS:doc/latex/oberdiek/test/atbegshi-test3.tex
%    TDS:source/latex/oberdiek/atbegshi.dtx
%
%<*ignore>
\begingroup
  \catcode123=1 %
  \catcode125=2 %
  \def\x{LaTeX2e}%
\expandafter\endgroup
\ifcase 0\ifx\install y1\fi\expandafter
         \ifx\csname processbatchFile\endcsname\relax\else1\fi
         \ifx\fmtname\x\else 1\fi\relax
\else\csname fi\endcsname
%</ignore>
%<*install>
\input docstrip.tex
\Msg{************************************************************************}
\Msg{* Installation}
\Msg{* Package: atbegshi 2016/06/09 v1.18 At begin shipout hook (HO)}
\Msg{************************************************************************}

\keepsilent
\askforoverwritefalse

\let\MetaPrefix\relax
\preamble

This is a generated file.

Project: atbegshi
Version: 2016/06/09 v1.18

Copyright (C) 2007-2011 by
   Heiko Oberdiek <heiko.oberdiek at googlemail.com>

This work may be distributed and/or modified under the
conditions of the LaTeX Project Public License, either
version 1.3c of this license or (at your option) any later
version. This version of this license is in
   http://www.latex-project.org/lppl/lppl-1-3c.txt
and the latest version of this license is in
   http://www.latex-project.org/lppl.txt
and version 1.3 or later is part of all distributions of
LaTeX version 2005/12/01 or later.

This work has the LPPL maintenance status "maintained".

This Current Maintainer of this work is Heiko Oberdiek.

The Base Interpreter refers to any `TeX-Format',
because some files are installed in TDS:tex/generic//.

This work consists of the main source file atbegshi.dtx
and the derived files
   atbegshi.sty, atbegshi.pdf, atbegshi.ins, atbegshi.drv,
   atbegshi-example1.tex, atbegshi-example2.tex,
   atbegshi-test1.tex, atbegshi-test2.tex, atbegshi-test3.tex.

\endpreamble
\let\MetaPrefix\DoubleperCent

\generate{%
  \file{atbegshi.ins}{\from{atbegshi.dtx}{install}}%
  \file{atbegshi.drv}{\from{atbegshi.dtx}{driver}}%
  \usedir{tex/generic/oberdiek}%
  \file{atbegshi.sty}{\from{atbegshi.dtx}{package}}%
  \usedir{doc/latex/oberdiek}%
  \file{atbegshi-example1.tex}{\from{atbegshi.dtx}{example1}}%
  \file{atbegshi-example2.tex}{\from{atbegshi.dtx}{example2}}%
  \usedir{doc/latex/oberdiek/test}%
  \file{atbegshi-test1.tex}{\from{atbegshi.dtx}{test1}}%
  \file{atbegshi-test2.tex}{\from{atbegshi.dtx}{test2}}%
  \file{atbegshi-test3.tex}{\from{atbegshi.dtx}{test3}}%
  \nopreamble
  \nopostamble
  \usedir{source/latex/oberdiek/catalogue}%
  \file{atbegshi.xml}{\from{atbegshi.dtx}{catalogue}}%
}

\catcode32=13\relax% active space
\let =\space%
\Msg{************************************************************************}
\Msg{*}
\Msg{* To finish the installation you have to move the following}
\Msg{* file into a directory searched by TeX:}
\Msg{*}
\Msg{*     atbegshi.sty}
\Msg{*}
\Msg{* To produce the documentation run the file `atbegshi.drv'}
\Msg{* through LaTeX.}
\Msg{*}
\Msg{* Happy TeXing!}
\Msg{*}
\Msg{************************************************************************}

\endbatchfile
%</install>
%<*ignore>
\fi
%</ignore>
%<*driver>
\NeedsTeXFormat{LaTeX2e}
\ProvidesFile{atbegshi.drv}%
  [2016/06/09 v1.18 At begin shipout hook (HO)]%
\documentclass{ltxdoc}
\usepackage{holtxdoc}[2011/11/22]
\begin{document}
  \DocInput{atbegshi.dtx}%
\end{document}
%</driver>
% \fi
%
%
% \CharacterTable
%  {Upper-case    \A\B\C\D\E\F\G\H\I\J\K\L\M\N\O\P\Q\R\S\T\U\V\W\X\Y\Z
%   Lower-case    \a\b\c\d\e\f\g\h\i\j\k\l\m\n\o\p\q\r\s\t\u\v\w\x\y\z
%   Digits        \0\1\2\3\4\5\6\7\8\9
%   Exclamation   \!     Double quote  \"     Hash (number) \#
%   Dollar        \$     Percent       \%     Ampersand     \&
%   Acute accent  \'     Left paren    \(     Right paren   \)
%   Asterisk      \*     Plus          \+     Comma         \,
%   Minus         \-     Point         \.     Solidus       \/
%   Colon         \:     Semicolon     \;     Less than     \<
%   Equals        \=     Greater than  \>     Question mark \?
%   Commercial at \@     Left bracket  \[     Backslash     \\
%   Right bracket \]     Circumflex    \^     Underscore    \_
%   Grave accent  \`     Left brace    \{     Vertical bar  \|
%   Right brace   \}     Tilde         \~}
%
% \GetFileInfo{atbegshi.drv}
%
% \title{The \xpackage{atbegshi} package}
% \date{2016/06/09 v1.18}
% \author{Heiko Oberdiek\thanks
% {Please report any issues at https://github.com/ho-tex/oberdiek/issues}\\
% \xemail{heiko.oberdiek at googlemail.com}}
%
% \maketitle
%
% \begin{abstract}
% This package is a modern reimplementation of package \xpackage{everyshi}
% without the burden of compatibility. It makes use of \eTeX's if available.
% Both \LaTeX\ and \plainTeX\ are supported.
% \end{abstract}
%
% \tableofcontents
%
% \section{Documentation}
%
% Package \xpackage{atbegshi} redefines \cs{shipout} to insert
% hooks for user code that is executed before the page is shipped out.
% The code may modify or even discard the output page.
% Three hooks are implemented:
% \begin{enumerate}
% \item A hook that is executed for every page, see\\
%       \cs{AtBeginShipout}
% \item A hook that is executed for the next page only, see\\
%       \cs{AtBeginShipoutNext}
% \item A hook that is only executed for the first page, see\\
%       \cs{AtBeginShipoutFirst}
% \end{enumerate}
% The hooks are executed in this order. The following three macros
% provide the user interface for adding code to these hooks:
%
% \begin{declcs}{AtBeginShipout} \M{code}\\
%   \cs{AtBeginShipoutBox}
% \end{declcs}
% Execute the \meta{code} for every page.
% The page contents is held in box register \cs{AtBeginShipoutBox}
% and may be modified. Use \cs{AtBeginShipoutDiscard} if you want to
% discard the page.
%
% \emph{Note:} Package \xpackage{everyshi} uses box register 255. With
% package \xpackage{atbegshi} you must use \cs{AtBeginShipoutBox} instead.
%
% If \LaTeX\ calls \cs{shipout} in \cs{@outputpage} (part of its output
% routine),
% the meaning of \cs{protect} is \cs{noexpand}. \LaTeX\ sets
% \cs{protect} to the appropriate \cs{@typeset@protect} in the box
% that is shipped out. This is too late for the hooks, they
% are called earlier in the redefined \cs{shipout}. Therefore
% package \xpackage{atbegshi} sets \cs{protect} to \cs{@typeset@protect}
% before it calls the hooks.
% (In \cs{EveryShipout} of package \xpackage{everyshi} the
% user is responsible for the correct setting of \cs{protect}.)
%
% \begin{declcs}{AtBeginShipoutNext} \M{code}
% \end{declcs}
% This reimplements package \xpackage{everyshi}'s \cs{AtNextShipout}.
% The \meta{code} is executed at shipout time of the next page only.
% It is just a convenience macro, it can be easily replaced by something
% like:
%\begin{quote}
%\begin{verbatim}
%\newcommand{\MyShipoutHook}{}%
%\AtBeginShipout{\MyShipoutHook}
%\gdef\MyShipoutHook{%
%  ... do something with next page ...
%  \gdef\MyShipoutHook{}%
% }
%\end{verbatim}
%\end{quote}
% (This can be necessary, if hook order does matter).
%
% \begin{declcs}{AtBeginShipoutFirst} \M{code}
% \end{declcs}
% This reimplements \LaTeX's \cs{AtBeginDvi}. This hook is
% usually used for \cs{special} commands that include PostScript
% header files. The \cs{code} is directly executed in a \cs{vbox}
% that is put at the beginning of the output page. Dealing with
% the output box \cs{AtBeginShipoutBox} is not necessary and
% not permitted here.
%
% \begin{declcs}{AtBeginShipoutDiscard}
% \end{declcs}
% This macro notifies package \xpackage{atbegshi} that
% the output page is discarded. The remaining hook code
% and the remaining hooks are not executed and the page is thrown away.
% Also \cs{deadcycles} is cleared to zero like an ordinary \cs{shipout}
% would do.
%
% \begin{declcs}{AtBeginShipoutInit}
% \end{declcs}
% Usually the redefinition of \cs{shipout} is delayed by
% \cs{AtBeginDocument} (if this macro exists). This can be
% too late, if other packages also redefines \cs{shipout}
% and the order does matter.
% \cs{AtBeginShipoutInit} forces the immediate redefinition
% of \cs{shipout}.
%
% \begin{declcs}{AtBeginShipoutAddToBox} \M{stuff}\\
%   \cs{AtBeginShipoutAddToBoxForeground} \M{stuff}
% \end{declcs}
% A quite common use case is the addition of \cs{special}
% or other whatsits to the page output box. Macro
% \cs{AtBeginShipoutAddToBox} puts \meta{stuff} in a
% box with zeroed dimensions. The box with the \meta{stuff}
% is put in the upper left corner of the shipout box
% \cs{AtBeginShipoutBox}. Macro \cs{AtBeginShipoutAddToBox}
% puts the \meta{stuff} in the background, the other macro
% \cs{AtBeginShipoutAddToBoxForeground} in the foreground
% after the original shipout box contents is set.
%
% A void shipout box (that means a discarded page)
% remains void that means \meta{stuff} is ignored in this case.
% The box type of \cs{AtBeginShipoutBox} is preserved.
% Also the box nesting level for the original contents
% of \cs{AtBeginShipoutBox} remains, for example, to avoid trouble
% with links across pages in case of \hologo{pdfTeX}.
%
% \begin{declcs}{AtBeginShipoutUpperLeft} \M{background material}
% \end{declcs}
% This is a macro that puts material in the background of
% box \cs{AtBeginShipoutBox}.
% The \meta{background material} is set in an \cs{hbox}, the
% reference point is the upper left corner of the output page.
% In case of \pdfTeX\ in PDF mode, the settings of
% \cs{pdfhorigin} and \cs{pdfvorigin} are respected.
%
% The macro \cs{AtBeginShipoutUpperLeft} is intended to be used
% in one of the hook setting macros, such as
% \cs{AtBeginShipout}, \cs{AtBeginShipoutFirst}, or
% \cs{AtBeginShipoutNext}.
%
% For \LaTeX\ users the \meta{background material} is set
% inside a \texttt{picture} environment:
% \begin{quote}
%   |\begin{picture}(0,0)|\\
%   \mbox{}\quad |\setlength{\unitlength}{1pt}%|\\
%   \mbox{}\quad \meta{background material}\\
%   |\end{picture}|
% \end{quote}
%
% \begin{declcs}{AtBeginShipoutUpperLeftForeground} \M{foreground material}
% \end{declcs}
% See \cs{AtBeginShipoutUpperLeft}. The difference is that the material
% is put in the foreground.
%
% \begin{declcs}{AtBeginShipoutOriginalShipout} \meta{box}
% \end{declcs}
% It stores the meaning of \cs{shipout} at the time this package
% is loaded.
%
% \begin{declcs}{AtBeginShipoutBoxWidth}\\
%   \cs{AtBeginShipoutBoxHeight}\\
%   \cs{AtBeginShipoutBoxDepth}\\
% \end{declcs}
% These macros store the dimensions of the output box \cs{AtBeginShipoutBox}
% before the original shipout is called. If \cs{shipout} is not
% redefined before the package loading or the box dimensions
% are not changed by the redefined \cs{shipout}, these macros
% contain the dimensions of the shipout box. These values can be
% remembered by \cs{label} and \cs{ref}. For example, this is done
% by the package module \xpackage{zref-pagelayout} of project \xpackage{zref}.
% The dimensions of the shipout page can be used in some \hologo{TeX}
% engines (\hologo{pdfTeX} in PDF mode, \hologo{XeTeX}) to calculate
% the media size of the shipout page if \cs{pdfpagewidth} and
% \cs{pdfpageheight} are not set.
%
% \subsection{Examples}
%
% \subsubsection{Example: circle in background}
%
% In this example we put a circle in the background in the middle
% of the paper.
%
%    \begin{macrocode}
%<*example1>
%    \end{macrocode}
%    \begin{macrocode}
\documentclass[a4paper]{article}
\usepackage{color}
\usepackage{atbegshi}
%    \end{macrocode}
%    Package \xpackage{picture} makes life a little easier, because
%    we can now also use length specifications in |picture|'s commands.
%    \begin{macrocode}
\usepackage{picture}
%    \end{macrocode}
%    Now we draw the circle in the middle of the paper.
%    \cs{put} moves downwards, because the origin is at the top
%    of the page, not at its bottom.
%    \begin{macrocode}
\AtBeginShipout{%
  \AtBeginShipoutUpperLeft{%
    \put(0.5\paperwidth,-0.5\paperheight){\circle{10}}%
  }%
}
\begin{document}
\section{Hello World}
\newpage
\AtBeginShipoutNext{%
  \AtBeginShipoutUpperLeft{%
    \color{red}%
    \put(0,-0.5\paperheight){\line(1,0){\paperwidth}}%
    \put(0.5\paperwidth, 0){\line(0,-1){\paperheight}}%
  }%
}
Only on this page we add a red cross.
\newpage
This page has the circle only.
\par
\vspace{\fill}
The next page will be discarded.
\newpage
\AtBeginShipoutNext{%
  \AtBeginShipoutDiscard
}
This page is discarded.
\newpage
The last page.
\end{document}
%    \end{macrocode}
%    \begin{macrocode}
%</example1>
%    \end{macrocode}
%
% \subsubsection{Example: adding TrimBox for dvipdfmx}
%
% Now an example from ``real life'' follows. Someone from the mailing  list
% for dvipdfmx wants to put a TrimBox on every page. If we use
% \cs{AtBeginShipout}, we have to put the \cs{special} inside the
% box \cs{AtBeginShipoutBox} that gets shipped out.
%    \begin{macrocode}
%<*example2>
\documentclass{minimal}
\usepackage{atbegshi}
\usepackage[
  dvipdfm,
  paperwidth=630bp,
  paperheight=810bp
]{geometry}
\AtBeginShipout{%
  \setbox\AtBeginShipoutBox=\hbox{%
    \special{pdf: put @thispage <</TrimBox[9 9 621 801]>>}%
    \box\AtBeginShipoutBox
  }%
}
\begin{document}
  First page
  \newpage
  Second page
\end{document}
%</example2>
%    \end{macrocode}
% Remember, in \cs{AtBeginShipoutBoxFirst} the \cs{setbox} wrapper code
% is implicitly given and the \cs{special} is used directly.
%
% \section{Method of \cs{shipout} overloading}
%
% \subsection{\cs{shipout}}
%
% The \TeX\ primitive command \cs{shipout} takes a box specification
% and puts the box as a new page in the output file.
% There are two kinds of box specifications:
% \begin{description}
% \item[Direct boxes:]
%   They are given by \cs{hbox}, \cs{vbox}, or \cs{vtop},\\
%   e.g. |\shipout\hbox{Hello World}|.
% \item[Indirect boxes:]
%   \cs{box} or \cs{copy} references a box register by number.
%   The box register contains the contents of the box.
%
%   \emph{Note:} \cs{box} also clears the box register globally.
% \end{description}
% Then we have to differentiate between void and empty boxes:
% \begin{description}
% \item[Void:]
%   Initially or after \cs{box} there is no box in the box register.
%   In this cases the box register is not empty, but \emph{void}.
% \item[Empty:]
%   A box with empty contents, such as
%   |\hbox{}| (= \cs{null}) or |\vbox{}| is an \emph{empty \cs{hbox}}
%   or \emph{empty \cs{vbox}}. If a box register holds such a
%   box, the box still exists, therefore the box register is
%   \emph{not void}.
% \end{description}
%
% \subsection{\cs{afterassignment}}
%
% We want to overload \cs{shipout} to do something with the box.
% It is quite impossible to do this reliable by catching the box
% using macro arguments. The
% variety of box specifications is too large, Examples:
% \begin{quote}
%   |\shipout\null|\\
%   |\shipout\vbox{...}|\\
%   |\shipout\vtop\bgroup ...\egroup|\\
%   |\shipout\box255|
% \end{quote}
% Even worse, the braces don't need to be balanced:
% \begin{quote}
%   |\shipout\hbox\bgroup}|\\
%   |\shipout\vbox{\egroup|
% \end{quote}
% Happily TeX provides a reliable way via \cs{afterassignment}.
% It takes a macro name and executes it just after the assignment.
%
% Now we can redefine \cs{shipout}. The box specification that
% follows \cs{shipout} is catched by \cs{setbox}. This is an
% assignment to a box register. \cs{afterassignment}
% notifies \TeX, that we want to call \cs{@test} right after
% the assignment:
%\begin{quote}
%\begin{verbatim}
%\shipout :=
%  \afterassignment\@test
%  \setbox\mybox=
%\end{verbatim}
%\end{quote}
% We have seen different box specifications. Indirect boxes are easy
% to understand:
% \begin{quote}
%   |\shipout\box0| $\Rightarrow$ |\setbox\mybox=\box0 |\fbox{\cs{@test}}
% \end{quote}
% However direct boxes can have arbitrary contents with lots of
% other assignments. It would be quite inpredictable if \TeX\ would
% put |\@test| after the first of such an assignment or after the
% box specification if the box lacks of assignments.
% Therefore \TeX\ puts |\@test| right at the beginning of the
% box specification, e.g:
% \begin{quote}
%   |\shipout\hbox{Hello World}|\\
%   $\Rightarrow$
%   |\setbox\mybox=\hbox{|\fbox{\cs{@test}}| Hello World}|
% \end{quote}
%
% \subsection{Test for direct or indirect boxes}
%
% Now we want to execute \cs{@test}, but where are we?
% We can be after the completed box assignment, if \cs{shipout}
% was called with an indirect box. Or we are right at the beginning
% of a direct box.
%
% \subsubsection{With \eTeX}
%
% With the \eTeX's extensions the answer is very easy:
% Being inside the direct box means that we are inside a new group.
% The new primitive command \cs{currentgrouplevel} tells how
% deeply the groups are currently nested. Macro \cs{@test} just
% compares the previously stored group level with the current one:
%\begin{quote}
%\begin{verbatim}
%\shipout :=
%  \edef\saved@grouplevel{\number\currentgrouplevel}
%  \afterassignment\@test
%  \setbox\mybox=
%
%\@test :=
%  \ifnum\saved@grouplevel=\currentgrouplevel
%    % case: indirect box, the assignment is completed
%    \@output
%  \else
%    % case: direct box, we are inside the box
%    \aftergroup\@outbox
%  \fi
%\end{verbatim}
%\end{quote}
%
% \subsubsection{Without \eTeX}
%
% Life becomes complicate without \eTeX. We cannot ask the
% group level. However, if we are inside a direct box,
% the box register \cs{mybox} is not yet changed by \cs{setbox}.
% Thus we need a special initial value and compare
% it in \cs{@test} with the current value of the box.
%
% What can be used as initial value? Arbitrary box contents cannot
% be compared. \TeX\ only tells us a few properties:
% \begin{itemize}
% \item Box type:\cs{ifhbox}, \cs{ifvbox}
% \item Dimensions: \cs{wd}, \cs{ht}, \cs{dp}
% \item Voidness: \cs{ifvoid}
% \end{itemize}
% Unhappily all these qualities even combined are not sufficient
% for constructing an initial box value, because \cs{shipout} can be
% called with a box that is accidently just the same as the choosen
% initial value.
%
% Nevertheless we have two alternatives for an initial value:
% \begin{itemize}
% \item A box of some type with some funny settings that are
% unlikely to occur in real life, e.g a height of |4911sp-\maxdimen|.
% \item A void box.
% \end{itemize}
% A collision between this initial value and an indirect \cs{shipout}
% box with just the same value is possible.
% Then \cs{@test} will make a wrong decision that it is executed
% inside a direct box and delays \cs{@output} by \cs{aftergroup}.
% Thus \cs{@output} is not called at the place we want.
% In contrary, the result is an uncertainty about the place:
% \begin{itemize}
% \item \cs{shipout} is used in a group that perhaps closes some
%   pages later. A bad place for \cs{@output}.
% \item Without a surrounding group \cs{aftergroup} effectively kills
%   its argument.
% \end{itemize}
% In the first case of a box with special dimensions we can even loose
% the page. However in the case of the void box, this effect is even
% desired, because the original \cs{shipout} does not output void boxes.
% All we have to do is to ensure that our box \cs{mybox} is always
% void except for the phase when the overloaded \cs{shipout} is executed.
% And secondly we must keep this semantics of \cs{shipout} for the void case
% in our macros, namely \cs{@output}.
%\begin{quote}
%\begin{verbatim}
%\shipout :=
%  % trick to get a void box \mybox
%  \begingroup
%    \setbox\mybox=\box\mybox
%  \endgroup
%  \afterassignment\@test
%  \setbox\mybox=
%
%\@test :=
%  \ifvoid\mybox
%    \aftergroup\@output
%  \else
%    \@output
%  \fi
%\end{verbatim}
%\end{quote}
% The nasty case is |\shipout\box\voidb@x| where the indirect box
% is void and that must not generate an output page.
% If a surrounding group is missing the output is ignored
% because of \cs{aftergroup}.
% Otherwise \@output is called some time
% later when the surrounding group closes.
% But \cs{mybox} is void outside the execution phase
% of the redefined \cs{shipout}. Also \cs{@output} checks for a
% void box and cancels the page output.
% The disadvantage remains that the hook in \cs{@output} is called
% for a page that will not be output.
%
% \subsubsection{\cs{lastkern} method}
%
% At the beginning of a new box, there is no \cs{kern}, the contents
% of the box is still empty and \cs{lastkern} returns 0\,pt.
% This can be used to distinguish between direct and indirect boxes:
% We execute \cs{setbox} in a box with a preceding non-zero kern.
% After an indirect box, \cs{lastkern} sees this kern, otherwise
% it returns 0\,pt.
%\begin{quote}
%\begin{verbatim}
%\shipout :=
%  \begingroup
%    \setbox\mybox=\hbox\bgroup
%      \kern1pt
%      \afterassignment\shipout@test
%      \global\setbox\mybox=
%\@test :=
%  \ifdim\lastkern=0pt
%    % direct box
%    \aftergroup\egroup
%    \aftergroup\endgroup
%    \aftergroup\@output
%  \else
%    \egroup
%    \endgroup
%    \@output
%  \fi
%\end{verbatim}
%\end{quote}
% We have two \cs{setbox} commands. The first creates a controlled
% context box where we can safely insert a \cs{kern}. We get rid of
% this temporarily used context box by putting the local \cs{setbox}
% in a group.
%
% After the group we want to have our shipout box in \cs{mybox}.
% Therefore we use a global assignment here.
%
% \subsection{Output}
%
% With or without \eTeX\ we ensure the original behaviour of
% \cs{shipout} that void boxes do not generate output pages.
%
% Now we can place the hook \cs{@hook} for the user code that
% wants to manipulate the output box.
%\begin{quote}
%\begin{verbatim}
%\@output :=
%  \ifvoid\mybox
%    % cancel output of void box
%  \else
%    \@hook
%    \ifvoid\mybox
%      % user code in \@hook could has voided the box
%    \else
%      \original@shipout\box\mybox
%    \fi
%  \fi
%\end{verbatim}
%\end{quote}
%
% \subsection{Separate box register}
%
% So far we have said nothing about the box number of \cs{mybox}.
% The following case that outputs the same page twice
% shows that we are not free in the use
% of the box register:
% \begin{quote}
%   |\shipout\copy<num> \shipout\box<num>|
% \end{quote}
% We manipulate the box by the hook and without \eTeX\ the box
% must even be voided. However, the use case above requires that
% the box contents does not change at all. Therefore we must
% reserve a separate box register to avoid collisions with
% user box registers.
%
% \emph{Note:} Box register number 255 is special for the output
% routine, because TeX complains if this box is not voided
% by the output routine.
% However, this requirement does not apply to \cs{shipout} at all.
% In fact \cs{shipout} does not change any box register. This is
% usually done by a call of \cs{box}, but the output routine
% can do it later \emph{after} invoking of \cs{shipout}.
%
% \subsection{Summary}
%
% \subsubsection{With \eTeX}
%
% Putting the pieces together we get for \eTeX:
%\begin{quote}
%\begin{verbatim}
%\newbox\mybox
%\let\original@shipout\shipout
%
%\shipout :=
%  \edef\saved@grouplevel{\number\currentgrouplevel}
%  \afterassignment\@test
%  \setbox\mybox=
%
%\@test :=
%  \ifnum\saved@grouplevel<\currentgrouplevel
%    \expandafter\aftergroup
%  \fi
%  \@output
%
%\@output :=
%  \ifvoid\mybox
%    % cancel output of void box
%  \else
%    \@hook
%    \ifvoid\mybox
%      % user code in \@hook could have voided the box
%    \else
%      \original@shipout\box\mybox
%    \fi
%  \fi
%\end{verbatim}
%\end{quote}
%
% \subsubsection{Without \eTeX, traditional way}
%
% And for \TeX\ without \eTeX:
%\begin{quote}
%\begin{verbatim}
%\newbox\mybox
%\begingroup
%   \setbox\mybox=\box\mybox % ensure \mybox is void
%\endgroup
%\let\original@shipout\shipout
%
%\shipout :=
%  % trick to get a void box \mybox
%  \begingroup
%    \setbox\mybox=\box\mybox
%  \endgroup
%  \afterassignment\@test
%  \setbox\mybox=
%
%\@test :=
%  \ifvoid\mybox
%    \expandafter\aftergroup
%  \fi
%  \@output
%
%\@output :=
%  \ifvoid\mybox
%    % cancel output of void box
%  \else
%    \@hook
%    \ifvoid\mybox
%      % user code in \@hook could have voided the box
%    \else
%      \original@shipout\box\mybox
%    \fi
%  \fi
%\end{verbatim}
%\end{quote}
%
% \subsubsection{\cs{lastkern} method}
%
% And for \TeX\ without \eTeX\ using the \cs{lastkern} method:
%\begin{quote}
%\begin{verbatim}
%\newbox\mybox
%\let\original@shipout\shipout
%
%\shipout :=
%  \begingroup
%  \setbox\mybox=\hbox\bgroup
%  \kern1pt
%  \afterassignment\@test
%  \setbox\mybox=
%
%\@test :=
%  \ifdim\lastkern=0pt
%    \expandafter\aftergroup
%  \fi
%  \@output
%
%\@output :=
%  \egroup
%  \endgroup
%  \ifvoid\mybox
%    % cancel output of void box
%  \else
%    \@hook
%    \ifvoid\mybox
%      % user code in \@hook could have voided the box
%    \else
%      \original@shipout\box\mybox
%    \fi
%  \fi
%\end{verbatim}
%\end{quote}
%
%
% \StopEventually{
% }
%
% \section{Implementation}
%
% Package \xpackage{atbegshi} uses \eTeX's \cs{currentgrouplevel},
% if it is available. Otherwise the \cs{lastkern} method is used.
%
%    \begin{macrocode}
%<*package>
%    \end{macrocode}
%
% \subsection{Reload check and package identification}
%    Reload check, especially if the package is not used with \LaTeX.
%    \begin{macrocode}
\begingroup\catcode61\catcode48\catcode32=10\relax%
  \catcode13=5 % ^^M
  \endlinechar=13 %
  \catcode35=6 % #
  \catcode39=12 % '
  \catcode44=12 % ,
  \catcode45=12 % -
  \catcode46=12 % .
  \catcode58=12 % :
  \catcode64=11 % @
  \catcode123=1 % {
  \catcode125=2 % }
  \expandafter\let\expandafter\x\csname ver@atbegshi.sty\endcsname
  \ifx\x\relax % plain-TeX, first loading
  \else
    \def\empty{}%
    \ifx\x\empty % LaTeX, first loading,
      % variable is initialized, but \ProvidesPackage not yet seen
    \else
      \expandafter\ifx\csname PackageInfo\endcsname\relax
        \def\x#1#2{%
          \immediate\write-1{Package #1 Info: #2.}%
        }%
      \else
        \def\x#1#2{\PackageInfo{#1}{#2, stopped}}%
      \fi
      \x{atbegshi}{The package is already loaded}%
      \aftergroup\endinput
    \fi
  \fi
\endgroup%
%    \end{macrocode}
%    Package identification:
%    \begin{macrocode}
\begingroup\catcode61\catcode48\catcode32=10\relax%
  \catcode13=5 % ^^M
  \endlinechar=13 %
  \catcode35=6 % #
  \catcode39=12 % '
  \catcode40=12 % (
  \catcode41=12 % )
  \catcode44=12 % ,
  \catcode45=12 % -
  \catcode46=12 % .
  \catcode47=12 % /
  \catcode58=12 % :
  \catcode64=11 % @
  \catcode91=12 % [
  \catcode93=12 % ]
  \catcode123=1 % {
  \catcode125=2 % }
  \expandafter\ifx\csname ProvidesPackage\endcsname\relax
    \def\x#1#2#3[#4]{\endgroup
      \immediate\write-1{Package: #3 #4}%
      \xdef#1{#4}%
    }%
  \else
    \def\x#1#2[#3]{\endgroup
      #2[{#3}]%
      \ifx#1\@undefined
        \xdef#1{#3}%
      \fi
      \ifx#1\relax
        \xdef#1{#3}%
      \fi
    }%
  \fi
\expandafter\x\csname ver@atbegshi.sty\endcsname
\ProvidesPackage{atbegshi}%
  [2016/06/09 v1.18 At begin shipout hook (HO)]%
%    \end{macrocode}
%
% \subsection{Catcodes}
%
%    \begin{macrocode}
\begingroup\catcode61\catcode48\catcode32=10\relax%
  \catcode13=5 % ^^M
  \endlinechar=13 %
  \catcode123=1 % {
  \catcode125=2 % }
  \catcode64=11 % @
  \def\x{\endgroup
    \expandafter\edef\csname AtBegShi@AtEnd\endcsname{%
      \endlinechar=\the\endlinechar\relax
      \catcode13=\the\catcode13\relax
      \catcode32=\the\catcode32\relax
      \catcode35=\the\catcode35\relax
      \catcode61=\the\catcode61\relax
      \catcode64=\the\catcode64\relax
      \catcode123=\the\catcode123\relax
      \catcode125=\the\catcode125\relax
    }%
  }%
\x\catcode61\catcode48\catcode32=10\relax%
\catcode13=5 % ^^M
\endlinechar=13 %
\catcode35=6 % #
\catcode64=11 % @
\catcode123=1 % {
\catcode125=2 % }
\def\TMP@EnsureCode#1#2{%
  \edef\AtBegShi@AtEnd{%
    \AtBegShi@AtEnd
    \catcode#1=\the\catcode#1\relax
  }%
  \catcode#1=#2\relax
}
\TMP@EnsureCode{40}{12}% (
\TMP@EnsureCode{41}{12}% )
\TMP@EnsureCode{44}{12}% ,
\TMP@EnsureCode{45}{12}% -
\TMP@EnsureCode{47}{12}% /
\TMP@EnsureCode{46}{12}% .
\TMP@EnsureCode{58}{12}% :
\TMP@EnsureCode{91}{12}% [
\TMP@EnsureCode{93}{12}% ]
\TMP@EnsureCode{94}{7}% ^(superscript)
\TMP@EnsureCode{96}{12}% `
\edef\AtBegShi@AtEnd{\AtBegShi@AtEnd\noexpand\endinput}
%    \end{macrocode}
%
% \subsection{Preparations}
%
%    \begin{macrocode}
\begingroup\expandafter\expandafter\expandafter\endgroup
\expandafter\ifx\csname RequirePackage\endcsname\relax
  \def\TMP@RequirePackage#1[#2]{%
    \begingroup\expandafter\expandafter\expandafter\endgroup
    \expandafter\ifx\csname ver@#1.sty\endcsname\relax
      \input #1.sty\relax
    \fi
  }%
  \TMP@RequirePackage{infwarerr}[2007/09/09]%
  \TMP@RequirePackage{ltxcmds}[2010/03/01]%
\else
  \RequirePackage{infwarerr}[2007/09/09]%
  \RequirePackage{ltxcmds}[2010/03/01]%
\fi
%    \end{macrocode}
%
%    \begin{macro}{\AtBegShi@CheckDefinable}
%    \begin{macrocode}
\begingroup\expandafter\expandafter\expandafter\endgroup
\expandafter\ifx\csname @ifdefinable\endcsname\relax
  \def\AtBegShi@CheckDefinable#1{%
    \ifcase\ifx#1\relax
             \ltx@one
           \else
             \ifx#1\@undefined
               \ltx@one
             \else
               \ltx@zero
             \fi
           \fi
      \@PackageError{atbegshi}{%
        \string#1\space is already defined%
      }\@ehd
    \fi
  }%
\else
  \def\AtBegShi@CheckDefinable#1{%
    \@ifdefinable{#1}{}%
  }%
\fi
%    \end{macrocode}
%    \end{macro}
%
%    \begin{macro}{\ifAtBegShi@Discarded}
%    \begin{macrocode}
\ltx@newif\ifAtBegShi@Discarded
%    \end{macrocode}
%    \end{macro}
%    \begin{macro}{\AtBeginShipoutDiscard}
%    \begin{macrocode}
\AtBegShi@CheckDefinable\AtBeginShipoutDiscard
\def\AtBeginShipoutDiscard{%
  \deadcycles=\ltx@zero
  \global\AtBegShi@Discardedtrue
}
%    \end{macrocode}
%    \end{macro}
%
%    \begin{macrocode}
\begingroup\expandafter\expandafter\expandafter\endgroup
\expandafter\ifx\csname currentgrouplevel\endcsname\relax
  \catcode`X=9  % ignore
  \catcode`E=14 % comment
\else
  \catcode`X=14 % comment
  \catcode`E=9  % ignore
\fi
%    \end{macrocode}
%
%    \begin{macro}{\AtBegShi@Shipout}
%    \begin{macrocode}
\def\AtBegShi@Shipout{%
X \begingroup
X \setbox\AtBeginShipoutBox=\hbox\bgroup
X \kern\p@
E \edef\AtBegShi@GroupLevel{\number\currentgrouplevel}%
  \afterassignment\AtBegShi@Test
X \global
  \setbox\AtBeginShipoutBox=%
}
%    \end{macrocode}
%    \end{macro}
%
%    \begin{macro}{\AtBegShi@Test}
%    \begin{macrocode}
\def\AtBegShi@Test{%
X \ifdim\lastkern=0pt %
E \ifnum\AtBegShi@GroupLevel<\currentgrouplevel
    \expandafter\aftergroup
  \fi
  \AtBegShi@Output
}
%    \end{macrocode}
%    \end{macro}
%
%    \begin{macro}{\AtBegShi@Output}
%    \begin{macrocode}
\def\AtBegShi@Output{%
X \egroup
X \endgroup
  \ifvoid\AtBeginShipoutBox
    \@PackageWarning{atbegshi}{Ignoring void shipout box}%
  \else
    \let\AtBegShi@OrgProtect\protect
    \csname set@typeset@protect\endcsname
    \global\AtBegShi@Discardedfalse
    \AtBegShi@Hook
    \expandafter\gdef\expandafter\AtBegShi@HookNext
    \expandafter{\expandafter}%
    \AtBegShi@HookNext
    \ifAtBegShi@Discarded
      \@PackageInfoNoLine{atbegshi}{Shipout page discarded}%
      \global\AtBegShi@Discardedfalse
      \begingroup
        \setbox\AtBeginShipoutBox\box\AtBeginShipoutBox
      \endgroup
      \let\protect\AtBegShi@OrgProtect
    \else
      \AtBegShi@First
      \let\protect\AtBegShi@OrgProtect
      \AtBegShi@GetBoxSize\AtBeginShipoutBox
      \ltx@ifundefined{AtNextShipout}{%
      }{%
        \AtNextShipout{\AtBegShi@GetBoxSize\@cclv}%
      }%
      \AtBeginShipoutOriginalShipout\box\AtBeginShipoutBox
    \fi
  \fi
}
%    \end{macrocode}
%    \end{macro}
%
%    \begin{macro}{\AtBegShi@GetBoxSize}
%    \begin{macrocode}
\def\AtBegShi@GetBoxSize#1{%
  \xdef\AtBeginShipoutBoxWidth{\the\wd#1}%
  \xdef\AtBeginShipoutBoxHeight{\the\ht#1}%
  \xdef\AtBeginShipoutBoxDepth{\the\dp#1}%
}
%    \end{macrocode}
%    \end{macro}
%    \begin{macro}{\AtBeginShipoutBoxWidth}
%    \begin{macrocode}
\def\AtBeginShipoutBoxWidth{0pt}
%    \end{macrocode}
%    \end{macro}
%    \begin{macro}{\AtBeginShipoutBoxHeight}
%    \begin{macrocode}
\def\AtBeginShipoutBoxHeight{0pt}
%    \end{macrocode}
%    \end{macro}
%    \begin{macro}{\AtBeginShipoutBoxDepth}
%    \begin{macrocode}
\def\AtBeginShipoutBoxDepth{0pt}
%    \end{macrocode}
%    \end{macro}
%
%    \begin{macrocode}
\catcode`\X=11 %
\catcode`\E=11 %
%    \end{macrocode}
%
%    \begin{macro}{\AtBegShi@First}
%    \begin{macrocode}
\def\AtBegShi@First{%
  \ifx\AtBegShi@HookFirst\ltx@empty
  \else
    \AtBeginShipoutAddToBox{\AtBegShi@HookFirst}%
  \fi
  \global\let\AtBegShi@First\ltx@empty
  \global\let\AtBeginShipoutFirst\AtBegShi@FirstDisabled
}
%    \end{macrocode}
%    \end{macro}
%
%    \begin{macro}{\AtBegShi@Hook}
%    \begin{macrocode}
\gdef\AtBegShi@Hook{}
%    \end{macrocode}
%    \end{macro}
%    \begin{macro}{\AtBegShi@HookNext}
%    \begin{macrocode}
\gdef\AtBegShi@HookNext{}
%    \end{macrocode}
%    \end{macro}
%    \begin{macro}{\AtBegShi@HookFirst}
%    \begin{macrocode}
\gdef\AtBegShi@HookFirst{}
%    \end{macrocode}
%    \end{macro}
%    \begin{macro}{\AtBeginShipout}
%    \begin{macrocode}
\AtBegShi@CheckDefinable\AtBeginShipout
\def\AtBeginShipout{%
  \AtBegShi@AddHook\AtBegShi@Hook
}
%    \end{macrocode}
%    \end{macro}
%    \begin{macro}{\AtBeginShipoutNext}
%    \begin{macrocode}
\AtBegShi@CheckDefinable\AtBeginShipoutNext
\def\AtBeginShipoutNext{%
  \AtBegShi@AddHook\AtBegShi@HookNext
}
%    \end{macrocode}
%    \end{macro}
%    \begin{macro}{\AtBeginShipoutFirst}
%    \begin{macrocode}
\AtBegShi@CheckDefinable\AtBeginShipoutFirst
\def\AtBeginShipoutFirst{%
  \AtBegShi@AddTo\AtBegShi@HookFirst
}
%    \end{macrocode}
%    \end{macro}
%    \begin{macro}{\AtBegShi@FirstDisabled}
%    \begin{macrocode}
\long\def\AtBegShi@FirstDisabled#1{%
  \@PackageWarning{atbegshi}{%
    First page is already shipped out, ignoring\MessageBreak
    \string\AtBeginShipoutFirst
  }%
}
%    \end{macrocode}
%    \end{macro}
%
%    \begin{macro}{\AtBegShi@AddTo}
%    \begin{macrocode}
\begingroup\expandafter\expandafter\expandafter\endgroup
\expandafter\ifx\csname g@addto@macro\endcsname\relax
  \long\def\AtBegShi@AddTo#1#2{%
    \begingroup
      \toks\ltx@zero\expandafter{#1#2}%
      \xdef#1{\the\toks\ltx@zero}%
    \endgroup
  }%
\else
  \let\AtBegShi@AddTo\g@addto@macro
\fi
%    \end{macrocode}
%    \end{macro}
%
%    \begin{macro}{\AtBegShi@AddHook}
%    \begin{macrocode}
\long\def\AtBegShi@AddHook#1#2{%
  \AtBegShi@AddTo#1{\AtBegShi@Item{#2}}%
}
%    \end{macrocode}
%    \end{macro}
%
%    \begin{macro}{\AtBegShi@Item}
%    \begin{macrocode}
\long\def\AtBegShi@Item#1{%
  \ifAtBegShi@Discarded
  \else
    #1%
    \ifAtBegShi@Discarded
    \else
      \ifvoid\AtBeginShipoutBox
        \@PackageWarning{atbegshi}{%
          Shipout box was voided by hook,\MessageBreak
          ignoring shipout box%
        }%
        \AtBeginShipoutDiscard
      \fi
    \fi
  \fi
}
%    \end{macrocode}
%    \end{macro}
%
%    \begin{macro}{\AtBeginShipoutInit}
%    \begin{macrocode}
\AtBegShi@CheckDefinable\AtBeginShipoutInit
\def\AtBeginShipoutInit{%
  \ltx@IfUndefined{newbox}{%
    \@PackageError{atbegshi}{%
      \string\AtBeginShipoutInit\space failed\MessageBreak
       because of missing \expandafter\string\csname newbox\endcsname
    }\@ehc
  }{%
    \csname newbox\endcsname\AtBeginShipoutBox
    \AtBegShi@CheckDefinable\AtBeginShipoutOriginalShipout
    \global\let\AtBeginShipoutOriginalShipout\shipout
    \global\let\shipout\AtBegShi@Shipout
  }%
  \gdef\AtBeginShipoutInit{}%
}
%    \end{macrocode}
%    \end{macro}
%    \begin{macrocode}
\begingroup\expandafter\expandafter\expandafter\endgroup
\expandafter\ifx\csname AtBeginDocument\endcsname\relax
  \AtBeginShipoutInit
\else
  \AtBeginDocument{\AtBeginShipoutInit}%
\fi
%    \end{macrocode}
%
% \subsection{Additions to the shipout box}
%
%    \begin{macro}{\AtBeginShipoutAddToBox}
%    \begin{macrocode}
\def\AtBeginShipoutAddToBox#1{%
  \ifhbox\AtBeginShipoutBox
    \edef\AtBegShi@restore{%
      \hfuzz=\the\hfuzz\relax
      \hbadness=\the\hbadness\relax
    }%
    \hfuzz=1073741823sp\relax
    \hbadness=2147483647\relax
    \setbox\AtBeginShipoutBox=\hbox to \wd\AtBeginShipoutBox{%
      \setbox\ltx@zero=\hbox{%
        \begingroup
          \AtBegShi@restore
          #1%
        \endgroup
      }%
      \wd\ltx@zero=0pt\relax
      \ht\ltx@zero=0pt\relax
      \dp\ltx@zero=0pt\relax
      \raise\ht\AtBeginShipoutBox\copy\ltx@zero
      \unhcopy\AtBeginShipoutBox
    }%
    \AtBegShi@restore
  \else
    \ifvbox\AtBeginShipoutBox
      \edef\AtBegShi@restore{%
        \vfuzz=\the\vfuzz\relax
        \vbadness=\the\vbadness\relax
        \dimen\ltx@zero=\the\dimen\ltx@zero\relax
      }%
      \edef\AtBegShi@restorebox{%
        \ht\AtBeginShipoutBox=\the\ht\AtBeginShipoutBox\relax
        \dp\AtBeginShipoutBox=\the\dp\AtBeginShipoutBox\relax
      }%
      \vfuzz=1073741823sp\relax
      \vbadness=2147483647\relax
      \dimen\ltx@zero=\ht\AtBeginShipoutBox
      \advance\dimen\ltx@zero by \dp\AtBeginShipoutBox
      \setbox\AtBeginShipoutBox=\vbox to \dimen\ltx@zero{%
        \setbox\ltx@zero=\hbox{%
          \begingroup
            \AtBegShi@restore
            #1%
          \endgroup
        }%
        \wd\ltx@zero=0pt\relax
        \ht\ltx@zero=0pt\relax
        \dp\ltx@zero=0pt\relax
        \baselineskip=0pt\relax
        \lineskip=0pt\relax
        \lineskiplimit=0pt\relax
        \copy\ltx@zero
        \unvbox\AtBeginShipoutBox
        \kern0pt%
      }%
      \AtBegShi@restore
      \AtBegShi@restorebox
    \fi
  \fi
}
%    \end{macrocode}
%    \end{macro}
%
%    \begin{macro}{\AtBeginShipoutAddToBoxForeground}
%    \begin{macrocode}
\def\AtBeginShipoutAddToBoxForeground#1{%
  \ifhbox\AtBeginShipoutBox
    \edef\AtBegShi@restore{%
      \hfuzz=\the\hfuzz\relax
      \hbadness=\the\hbadness\relax
    }%
    \hfuzz=1073741823sp\relax
    \hbadness=2147483647\relax
    \setbox\AtBeginShipoutBox=\hbox to \wd\AtBeginShipoutBox{%
      \unhcopy\AtBeginShipoutBox
      \kern-\wd\AtBeginShipoutBox
      \setbox\ltx@zero=\hbox{%
        \begingroup
          \AtBegShi@restore
          #1%
        \endgroup
      }%
      \wd\ltx@zero=0pt\relax
      \ht\ltx@zero=0pt\relax
      \dp\ltx@zero=0pt\relax
      \raise\ht\AtBeginShipoutBox\copy\ltx@zero
      \kern\wd\AtBeginShipoutBox
    }%
    \AtBegShi@restore
  \else
    \ifvbox\AtBeginShipoutBox
      \edef\AtBegShi@restore{%
        \vfuzz=\the\vfuzz\relax
        \vbadness=\the\vbadness\relax
        \dimen\ltx@zero=\the\dimen\ltx@zero\relax
      }%
      \edef\AtBegShi@restorebox{%
        \ht\AtBeginShipoutBox=\the\ht\AtBeginShipoutBox\relax
        \dp\AtBeginShipoutBox=\the\dp\AtBeginShipoutBox\relax
      }%
      \vfuzz=1073741823sp\relax
      \vbadness=2147483647\relax
      \dimen\ltx@zero=\ht\AtBeginShipoutBox
      \advance\dimen\ltx@zero by \dp\AtBeginShipoutBox
      \setbox\AtBeginShipoutBox=\vbox to \dimen\ltx@zero{%
        \setbox\ltx@zero=\hbox{%
          \begingroup
            \AtBegShi@restore
            #1%
          \endgroup
        }%
        \wd\ltx@zero=0pt\relax
        \ht\ltx@zero=0pt\relax
        \dp\ltx@zero=0pt\relax
        \baselineskip=0pt\relax
        \lineskip=0pt\relax
        \lineskiplimit=0pt\relax
        \unvbox\AtBeginShipoutBox
        \kern-\dimen\ltx@zero
        \copy\ltx@zero
        \kern\dimen\ltx@zero
      }%
      \AtBegShi@restore
      \AtBegShi@restorebox
    \fi
  \fi
}
%    \end{macrocode}
%    \end{macro}
%
% \subsection{Positioning}
%
%    \begin{macrocode}
\begingroup\expandafter\expandafter\expandafter\endgroup
\expandafter\ifx\csname RequirePackage\endcsname\relax
  \def\TMP@RequirePackage#1[#2]{%
    \begingroup\expandafter\expandafter\expandafter\endgroup
    \expandafter\ifx\csname ver@#1.sty\endcsname\relax
      \input #1.sty\relax
    \fi
  }%
  \TMP@RequirePackage{ifpdf}[2011/01/30]%
\else
  \RequirePackage{ifpdf}[2011/01/30]%
\fi
%    \end{macrocode}
%
%    \begin{macrocode}
\ifpdf
  \def\AtBegShi@horigin{%
    \ifx\pdfhorigin\@undefined\pdfvariable horigin\else\pdfhorigin\fi}%
  \def\AtBegShi@vorigin{%
    \ifx\pdfvorigin\@undefined\pdfvariable vorigin\else\pdfvorigin\fi}%
\else
  \def\AtBegShi@horigin{72.27pt}%
  \def\AtBegShi@vorigin{72.27pt}%
\fi
%    \end{macrocode}
%
%    \begin{macrocode}
\begingroup
\ifcase
    \expandafter\ifx\csname picture\endcsname\relax
      1%
    \else
      \expandafter\ifx\csname endpicture\endcsname\relax
        1%
      \else
        0%
      \fi
    \fi
  \endgroup
  \def\AtBegShi@BeginPicture{%
    \begingroup
    \picture(0,0)\relax
    \begingroup\expandafter\expandafter\expandafter\endgroup
    \expandafter\ifx\csname unitlength\endcsname\relax
    \else
      \unitlength=1pt\relax
    \fi
    \ignorespaces
  }%
  \def\AtBegShi@EndPicture{%
    \endpicture
    \endgroup
  }%
\else
  \endgroup
  \def\AtBegShi@BeginPicture{%
    \setbox\ltx@zero=\hbox\bgroup
    \begingroup
    \ignorespaces
  }%
  \def\AtBegShi@EndPicture{%
    \endgroup
    \egroup
    \ht\ltx@zero=0pt\relax
    \dp\ltx@zero=0pt\relax
    \copy\ltx@zero
  }%
\fi
%    \end{macrocode}
%
%    \begin{macro}{\AtBeginShipoutUpperLeft}
%    A surrounding \cs{rlap} is not necessary, because
%    the stuff is put in an \cs{hbox} with zero width.
%    \begin{macrocode}
\def\AtBeginShipoutUpperLeft#1{%
  \AtBeginShipoutAddToBox{%
    \kern-\AtBegShi@horigin\relax
    \vbox to 0pt{%
      \kern-\AtBegShi@vorigin\relax
      \AtBegShi@BeginPicture
      #1%
      \AtBegShi@EndPicture
      \vss
    }%
  }%
}
%    \end{macrocode}
%    \end{macro}
%
%    \begin{macro}{\AtBeginShipoutUpperLeftForeground}
%    \begin{macrocode}
\def\AtBeginShipoutUpperLeftForeground#1{%
  \AtBeginShipoutAddToBoxForeground{%
    \kern-\AtBegShi@horigin\relax
    \vbox to 0pt{%
      \kern-\AtBegShi@vorigin\relax
      \AtBegShi@BeginPicture
      #1%
      \AtBegShi@EndPicture
      \vss
    }%
  }%
}
%    \end{macrocode}
%    \end{macro}
%
% \subsection{Patches}
%
%    Patches for \LaTeX\ packages that redefine \cs{shipout}.
%    \LaTeX\ is now supposed to use \eTeX. Thus we do not patch,
%    without \LaTeX\ and \eTeX.
%    \begin{macrocode}
\def\AtBegShi@AbortIfUndefined#1{%
  \begingroup\expandafter\expandafter\expandafter\endgroup
  \expandafter\ifx\csname#1\endcsname\relax
    \expandafter\AtBegShi@AtEnd
  \fi
}
\AtBegShi@AbortIfUndefined{currentgrouplevel}%
\AtBegShi@AbortIfUndefined{AtBeginDocument}%
\AtBegShi@AbortIfUndefined{@ifpackageloaded}%
\AtBegShi@AbortIfUndefined{@ifclassloaded}%
%    \end{macrocode}
%
% \subsubsection{Package \xpackage{crop}}
%
% Fix of method and box.
%    \begin{macrocode}
\def\AtBegShi@PatchCrop{%
  \begingroup
    \def\AtBegShi@Crop@shipout{%
      \afterassignment\CROP@ship
      \setbox\@cclv=%
    }%
    \def\AtBegShi@Crop@ship{%
      \ifvoid\@cclv
        \expandafter\aftergroup
      \fi
      \CROP@@ship
    }%
    \def\AtBegShi@Crop@shiplist{%
      \lineskip\z@
      \lineskiplimit\z@
      \baselineskip\z@
      \CROP@kernel
      \box\@cclv
    }%
    \def\AtBegShi@Crop@@ship{%
      \CROP@shipout\vbox{%
        \CROP@shiplist
      }%
    }%
    \ifx\AtBegShi@Crop@ship\CROP@ship
      \ifx\AtBegShi@Crop@shiplist\CROP@shiplist
        \ifx\AtBegShi@Crop@@ship\CROP@@ship
          \let\AtBegShi@found\relax
          \ifx\shipout\AtBegShi@Crop@shipout
            \def\AtBegShi@found{\shipout}%
          \else\ifx\AtBeginShipoutOriginalShipout\AtBegShi@Crop@shipout
            \def\AtBegShi@found{\AtBeginShipoutOriginalShipout}%
          \else\ifx\@EveryShipout@Org@Shipout\AtBegShi@Crop@shipout
            \def\AtBegShi@found{\@EveryShipout@Org@Shipout}%
          \else\ifx\GPTorg@shipout\AtBegShi@Crop@shipout
            \def\AtBegShi@found{\GPTorg@shipout}%
          \else\ifx\THBorg@shipout\AtBegShi@Crop@shipout
            \def\AtBegShi@found{\THBorg@shipout}%
          \else\ifx\mem@oldshipout\AtBegShi@Crop@shipout
            \def\AtBegShi@found{\mem@oldshipout}%
          \fi\fi\fi\fi\fi\fi
          \ifx\AtBegShi@found\relax
          \else
            \expandafter\endgroup
            \expandafter\def\AtBegShi@found{%
              \edef\AtBegShi@GroupLevel{\number\currentgrouplevel}%
              \afterassignment\CROP@ship
              \setbox\AtBeginShipoutBox=%
            }%
            \def\CROP@ship{%
              \ifnum\AtBegShi@GroupLevel=\currentgrouplevel
              \else
                \expandafter\aftergroup
              \fi
              \CROP@@ship
            }%
            \def\CROP@shiplist{%
              \lineskip 0pt\relax
              \lineskiplimit 0pt\relax
              \baselineskip 0pt\relax
              \CROP@kernel
              \box\AtBeginShipoutBox
            }%
            \def\CROP@@ship{%
              \ifvoid\AtBeginShipoutBox
              \else
                \setbox\AtBeginShipoutBox=\vbox{%
                  \CROP@shiplist
                }%
                \AtBegShi@GetBoxSize\AtBeginShipoutBox
                \expandafter\CROP@shipout
                \expandafter\box
                \expandafter\AtBeginShipoutBox
              \fi
            }%
            \@PackageInfoNoLine{atbegshi}{Package `crop' patched}%
            \begingroup
          \fi
        \fi
      \fi
    \fi
  \endgroup
  \let\AtBegShi@PatchCrop\relax
}
\@ifpackageloaded{crop}{%
  \AtBegShi@PatchCrop
}{%
  \AtBeginDocument{\AtBegShi@PatchCrop}%
}
%    \end{macrocode}
%
% \subsubsection{Package \xpackage{everyshi}}
%
%     Fix of method. Use of box 255 is not changed.
%    \begin{macrocode}
\def\AtBegShi@PatchEveryshi{%
  \begingroup
    \long\def\AtBegShi@Everyshi@shipout{%
      \afterassignment\@EveryShipout@Test
      \global\setbox\@cclv= %
    }%
    \long\def\AtBegShi@Everyshi@Test{%
      \ifvoid\@cclv\relax
        \aftergroup\@EveryShipout@Output
      \else
        \@EveryShipout@Output
      \fi
    }%
    \ifx\AtBegShi@Everyshi@Test\@EveryShipout@Test
      \let\AtBegShi@found\relax
      \ifx\shipout\AtBegShi@Everyshi@shipout
        \def\AtBegShi@found{\shipout}%
      \else\ifx\AtBeginShipoutOriginalShipout\AtBegShi@Everyshi@shipout
        \def\AtBegShi@found{\AtBeginShipoutOriginalShipout}%
      \else\ifx\CROP@shipout\AtBegShi@Everyshi@shipout
        \def\AtBegShi@found{\CROP@shipout}%
      \else\ifx\GPTorg@shipout\AtBegShi@Everyshi@shipout
        \def\AtBegShi@found{\GPTorg@shipout}%
      \else\ifx\THBorg@shipout\AtBegShi@Everyshi@shipout
        \def\AtBegShi@found{\THBorg@shipout}%
      \else\ifx\mem@oldshipout\AtBegShi@Everyshi@shipout
        \def\AtBegShi@found{\mem@oldshipout}%
      \else
        \expandafter\ifx\csname @EveryShipout@Org@Shipout\endcsname
                        \relax
           \ifx\@EveryShipout@Shipout\AtBegShi@Everyshi@shipout
             \def\AtBegShi@found{\@EveryShipout@Shipout}%
           \fi
        \fi
      \fi\fi\fi\fi\fi\fi
      \ifx\AtBegShi@found\relax
      \else
        \expandafter\endgroup
        \expandafter\def\AtBegShi@found{%
          \edef\AtBegShi@GroupLevel{\number\currentgrouplevel}%
          \afterassignment\@EveryShipout@Test
          \setbox\AtBeginShipoutBox=%
        }%
        \def\@EveryShipout@Test{%
          \ifnum\AtBegShi@GroupLevel=\currentgrouplevel
          \else
            \expandafter\aftergroup
          \fi
          \AtBegShi@Everyshi@Output
        }%
        \def\AtBegShi@Everyshi@Output{%
          \ifvoid\AtBeginShipoutBox
          \else
            \global\setbox\ltx@cclv\box\AtBeginShipoutBox
            \expandafter\@EveryShipout@Output
          \fi
        }%
        \@PackageInfoNoLine{atbegshi}{Package `everyshi' patched}%
        \begingroup
      \fi
    \fi
  \endgroup
  \let\AtBegShi@PatchEveryshi\relax
}
\@ifpackageloaded{everyshi}{%
  \AtBegShi@PatchEveryshi
}{%
  \AtBeginDocument{\AtBegShi@PatchEveryshi}%
}
%    \end{macrocode}
%
% \subsubsection{Class \xpackage{memoir}}
%
%     Fix of method and box.
%    \begin{macrocode}
\def\AtBegShi@PatchMemoir{%
  \begingroup
    \def\AtBegShi@Memoir@shipout{%
      \afterassignment\mem@shipi
      \setbox\@cclv=%
    }%
    \def\AtBegShi@Memoir@shipi{%
      \ifvoid\@cclv
        \expandafter\aftergroup
      \fi
      \mem@shipii
    }%
    \def\AtBegShi@Memoir@shipiiA{%
      \mem@oldshipout\vbox{%
        \trimmarks
        \unvbox\@cclv
      }%
    }%
    \def\AtBegShi@Memoir@shipiiB{%
      \ifvoid\@cclv
        \mem@oldshipout\box\@cclv
      \else
        \mem@oldshipout\vbox{%
          \trimmarks
          \unvbox\@cclv
        }%
      \fi
    }%
    \def\AtBegShi@Memoir@PatchAB{%
      \ifvoid\AtBeginShipoutBox
      \else
        \setbox\AtBeginShipoutBox=\vbox{%
          \trimmarks
          \ifvbox\AtBeginShipoutBox
            \unvbox\AtBeginShipoutBox
          \else
            \box\AtBeginShipoutBox
          \fi
        }%
        \AtBegShi@GetBoxSize\AtBeginShipoutBox
        \expandafter\mem@oldshipout
        \expandafter\box
        \expandafter\AtBeginShipoutBox
      \fi
    }%
    \def\AtBegShi@Memoir@shipiiC{% 2008/08/07 v1.6180339a
      \ifvoid\@cclv
        \mem@oldshipout\box\@cclv
      \else
        \ifshowtrims
          \mem@oldshipout\vbox{\trimmarks\unvbox\@cclv}%
        \else
          \mem@oldshipout\box\@cclv
        \fi
      \fi
    }%
    \def\AtBegShi@Memoir@shipiiD{% 2011/03/06 v3.6j
      \ifvoid\@cclv
        \mem@oldshipout\box\@cclv
      \else
        \ifshowtrims
          \mem@oldshipout\vbox{%
            \trimmarks
            \nointerlineskip
            \box\@cclv
          }%
        \else
          \mem@oldshipout\box\@cclv
        \fi
      \fi
    }%
    \def\AtBegShi@Memoir@PatchCD{%
      \ifvoid\AtBeginShipoutBox
      \else
        \ifshowtrims
          \setbox\AtBeginShipoutBox=\vbox{%
            \trimmarks
            \nointerlineskip
            \box\AtBeginShipoutBox
          }%
        \fi
        \AtBegShi@GetBoxSize\AtBeginShipoutBox
        \expandafter\mem@oldshipout
        \expandafter\box
        \expandafter\AtBeginShipoutBox
      \fi
    }%
    \ifx\AtBegShi@Memoir@shipi\mem@shipi
      \let\AtBegShi@found\ltx@one
      \ifx\AtBegShi@Memoir@shipiiA\mem@shipii
        \let\AtBegShi@found\ltx@zero
        \global\let\AtBegShi@Memoir@PatchX\AtBegShi@Memoir@PatchAB
      \else\ifx\AtBegShi@Memoir@shipiiB\mem@shipii
        \let\AtBegShi@found\ltx@zero
        \global\let\AtBegShi@Memoir@PatchX\AtBegShi@Memoir@PatchAB
      \else\ifx\AtBegShi@Memoir@shipiiC\mem@shipii
        \let\AtBegShi@found\ltx@zero
        \global\let\AtBegShi@Memoir@PatchX\AtBegShi@Memoir@PatchCD
      \else\ifx\AtBegShi@Memoir@shipiiD\mem@shipii
        \let\AtBegShi@found\ltx@zero
        \global\let\AtBegShi@Memoir@PatchX\AtBegShi@Memoir@PatchCD
      \fi\fi\fi\fi
      \ifcase\AtBegShi@found
        \let\AtBegShi@found\relax
        \ifx\shipout\AtBegShi@Memoir@shipout
          \def\AtBegShi@found{\shipout}%
        \else\ifx\AtBeginShipoutOriginalShipout\AtBegShi@Memoir@shipout
          \def\AtBegShi@found{\AtBeginShipoutOriginalShipout}%
        \else\ifx\CROP@shipout\AtBegShi@Memoir@shipout
          \def\AtBegShi@found{\CROP@shipout}%
        \else\ifx\GPTorg@shipout\AtBegShi@Memoir@shipout
          \def\AtBegShi@found{\GPTorg@shipout}%
        \else\ifx\THBorg@shipout\AtBegShi@Memoir@shipout
          \def\AtBegShi@found{\THBorg@shipout}%
        \else\ifx\@EveryShipout@Org@Shipout\AtBegShi@Memoir@shipout
          \def\AtBegShi@found{\@EveryShipout@Org@Shipout}%
        \fi\fi\fi\fi\fi\fi
        \ifx\AtBegShi@found\relax
        \else
          \expandafter\endgroup
          \expandafter\def\AtBegShi@found{%
            \edef\AtBegShi@GroupLevel{\number\currentgrouplevel}%
            \afterassignment\mem@shipi
            \setbox\AtBeginShipoutBox=%
          }%
          \def\mem@shipi{%
            \ifnum\AtBegShi@GroupLevel=\currentgrouplevel
            \else
              \expandafter\aftergroup
            \fi
            \mem@shipii
          }%
          \let\mem@shipii\AtBegShi@Memoir@PatchX
          \@PackageInfoNoLine{atbegshi}{Class `memoir' patched}%
          \begingroup
        \fi
      \fi
    \fi
  \endgroup
  \let\AtBegShi@PatchMemoir\relax
}
\@ifclassloaded{memoir}{%
  \AtBegShi@PatchMemoir
}{%
  \AtBeginDocument{\AtBegShi@PatchMemoir}%
}
%    \end{macrocode}
%
%    \begin{macrocode}
\AtBegShi@AtEnd%
%</package>
%    \end{macrocode}
%
% \section{Test}
%
% \subsection{Catcode checks for loading}
%
%    \begin{macrocode}
%<*test1>
%    \end{macrocode}
%    \begin{macrocode}
\catcode`\{=1 %
\catcode`\}=2 %
\catcode`\#=6 %
\catcode`\@=11 %
\expandafter\ifx\csname count@\endcsname\relax
  \countdef\count@=255 %
\fi
\expandafter\ifx\csname @gobble\endcsname\relax
  \long\def\@gobble#1{}%
\fi
\expandafter\ifx\csname @firstofone\endcsname\relax
  \long\def\@firstofone#1{#1}%
\fi
\expandafter\ifx\csname loop\endcsname\relax
  \expandafter\@firstofone
\else
  \expandafter\@gobble
\fi
{%
  \def\loop#1\repeat{%
    \def\body{#1}%
    \iterate
  }%
  \def\iterate{%
    \body
      \let\next\iterate
    \else
      \let\next\relax
    \fi
    \next
  }%
  \let\repeat=\fi
}%
\def\RestoreCatcodes{}
\count@=0 %
\loop
  \edef\RestoreCatcodes{%
    \RestoreCatcodes
    \catcode\the\count@=\the\catcode\count@\relax
  }%
\ifnum\count@<255 %
  \advance\count@ 1 %
\repeat

\def\RangeCatcodeInvalid#1#2{%
  \count@=#1\relax
  \loop
    \catcode\count@=15 %
  \ifnum\count@<#2\relax
    \advance\count@ 1 %
  \repeat
}
\def\RangeCatcodeCheck#1#2#3{%
  \count@=#1\relax
  \loop
    \ifnum#3=\catcode\count@
    \else
      \errmessage{%
        Character \the\count@\space
        with wrong catcode \the\catcode\count@\space
        instead of \number#3%
      }%
    \fi
  \ifnum\count@<#2\relax
    \advance\count@ 1 %
  \repeat
}
\def\space{ }
\expandafter\ifx\csname LoadCommand\endcsname\relax
  \def\LoadCommand{\input atbegshi.sty\relax}%
\fi
\def\Test{%
  \RangeCatcodeInvalid{0}{47}%
  \RangeCatcodeInvalid{58}{64}%
  \RangeCatcodeInvalid{91}{96}%
  \RangeCatcodeInvalid{123}{255}%
  \catcode`\@=12 %
  \catcode`\\=0 %
  \catcode`\%=14 %
  \LoadCommand
  \RangeCatcodeCheck{0}{36}{15}%
  \RangeCatcodeCheck{37}{37}{14}%
  \RangeCatcodeCheck{38}{47}{15}%
  \RangeCatcodeCheck{48}{57}{12}%
  \RangeCatcodeCheck{58}{63}{15}%
  \RangeCatcodeCheck{64}{64}{12}%
  \RangeCatcodeCheck{65}{90}{11}%
  \RangeCatcodeCheck{91}{91}{15}%
  \RangeCatcodeCheck{92}{92}{0}%
  \RangeCatcodeCheck{93}{96}{15}%
  \RangeCatcodeCheck{97}{122}{11}%
  \RangeCatcodeCheck{123}{255}{15}%
  \RestoreCatcodes
}
\Test
\csname @@end\endcsname
\end
%    \end{macrocode}
%    \begin{macrocode}
%</test1>
%    \end{macrocode}
%
%    \begin{macrocode}
%<*test2>
\input atbegshi.sty\relax
\def\msg#{\immediate\write16}
\msg{File: atbegshi-test2.tex 2016/06/09 v1.18 Test file for plain-TeX}
\def\testmsg#1#2{%
  \msg{}%
  \msg{*** Test with box (#1), expected page output [#2]}% hash-ok
}

\newbox\voidbox
\def\void{\box\voidbox}
\begingroup
  \setbox\voidbox=\void
\endgroup

\count0=0\relax
\AtBeginShipout{%
  \global\advance\count0 by 1\relax
  \msg{* Inside \string\AtBeginShipout: [\the\count0]}%
}

\AtBeginShipoutFirst{%
  \msg{* Inside \string\AtBeginShipoutFirst}%
  Hello World%
}

\testmsg{\string\null}{1}
\shipout\null

\AtBeginShipoutFirst{%
  This is too late%
}

\testmsg{void}{}
\shipout\void

\testmsg{\string\copy255 (not void)}{2}
\setbox255\hbox{\vrule height 10bp width 10bp}
\shipout\copy255 %

\testmsg{\string\copy255 (again)}{3}
\shipout\copy255 %

\testmsg{\string\box255}{4}
\shipout\box255 %

\testmsg{\string\box255 (again)}{}
\shipout\box255 %

\testmsg{\string\hbox}{5}
\shipout\hbox{\vrule height 5bp width 20bp}

\testmsg{\string\vbox}{6}
\shipout\vbox{\hrule height 20bp width 5bp}

\testmsg{\string\null, voided by hook}{}
\def\VoidBox{%
  \begingroup
    \setbox\AtBeginShipoutBox=\box\AtBeginShipoutBox
  \endgroup
}
\AtBeginShipout{\VoidBox}
\shipout\null
\def\VoidBox{}

\msg{*** \string\begingroup}
\begingroup
  \testmsg{void}{}%
  \shipout\void
\msg{*** \string\endgroup}
\endgroup

\msg{*** \string\begingroup}
\begingroup
  \testmsg{void}{}%
  \shipout\void
  \testmsg{\string\null}{8}%
  \shipout\null
\msg{*** \string\endgroup}
\endgroup

\testmsg{output routine}{9}
Hello World
\vfill
\eject

\testmsg{\string\null\space(discarded)}{}
\AtBeginShipout{%
  \msg{* Inside \string\AtBeginShipout: DISCARD}%
  \AtBeginShipoutDiscard
}
\shipout\null

\end
%</test2>
%    \end{macrocode}
%
%    \begin{macrocode}
%<*test3>
\NeedsTeXFormat{LaTeX2e}
\ProvidesFile{atbegshi-test3.tex}[2016/06/09 v1.18 Test file for LaTeX]
\RequirePackage{color}
\pagecolor{yellow}
\documentclass[a5paper,showtrims]{memoir}
\usepackage{atbegshi}
\AtBeginShipout{%
  \setbox\AtBeginShipoutBox=\vbox{%
    \vbox to 0pt{%
      \kern-1.5in %
      \hbox to 0pt{%
        \kern-1.5in %
        \color{blue}%
        \rule{1in}{1in}%
        \hss
      }%
      \vss
    }%
    \hrule
    \hbox{\vrule\box\AtBeginShipoutBox\vrule}%
    \hrule
  }%
}
\usepackage{eso-pic}
\makeatletter
\@ifundefined{@EveryShipout@Init}{%
  \typeout{Test skipped}%
  \@@end
}{}
\@EveryShipout@Init
\let\@EveryShipout@Init\relax
\makeatother
\AddToShipoutPicture{%
  \hspace{.52\paperwidth}%
  \colorbox{cyan}{%
    \rule{0mm}{\paperheight}%
    \hspace{.48\paperwidth}%
  }%
}
%    \end{macrocode}
%    Newer versions of class \xclass{memoir} emulate package
%    \xpackage{crop} and prevents its loading. This is undone
%    in next line for this test file.
%    \begin{macrocode}
\expandafter\let\csname ver@crop.sty\endcsname\relax
\usepackage[color=red,cross,a4,center]{crop}
\begin{document}
\shipout\null
\shipout\box\csname voidb@x\endcsname
\section{Hello World}
\end{document}
%</test3>
%    \end{macrocode}
%
% \section{Installation}
%
% \subsection{Download}
%
% \paragraph{Package.} This package is available on
% CTAN\footnote{\url{http://ctan.org/pkg/atbegshi}}:
% \begin{description}
% \item[\CTAN{macros/latex/contrib/oberdiek/atbegshi.dtx}] The source file.
% \item[\CTAN{macros/latex/contrib/oberdiek/atbegshi.pdf}] Documentation.
% \end{description}
%
%
% \paragraph{Bundle.} All the packages of the bundle `oberdiek'
% are also available in a TDS compliant ZIP archive. There
% the packages are already unpacked and the documentation files
% are generated. The files and directories obey the TDS standard.
% \begin{description}
% \item[\CTAN{install/macros/latex/contrib/oberdiek.tds.zip}]
% \end{description}
% \emph{TDS} refers to the standard ``A Directory Structure
% for \TeX\ Files'' (\CTAN{tds/tds.pdf}). Directories
% with \xfile{texmf} in their name are usually organized this way.
%
% \subsection{Bundle installation}
%
% \paragraph{Unpacking.} Unpack the \xfile{oberdiek.tds.zip} in the
% TDS tree (also known as \xfile{texmf} tree) of your choice.
% Example (linux):
% \begin{quote}
%   |unzip oberdiek.tds.zip -d ~/texmf|
% \end{quote}
%
% \paragraph{Script installation.}
% Check the directory \xfile{TDS:scripts/oberdiek/} for
% scripts that need further installation steps.
% Package \xpackage{attachfile2} comes with the Perl script
% \xfile{pdfatfi.pl} that should be installed in such a way
% that it can be called as \texttt{pdfatfi}.
% Example (linux):
% \begin{quote}
%   |chmod +x scripts/oberdiek/pdfatfi.pl|\\
%   |cp scripts/oberdiek/pdfatfi.pl /usr/local/bin/|
% \end{quote}
%
% \subsection{Package installation}
%
% \paragraph{Unpacking.} The \xfile{.dtx} file is a self-extracting
% \docstrip\ archive. The files are extracted by running the
% \xfile{.dtx} through \plainTeX:
% \begin{quote}
%   \verb|tex atbegshi.dtx|
% \end{quote}
%
% \paragraph{TDS.} Now the different files must be moved into
% the different directories in your installation TDS tree
% (also known as \xfile{texmf} tree):
% \begin{quote}
% \def\t{^^A
% \begin{tabular}{@{}>{\ttfamily}l@{ $\rightarrow$ }>{\ttfamily}l@{}}
%   atbegshi.sty & tex/generic/oberdiek/atbegshi.sty\\
%   atbegshi.pdf & doc/latex/oberdiek/atbegshi.pdf\\
%   atbegshi-example1.tex & doc/latex/oberdiek/atbegshi-example1.tex\\
%   atbegshi-example2.tex & doc/latex/oberdiek/atbegshi-example2.tex\\
%   test/atbegshi-test1.tex & doc/latex/oberdiek/test/atbegshi-test1.tex\\
%   test/atbegshi-test2.tex & doc/latex/oberdiek/test/atbegshi-test2.tex\\
%   test/atbegshi-test3.tex & doc/latex/oberdiek/test/atbegshi-test3.tex\\
%   atbegshi.dtx & source/latex/oberdiek/atbegshi.dtx\\
% \end{tabular}^^A
% }^^A
% \sbox0{\t}^^A
% \ifdim\wd0>\linewidth
%   \begingroup
%     \advance\linewidth by\leftmargin
%     \advance\linewidth by\rightmargin
%   \edef\x{\endgroup
%     \def\noexpand\lw{\the\linewidth}^^A
%   }\x
%   \def\lwbox{^^A
%     \leavevmode
%     \hbox to \linewidth{^^A
%       \kern-\leftmargin\relax
%       \hss
%       \usebox0
%       \hss
%       \kern-\rightmargin\relax
%     }^^A
%   }^^A
%   \ifdim\wd0>\lw
%     \sbox0{\small\t}^^A
%     \ifdim\wd0>\linewidth
%       \ifdim\wd0>\lw
%         \sbox0{\footnotesize\t}^^A
%         \ifdim\wd0>\linewidth
%           \ifdim\wd0>\lw
%             \sbox0{\scriptsize\t}^^A
%             \ifdim\wd0>\linewidth
%               \ifdim\wd0>\lw
%                 \sbox0{\tiny\t}^^A
%                 \ifdim\wd0>\linewidth
%                   \lwbox
%                 \else
%                   \usebox0
%                 \fi
%               \else
%                 \lwbox
%               \fi
%             \else
%               \usebox0
%             \fi
%           \else
%             \lwbox
%           \fi
%         \else
%           \usebox0
%         \fi
%       \else
%         \lwbox
%       \fi
%     \else
%       \usebox0
%     \fi
%   \else
%     \lwbox
%   \fi
% \else
%   \usebox0
% \fi
% \end{quote}
% If you have a \xfile{docstrip.cfg} that configures and enables \docstrip's
% TDS installing feature, then some files can already be in the right
% place, see the documentation of \docstrip.
%
% \subsection{Refresh file name databases}
%
% If your \TeX~distribution
% (\teTeX, \mikTeX, \dots) relies on file name databases, you must refresh
% these. For example, \teTeX\ users run \verb|texhash| or
% \verb|mktexlsr|.
%
% \subsection{Some details for the interested}
%
% \paragraph{Attached source.}
%
% The PDF documentation on CTAN also includes the
% \xfile{.dtx} source file. It can be extracted by
% AcrobatReader 6 or higher. Another option is \textsf{pdftk},
% e.g. unpack the file into the current directory:
% \begin{quote}
%   \verb|pdftk atbegshi.pdf unpack_files output .|
% \end{quote}
%
% \paragraph{Unpacking with \LaTeX.}
% The \xfile{.dtx} chooses its action depending on the format:
% \begin{description}
% \item[\plainTeX:] Run \docstrip\ and extract the files.
% \item[\LaTeX:] Generate the documentation.
% \end{description}
% If you insist on using \LaTeX\ for \docstrip\ (really,
% \docstrip\ does not need \LaTeX), then inform the autodetect routine
% about your intention:
% \begin{quote}
%   \verb|latex \let\install=y% \iffalse meta-comment
%
% File: atbegshi.dtx
% Version: 2016/06/09 v1.18
% Info: At begin shipout hook
%
% Copyright (C) 2007-2011 by
%    Heiko Oberdiek <heiko.oberdiek at googlemail.com>
%    2016
%    https://github.com/ho-tex/oberdiek/issues
%
% This work may be distributed and/or modified under the
% conditions of the LaTeX Project Public License, either
% version 1.3c of this license or (at your option) any later
% version. This version of this license is in
%    http://www.latex-project.org/lppl/lppl-1-3c.txt
% and the latest version of this license is in
%    http://www.latex-project.org/lppl.txt
% and version 1.3 or later is part of all distributions of
% LaTeX version 2005/12/01 or later.
%
% This work has the LPPL maintenance status "maintained".
%
% This Current Maintainer of this work is Heiko Oberdiek.
%
% The Base Interpreter refers to any `TeX-Format',
% because some files are installed in TDS:tex/generic//.
%
% This work consists of the main source file atbegshi.dtx
% and the derived files
%    atbegshi.sty, atbegshi.pdf, atbegshi.ins, atbegshi.drv,
%    atbegshi-example1.tex, atbegshi-example2.tex,
%    atbegshi-test1.tex, atbegshi-test2.tex, atbegshi-test3.tex.
%
% Distribution:
%    CTAN:macros/latex/contrib/oberdiek/atbegshi.dtx
%    CTAN:macros/latex/contrib/oberdiek/atbegshi.pdf
%
% Unpacking:
%    (a) If atbegshi.ins is present:
%           tex atbegshi.ins
%    (b) Without atbegshi.ins:
%           tex atbegshi.dtx
%    (c) If you insist on using LaTeX
%           latex \let\install=y% \iffalse meta-comment
%
% File: atbegshi.dtx
% Version: 2016/06/09 v1.18
% Info: At begin shipout hook
%
% Copyright (C) 2007-2011 by
%    Heiko Oberdiek <heiko.oberdiek at googlemail.com>
%    2016
%    https://github.com/ho-tex/oberdiek/issues
%
% This work may be distributed and/or modified under the
% conditions of the LaTeX Project Public License, either
% version 1.3c of this license or (at your option) any later
% version. This version of this license is in
%    http://www.latex-project.org/lppl/lppl-1-3c.txt
% and the latest version of this license is in
%    http://www.latex-project.org/lppl.txt
% and version 1.3 or later is part of all distributions of
% LaTeX version 2005/12/01 or later.
%
% This work has the LPPL maintenance status "maintained".
%
% This Current Maintainer of this work is Heiko Oberdiek.
%
% The Base Interpreter refers to any `TeX-Format',
% because some files are installed in TDS:tex/generic//.
%
% This work consists of the main source file atbegshi.dtx
% and the derived files
%    atbegshi.sty, atbegshi.pdf, atbegshi.ins, atbegshi.drv,
%    atbegshi-example1.tex, atbegshi-example2.tex,
%    atbegshi-test1.tex, atbegshi-test2.tex, atbegshi-test3.tex.
%
% Distribution:
%    CTAN:macros/latex/contrib/oberdiek/atbegshi.dtx
%    CTAN:macros/latex/contrib/oberdiek/atbegshi.pdf
%
% Unpacking:
%    (a) If atbegshi.ins is present:
%           tex atbegshi.ins
%    (b) Without atbegshi.ins:
%           tex atbegshi.dtx
%    (c) If you insist on using LaTeX
%           latex \let\install=y\input{atbegshi.dtx}
%        (quote the arguments according to the demands of your shell)
%
% Documentation:
%    (a) If atbegshi.drv is present:
%           latex atbegshi.drv
%    (b) Without atbegshi.drv:
%           latex atbegshi.dtx; ...
%    The class ltxdoc loads the configuration file ltxdoc.cfg
%    if available. Here you can specify further options, e.g.
%    use A4 as paper format:
%       \PassOptionsToClass{a4paper}{article}
%
%    Programm calls to get the documentation (example):
%       pdflatex atbegshi.dtx
%       makeindex -s gind.ist atbegshi.idx
%       pdflatex atbegshi.dtx
%       makeindex -s gind.ist atbegshi.idx
%       pdflatex atbegshi.dtx
%
% Installation:
%    TDS:tex/generic/oberdiek/atbegshi.sty
%    TDS:doc/latex/oberdiek/atbegshi.pdf
%    TDS:doc/latex/oberdiek/atbegshi-example1.tex
%    TDS:doc/latex/oberdiek/atbegshi-example2.tex
%    TDS:doc/latex/oberdiek/test/atbegshi-test1.tex
%    TDS:doc/latex/oberdiek/test/atbegshi-test2.tex
%    TDS:doc/latex/oberdiek/test/atbegshi-test3.tex
%    TDS:source/latex/oberdiek/atbegshi.dtx
%
%<*ignore>
\begingroup
  \catcode123=1 %
  \catcode125=2 %
  \def\x{LaTeX2e}%
\expandafter\endgroup
\ifcase 0\ifx\install y1\fi\expandafter
         \ifx\csname processbatchFile\endcsname\relax\else1\fi
         \ifx\fmtname\x\else 1\fi\relax
\else\csname fi\endcsname
%</ignore>
%<*install>
\input docstrip.tex
\Msg{************************************************************************}
\Msg{* Installation}
\Msg{* Package: atbegshi 2016/06/09 v1.18 At begin shipout hook (HO)}
\Msg{************************************************************************}

\keepsilent
\askforoverwritefalse

\let\MetaPrefix\relax
\preamble

This is a generated file.

Project: atbegshi
Version: 2016/06/09 v1.18

Copyright (C) 2007-2011 by
   Heiko Oberdiek <heiko.oberdiek at googlemail.com>

This work may be distributed and/or modified under the
conditions of the LaTeX Project Public License, either
version 1.3c of this license or (at your option) any later
version. This version of this license is in
   http://www.latex-project.org/lppl/lppl-1-3c.txt
and the latest version of this license is in
   http://www.latex-project.org/lppl.txt
and version 1.3 or later is part of all distributions of
LaTeX version 2005/12/01 or later.

This work has the LPPL maintenance status "maintained".

This Current Maintainer of this work is Heiko Oberdiek.

The Base Interpreter refers to any `TeX-Format',
because some files are installed in TDS:tex/generic//.

This work consists of the main source file atbegshi.dtx
and the derived files
   atbegshi.sty, atbegshi.pdf, atbegshi.ins, atbegshi.drv,
   atbegshi-example1.tex, atbegshi-example2.tex,
   atbegshi-test1.tex, atbegshi-test2.tex, atbegshi-test3.tex.

\endpreamble
\let\MetaPrefix\DoubleperCent

\generate{%
  \file{atbegshi.ins}{\from{atbegshi.dtx}{install}}%
  \file{atbegshi.drv}{\from{atbegshi.dtx}{driver}}%
  \usedir{tex/generic/oberdiek}%
  \file{atbegshi.sty}{\from{atbegshi.dtx}{package}}%
  \usedir{doc/latex/oberdiek}%
  \file{atbegshi-example1.tex}{\from{atbegshi.dtx}{example1}}%
  \file{atbegshi-example2.tex}{\from{atbegshi.dtx}{example2}}%
  \usedir{doc/latex/oberdiek/test}%
  \file{atbegshi-test1.tex}{\from{atbegshi.dtx}{test1}}%
  \file{atbegshi-test2.tex}{\from{atbegshi.dtx}{test2}}%
  \file{atbegshi-test3.tex}{\from{atbegshi.dtx}{test3}}%
  \nopreamble
  \nopostamble
  \usedir{source/latex/oberdiek/catalogue}%
  \file{atbegshi.xml}{\from{atbegshi.dtx}{catalogue}}%
}

\catcode32=13\relax% active space
\let =\space%
\Msg{************************************************************************}
\Msg{*}
\Msg{* To finish the installation you have to move the following}
\Msg{* file into a directory searched by TeX:}
\Msg{*}
\Msg{*     atbegshi.sty}
\Msg{*}
\Msg{* To produce the documentation run the file `atbegshi.drv'}
\Msg{* through LaTeX.}
\Msg{*}
\Msg{* Happy TeXing!}
\Msg{*}
\Msg{************************************************************************}

\endbatchfile
%</install>
%<*ignore>
\fi
%</ignore>
%<*driver>
\NeedsTeXFormat{LaTeX2e}
\ProvidesFile{atbegshi.drv}%
  [2016/06/09 v1.18 At begin shipout hook (HO)]%
\documentclass{ltxdoc}
\usepackage{holtxdoc}[2011/11/22]
\begin{document}
  \DocInput{atbegshi.dtx}%
\end{document}
%</driver>
% \fi
%
%
% \CharacterTable
%  {Upper-case    \A\B\C\D\E\F\G\H\I\J\K\L\M\N\O\P\Q\R\S\T\U\V\W\X\Y\Z
%   Lower-case    \a\b\c\d\e\f\g\h\i\j\k\l\m\n\o\p\q\r\s\t\u\v\w\x\y\z
%   Digits        \0\1\2\3\4\5\6\7\8\9
%   Exclamation   \!     Double quote  \"     Hash (number) \#
%   Dollar        \$     Percent       \%     Ampersand     \&
%   Acute accent  \'     Left paren    \(     Right paren   \)
%   Asterisk      \*     Plus          \+     Comma         \,
%   Minus         \-     Point         \.     Solidus       \/
%   Colon         \:     Semicolon     \;     Less than     \<
%   Equals        \=     Greater than  \>     Question mark \?
%   Commercial at \@     Left bracket  \[     Backslash     \\
%   Right bracket \]     Circumflex    \^     Underscore    \_
%   Grave accent  \`     Left brace    \{     Vertical bar  \|
%   Right brace   \}     Tilde         \~}
%
% \GetFileInfo{atbegshi.drv}
%
% \title{The \xpackage{atbegshi} package}
% \date{2016/06/09 v1.18}
% \author{Heiko Oberdiek\thanks
% {Please report any issues at https://github.com/ho-tex/oberdiek/issues}\\
% \xemail{heiko.oberdiek at googlemail.com}}
%
% \maketitle
%
% \begin{abstract}
% This package is a modern reimplementation of package \xpackage{everyshi}
% without the burden of compatibility. It makes use of \eTeX's if available.
% Both \LaTeX\ and \plainTeX\ are supported.
% \end{abstract}
%
% \tableofcontents
%
% \section{Documentation}
%
% Package \xpackage{atbegshi} redefines \cs{shipout} to insert
% hooks for user code that is executed before the page is shipped out.
% The code may modify or even discard the output page.
% Three hooks are implemented:
% \begin{enumerate}
% \item A hook that is executed for every page, see\\
%       \cs{AtBeginShipout}
% \item A hook that is executed for the next page only, see\\
%       \cs{AtBeginShipoutNext}
% \item A hook that is only executed for the first page, see\\
%       \cs{AtBeginShipoutFirst}
% \end{enumerate}
% The hooks are executed in this order. The following three macros
% provide the user interface for adding code to these hooks:
%
% \begin{declcs}{AtBeginShipout} \M{code}\\
%   \cs{AtBeginShipoutBox}
% \end{declcs}
% Execute the \meta{code} for every page.
% The page contents is held in box register \cs{AtBeginShipoutBox}
% and may be modified. Use \cs{AtBeginShipoutDiscard} if you want to
% discard the page.
%
% \emph{Note:} Package \xpackage{everyshi} uses box register 255. With
% package \xpackage{atbegshi} you must use \cs{AtBeginShipoutBox} instead.
%
% If \LaTeX\ calls \cs{shipout} in \cs{@outputpage} (part of its output
% routine),
% the meaning of \cs{protect} is \cs{noexpand}. \LaTeX\ sets
% \cs{protect} to the appropriate \cs{@typeset@protect} in the box
% that is shipped out. This is too late for the hooks, they
% are called earlier in the redefined \cs{shipout}. Therefore
% package \xpackage{atbegshi} sets \cs{protect} to \cs{@typeset@protect}
% before it calls the hooks.
% (In \cs{EveryShipout} of package \xpackage{everyshi} the
% user is responsible for the correct setting of \cs{protect}.)
%
% \begin{declcs}{AtBeginShipoutNext} \M{code}
% \end{declcs}
% This reimplements package \xpackage{everyshi}'s \cs{AtNextShipout}.
% The \meta{code} is executed at shipout time of the next page only.
% It is just a convenience macro, it can be easily replaced by something
% like:
%\begin{quote}
%\begin{verbatim}
%\newcommand{\MyShipoutHook}{}%
%\AtBeginShipout{\MyShipoutHook}
%\gdef\MyShipoutHook{%
%  ... do something with next page ...
%  \gdef\MyShipoutHook{}%
% }
%\end{verbatim}
%\end{quote}
% (This can be necessary, if hook order does matter).
%
% \begin{declcs}{AtBeginShipoutFirst} \M{code}
% \end{declcs}
% This reimplements \LaTeX's \cs{AtBeginDvi}. This hook is
% usually used for \cs{special} commands that include PostScript
% header files. The \cs{code} is directly executed in a \cs{vbox}
% that is put at the beginning of the output page. Dealing with
% the output box \cs{AtBeginShipoutBox} is not necessary and
% not permitted here.
%
% \begin{declcs}{AtBeginShipoutDiscard}
% \end{declcs}
% This macro notifies package \xpackage{atbegshi} that
% the output page is discarded. The remaining hook code
% and the remaining hooks are not executed and the page is thrown away.
% Also \cs{deadcycles} is cleared to zero like an ordinary \cs{shipout}
% would do.
%
% \begin{declcs}{AtBeginShipoutInit}
% \end{declcs}
% Usually the redefinition of \cs{shipout} is delayed by
% \cs{AtBeginDocument} (if this macro exists). This can be
% too late, if other packages also redefines \cs{shipout}
% and the order does matter.
% \cs{AtBeginShipoutInit} forces the immediate redefinition
% of \cs{shipout}.
%
% \begin{declcs}{AtBeginShipoutAddToBox} \M{stuff}\\
%   \cs{AtBeginShipoutAddToBoxForeground} \M{stuff}
% \end{declcs}
% A quite common use case is the addition of \cs{special}
% or other whatsits to the page output box. Macro
% \cs{AtBeginShipoutAddToBox} puts \meta{stuff} in a
% box with zeroed dimensions. The box with the \meta{stuff}
% is put in the upper left corner of the shipout box
% \cs{AtBeginShipoutBox}. Macro \cs{AtBeginShipoutAddToBox}
% puts the \meta{stuff} in the background, the other macro
% \cs{AtBeginShipoutAddToBoxForeground} in the foreground
% after the original shipout box contents is set.
%
% A void shipout box (that means a discarded page)
% remains void that means \meta{stuff} is ignored in this case.
% The box type of \cs{AtBeginShipoutBox} is preserved.
% Also the box nesting level for the original contents
% of \cs{AtBeginShipoutBox} remains, for example, to avoid trouble
% with links across pages in case of \hologo{pdfTeX}.
%
% \begin{declcs}{AtBeginShipoutUpperLeft} \M{background material}
% \end{declcs}
% This is a macro that puts material in the background of
% box \cs{AtBeginShipoutBox}.
% The \meta{background material} is set in an \cs{hbox}, the
% reference point is the upper left corner of the output page.
% In case of \pdfTeX\ in PDF mode, the settings of
% \cs{pdfhorigin} and \cs{pdfvorigin} are respected.
%
% The macro \cs{AtBeginShipoutUpperLeft} is intended to be used
% in one of the hook setting macros, such as
% \cs{AtBeginShipout}, \cs{AtBeginShipoutFirst}, or
% \cs{AtBeginShipoutNext}.
%
% For \LaTeX\ users the \meta{background material} is set
% inside a \texttt{picture} environment:
% \begin{quote}
%   |\begin{picture}(0,0)|\\
%   \mbox{}\quad |\setlength{\unitlength}{1pt}%|\\
%   \mbox{}\quad \meta{background material}\\
%   |\end{picture}|
% \end{quote}
%
% \begin{declcs}{AtBeginShipoutUpperLeftForeground} \M{foreground material}
% \end{declcs}
% See \cs{AtBeginShipoutUpperLeft}. The difference is that the material
% is put in the foreground.
%
% \begin{declcs}{AtBeginShipoutOriginalShipout} \meta{box}
% \end{declcs}
% It stores the meaning of \cs{shipout} at the time this package
% is loaded.
%
% \begin{declcs}{AtBeginShipoutBoxWidth}\\
%   \cs{AtBeginShipoutBoxHeight}\\
%   \cs{AtBeginShipoutBoxDepth}\\
% \end{declcs}
% These macros store the dimensions of the output box \cs{AtBeginShipoutBox}
% before the original shipout is called. If \cs{shipout} is not
% redefined before the package loading or the box dimensions
% are not changed by the redefined \cs{shipout}, these macros
% contain the dimensions of the shipout box. These values can be
% remembered by \cs{label} and \cs{ref}. For example, this is done
% by the package module \xpackage{zref-pagelayout} of project \xpackage{zref}.
% The dimensions of the shipout page can be used in some \hologo{TeX}
% engines (\hologo{pdfTeX} in PDF mode, \hologo{XeTeX}) to calculate
% the media size of the shipout page if \cs{pdfpagewidth} and
% \cs{pdfpageheight} are not set.
%
% \subsection{Examples}
%
% \subsubsection{Example: circle in background}
%
% In this example we put a circle in the background in the middle
% of the paper.
%
%    \begin{macrocode}
%<*example1>
%    \end{macrocode}
%    \begin{macrocode}
\documentclass[a4paper]{article}
\usepackage{color}
\usepackage{atbegshi}
%    \end{macrocode}
%    Package \xpackage{picture} makes life a little easier, because
%    we can now also use length specifications in |picture|'s commands.
%    \begin{macrocode}
\usepackage{picture}
%    \end{macrocode}
%    Now we draw the circle in the middle of the paper.
%    \cs{put} moves downwards, because the origin is at the top
%    of the page, not at its bottom.
%    \begin{macrocode}
\AtBeginShipout{%
  \AtBeginShipoutUpperLeft{%
    \put(0.5\paperwidth,-0.5\paperheight){\circle{10}}%
  }%
}
\begin{document}
\section{Hello World}
\newpage
\AtBeginShipoutNext{%
  \AtBeginShipoutUpperLeft{%
    \color{red}%
    \put(0,-0.5\paperheight){\line(1,0){\paperwidth}}%
    \put(0.5\paperwidth, 0){\line(0,-1){\paperheight}}%
  }%
}
Only on this page we add a red cross.
\newpage
This page has the circle only.
\par
\vspace{\fill}
The next page will be discarded.
\newpage
\AtBeginShipoutNext{%
  \AtBeginShipoutDiscard
}
This page is discarded.
\newpage
The last page.
\end{document}
%    \end{macrocode}
%    \begin{macrocode}
%</example1>
%    \end{macrocode}
%
% \subsubsection{Example: adding TrimBox for dvipdfmx}
%
% Now an example from ``real life'' follows. Someone from the mailing  list
% for dvipdfmx wants to put a TrimBox on every page. If we use
% \cs{AtBeginShipout}, we have to put the \cs{special} inside the
% box \cs{AtBeginShipoutBox} that gets shipped out.
%    \begin{macrocode}
%<*example2>
\documentclass{minimal}
\usepackage{atbegshi}
\usepackage[
  dvipdfm,
  paperwidth=630bp,
  paperheight=810bp
]{geometry}
\AtBeginShipout{%
  \setbox\AtBeginShipoutBox=\hbox{%
    \special{pdf: put @thispage <</TrimBox[9 9 621 801]>>}%
    \box\AtBeginShipoutBox
  }%
}
\begin{document}
  First page
  \newpage
  Second page
\end{document}
%</example2>
%    \end{macrocode}
% Remember, in \cs{AtBeginShipoutBoxFirst} the \cs{setbox} wrapper code
% is implicitly given and the \cs{special} is used directly.
%
% \section{Method of \cs{shipout} overloading}
%
% \subsection{\cs{shipout}}
%
% The \TeX\ primitive command \cs{shipout} takes a box specification
% and puts the box as a new page in the output file.
% There are two kinds of box specifications:
% \begin{description}
% \item[Direct boxes:]
%   They are given by \cs{hbox}, \cs{vbox}, or \cs{vtop},\\
%   e.g. |\shipout\hbox{Hello World}|.
% \item[Indirect boxes:]
%   \cs{box} or \cs{copy} references a box register by number.
%   The box register contains the contents of the box.
%
%   \emph{Note:} \cs{box} also clears the box register globally.
% \end{description}
% Then we have to differentiate between void and empty boxes:
% \begin{description}
% \item[Void:]
%   Initially or after \cs{box} there is no box in the box register.
%   In this cases the box register is not empty, but \emph{void}.
% \item[Empty:]
%   A box with empty contents, such as
%   |\hbox{}| (= \cs{null}) or |\vbox{}| is an \emph{empty \cs{hbox}}
%   or \emph{empty \cs{vbox}}. If a box register holds such a
%   box, the box still exists, therefore the box register is
%   \emph{not void}.
% \end{description}
%
% \subsection{\cs{afterassignment}}
%
% We want to overload \cs{shipout} to do something with the box.
% It is quite impossible to do this reliable by catching the box
% using macro arguments. The
% variety of box specifications is too large, Examples:
% \begin{quote}
%   |\shipout\null|\\
%   |\shipout\vbox{...}|\\
%   |\shipout\vtop\bgroup ...\egroup|\\
%   |\shipout\box255|
% \end{quote}
% Even worse, the braces don't need to be balanced:
% \begin{quote}
%   |\shipout\hbox\bgroup}|\\
%   |\shipout\vbox{\egroup|
% \end{quote}
% Happily TeX provides a reliable way via \cs{afterassignment}.
% It takes a macro name and executes it just after the assignment.
%
% Now we can redefine \cs{shipout}. The box specification that
% follows \cs{shipout} is catched by \cs{setbox}. This is an
% assignment to a box register. \cs{afterassignment}
% notifies \TeX, that we want to call \cs{@test} right after
% the assignment:
%\begin{quote}
%\begin{verbatim}
%\shipout :=
%  \afterassignment\@test
%  \setbox\mybox=
%\end{verbatim}
%\end{quote}
% We have seen different box specifications. Indirect boxes are easy
% to understand:
% \begin{quote}
%   |\shipout\box0| $\Rightarrow$ |\setbox\mybox=\box0 |\fbox{\cs{@test}}
% \end{quote}
% However direct boxes can have arbitrary contents with lots of
% other assignments. It would be quite inpredictable if \TeX\ would
% put |\@test| after the first of such an assignment or after the
% box specification if the box lacks of assignments.
% Therefore \TeX\ puts |\@test| right at the beginning of the
% box specification, e.g:
% \begin{quote}
%   |\shipout\hbox{Hello World}|\\
%   $\Rightarrow$
%   |\setbox\mybox=\hbox{|\fbox{\cs{@test}}| Hello World}|
% \end{quote}
%
% \subsection{Test for direct or indirect boxes}
%
% Now we want to execute \cs{@test}, but where are we?
% We can be after the completed box assignment, if \cs{shipout}
% was called with an indirect box. Or we are right at the beginning
% of a direct box.
%
% \subsubsection{With \eTeX}
%
% With the \eTeX's extensions the answer is very easy:
% Being inside the direct box means that we are inside a new group.
% The new primitive command \cs{currentgrouplevel} tells how
% deeply the groups are currently nested. Macro \cs{@test} just
% compares the previously stored group level with the current one:
%\begin{quote}
%\begin{verbatim}
%\shipout :=
%  \edef\saved@grouplevel{\number\currentgrouplevel}
%  \afterassignment\@test
%  \setbox\mybox=
%
%\@test :=
%  \ifnum\saved@grouplevel=\currentgrouplevel
%    % case: indirect box, the assignment is completed
%    \@output
%  \else
%    % case: direct box, we are inside the box
%    \aftergroup\@outbox
%  \fi
%\end{verbatim}
%\end{quote}
%
% \subsubsection{Without \eTeX}
%
% Life becomes complicate without \eTeX. We cannot ask the
% group level. However, if we are inside a direct box,
% the box register \cs{mybox} is not yet changed by \cs{setbox}.
% Thus we need a special initial value and compare
% it in \cs{@test} with the current value of the box.
%
% What can be used as initial value? Arbitrary box contents cannot
% be compared. \TeX\ only tells us a few properties:
% \begin{itemize}
% \item Box type:\cs{ifhbox}, \cs{ifvbox}
% \item Dimensions: \cs{wd}, \cs{ht}, \cs{dp}
% \item Voidness: \cs{ifvoid}
% \end{itemize}
% Unhappily all these qualities even combined are not sufficient
% for constructing an initial box value, because \cs{shipout} can be
% called with a box that is accidently just the same as the choosen
% initial value.
%
% Nevertheless we have two alternatives for an initial value:
% \begin{itemize}
% \item A box of some type with some funny settings that are
% unlikely to occur in real life, e.g a height of |4911sp-\maxdimen|.
% \item A void box.
% \end{itemize}
% A collision between this initial value and an indirect \cs{shipout}
% box with just the same value is possible.
% Then \cs{@test} will make a wrong decision that it is executed
% inside a direct box and delays \cs{@output} by \cs{aftergroup}.
% Thus \cs{@output} is not called at the place we want.
% In contrary, the result is an uncertainty about the place:
% \begin{itemize}
% \item \cs{shipout} is used in a group that perhaps closes some
%   pages later. A bad place for \cs{@output}.
% \item Without a surrounding group \cs{aftergroup} effectively kills
%   its argument.
% \end{itemize}
% In the first case of a box with special dimensions we can even loose
% the page. However in the case of the void box, this effect is even
% desired, because the original \cs{shipout} does not output void boxes.
% All we have to do is to ensure that our box \cs{mybox} is always
% void except for the phase when the overloaded \cs{shipout} is executed.
% And secondly we must keep this semantics of \cs{shipout} for the void case
% in our macros, namely \cs{@output}.
%\begin{quote}
%\begin{verbatim}
%\shipout :=
%  % trick to get a void box \mybox
%  \begingroup
%    \setbox\mybox=\box\mybox
%  \endgroup
%  \afterassignment\@test
%  \setbox\mybox=
%
%\@test :=
%  \ifvoid\mybox
%    \aftergroup\@output
%  \else
%    \@output
%  \fi
%\end{verbatim}
%\end{quote}
% The nasty case is |\shipout\box\voidb@x| where the indirect box
% is void and that must not generate an output page.
% If a surrounding group is missing the output is ignored
% because of \cs{aftergroup}.
% Otherwise \@output is called some time
% later when the surrounding group closes.
% But \cs{mybox} is void outside the execution phase
% of the redefined \cs{shipout}. Also \cs{@output} checks for a
% void box and cancels the page output.
% The disadvantage remains that the hook in \cs{@output} is called
% for a page that will not be output.
%
% \subsubsection{\cs{lastkern} method}
%
% At the beginning of a new box, there is no \cs{kern}, the contents
% of the box is still empty and \cs{lastkern} returns 0\,pt.
% This can be used to distinguish between direct and indirect boxes:
% We execute \cs{setbox} in a box with a preceding non-zero kern.
% After an indirect box, \cs{lastkern} sees this kern, otherwise
% it returns 0\,pt.
%\begin{quote}
%\begin{verbatim}
%\shipout :=
%  \begingroup
%    \setbox\mybox=\hbox\bgroup
%      \kern1pt
%      \afterassignment\shipout@test
%      \global\setbox\mybox=
%\@test :=
%  \ifdim\lastkern=0pt
%    % direct box
%    \aftergroup\egroup
%    \aftergroup\endgroup
%    \aftergroup\@output
%  \else
%    \egroup
%    \endgroup
%    \@output
%  \fi
%\end{verbatim}
%\end{quote}
% We have two \cs{setbox} commands. The first creates a controlled
% context box where we can safely insert a \cs{kern}. We get rid of
% this temporarily used context box by putting the local \cs{setbox}
% in a group.
%
% After the group we want to have our shipout box in \cs{mybox}.
% Therefore we use a global assignment here.
%
% \subsection{Output}
%
% With or without \eTeX\ we ensure the original behaviour of
% \cs{shipout} that void boxes do not generate output pages.
%
% Now we can place the hook \cs{@hook} for the user code that
% wants to manipulate the output box.
%\begin{quote}
%\begin{verbatim}
%\@output :=
%  \ifvoid\mybox
%    % cancel output of void box
%  \else
%    \@hook
%    \ifvoid\mybox
%      % user code in \@hook could has voided the box
%    \else
%      \original@shipout\box\mybox
%    \fi
%  \fi
%\end{verbatim}
%\end{quote}
%
% \subsection{Separate box register}
%
% So far we have said nothing about the box number of \cs{mybox}.
% The following case that outputs the same page twice
% shows that we are not free in the use
% of the box register:
% \begin{quote}
%   |\shipout\copy<num> \shipout\box<num>|
% \end{quote}
% We manipulate the box by the hook and without \eTeX\ the box
% must even be voided. However, the use case above requires that
% the box contents does not change at all. Therefore we must
% reserve a separate box register to avoid collisions with
% user box registers.
%
% \emph{Note:} Box register number 255 is special for the output
% routine, because TeX complains if this box is not voided
% by the output routine.
% However, this requirement does not apply to \cs{shipout} at all.
% In fact \cs{shipout} does not change any box register. This is
% usually done by a call of \cs{box}, but the output routine
% can do it later \emph{after} invoking of \cs{shipout}.
%
% \subsection{Summary}
%
% \subsubsection{With \eTeX}
%
% Putting the pieces together we get for \eTeX:
%\begin{quote}
%\begin{verbatim}
%\newbox\mybox
%\let\original@shipout\shipout
%
%\shipout :=
%  \edef\saved@grouplevel{\number\currentgrouplevel}
%  \afterassignment\@test
%  \setbox\mybox=
%
%\@test :=
%  \ifnum\saved@grouplevel<\currentgrouplevel
%    \expandafter\aftergroup
%  \fi
%  \@output
%
%\@output :=
%  \ifvoid\mybox
%    % cancel output of void box
%  \else
%    \@hook
%    \ifvoid\mybox
%      % user code in \@hook could have voided the box
%    \else
%      \original@shipout\box\mybox
%    \fi
%  \fi
%\end{verbatim}
%\end{quote}
%
% \subsubsection{Without \eTeX, traditional way}
%
% And for \TeX\ without \eTeX:
%\begin{quote}
%\begin{verbatim}
%\newbox\mybox
%\begingroup
%   \setbox\mybox=\box\mybox % ensure \mybox is void
%\endgroup
%\let\original@shipout\shipout
%
%\shipout :=
%  % trick to get a void box \mybox
%  \begingroup
%    \setbox\mybox=\box\mybox
%  \endgroup
%  \afterassignment\@test
%  \setbox\mybox=
%
%\@test :=
%  \ifvoid\mybox
%    \expandafter\aftergroup
%  \fi
%  \@output
%
%\@output :=
%  \ifvoid\mybox
%    % cancel output of void box
%  \else
%    \@hook
%    \ifvoid\mybox
%      % user code in \@hook could have voided the box
%    \else
%      \original@shipout\box\mybox
%    \fi
%  \fi
%\end{verbatim}
%\end{quote}
%
% \subsubsection{\cs{lastkern} method}
%
% And for \TeX\ without \eTeX\ using the \cs{lastkern} method:
%\begin{quote}
%\begin{verbatim}
%\newbox\mybox
%\let\original@shipout\shipout
%
%\shipout :=
%  \begingroup
%  \setbox\mybox=\hbox\bgroup
%  \kern1pt
%  \afterassignment\@test
%  \setbox\mybox=
%
%\@test :=
%  \ifdim\lastkern=0pt
%    \expandafter\aftergroup
%  \fi
%  \@output
%
%\@output :=
%  \egroup
%  \endgroup
%  \ifvoid\mybox
%    % cancel output of void box
%  \else
%    \@hook
%    \ifvoid\mybox
%      % user code in \@hook could have voided the box
%    \else
%      \original@shipout\box\mybox
%    \fi
%  \fi
%\end{verbatim}
%\end{quote}
%
%
% \StopEventually{
% }
%
% \section{Implementation}
%
% Package \xpackage{atbegshi} uses \eTeX's \cs{currentgrouplevel},
% if it is available. Otherwise the \cs{lastkern} method is used.
%
%    \begin{macrocode}
%<*package>
%    \end{macrocode}
%
% \subsection{Reload check and package identification}
%    Reload check, especially if the package is not used with \LaTeX.
%    \begin{macrocode}
\begingroup\catcode61\catcode48\catcode32=10\relax%
  \catcode13=5 % ^^M
  \endlinechar=13 %
  \catcode35=6 % #
  \catcode39=12 % '
  \catcode44=12 % ,
  \catcode45=12 % -
  \catcode46=12 % .
  \catcode58=12 % :
  \catcode64=11 % @
  \catcode123=1 % {
  \catcode125=2 % }
  \expandafter\let\expandafter\x\csname ver@atbegshi.sty\endcsname
  \ifx\x\relax % plain-TeX, first loading
  \else
    \def\empty{}%
    \ifx\x\empty % LaTeX, first loading,
      % variable is initialized, but \ProvidesPackage not yet seen
    \else
      \expandafter\ifx\csname PackageInfo\endcsname\relax
        \def\x#1#2{%
          \immediate\write-1{Package #1 Info: #2.}%
        }%
      \else
        \def\x#1#2{\PackageInfo{#1}{#2, stopped}}%
      \fi
      \x{atbegshi}{The package is already loaded}%
      \aftergroup\endinput
    \fi
  \fi
\endgroup%
%    \end{macrocode}
%    Package identification:
%    \begin{macrocode}
\begingroup\catcode61\catcode48\catcode32=10\relax%
  \catcode13=5 % ^^M
  \endlinechar=13 %
  \catcode35=6 % #
  \catcode39=12 % '
  \catcode40=12 % (
  \catcode41=12 % )
  \catcode44=12 % ,
  \catcode45=12 % -
  \catcode46=12 % .
  \catcode47=12 % /
  \catcode58=12 % :
  \catcode64=11 % @
  \catcode91=12 % [
  \catcode93=12 % ]
  \catcode123=1 % {
  \catcode125=2 % }
  \expandafter\ifx\csname ProvidesPackage\endcsname\relax
    \def\x#1#2#3[#4]{\endgroup
      \immediate\write-1{Package: #3 #4}%
      \xdef#1{#4}%
    }%
  \else
    \def\x#1#2[#3]{\endgroup
      #2[{#3}]%
      \ifx#1\@undefined
        \xdef#1{#3}%
      \fi
      \ifx#1\relax
        \xdef#1{#3}%
      \fi
    }%
  \fi
\expandafter\x\csname ver@atbegshi.sty\endcsname
\ProvidesPackage{atbegshi}%
  [2016/06/09 v1.18 At begin shipout hook (HO)]%
%    \end{macrocode}
%
% \subsection{Catcodes}
%
%    \begin{macrocode}
\begingroup\catcode61\catcode48\catcode32=10\relax%
  \catcode13=5 % ^^M
  \endlinechar=13 %
  \catcode123=1 % {
  \catcode125=2 % }
  \catcode64=11 % @
  \def\x{\endgroup
    \expandafter\edef\csname AtBegShi@AtEnd\endcsname{%
      \endlinechar=\the\endlinechar\relax
      \catcode13=\the\catcode13\relax
      \catcode32=\the\catcode32\relax
      \catcode35=\the\catcode35\relax
      \catcode61=\the\catcode61\relax
      \catcode64=\the\catcode64\relax
      \catcode123=\the\catcode123\relax
      \catcode125=\the\catcode125\relax
    }%
  }%
\x\catcode61\catcode48\catcode32=10\relax%
\catcode13=5 % ^^M
\endlinechar=13 %
\catcode35=6 % #
\catcode64=11 % @
\catcode123=1 % {
\catcode125=2 % }
\def\TMP@EnsureCode#1#2{%
  \edef\AtBegShi@AtEnd{%
    \AtBegShi@AtEnd
    \catcode#1=\the\catcode#1\relax
  }%
  \catcode#1=#2\relax
}
\TMP@EnsureCode{40}{12}% (
\TMP@EnsureCode{41}{12}% )
\TMP@EnsureCode{44}{12}% ,
\TMP@EnsureCode{45}{12}% -
\TMP@EnsureCode{47}{12}% /
\TMP@EnsureCode{46}{12}% .
\TMP@EnsureCode{58}{12}% :
\TMP@EnsureCode{91}{12}% [
\TMP@EnsureCode{93}{12}% ]
\TMP@EnsureCode{94}{7}% ^(superscript)
\TMP@EnsureCode{96}{12}% `
\edef\AtBegShi@AtEnd{\AtBegShi@AtEnd\noexpand\endinput}
%    \end{macrocode}
%
% \subsection{Preparations}
%
%    \begin{macrocode}
\begingroup\expandafter\expandafter\expandafter\endgroup
\expandafter\ifx\csname RequirePackage\endcsname\relax
  \def\TMP@RequirePackage#1[#2]{%
    \begingroup\expandafter\expandafter\expandafter\endgroup
    \expandafter\ifx\csname ver@#1.sty\endcsname\relax
      \input #1.sty\relax
    \fi
  }%
  \TMP@RequirePackage{infwarerr}[2007/09/09]%
  \TMP@RequirePackage{ltxcmds}[2010/03/01]%
\else
  \RequirePackage{infwarerr}[2007/09/09]%
  \RequirePackage{ltxcmds}[2010/03/01]%
\fi
%    \end{macrocode}
%
%    \begin{macro}{\AtBegShi@CheckDefinable}
%    \begin{macrocode}
\begingroup\expandafter\expandafter\expandafter\endgroup
\expandafter\ifx\csname @ifdefinable\endcsname\relax
  \def\AtBegShi@CheckDefinable#1{%
    \ifcase\ifx#1\relax
             \ltx@one
           \else
             \ifx#1\@undefined
               \ltx@one
             \else
               \ltx@zero
             \fi
           \fi
      \@PackageError{atbegshi}{%
        \string#1\space is already defined%
      }\@ehd
    \fi
  }%
\else
  \def\AtBegShi@CheckDefinable#1{%
    \@ifdefinable{#1}{}%
  }%
\fi
%    \end{macrocode}
%    \end{macro}
%
%    \begin{macro}{\ifAtBegShi@Discarded}
%    \begin{macrocode}
\ltx@newif\ifAtBegShi@Discarded
%    \end{macrocode}
%    \end{macro}
%    \begin{macro}{\AtBeginShipoutDiscard}
%    \begin{macrocode}
\AtBegShi@CheckDefinable\AtBeginShipoutDiscard
\def\AtBeginShipoutDiscard{%
  \deadcycles=\ltx@zero
  \global\AtBegShi@Discardedtrue
}
%    \end{macrocode}
%    \end{macro}
%
%    \begin{macrocode}
\begingroup\expandafter\expandafter\expandafter\endgroup
\expandafter\ifx\csname currentgrouplevel\endcsname\relax
  \catcode`X=9  % ignore
  \catcode`E=14 % comment
\else
  \catcode`X=14 % comment
  \catcode`E=9  % ignore
\fi
%    \end{macrocode}
%
%    \begin{macro}{\AtBegShi@Shipout}
%    \begin{macrocode}
\def\AtBegShi@Shipout{%
X \begingroup
X \setbox\AtBeginShipoutBox=\hbox\bgroup
X \kern\p@
E \edef\AtBegShi@GroupLevel{\number\currentgrouplevel}%
  \afterassignment\AtBegShi@Test
X \global
  \setbox\AtBeginShipoutBox=%
}
%    \end{macrocode}
%    \end{macro}
%
%    \begin{macro}{\AtBegShi@Test}
%    \begin{macrocode}
\def\AtBegShi@Test{%
X \ifdim\lastkern=0pt %
E \ifnum\AtBegShi@GroupLevel<\currentgrouplevel
    \expandafter\aftergroup
  \fi
  \AtBegShi@Output
}
%    \end{macrocode}
%    \end{macro}
%
%    \begin{macro}{\AtBegShi@Output}
%    \begin{macrocode}
\def\AtBegShi@Output{%
X \egroup
X \endgroup
  \ifvoid\AtBeginShipoutBox
    \@PackageWarning{atbegshi}{Ignoring void shipout box}%
  \else
    \let\AtBegShi@OrgProtect\protect
    \csname set@typeset@protect\endcsname
    \global\AtBegShi@Discardedfalse
    \AtBegShi@Hook
    \expandafter\gdef\expandafter\AtBegShi@HookNext
    \expandafter{\expandafter}%
    \AtBegShi@HookNext
    \ifAtBegShi@Discarded
      \@PackageInfoNoLine{atbegshi}{Shipout page discarded}%
      \global\AtBegShi@Discardedfalse
      \begingroup
        \setbox\AtBeginShipoutBox\box\AtBeginShipoutBox
      \endgroup
      \let\protect\AtBegShi@OrgProtect
    \else
      \AtBegShi@First
      \let\protect\AtBegShi@OrgProtect
      \AtBegShi@GetBoxSize\AtBeginShipoutBox
      \ltx@ifundefined{AtNextShipout}{%
      }{%
        \AtNextShipout{\AtBegShi@GetBoxSize\@cclv}%
      }%
      \AtBeginShipoutOriginalShipout\box\AtBeginShipoutBox
    \fi
  \fi
}
%    \end{macrocode}
%    \end{macro}
%
%    \begin{macro}{\AtBegShi@GetBoxSize}
%    \begin{macrocode}
\def\AtBegShi@GetBoxSize#1{%
  \xdef\AtBeginShipoutBoxWidth{\the\wd#1}%
  \xdef\AtBeginShipoutBoxHeight{\the\ht#1}%
  \xdef\AtBeginShipoutBoxDepth{\the\dp#1}%
}
%    \end{macrocode}
%    \end{macro}
%    \begin{macro}{\AtBeginShipoutBoxWidth}
%    \begin{macrocode}
\def\AtBeginShipoutBoxWidth{0pt}
%    \end{macrocode}
%    \end{macro}
%    \begin{macro}{\AtBeginShipoutBoxHeight}
%    \begin{macrocode}
\def\AtBeginShipoutBoxHeight{0pt}
%    \end{macrocode}
%    \end{macro}
%    \begin{macro}{\AtBeginShipoutBoxDepth}
%    \begin{macrocode}
\def\AtBeginShipoutBoxDepth{0pt}
%    \end{macrocode}
%    \end{macro}
%
%    \begin{macrocode}
\catcode`\X=11 %
\catcode`\E=11 %
%    \end{macrocode}
%
%    \begin{macro}{\AtBegShi@First}
%    \begin{macrocode}
\def\AtBegShi@First{%
  \ifx\AtBegShi@HookFirst\ltx@empty
  \else
    \AtBeginShipoutAddToBox{\AtBegShi@HookFirst}%
  \fi
  \global\let\AtBegShi@First\ltx@empty
  \global\let\AtBeginShipoutFirst\AtBegShi@FirstDisabled
}
%    \end{macrocode}
%    \end{macro}
%
%    \begin{macro}{\AtBegShi@Hook}
%    \begin{macrocode}
\gdef\AtBegShi@Hook{}
%    \end{macrocode}
%    \end{macro}
%    \begin{macro}{\AtBegShi@HookNext}
%    \begin{macrocode}
\gdef\AtBegShi@HookNext{}
%    \end{macrocode}
%    \end{macro}
%    \begin{macro}{\AtBegShi@HookFirst}
%    \begin{macrocode}
\gdef\AtBegShi@HookFirst{}
%    \end{macrocode}
%    \end{macro}
%    \begin{macro}{\AtBeginShipout}
%    \begin{macrocode}
\AtBegShi@CheckDefinable\AtBeginShipout
\def\AtBeginShipout{%
  \AtBegShi@AddHook\AtBegShi@Hook
}
%    \end{macrocode}
%    \end{macro}
%    \begin{macro}{\AtBeginShipoutNext}
%    \begin{macrocode}
\AtBegShi@CheckDefinable\AtBeginShipoutNext
\def\AtBeginShipoutNext{%
  \AtBegShi@AddHook\AtBegShi@HookNext
}
%    \end{macrocode}
%    \end{macro}
%    \begin{macro}{\AtBeginShipoutFirst}
%    \begin{macrocode}
\AtBegShi@CheckDefinable\AtBeginShipoutFirst
\def\AtBeginShipoutFirst{%
  \AtBegShi@AddTo\AtBegShi@HookFirst
}
%    \end{macrocode}
%    \end{macro}
%    \begin{macro}{\AtBegShi@FirstDisabled}
%    \begin{macrocode}
\long\def\AtBegShi@FirstDisabled#1{%
  \@PackageWarning{atbegshi}{%
    First page is already shipped out, ignoring\MessageBreak
    \string\AtBeginShipoutFirst
  }%
}
%    \end{macrocode}
%    \end{macro}
%
%    \begin{macro}{\AtBegShi@AddTo}
%    \begin{macrocode}
\begingroup\expandafter\expandafter\expandafter\endgroup
\expandafter\ifx\csname g@addto@macro\endcsname\relax
  \long\def\AtBegShi@AddTo#1#2{%
    \begingroup
      \toks\ltx@zero\expandafter{#1#2}%
      \xdef#1{\the\toks\ltx@zero}%
    \endgroup
  }%
\else
  \let\AtBegShi@AddTo\g@addto@macro
\fi
%    \end{macrocode}
%    \end{macro}
%
%    \begin{macro}{\AtBegShi@AddHook}
%    \begin{macrocode}
\long\def\AtBegShi@AddHook#1#2{%
  \AtBegShi@AddTo#1{\AtBegShi@Item{#2}}%
}
%    \end{macrocode}
%    \end{macro}
%
%    \begin{macro}{\AtBegShi@Item}
%    \begin{macrocode}
\long\def\AtBegShi@Item#1{%
  \ifAtBegShi@Discarded
  \else
    #1%
    \ifAtBegShi@Discarded
    \else
      \ifvoid\AtBeginShipoutBox
        \@PackageWarning{atbegshi}{%
          Shipout box was voided by hook,\MessageBreak
          ignoring shipout box%
        }%
        \AtBeginShipoutDiscard
      \fi
    \fi
  \fi
}
%    \end{macrocode}
%    \end{macro}
%
%    \begin{macro}{\AtBeginShipoutInit}
%    \begin{macrocode}
\AtBegShi@CheckDefinable\AtBeginShipoutInit
\def\AtBeginShipoutInit{%
  \ltx@IfUndefined{newbox}{%
    \@PackageError{atbegshi}{%
      \string\AtBeginShipoutInit\space failed\MessageBreak
       because of missing \expandafter\string\csname newbox\endcsname
    }\@ehc
  }{%
    \csname newbox\endcsname\AtBeginShipoutBox
    \AtBegShi@CheckDefinable\AtBeginShipoutOriginalShipout
    \global\let\AtBeginShipoutOriginalShipout\shipout
    \global\let\shipout\AtBegShi@Shipout
  }%
  \gdef\AtBeginShipoutInit{}%
}
%    \end{macrocode}
%    \end{macro}
%    \begin{macrocode}
\begingroup\expandafter\expandafter\expandafter\endgroup
\expandafter\ifx\csname AtBeginDocument\endcsname\relax
  \AtBeginShipoutInit
\else
  \AtBeginDocument{\AtBeginShipoutInit}%
\fi
%    \end{macrocode}
%
% \subsection{Additions to the shipout box}
%
%    \begin{macro}{\AtBeginShipoutAddToBox}
%    \begin{macrocode}
\def\AtBeginShipoutAddToBox#1{%
  \ifhbox\AtBeginShipoutBox
    \edef\AtBegShi@restore{%
      \hfuzz=\the\hfuzz\relax
      \hbadness=\the\hbadness\relax
    }%
    \hfuzz=1073741823sp\relax
    \hbadness=2147483647\relax
    \setbox\AtBeginShipoutBox=\hbox to \wd\AtBeginShipoutBox{%
      \setbox\ltx@zero=\hbox{%
        \begingroup
          \AtBegShi@restore
          #1%
        \endgroup
      }%
      \wd\ltx@zero=0pt\relax
      \ht\ltx@zero=0pt\relax
      \dp\ltx@zero=0pt\relax
      \raise\ht\AtBeginShipoutBox\copy\ltx@zero
      \unhcopy\AtBeginShipoutBox
    }%
    \AtBegShi@restore
  \else
    \ifvbox\AtBeginShipoutBox
      \edef\AtBegShi@restore{%
        \vfuzz=\the\vfuzz\relax
        \vbadness=\the\vbadness\relax
        \dimen\ltx@zero=\the\dimen\ltx@zero\relax
      }%
      \edef\AtBegShi@restorebox{%
        \ht\AtBeginShipoutBox=\the\ht\AtBeginShipoutBox\relax
        \dp\AtBeginShipoutBox=\the\dp\AtBeginShipoutBox\relax
      }%
      \vfuzz=1073741823sp\relax
      \vbadness=2147483647\relax
      \dimen\ltx@zero=\ht\AtBeginShipoutBox
      \advance\dimen\ltx@zero by \dp\AtBeginShipoutBox
      \setbox\AtBeginShipoutBox=\vbox to \dimen\ltx@zero{%
        \setbox\ltx@zero=\hbox{%
          \begingroup
            \AtBegShi@restore
            #1%
          \endgroup
        }%
        \wd\ltx@zero=0pt\relax
        \ht\ltx@zero=0pt\relax
        \dp\ltx@zero=0pt\relax
        \baselineskip=0pt\relax
        \lineskip=0pt\relax
        \lineskiplimit=0pt\relax
        \copy\ltx@zero
        \unvbox\AtBeginShipoutBox
        \kern0pt%
      }%
      \AtBegShi@restore
      \AtBegShi@restorebox
    \fi
  \fi
}
%    \end{macrocode}
%    \end{macro}
%
%    \begin{macro}{\AtBeginShipoutAddToBoxForeground}
%    \begin{macrocode}
\def\AtBeginShipoutAddToBoxForeground#1{%
  \ifhbox\AtBeginShipoutBox
    \edef\AtBegShi@restore{%
      \hfuzz=\the\hfuzz\relax
      \hbadness=\the\hbadness\relax
    }%
    \hfuzz=1073741823sp\relax
    \hbadness=2147483647\relax
    \setbox\AtBeginShipoutBox=\hbox to \wd\AtBeginShipoutBox{%
      \unhcopy\AtBeginShipoutBox
      \kern-\wd\AtBeginShipoutBox
      \setbox\ltx@zero=\hbox{%
        \begingroup
          \AtBegShi@restore
          #1%
        \endgroup
      }%
      \wd\ltx@zero=0pt\relax
      \ht\ltx@zero=0pt\relax
      \dp\ltx@zero=0pt\relax
      \raise\ht\AtBeginShipoutBox\copy\ltx@zero
      \kern\wd\AtBeginShipoutBox
    }%
    \AtBegShi@restore
  \else
    \ifvbox\AtBeginShipoutBox
      \edef\AtBegShi@restore{%
        \vfuzz=\the\vfuzz\relax
        \vbadness=\the\vbadness\relax
        \dimen\ltx@zero=\the\dimen\ltx@zero\relax
      }%
      \edef\AtBegShi@restorebox{%
        \ht\AtBeginShipoutBox=\the\ht\AtBeginShipoutBox\relax
        \dp\AtBeginShipoutBox=\the\dp\AtBeginShipoutBox\relax
      }%
      \vfuzz=1073741823sp\relax
      \vbadness=2147483647\relax
      \dimen\ltx@zero=\ht\AtBeginShipoutBox
      \advance\dimen\ltx@zero by \dp\AtBeginShipoutBox
      \setbox\AtBeginShipoutBox=\vbox to \dimen\ltx@zero{%
        \setbox\ltx@zero=\hbox{%
          \begingroup
            \AtBegShi@restore
            #1%
          \endgroup
        }%
        \wd\ltx@zero=0pt\relax
        \ht\ltx@zero=0pt\relax
        \dp\ltx@zero=0pt\relax
        \baselineskip=0pt\relax
        \lineskip=0pt\relax
        \lineskiplimit=0pt\relax
        \unvbox\AtBeginShipoutBox
        \kern-\dimen\ltx@zero
        \copy\ltx@zero
        \kern\dimen\ltx@zero
      }%
      \AtBegShi@restore
      \AtBegShi@restorebox
    \fi
  \fi
}
%    \end{macrocode}
%    \end{macro}
%
% \subsection{Positioning}
%
%    \begin{macrocode}
\begingroup\expandafter\expandafter\expandafter\endgroup
\expandafter\ifx\csname RequirePackage\endcsname\relax
  \def\TMP@RequirePackage#1[#2]{%
    \begingroup\expandafter\expandafter\expandafter\endgroup
    \expandafter\ifx\csname ver@#1.sty\endcsname\relax
      \input #1.sty\relax
    \fi
  }%
  \TMP@RequirePackage{ifpdf}[2011/01/30]%
\else
  \RequirePackage{ifpdf}[2011/01/30]%
\fi
%    \end{macrocode}
%
%    \begin{macrocode}
\ifpdf
  \def\AtBegShi@horigin{%
    \ifx\pdfhorigin\@undefined\pdfvariable horigin\else\pdfhorigin\fi}%
  \def\AtBegShi@vorigin{%
    \ifx\pdfvorigin\@undefined\pdfvariable vorigin\else\pdfvorigin\fi}%
\else
  \def\AtBegShi@horigin{72.27pt}%
  \def\AtBegShi@vorigin{72.27pt}%
\fi
%    \end{macrocode}
%
%    \begin{macrocode}
\begingroup
\ifcase
    \expandafter\ifx\csname picture\endcsname\relax
      1%
    \else
      \expandafter\ifx\csname endpicture\endcsname\relax
        1%
      \else
        0%
      \fi
    \fi
  \endgroup
  \def\AtBegShi@BeginPicture{%
    \begingroup
    \picture(0,0)\relax
    \begingroup\expandafter\expandafter\expandafter\endgroup
    \expandafter\ifx\csname unitlength\endcsname\relax
    \else
      \unitlength=1pt\relax
    \fi
    \ignorespaces
  }%
  \def\AtBegShi@EndPicture{%
    \endpicture
    \endgroup
  }%
\else
  \endgroup
  \def\AtBegShi@BeginPicture{%
    \setbox\ltx@zero=\hbox\bgroup
    \begingroup
    \ignorespaces
  }%
  \def\AtBegShi@EndPicture{%
    \endgroup
    \egroup
    \ht\ltx@zero=0pt\relax
    \dp\ltx@zero=0pt\relax
    \copy\ltx@zero
  }%
\fi
%    \end{macrocode}
%
%    \begin{macro}{\AtBeginShipoutUpperLeft}
%    A surrounding \cs{rlap} is not necessary, because
%    the stuff is put in an \cs{hbox} with zero width.
%    \begin{macrocode}
\def\AtBeginShipoutUpperLeft#1{%
  \AtBeginShipoutAddToBox{%
    \kern-\AtBegShi@horigin\relax
    \vbox to 0pt{%
      \kern-\AtBegShi@vorigin\relax
      \AtBegShi@BeginPicture
      #1%
      \AtBegShi@EndPicture
      \vss
    }%
  }%
}
%    \end{macrocode}
%    \end{macro}
%
%    \begin{macro}{\AtBeginShipoutUpperLeftForeground}
%    \begin{macrocode}
\def\AtBeginShipoutUpperLeftForeground#1{%
  \AtBeginShipoutAddToBoxForeground{%
    \kern-\AtBegShi@horigin\relax
    \vbox to 0pt{%
      \kern-\AtBegShi@vorigin\relax
      \AtBegShi@BeginPicture
      #1%
      \AtBegShi@EndPicture
      \vss
    }%
  }%
}
%    \end{macrocode}
%    \end{macro}
%
% \subsection{Patches}
%
%    Patches for \LaTeX\ packages that redefine \cs{shipout}.
%    \LaTeX\ is now supposed to use \eTeX. Thus we do not patch,
%    without \LaTeX\ and \eTeX.
%    \begin{macrocode}
\def\AtBegShi@AbortIfUndefined#1{%
  \begingroup\expandafter\expandafter\expandafter\endgroup
  \expandafter\ifx\csname#1\endcsname\relax
    \expandafter\AtBegShi@AtEnd
  \fi
}
\AtBegShi@AbortIfUndefined{currentgrouplevel}%
\AtBegShi@AbortIfUndefined{AtBeginDocument}%
\AtBegShi@AbortIfUndefined{@ifpackageloaded}%
\AtBegShi@AbortIfUndefined{@ifclassloaded}%
%    \end{macrocode}
%
% \subsubsection{Package \xpackage{crop}}
%
% Fix of method and box.
%    \begin{macrocode}
\def\AtBegShi@PatchCrop{%
  \begingroup
    \def\AtBegShi@Crop@shipout{%
      \afterassignment\CROP@ship
      \setbox\@cclv=%
    }%
    \def\AtBegShi@Crop@ship{%
      \ifvoid\@cclv
        \expandafter\aftergroup
      \fi
      \CROP@@ship
    }%
    \def\AtBegShi@Crop@shiplist{%
      \lineskip\z@
      \lineskiplimit\z@
      \baselineskip\z@
      \CROP@kernel
      \box\@cclv
    }%
    \def\AtBegShi@Crop@@ship{%
      \CROP@shipout\vbox{%
        \CROP@shiplist
      }%
    }%
    \ifx\AtBegShi@Crop@ship\CROP@ship
      \ifx\AtBegShi@Crop@shiplist\CROP@shiplist
        \ifx\AtBegShi@Crop@@ship\CROP@@ship
          \let\AtBegShi@found\relax
          \ifx\shipout\AtBegShi@Crop@shipout
            \def\AtBegShi@found{\shipout}%
          \else\ifx\AtBeginShipoutOriginalShipout\AtBegShi@Crop@shipout
            \def\AtBegShi@found{\AtBeginShipoutOriginalShipout}%
          \else\ifx\@EveryShipout@Org@Shipout\AtBegShi@Crop@shipout
            \def\AtBegShi@found{\@EveryShipout@Org@Shipout}%
          \else\ifx\GPTorg@shipout\AtBegShi@Crop@shipout
            \def\AtBegShi@found{\GPTorg@shipout}%
          \else\ifx\THBorg@shipout\AtBegShi@Crop@shipout
            \def\AtBegShi@found{\THBorg@shipout}%
          \else\ifx\mem@oldshipout\AtBegShi@Crop@shipout
            \def\AtBegShi@found{\mem@oldshipout}%
          \fi\fi\fi\fi\fi\fi
          \ifx\AtBegShi@found\relax
          \else
            \expandafter\endgroup
            \expandafter\def\AtBegShi@found{%
              \edef\AtBegShi@GroupLevel{\number\currentgrouplevel}%
              \afterassignment\CROP@ship
              \setbox\AtBeginShipoutBox=%
            }%
            \def\CROP@ship{%
              \ifnum\AtBegShi@GroupLevel=\currentgrouplevel
              \else
                \expandafter\aftergroup
              \fi
              \CROP@@ship
            }%
            \def\CROP@shiplist{%
              \lineskip 0pt\relax
              \lineskiplimit 0pt\relax
              \baselineskip 0pt\relax
              \CROP@kernel
              \box\AtBeginShipoutBox
            }%
            \def\CROP@@ship{%
              \ifvoid\AtBeginShipoutBox
              \else
                \setbox\AtBeginShipoutBox=\vbox{%
                  \CROP@shiplist
                }%
                \AtBegShi@GetBoxSize\AtBeginShipoutBox
                \expandafter\CROP@shipout
                \expandafter\box
                \expandafter\AtBeginShipoutBox
              \fi
            }%
            \@PackageInfoNoLine{atbegshi}{Package `crop' patched}%
            \begingroup
          \fi
        \fi
      \fi
    \fi
  \endgroup
  \let\AtBegShi@PatchCrop\relax
}
\@ifpackageloaded{crop}{%
  \AtBegShi@PatchCrop
}{%
  \AtBeginDocument{\AtBegShi@PatchCrop}%
}
%    \end{macrocode}
%
% \subsubsection{Package \xpackage{everyshi}}
%
%     Fix of method. Use of box 255 is not changed.
%    \begin{macrocode}
\def\AtBegShi@PatchEveryshi{%
  \begingroup
    \long\def\AtBegShi@Everyshi@shipout{%
      \afterassignment\@EveryShipout@Test
      \global\setbox\@cclv= %
    }%
    \long\def\AtBegShi@Everyshi@Test{%
      \ifvoid\@cclv\relax
        \aftergroup\@EveryShipout@Output
      \else
        \@EveryShipout@Output
      \fi
    }%
    \ifx\AtBegShi@Everyshi@Test\@EveryShipout@Test
      \let\AtBegShi@found\relax
      \ifx\shipout\AtBegShi@Everyshi@shipout
        \def\AtBegShi@found{\shipout}%
      \else\ifx\AtBeginShipoutOriginalShipout\AtBegShi@Everyshi@shipout
        \def\AtBegShi@found{\AtBeginShipoutOriginalShipout}%
      \else\ifx\CROP@shipout\AtBegShi@Everyshi@shipout
        \def\AtBegShi@found{\CROP@shipout}%
      \else\ifx\GPTorg@shipout\AtBegShi@Everyshi@shipout
        \def\AtBegShi@found{\GPTorg@shipout}%
      \else\ifx\THBorg@shipout\AtBegShi@Everyshi@shipout
        \def\AtBegShi@found{\THBorg@shipout}%
      \else\ifx\mem@oldshipout\AtBegShi@Everyshi@shipout
        \def\AtBegShi@found{\mem@oldshipout}%
      \else
        \expandafter\ifx\csname @EveryShipout@Org@Shipout\endcsname
                        \relax
           \ifx\@EveryShipout@Shipout\AtBegShi@Everyshi@shipout
             \def\AtBegShi@found{\@EveryShipout@Shipout}%
           \fi
        \fi
      \fi\fi\fi\fi\fi\fi
      \ifx\AtBegShi@found\relax
      \else
        \expandafter\endgroup
        \expandafter\def\AtBegShi@found{%
          \edef\AtBegShi@GroupLevel{\number\currentgrouplevel}%
          \afterassignment\@EveryShipout@Test
          \setbox\AtBeginShipoutBox=%
        }%
        \def\@EveryShipout@Test{%
          \ifnum\AtBegShi@GroupLevel=\currentgrouplevel
          \else
            \expandafter\aftergroup
          \fi
          \AtBegShi@Everyshi@Output
        }%
        \def\AtBegShi@Everyshi@Output{%
          \ifvoid\AtBeginShipoutBox
          \else
            \global\setbox\ltx@cclv\box\AtBeginShipoutBox
            \expandafter\@EveryShipout@Output
          \fi
        }%
        \@PackageInfoNoLine{atbegshi}{Package `everyshi' patched}%
        \begingroup
      \fi
    \fi
  \endgroup
  \let\AtBegShi@PatchEveryshi\relax
}
\@ifpackageloaded{everyshi}{%
  \AtBegShi@PatchEveryshi
}{%
  \AtBeginDocument{\AtBegShi@PatchEveryshi}%
}
%    \end{macrocode}
%
% \subsubsection{Class \xpackage{memoir}}
%
%     Fix of method and box.
%    \begin{macrocode}
\def\AtBegShi@PatchMemoir{%
  \begingroup
    \def\AtBegShi@Memoir@shipout{%
      \afterassignment\mem@shipi
      \setbox\@cclv=%
    }%
    \def\AtBegShi@Memoir@shipi{%
      \ifvoid\@cclv
        \expandafter\aftergroup
      \fi
      \mem@shipii
    }%
    \def\AtBegShi@Memoir@shipiiA{%
      \mem@oldshipout\vbox{%
        \trimmarks
        \unvbox\@cclv
      }%
    }%
    \def\AtBegShi@Memoir@shipiiB{%
      \ifvoid\@cclv
        \mem@oldshipout\box\@cclv
      \else
        \mem@oldshipout\vbox{%
          \trimmarks
          \unvbox\@cclv
        }%
      \fi
    }%
    \def\AtBegShi@Memoir@PatchAB{%
      \ifvoid\AtBeginShipoutBox
      \else
        \setbox\AtBeginShipoutBox=\vbox{%
          \trimmarks
          \ifvbox\AtBeginShipoutBox
            \unvbox\AtBeginShipoutBox
          \else
            \box\AtBeginShipoutBox
          \fi
        }%
        \AtBegShi@GetBoxSize\AtBeginShipoutBox
        \expandafter\mem@oldshipout
        \expandafter\box
        \expandafter\AtBeginShipoutBox
      \fi
    }%
    \def\AtBegShi@Memoir@shipiiC{% 2008/08/07 v1.6180339a
      \ifvoid\@cclv
        \mem@oldshipout\box\@cclv
      \else
        \ifshowtrims
          \mem@oldshipout\vbox{\trimmarks\unvbox\@cclv}%
        \else
          \mem@oldshipout\box\@cclv
        \fi
      \fi
    }%
    \def\AtBegShi@Memoir@shipiiD{% 2011/03/06 v3.6j
      \ifvoid\@cclv
        \mem@oldshipout\box\@cclv
      \else
        \ifshowtrims
          \mem@oldshipout\vbox{%
            \trimmarks
            \nointerlineskip
            \box\@cclv
          }%
        \else
          \mem@oldshipout\box\@cclv
        \fi
      \fi
    }%
    \def\AtBegShi@Memoir@PatchCD{%
      \ifvoid\AtBeginShipoutBox
      \else
        \ifshowtrims
          \setbox\AtBeginShipoutBox=\vbox{%
            \trimmarks
            \nointerlineskip
            \box\AtBeginShipoutBox
          }%
        \fi
        \AtBegShi@GetBoxSize\AtBeginShipoutBox
        \expandafter\mem@oldshipout
        \expandafter\box
        \expandafter\AtBeginShipoutBox
      \fi
    }%
    \ifx\AtBegShi@Memoir@shipi\mem@shipi
      \let\AtBegShi@found\ltx@one
      \ifx\AtBegShi@Memoir@shipiiA\mem@shipii
        \let\AtBegShi@found\ltx@zero
        \global\let\AtBegShi@Memoir@PatchX\AtBegShi@Memoir@PatchAB
      \else\ifx\AtBegShi@Memoir@shipiiB\mem@shipii
        \let\AtBegShi@found\ltx@zero
        \global\let\AtBegShi@Memoir@PatchX\AtBegShi@Memoir@PatchAB
      \else\ifx\AtBegShi@Memoir@shipiiC\mem@shipii
        \let\AtBegShi@found\ltx@zero
        \global\let\AtBegShi@Memoir@PatchX\AtBegShi@Memoir@PatchCD
      \else\ifx\AtBegShi@Memoir@shipiiD\mem@shipii
        \let\AtBegShi@found\ltx@zero
        \global\let\AtBegShi@Memoir@PatchX\AtBegShi@Memoir@PatchCD
      \fi\fi\fi\fi
      \ifcase\AtBegShi@found
        \let\AtBegShi@found\relax
        \ifx\shipout\AtBegShi@Memoir@shipout
          \def\AtBegShi@found{\shipout}%
        \else\ifx\AtBeginShipoutOriginalShipout\AtBegShi@Memoir@shipout
          \def\AtBegShi@found{\AtBeginShipoutOriginalShipout}%
        \else\ifx\CROP@shipout\AtBegShi@Memoir@shipout
          \def\AtBegShi@found{\CROP@shipout}%
        \else\ifx\GPTorg@shipout\AtBegShi@Memoir@shipout
          \def\AtBegShi@found{\GPTorg@shipout}%
        \else\ifx\THBorg@shipout\AtBegShi@Memoir@shipout
          \def\AtBegShi@found{\THBorg@shipout}%
        \else\ifx\@EveryShipout@Org@Shipout\AtBegShi@Memoir@shipout
          \def\AtBegShi@found{\@EveryShipout@Org@Shipout}%
        \fi\fi\fi\fi\fi\fi
        \ifx\AtBegShi@found\relax
        \else
          \expandafter\endgroup
          \expandafter\def\AtBegShi@found{%
            \edef\AtBegShi@GroupLevel{\number\currentgrouplevel}%
            \afterassignment\mem@shipi
            \setbox\AtBeginShipoutBox=%
          }%
          \def\mem@shipi{%
            \ifnum\AtBegShi@GroupLevel=\currentgrouplevel
            \else
              \expandafter\aftergroup
            \fi
            \mem@shipii
          }%
          \let\mem@shipii\AtBegShi@Memoir@PatchX
          \@PackageInfoNoLine{atbegshi}{Class `memoir' patched}%
          \begingroup
        \fi
      \fi
    \fi
  \endgroup
  \let\AtBegShi@PatchMemoir\relax
}
\@ifclassloaded{memoir}{%
  \AtBegShi@PatchMemoir
}{%
  \AtBeginDocument{\AtBegShi@PatchMemoir}%
}
%    \end{macrocode}
%
%    \begin{macrocode}
\AtBegShi@AtEnd%
%</package>
%    \end{macrocode}
%
% \section{Test}
%
% \subsection{Catcode checks for loading}
%
%    \begin{macrocode}
%<*test1>
%    \end{macrocode}
%    \begin{macrocode}
\catcode`\{=1 %
\catcode`\}=2 %
\catcode`\#=6 %
\catcode`\@=11 %
\expandafter\ifx\csname count@\endcsname\relax
  \countdef\count@=255 %
\fi
\expandafter\ifx\csname @gobble\endcsname\relax
  \long\def\@gobble#1{}%
\fi
\expandafter\ifx\csname @firstofone\endcsname\relax
  \long\def\@firstofone#1{#1}%
\fi
\expandafter\ifx\csname loop\endcsname\relax
  \expandafter\@firstofone
\else
  \expandafter\@gobble
\fi
{%
  \def\loop#1\repeat{%
    \def\body{#1}%
    \iterate
  }%
  \def\iterate{%
    \body
      \let\next\iterate
    \else
      \let\next\relax
    \fi
    \next
  }%
  \let\repeat=\fi
}%
\def\RestoreCatcodes{}
\count@=0 %
\loop
  \edef\RestoreCatcodes{%
    \RestoreCatcodes
    \catcode\the\count@=\the\catcode\count@\relax
  }%
\ifnum\count@<255 %
  \advance\count@ 1 %
\repeat

\def\RangeCatcodeInvalid#1#2{%
  \count@=#1\relax
  \loop
    \catcode\count@=15 %
  \ifnum\count@<#2\relax
    \advance\count@ 1 %
  \repeat
}
\def\RangeCatcodeCheck#1#2#3{%
  \count@=#1\relax
  \loop
    \ifnum#3=\catcode\count@
    \else
      \errmessage{%
        Character \the\count@\space
        with wrong catcode \the\catcode\count@\space
        instead of \number#3%
      }%
    \fi
  \ifnum\count@<#2\relax
    \advance\count@ 1 %
  \repeat
}
\def\space{ }
\expandafter\ifx\csname LoadCommand\endcsname\relax
  \def\LoadCommand{\input atbegshi.sty\relax}%
\fi
\def\Test{%
  \RangeCatcodeInvalid{0}{47}%
  \RangeCatcodeInvalid{58}{64}%
  \RangeCatcodeInvalid{91}{96}%
  \RangeCatcodeInvalid{123}{255}%
  \catcode`\@=12 %
  \catcode`\\=0 %
  \catcode`\%=14 %
  \LoadCommand
  \RangeCatcodeCheck{0}{36}{15}%
  \RangeCatcodeCheck{37}{37}{14}%
  \RangeCatcodeCheck{38}{47}{15}%
  \RangeCatcodeCheck{48}{57}{12}%
  \RangeCatcodeCheck{58}{63}{15}%
  \RangeCatcodeCheck{64}{64}{12}%
  \RangeCatcodeCheck{65}{90}{11}%
  \RangeCatcodeCheck{91}{91}{15}%
  \RangeCatcodeCheck{92}{92}{0}%
  \RangeCatcodeCheck{93}{96}{15}%
  \RangeCatcodeCheck{97}{122}{11}%
  \RangeCatcodeCheck{123}{255}{15}%
  \RestoreCatcodes
}
\Test
\csname @@end\endcsname
\end
%    \end{macrocode}
%    \begin{macrocode}
%</test1>
%    \end{macrocode}
%
%    \begin{macrocode}
%<*test2>
\input atbegshi.sty\relax
\def\msg#{\immediate\write16}
\msg{File: atbegshi-test2.tex 2016/06/09 v1.18 Test file for plain-TeX}
\def\testmsg#1#2{%
  \msg{}%
  \msg{*** Test with box (#1), expected page output [#2]}% hash-ok
}

\newbox\voidbox
\def\void{\box\voidbox}
\begingroup
  \setbox\voidbox=\void
\endgroup

\count0=0\relax
\AtBeginShipout{%
  \global\advance\count0 by 1\relax
  \msg{* Inside \string\AtBeginShipout: [\the\count0]}%
}

\AtBeginShipoutFirst{%
  \msg{* Inside \string\AtBeginShipoutFirst}%
  Hello World%
}

\testmsg{\string\null}{1}
\shipout\null

\AtBeginShipoutFirst{%
  This is too late%
}

\testmsg{void}{}
\shipout\void

\testmsg{\string\copy255 (not void)}{2}
\setbox255\hbox{\vrule height 10bp width 10bp}
\shipout\copy255 %

\testmsg{\string\copy255 (again)}{3}
\shipout\copy255 %

\testmsg{\string\box255}{4}
\shipout\box255 %

\testmsg{\string\box255 (again)}{}
\shipout\box255 %

\testmsg{\string\hbox}{5}
\shipout\hbox{\vrule height 5bp width 20bp}

\testmsg{\string\vbox}{6}
\shipout\vbox{\hrule height 20bp width 5bp}

\testmsg{\string\null, voided by hook}{}
\def\VoidBox{%
  \begingroup
    \setbox\AtBeginShipoutBox=\box\AtBeginShipoutBox
  \endgroup
}
\AtBeginShipout{\VoidBox}
\shipout\null
\def\VoidBox{}

\msg{*** \string\begingroup}
\begingroup
  \testmsg{void}{}%
  \shipout\void
\msg{*** \string\endgroup}
\endgroup

\msg{*** \string\begingroup}
\begingroup
  \testmsg{void}{}%
  \shipout\void
  \testmsg{\string\null}{8}%
  \shipout\null
\msg{*** \string\endgroup}
\endgroup

\testmsg{output routine}{9}
Hello World
\vfill
\eject

\testmsg{\string\null\space(discarded)}{}
\AtBeginShipout{%
  \msg{* Inside \string\AtBeginShipout: DISCARD}%
  \AtBeginShipoutDiscard
}
\shipout\null

\end
%</test2>
%    \end{macrocode}
%
%    \begin{macrocode}
%<*test3>
\NeedsTeXFormat{LaTeX2e}
\ProvidesFile{atbegshi-test3.tex}[2016/06/09 v1.18 Test file for LaTeX]
\RequirePackage{color}
\pagecolor{yellow}
\documentclass[a5paper,showtrims]{memoir}
\usepackage{atbegshi}
\AtBeginShipout{%
  \setbox\AtBeginShipoutBox=\vbox{%
    \vbox to 0pt{%
      \kern-1.5in %
      \hbox to 0pt{%
        \kern-1.5in %
        \color{blue}%
        \rule{1in}{1in}%
        \hss
      }%
      \vss
    }%
    \hrule
    \hbox{\vrule\box\AtBeginShipoutBox\vrule}%
    \hrule
  }%
}
\usepackage{eso-pic}
\makeatletter
\@ifundefined{@EveryShipout@Init}{%
  \typeout{Test skipped}%
  \@@end
}{}
\@EveryShipout@Init
\let\@EveryShipout@Init\relax
\makeatother
\AddToShipoutPicture{%
  \hspace{.52\paperwidth}%
  \colorbox{cyan}{%
    \rule{0mm}{\paperheight}%
    \hspace{.48\paperwidth}%
  }%
}
%    \end{macrocode}
%    Newer versions of class \xclass{memoir} emulate package
%    \xpackage{crop} and prevents its loading. This is undone
%    in next line for this test file.
%    \begin{macrocode}
\expandafter\let\csname ver@crop.sty\endcsname\relax
\usepackage[color=red,cross,a4,center]{crop}
\begin{document}
\shipout\null
\shipout\box\csname voidb@x\endcsname
\section{Hello World}
\end{document}
%</test3>
%    \end{macrocode}
%
% \section{Installation}
%
% \subsection{Download}
%
% \paragraph{Package.} This package is available on
% CTAN\footnote{\url{http://ctan.org/pkg/atbegshi}}:
% \begin{description}
% \item[\CTAN{macros/latex/contrib/oberdiek/atbegshi.dtx}] The source file.
% \item[\CTAN{macros/latex/contrib/oberdiek/atbegshi.pdf}] Documentation.
% \end{description}
%
%
% \paragraph{Bundle.} All the packages of the bundle `oberdiek'
% are also available in a TDS compliant ZIP archive. There
% the packages are already unpacked and the documentation files
% are generated. The files and directories obey the TDS standard.
% \begin{description}
% \item[\CTAN{install/macros/latex/contrib/oberdiek.tds.zip}]
% \end{description}
% \emph{TDS} refers to the standard ``A Directory Structure
% for \TeX\ Files'' (\CTAN{tds/tds.pdf}). Directories
% with \xfile{texmf} in their name are usually organized this way.
%
% \subsection{Bundle installation}
%
% \paragraph{Unpacking.} Unpack the \xfile{oberdiek.tds.zip} in the
% TDS tree (also known as \xfile{texmf} tree) of your choice.
% Example (linux):
% \begin{quote}
%   |unzip oberdiek.tds.zip -d ~/texmf|
% \end{quote}
%
% \paragraph{Script installation.}
% Check the directory \xfile{TDS:scripts/oberdiek/} for
% scripts that need further installation steps.
% Package \xpackage{attachfile2} comes with the Perl script
% \xfile{pdfatfi.pl} that should be installed in such a way
% that it can be called as \texttt{pdfatfi}.
% Example (linux):
% \begin{quote}
%   |chmod +x scripts/oberdiek/pdfatfi.pl|\\
%   |cp scripts/oberdiek/pdfatfi.pl /usr/local/bin/|
% \end{quote}
%
% \subsection{Package installation}
%
% \paragraph{Unpacking.} The \xfile{.dtx} file is a self-extracting
% \docstrip\ archive. The files are extracted by running the
% \xfile{.dtx} through \plainTeX:
% \begin{quote}
%   \verb|tex atbegshi.dtx|
% \end{quote}
%
% \paragraph{TDS.} Now the different files must be moved into
% the different directories in your installation TDS tree
% (also known as \xfile{texmf} tree):
% \begin{quote}
% \def\t{^^A
% \begin{tabular}{@{}>{\ttfamily}l@{ $\rightarrow$ }>{\ttfamily}l@{}}
%   atbegshi.sty & tex/generic/oberdiek/atbegshi.sty\\
%   atbegshi.pdf & doc/latex/oberdiek/atbegshi.pdf\\
%   atbegshi-example1.tex & doc/latex/oberdiek/atbegshi-example1.tex\\
%   atbegshi-example2.tex & doc/latex/oberdiek/atbegshi-example2.tex\\
%   test/atbegshi-test1.tex & doc/latex/oberdiek/test/atbegshi-test1.tex\\
%   test/atbegshi-test2.tex & doc/latex/oberdiek/test/atbegshi-test2.tex\\
%   test/atbegshi-test3.tex & doc/latex/oberdiek/test/atbegshi-test3.tex\\
%   atbegshi.dtx & source/latex/oberdiek/atbegshi.dtx\\
% \end{tabular}^^A
% }^^A
% \sbox0{\t}^^A
% \ifdim\wd0>\linewidth
%   \begingroup
%     \advance\linewidth by\leftmargin
%     \advance\linewidth by\rightmargin
%   \edef\x{\endgroup
%     \def\noexpand\lw{\the\linewidth}^^A
%   }\x
%   \def\lwbox{^^A
%     \leavevmode
%     \hbox to \linewidth{^^A
%       \kern-\leftmargin\relax
%       \hss
%       \usebox0
%       \hss
%       \kern-\rightmargin\relax
%     }^^A
%   }^^A
%   \ifdim\wd0>\lw
%     \sbox0{\small\t}^^A
%     \ifdim\wd0>\linewidth
%       \ifdim\wd0>\lw
%         \sbox0{\footnotesize\t}^^A
%         \ifdim\wd0>\linewidth
%           \ifdim\wd0>\lw
%             \sbox0{\scriptsize\t}^^A
%             \ifdim\wd0>\linewidth
%               \ifdim\wd0>\lw
%                 \sbox0{\tiny\t}^^A
%                 \ifdim\wd0>\linewidth
%                   \lwbox
%                 \else
%                   \usebox0
%                 \fi
%               \else
%                 \lwbox
%               \fi
%             \else
%               \usebox0
%             \fi
%           \else
%             \lwbox
%           \fi
%         \else
%           \usebox0
%         \fi
%       \else
%         \lwbox
%       \fi
%     \else
%       \usebox0
%     \fi
%   \else
%     \lwbox
%   \fi
% \else
%   \usebox0
% \fi
% \end{quote}
% If you have a \xfile{docstrip.cfg} that configures and enables \docstrip's
% TDS installing feature, then some files can already be in the right
% place, see the documentation of \docstrip.
%
% \subsection{Refresh file name databases}
%
% If your \TeX~distribution
% (\teTeX, \mikTeX, \dots) relies on file name databases, you must refresh
% these. For example, \teTeX\ users run \verb|texhash| or
% \verb|mktexlsr|.
%
% \subsection{Some details for the interested}
%
% \paragraph{Attached source.}
%
% The PDF documentation on CTAN also includes the
% \xfile{.dtx} source file. It can be extracted by
% AcrobatReader 6 or higher. Another option is \textsf{pdftk},
% e.g. unpack the file into the current directory:
% \begin{quote}
%   \verb|pdftk atbegshi.pdf unpack_files output .|
% \end{quote}
%
% \paragraph{Unpacking with \LaTeX.}
% The \xfile{.dtx} chooses its action depending on the format:
% \begin{description}
% \item[\plainTeX:] Run \docstrip\ and extract the files.
% \item[\LaTeX:] Generate the documentation.
% \end{description}
% If you insist on using \LaTeX\ for \docstrip\ (really,
% \docstrip\ does not need \LaTeX), then inform the autodetect routine
% about your intention:
% \begin{quote}
%   \verb|latex \let\install=y\input{atbegshi.dtx}|
% \end{quote}
% Do not forget to quote the argument according to the demands
% of your shell.
%
% \paragraph{Generating the documentation.}
% You can use both the \xfile{.dtx} or the \xfile{.drv} to generate
% the documentation. The process can be configured by the
% configuration file \xfile{ltxdoc.cfg}. For instance, put this
% line into this file, if you want to have A4 as paper format:
% \begin{quote}
%   \verb|\PassOptionsToClass{a4paper}{article}|
% \end{quote}
% An example follows how to generate the
% documentation with pdf\LaTeX:
% \begin{quote}
%\begin{verbatim}
%pdflatex atbegshi.dtx
%makeindex -s gind.ist atbegshi.idx
%pdflatex atbegshi.dtx
%makeindex -s gind.ist atbegshi.idx
%pdflatex atbegshi.dtx
%\end{verbatim}
% \end{quote}
%
% \section{Catalogue}
%
% The following XML file can be used as source for the
% \href{http://mirror.ctan.org/help/Catalogue/catalogue.html}{\TeX\ Catalogue}.
% The elements \texttt{caption} and \texttt{description} are imported
% from the original XML file from the Catalogue.
% The name of the XML file in the Catalogue is \xfile{atbegshi.xml}.
%    \begin{macrocode}
%<*catalogue>
<?xml version='1.0' encoding='us-ascii'?>
<!DOCTYPE entry SYSTEM 'catalogue.dtd'>
<entry datestamp='$Date$' modifier='$Author$' id='atbegshi'>
  <name>atbegshi</name>
  <caption>Execute stuff at \shipout time.</caption>
  <authorref id='auth:oberdiek'/>
  <copyright owner='Heiko Oberdiek' year='2007-2011'/>
  <license type='lppl1.3'/>
  <version number='1.18'/>
  <description>
    This package is a modern reimplementation of package
    <xref refid='everyshi'>everyshi</xref>, providing various commands
    to be executed before a <tt>\shipout</tt> command.  It makes use of
    e-TeX&#x2019;s facilities if they are available.  The package may
    be used either with LaTeX or with plain TeX.
    <p/>
    The package is part of the <xref refid='oberdiek'>oberdiek</xref> bundle.
  </description>
  <documentation details='Package documentation'
      href='ctan:/macros/latex/contrib/oberdiek/atbegshi.pdf'/>
  <ctan file='true' path='/macros/latex/contrib/oberdiek/atbegshi.dtx'/>
  <miktex location='oberdiek'/>
  <texlive location='oberdiek'/>
  <install path='/macros/latex/contrib/oberdiek/oberdiek.tds.zip'/>
</entry>
%</catalogue>
%    \end{macrocode}
%
% \begin{History}
%   \begin{Version}{2007/04/17 v1.0}
%   \item
%     First version.
%   \end{Version}
%   \begin{Version}{2007/04/18 v1.1}
%   \item
%     New method based on \cs{lastkern} is used if \eTeX\ is missing.
%   \item
%     \cs{AtBeginShipoutDiscard} also resets \cs{deadcycles}.
%   \end{Version}
%   \begin{Version}{2007/04/19 v1.2}
%   \item
%     \cs{AtBeginShipoutEarly} removed for simplification reasons.
%   \item
%     Forgotten definition of \cs{AtBegShi@Info} added.
%   \item
%     Patches for packages \xpackage{crop} and \xpackage{everyshi}
%     and class \xclass{memoir} added.
%   \end{Version}
%   \begin{Version}{2007/04/26 v1.3}
%   \item
%     Use of package \xpackage{infwarerr}.
%   \item
%     Catcode section after generic header.
%   \end{Version}
%   \begin{Version}{2007/04/27 v1.4}
%   \item
%     Small optimizations.
%   \end{Version}
%   \begin{Version}{2007/06/06 v1.5}
%   \item
%     \cs{AtBeginShipoutUpperLeft} added.
%   \item
%     Example added.
%   \item
%     Fix in second test file for newer version of \xclass{memoir}.
%   \end{Version}
%   \begin{Version}{2007/09/09 v1.6}
%   \item
%     Catcode section rewritten.
%   \end{Version}
%   \begin{Version}{2008/07/18 v1.7}
%   \item
%     Documentation of \cs{AtBeginShipoutUpperLeft} fixed and extended.
%   \end{Version}
%   \begin{Version}{2008/07/19 v1.8}
%   \item
%     \cs{AtBeginShipoutUpperLeftForeground} added.
%   \end{Version}
%   \begin{Version}{2008/07/31 v1.9}
%   \item
%     Second example (TrimBox for dvipdfmx) added.
%   \item
%     No changes in package code.
%   \end{Version}
%   \begin{Version}{2009/12/02 v1.10}
%   \item
%     \cs{AtBeginShipoutOriginalShipout} added.
%   \item
%     Test file fixed.
%   \end{Version}
%   \begin{Version}{2010/03/01 v1.11}
%   \item
%     Compatibility with \iniTeX\ except for \cs{newbox}.
%   \end{Version}
%   \begin{Version}{2010/03/25 v1.12}
%   \item
%     \cs{AtBeginShipoutNext} can now be used inside
%     \cs{AtBeginShipoutNext}.
%   \end{Version}
%   \begin{Version}{2010/08/18 v1.13}
%   \item
%     Fixes for \cs{AtBegShi@CheckDefinable}.
%   \end{Version}
%   \begin{Version}{2010/12/02 v1.14}
%   \item
%     Remove the warning because of void box if the hook
%     calls \AtBeginShipoutDiscard.
%   \end{Version}
%   \begin{Version}{2011/01/30 v1.15}
%   \item
%     Already loaded package files are not input in \hologo{plainTeX}.
%   \end{Version}
%   \begin{Version}{2011/10/05 v1.16}
%   \item
%     \cs{AtBeginShipoutAddToBox}, \cs{AtBeginShipoutAddToBoxForeground} added.
%   \item
%     \cs{AtBeginShipoutBoxWidth}, \cs{AtBeginShipoutBoxHeight},
%     \cs{AtBeginShipoutBoxDepth} added.
%   \item
%     Updates for patches of class \xclass{memoir}.
%   \end{Version}
%   \begin{Version}{2016/05/16 v1.17}
%   \item
%     Documentation updates.
%   \end{Version}
%   \begin{Version}{2016/06/09 v1.18}
%   \item
%     Update for \cs{pdfhorign} in new LuaTeX.
%   \end{Version}
% \end{History}
%
% \PrintIndex
%
% \Finale
\endinput

%        (quote the arguments according to the demands of your shell)
%
% Documentation:
%    (a) If atbegshi.drv is present:
%           latex atbegshi.drv
%    (b) Without atbegshi.drv:
%           latex atbegshi.dtx; ...
%    The class ltxdoc loads the configuration file ltxdoc.cfg
%    if available. Here you can specify further options, e.g.
%    use A4 as paper format:
%       \PassOptionsToClass{a4paper}{article}
%
%    Programm calls to get the documentation (example):
%       pdflatex atbegshi.dtx
%       makeindex -s gind.ist atbegshi.idx
%       pdflatex atbegshi.dtx
%       makeindex -s gind.ist atbegshi.idx
%       pdflatex atbegshi.dtx
%
% Installation:
%    TDS:tex/generic/oberdiek/atbegshi.sty
%    TDS:doc/latex/oberdiek/atbegshi.pdf
%    TDS:doc/latex/oberdiek/atbegshi-example1.tex
%    TDS:doc/latex/oberdiek/atbegshi-example2.tex
%    TDS:doc/latex/oberdiek/test/atbegshi-test1.tex
%    TDS:doc/latex/oberdiek/test/atbegshi-test2.tex
%    TDS:doc/latex/oberdiek/test/atbegshi-test3.tex
%    TDS:source/latex/oberdiek/atbegshi.dtx
%
%<*ignore>
\begingroup
  \catcode123=1 %
  \catcode125=2 %
  \def\x{LaTeX2e}%
\expandafter\endgroup
\ifcase 0\ifx\install y1\fi\expandafter
         \ifx\csname processbatchFile\endcsname\relax\else1\fi
         \ifx\fmtname\x\else 1\fi\relax
\else\csname fi\endcsname
%</ignore>
%<*install>
\input docstrip.tex
\Msg{************************************************************************}
\Msg{* Installation}
\Msg{* Package: atbegshi 2016/06/09 v1.18 At begin shipout hook (HO)}
\Msg{************************************************************************}

\keepsilent
\askforoverwritefalse

\let\MetaPrefix\relax
\preamble

This is a generated file.

Project: atbegshi
Version: 2016/06/09 v1.18

Copyright (C) 2007-2011 by
   Heiko Oberdiek <heiko.oberdiek at googlemail.com>

This work may be distributed and/or modified under the
conditions of the LaTeX Project Public License, either
version 1.3c of this license or (at your option) any later
version. This version of this license is in
   http://www.latex-project.org/lppl/lppl-1-3c.txt
and the latest version of this license is in
   http://www.latex-project.org/lppl.txt
and version 1.3 or later is part of all distributions of
LaTeX version 2005/12/01 or later.

This work has the LPPL maintenance status "maintained".

This Current Maintainer of this work is Heiko Oberdiek.

The Base Interpreter refers to any `TeX-Format',
because some files are installed in TDS:tex/generic//.

This work consists of the main source file atbegshi.dtx
and the derived files
   atbegshi.sty, atbegshi.pdf, atbegshi.ins, atbegshi.drv,
   atbegshi-example1.tex, atbegshi-example2.tex,
   atbegshi-test1.tex, atbegshi-test2.tex, atbegshi-test3.tex.

\endpreamble
\let\MetaPrefix\DoubleperCent

\generate{%
  \file{atbegshi.ins}{\from{atbegshi.dtx}{install}}%
  \file{atbegshi.drv}{\from{atbegshi.dtx}{driver}}%
  \usedir{tex/generic/oberdiek}%
  \file{atbegshi.sty}{\from{atbegshi.dtx}{package}}%
  \usedir{doc/latex/oberdiek}%
  \file{atbegshi-example1.tex}{\from{atbegshi.dtx}{example1}}%
  \file{atbegshi-example2.tex}{\from{atbegshi.dtx}{example2}}%
  \usedir{doc/latex/oberdiek/test}%
  \file{atbegshi-test1.tex}{\from{atbegshi.dtx}{test1}}%
  \file{atbegshi-test2.tex}{\from{atbegshi.dtx}{test2}}%
  \file{atbegshi-test3.tex}{\from{atbegshi.dtx}{test3}}%
  \nopreamble
  \nopostamble
  \usedir{source/latex/oberdiek/catalogue}%
  \file{atbegshi.xml}{\from{atbegshi.dtx}{catalogue}}%
}

\catcode32=13\relax% active space
\let =\space%
\Msg{************************************************************************}
\Msg{*}
\Msg{* To finish the installation you have to move the following}
\Msg{* file into a directory searched by TeX:}
\Msg{*}
\Msg{*     atbegshi.sty}
\Msg{*}
\Msg{* To produce the documentation run the file `atbegshi.drv'}
\Msg{* through LaTeX.}
\Msg{*}
\Msg{* Happy TeXing!}
\Msg{*}
\Msg{************************************************************************}

\endbatchfile
%</install>
%<*ignore>
\fi
%</ignore>
%<*driver>
\NeedsTeXFormat{LaTeX2e}
\ProvidesFile{atbegshi.drv}%
  [2016/06/09 v1.18 At begin shipout hook (HO)]%
\documentclass{ltxdoc}
\usepackage{holtxdoc}[2011/11/22]
\begin{document}
  \DocInput{atbegshi.dtx}%
\end{document}
%</driver>
% \fi
%
%
% \CharacterTable
%  {Upper-case    \A\B\C\D\E\F\G\H\I\J\K\L\M\N\O\P\Q\R\S\T\U\V\W\X\Y\Z
%   Lower-case    \a\b\c\d\e\f\g\h\i\j\k\l\m\n\o\p\q\r\s\t\u\v\w\x\y\z
%   Digits        \0\1\2\3\4\5\6\7\8\9
%   Exclamation   \!     Double quote  \"     Hash (number) \#
%   Dollar        \$     Percent       \%     Ampersand     \&
%   Acute accent  \'     Left paren    \(     Right paren   \)
%   Asterisk      \*     Plus          \+     Comma         \,
%   Minus         \-     Point         \.     Solidus       \/
%   Colon         \:     Semicolon     \;     Less than     \<
%   Equals        \=     Greater than  \>     Question mark \?
%   Commercial at \@     Left bracket  \[     Backslash     \\
%   Right bracket \]     Circumflex    \^     Underscore    \_
%   Grave accent  \`     Left brace    \{     Vertical bar  \|
%   Right brace   \}     Tilde         \~}
%
% \GetFileInfo{atbegshi.drv}
%
% \title{The \xpackage{atbegshi} package}
% \date{2016/06/09 v1.18}
% \author{Heiko Oberdiek\thanks
% {Please report any issues at https://github.com/ho-tex/oberdiek/issues}\\
% \xemail{heiko.oberdiek at googlemail.com}}
%
% \maketitle
%
% \begin{abstract}
% This package is a modern reimplementation of package \xpackage{everyshi}
% without the burden of compatibility. It makes use of \eTeX's if available.
% Both \LaTeX\ and \plainTeX\ are supported.
% \end{abstract}
%
% \tableofcontents
%
% \section{Documentation}
%
% Package \xpackage{atbegshi} redefines \cs{shipout} to insert
% hooks for user code that is executed before the page is shipped out.
% The code may modify or even discard the output page.
% Three hooks are implemented:
% \begin{enumerate}
% \item A hook that is executed for every page, see\\
%       \cs{AtBeginShipout}
% \item A hook that is executed for the next page only, see\\
%       \cs{AtBeginShipoutNext}
% \item A hook that is only executed for the first page, see\\
%       \cs{AtBeginShipoutFirst}
% \end{enumerate}
% The hooks are executed in this order. The following three macros
% provide the user interface for adding code to these hooks:
%
% \begin{declcs}{AtBeginShipout} \M{code}\\
%   \cs{AtBeginShipoutBox}
% \end{declcs}
% Execute the \meta{code} for every page.
% The page contents is held in box register \cs{AtBeginShipoutBox}
% and may be modified. Use \cs{AtBeginShipoutDiscard} if you want to
% discard the page.
%
% \emph{Note:} Package \xpackage{everyshi} uses box register 255. With
% package \xpackage{atbegshi} you must use \cs{AtBeginShipoutBox} instead.
%
% If \LaTeX\ calls \cs{shipout} in \cs{@outputpage} (part of its output
% routine),
% the meaning of \cs{protect} is \cs{noexpand}. \LaTeX\ sets
% \cs{protect} to the appropriate \cs{@typeset@protect} in the box
% that is shipped out. This is too late for the hooks, they
% are called earlier in the redefined \cs{shipout}. Therefore
% package \xpackage{atbegshi} sets \cs{protect} to \cs{@typeset@protect}
% before it calls the hooks.
% (In \cs{EveryShipout} of package \xpackage{everyshi} the
% user is responsible for the correct setting of \cs{protect}.)
%
% \begin{declcs}{AtBeginShipoutNext} \M{code}
% \end{declcs}
% This reimplements package \xpackage{everyshi}'s \cs{AtNextShipout}.
% The \meta{code} is executed at shipout time of the next page only.
% It is just a convenience macro, it can be easily replaced by something
% like:
%\begin{quote}
%\begin{verbatim}
%\newcommand{\MyShipoutHook}{}%
%\AtBeginShipout{\MyShipoutHook}
%\gdef\MyShipoutHook{%
%  ... do something with next page ...
%  \gdef\MyShipoutHook{}%
% }
%\end{verbatim}
%\end{quote}
% (This can be necessary, if hook order does matter).
%
% \begin{declcs}{AtBeginShipoutFirst} \M{code}
% \end{declcs}
% This reimplements \LaTeX's \cs{AtBeginDvi}. This hook is
% usually used for \cs{special} commands that include PostScript
% header files. The \cs{code} is directly executed in a \cs{vbox}
% that is put at the beginning of the output page. Dealing with
% the output box \cs{AtBeginShipoutBox} is not necessary and
% not permitted here.
%
% \begin{declcs}{AtBeginShipoutDiscard}
% \end{declcs}
% This macro notifies package \xpackage{atbegshi} that
% the output page is discarded. The remaining hook code
% and the remaining hooks are not executed and the page is thrown away.
% Also \cs{deadcycles} is cleared to zero like an ordinary \cs{shipout}
% would do.
%
% \begin{declcs}{AtBeginShipoutInit}
% \end{declcs}
% Usually the redefinition of \cs{shipout} is delayed by
% \cs{AtBeginDocument} (if this macro exists). This can be
% too late, if other packages also redefines \cs{shipout}
% and the order does matter.
% \cs{AtBeginShipoutInit} forces the immediate redefinition
% of \cs{shipout}.
%
% \begin{declcs}{AtBeginShipoutAddToBox} \M{stuff}\\
%   \cs{AtBeginShipoutAddToBoxForeground} \M{stuff}
% \end{declcs}
% A quite common use case is the addition of \cs{special}
% or other whatsits to the page output box. Macro
% \cs{AtBeginShipoutAddToBox} puts \meta{stuff} in a
% box with zeroed dimensions. The box with the \meta{stuff}
% is put in the upper left corner of the shipout box
% \cs{AtBeginShipoutBox}. Macro \cs{AtBeginShipoutAddToBox}
% puts the \meta{stuff} in the background, the other macro
% \cs{AtBeginShipoutAddToBoxForeground} in the foreground
% after the original shipout box contents is set.
%
% A void shipout box (that means a discarded page)
% remains void that means \meta{stuff} is ignored in this case.
% The box type of \cs{AtBeginShipoutBox} is preserved.
% Also the box nesting level for the original contents
% of \cs{AtBeginShipoutBox} remains, for example, to avoid trouble
% with links across pages in case of \hologo{pdfTeX}.
%
% \begin{declcs}{AtBeginShipoutUpperLeft} \M{background material}
% \end{declcs}
% This is a macro that puts material in the background of
% box \cs{AtBeginShipoutBox}.
% The \meta{background material} is set in an \cs{hbox}, the
% reference point is the upper left corner of the output page.
% In case of \pdfTeX\ in PDF mode, the settings of
% \cs{pdfhorigin} and \cs{pdfvorigin} are respected.
%
% The macro \cs{AtBeginShipoutUpperLeft} is intended to be used
% in one of the hook setting macros, such as
% \cs{AtBeginShipout}, \cs{AtBeginShipoutFirst}, or
% \cs{AtBeginShipoutNext}.
%
% For \LaTeX\ users the \meta{background material} is set
% inside a \texttt{picture} environment:
% \begin{quote}
%   |\begin{picture}(0,0)|\\
%   \mbox{}\quad |\setlength{\unitlength}{1pt}%|\\
%   \mbox{}\quad \meta{background material}\\
%   |\end{picture}|
% \end{quote}
%
% \begin{declcs}{AtBeginShipoutUpperLeftForeground} \M{foreground material}
% \end{declcs}
% See \cs{AtBeginShipoutUpperLeft}. The difference is that the material
% is put in the foreground.
%
% \begin{declcs}{AtBeginShipoutOriginalShipout} \meta{box}
% \end{declcs}
% It stores the meaning of \cs{shipout} at the time this package
% is loaded.
%
% \begin{declcs}{AtBeginShipoutBoxWidth}\\
%   \cs{AtBeginShipoutBoxHeight}\\
%   \cs{AtBeginShipoutBoxDepth}\\
% \end{declcs}
% These macros store the dimensions of the output box \cs{AtBeginShipoutBox}
% before the original shipout is called. If \cs{shipout} is not
% redefined before the package loading or the box dimensions
% are not changed by the redefined \cs{shipout}, these macros
% contain the dimensions of the shipout box. These values can be
% remembered by \cs{label} and \cs{ref}. For example, this is done
% by the package module \xpackage{zref-pagelayout} of project \xpackage{zref}.
% The dimensions of the shipout page can be used in some \hologo{TeX}
% engines (\hologo{pdfTeX} in PDF mode, \hologo{XeTeX}) to calculate
% the media size of the shipout page if \cs{pdfpagewidth} and
% \cs{pdfpageheight} are not set.
%
% \subsection{Examples}
%
% \subsubsection{Example: circle in background}
%
% In this example we put a circle in the background in the middle
% of the paper.
%
%    \begin{macrocode}
%<*example1>
%    \end{macrocode}
%    \begin{macrocode}
\documentclass[a4paper]{article}
\usepackage{color}
\usepackage{atbegshi}
%    \end{macrocode}
%    Package \xpackage{picture} makes life a little easier, because
%    we can now also use length specifications in |picture|'s commands.
%    \begin{macrocode}
\usepackage{picture}
%    \end{macrocode}
%    Now we draw the circle in the middle of the paper.
%    \cs{put} moves downwards, because the origin is at the top
%    of the page, not at its bottom.
%    \begin{macrocode}
\AtBeginShipout{%
  \AtBeginShipoutUpperLeft{%
    \put(0.5\paperwidth,-0.5\paperheight){\circle{10}}%
  }%
}
\begin{document}
\section{Hello World}
\newpage
\AtBeginShipoutNext{%
  \AtBeginShipoutUpperLeft{%
    \color{red}%
    \put(0,-0.5\paperheight){\line(1,0){\paperwidth}}%
    \put(0.5\paperwidth, 0){\line(0,-1){\paperheight}}%
  }%
}
Only on this page we add a red cross.
\newpage
This page has the circle only.
\par
\vspace{\fill}
The next page will be discarded.
\newpage
\AtBeginShipoutNext{%
  \AtBeginShipoutDiscard
}
This page is discarded.
\newpage
The last page.
\end{document}
%    \end{macrocode}
%    \begin{macrocode}
%</example1>
%    \end{macrocode}
%
% \subsubsection{Example: adding TrimBox for dvipdfmx}
%
% Now an example from ``real life'' follows. Someone from the mailing  list
% for dvipdfmx wants to put a TrimBox on every page. If we use
% \cs{AtBeginShipout}, we have to put the \cs{special} inside the
% box \cs{AtBeginShipoutBox} that gets shipped out.
%    \begin{macrocode}
%<*example2>
\documentclass{minimal}
\usepackage{atbegshi}
\usepackage[
  dvipdfm,
  paperwidth=630bp,
  paperheight=810bp
]{geometry}
\AtBeginShipout{%
  \setbox\AtBeginShipoutBox=\hbox{%
    \special{pdf: put @thispage <</TrimBox[9 9 621 801]>>}%
    \box\AtBeginShipoutBox
  }%
}
\begin{document}
  First page
  \newpage
  Second page
\end{document}
%</example2>
%    \end{macrocode}
% Remember, in \cs{AtBeginShipoutBoxFirst} the \cs{setbox} wrapper code
% is implicitly given and the \cs{special} is used directly.
%
% \section{Method of \cs{shipout} overloading}
%
% \subsection{\cs{shipout}}
%
% The \TeX\ primitive command \cs{shipout} takes a box specification
% and puts the box as a new page in the output file.
% There are two kinds of box specifications:
% \begin{description}
% \item[Direct boxes:]
%   They are given by \cs{hbox}, \cs{vbox}, or \cs{vtop},\\
%   e.g. |\shipout\hbox{Hello World}|.
% \item[Indirect boxes:]
%   \cs{box} or \cs{copy} references a box register by number.
%   The box register contains the contents of the box.
%
%   \emph{Note:} \cs{box} also clears the box register globally.
% \end{description}
% Then we have to differentiate between void and empty boxes:
% \begin{description}
% \item[Void:]
%   Initially or after \cs{box} there is no box in the box register.
%   In this cases the box register is not empty, but \emph{void}.
% \item[Empty:]
%   A box with empty contents, such as
%   |\hbox{}| (= \cs{null}) or |\vbox{}| is an \emph{empty \cs{hbox}}
%   or \emph{empty \cs{vbox}}. If a box register holds such a
%   box, the box still exists, therefore the box register is
%   \emph{not void}.
% \end{description}
%
% \subsection{\cs{afterassignment}}
%
% We want to overload \cs{shipout} to do something with the box.
% It is quite impossible to do this reliable by catching the box
% using macro arguments. The
% variety of box specifications is too large, Examples:
% \begin{quote}
%   |\shipout\null|\\
%   |\shipout\vbox{...}|\\
%   |\shipout\vtop\bgroup ...\egroup|\\
%   |\shipout\box255|
% \end{quote}
% Even worse, the braces don't need to be balanced:
% \begin{quote}
%   |\shipout\hbox\bgroup}|\\
%   |\shipout\vbox{\egroup|
% \end{quote}
% Happily TeX provides a reliable way via \cs{afterassignment}.
% It takes a macro name and executes it just after the assignment.
%
% Now we can redefine \cs{shipout}. The box specification that
% follows \cs{shipout} is catched by \cs{setbox}. This is an
% assignment to a box register. \cs{afterassignment}
% notifies \TeX, that we want to call \cs{@test} right after
% the assignment:
%\begin{quote}
%\begin{verbatim}
%\shipout :=
%  \afterassignment\@test
%  \setbox\mybox=
%\end{verbatim}
%\end{quote}
% We have seen different box specifications. Indirect boxes are easy
% to understand:
% \begin{quote}
%   |\shipout\box0| $\Rightarrow$ |\setbox\mybox=\box0 |\fbox{\cs{@test}}
% \end{quote}
% However direct boxes can have arbitrary contents with lots of
% other assignments. It would be quite inpredictable if \TeX\ would
% put |\@test| after the first of such an assignment or after the
% box specification if the box lacks of assignments.
% Therefore \TeX\ puts |\@test| right at the beginning of the
% box specification, e.g:
% \begin{quote}
%   |\shipout\hbox{Hello World}|\\
%   $\Rightarrow$
%   |\setbox\mybox=\hbox{|\fbox{\cs{@test}}| Hello World}|
% \end{quote}
%
% \subsection{Test for direct or indirect boxes}
%
% Now we want to execute \cs{@test}, but where are we?
% We can be after the completed box assignment, if \cs{shipout}
% was called with an indirect box. Or we are right at the beginning
% of a direct box.
%
% \subsubsection{With \eTeX}
%
% With the \eTeX's extensions the answer is very easy:
% Being inside the direct box means that we are inside a new group.
% The new primitive command \cs{currentgrouplevel} tells how
% deeply the groups are currently nested. Macro \cs{@test} just
% compares the previously stored group level with the current one:
%\begin{quote}
%\begin{verbatim}
%\shipout :=
%  \edef\saved@grouplevel{\number\currentgrouplevel}
%  \afterassignment\@test
%  \setbox\mybox=
%
%\@test :=
%  \ifnum\saved@grouplevel=\currentgrouplevel
%    % case: indirect box, the assignment is completed
%    \@output
%  \else
%    % case: direct box, we are inside the box
%    \aftergroup\@outbox
%  \fi
%\end{verbatim}
%\end{quote}
%
% \subsubsection{Without \eTeX}
%
% Life becomes complicate without \eTeX. We cannot ask the
% group level. However, if we are inside a direct box,
% the box register \cs{mybox} is not yet changed by \cs{setbox}.
% Thus we need a special initial value and compare
% it in \cs{@test} with the current value of the box.
%
% What can be used as initial value? Arbitrary box contents cannot
% be compared. \TeX\ only tells us a few properties:
% \begin{itemize}
% \item Box type:\cs{ifhbox}, \cs{ifvbox}
% \item Dimensions: \cs{wd}, \cs{ht}, \cs{dp}
% \item Voidness: \cs{ifvoid}
% \end{itemize}
% Unhappily all these qualities even combined are not sufficient
% for constructing an initial box value, because \cs{shipout} can be
% called with a box that is accidently just the same as the choosen
% initial value.
%
% Nevertheless we have two alternatives for an initial value:
% \begin{itemize}
% \item A box of some type with some funny settings that are
% unlikely to occur in real life, e.g a height of |4911sp-\maxdimen|.
% \item A void box.
% \end{itemize}
% A collision between this initial value and an indirect \cs{shipout}
% box with just the same value is possible.
% Then \cs{@test} will make a wrong decision that it is executed
% inside a direct box and delays \cs{@output} by \cs{aftergroup}.
% Thus \cs{@output} is not called at the place we want.
% In contrary, the result is an uncertainty about the place:
% \begin{itemize}
% \item \cs{shipout} is used in a group that perhaps closes some
%   pages later. A bad place for \cs{@output}.
% \item Without a surrounding group \cs{aftergroup} effectively kills
%   its argument.
% \end{itemize}
% In the first case of a box with special dimensions we can even loose
% the page. However in the case of the void box, this effect is even
% desired, because the original \cs{shipout} does not output void boxes.
% All we have to do is to ensure that our box \cs{mybox} is always
% void except for the phase when the overloaded \cs{shipout} is executed.
% And secondly we must keep this semantics of \cs{shipout} for the void case
% in our macros, namely \cs{@output}.
%\begin{quote}
%\begin{verbatim}
%\shipout :=
%  % trick to get a void box \mybox
%  \begingroup
%    \setbox\mybox=\box\mybox
%  \endgroup
%  \afterassignment\@test
%  \setbox\mybox=
%
%\@test :=
%  \ifvoid\mybox
%    \aftergroup\@output
%  \else
%    \@output
%  \fi
%\end{verbatim}
%\end{quote}
% The nasty case is |\shipout\box\voidb@x| where the indirect box
% is void and that must not generate an output page.
% If a surrounding group is missing the output is ignored
% because of \cs{aftergroup}.
% Otherwise \@output is called some time
% later when the surrounding group closes.
% But \cs{mybox} is void outside the execution phase
% of the redefined \cs{shipout}. Also \cs{@output} checks for a
% void box and cancels the page output.
% The disadvantage remains that the hook in \cs{@output} is called
% for a page that will not be output.
%
% \subsubsection{\cs{lastkern} method}
%
% At the beginning of a new box, there is no \cs{kern}, the contents
% of the box is still empty and \cs{lastkern} returns 0\,pt.
% This can be used to distinguish between direct and indirect boxes:
% We execute \cs{setbox} in a box with a preceding non-zero kern.
% After an indirect box, \cs{lastkern} sees this kern, otherwise
% it returns 0\,pt.
%\begin{quote}
%\begin{verbatim}
%\shipout :=
%  \begingroup
%    \setbox\mybox=\hbox\bgroup
%      \kern1pt
%      \afterassignment\shipout@test
%      \global\setbox\mybox=
%\@test :=
%  \ifdim\lastkern=0pt
%    % direct box
%    \aftergroup\egroup
%    \aftergroup\endgroup
%    \aftergroup\@output
%  \else
%    \egroup
%    \endgroup
%    \@output
%  \fi
%\end{verbatim}
%\end{quote}
% We have two \cs{setbox} commands. The first creates a controlled
% context box where we can safely insert a \cs{kern}. We get rid of
% this temporarily used context box by putting the local \cs{setbox}
% in a group.
%
% After the group we want to have our shipout box in \cs{mybox}.
% Therefore we use a global assignment here.
%
% \subsection{Output}
%
% With or without \eTeX\ we ensure the original behaviour of
% \cs{shipout} that void boxes do not generate output pages.
%
% Now we can place the hook \cs{@hook} for the user code that
% wants to manipulate the output box.
%\begin{quote}
%\begin{verbatim}
%\@output :=
%  \ifvoid\mybox
%    % cancel output of void box
%  \else
%    \@hook
%    \ifvoid\mybox
%      % user code in \@hook could has voided the box
%    \else
%      \original@shipout\box\mybox
%    \fi
%  \fi
%\end{verbatim}
%\end{quote}
%
% \subsection{Separate box register}
%
% So far we have said nothing about the box number of \cs{mybox}.
% The following case that outputs the same page twice
% shows that we are not free in the use
% of the box register:
% \begin{quote}
%   |\shipout\copy<num> \shipout\box<num>|
% \end{quote}
% We manipulate the box by the hook and without \eTeX\ the box
% must even be voided. However, the use case above requires that
% the box contents does not change at all. Therefore we must
% reserve a separate box register to avoid collisions with
% user box registers.
%
% \emph{Note:} Box register number 255 is special for the output
% routine, because TeX complains if this box is not voided
% by the output routine.
% However, this requirement does not apply to \cs{shipout} at all.
% In fact \cs{shipout} does not change any box register. This is
% usually done by a call of \cs{box}, but the output routine
% can do it later \emph{after} invoking of \cs{shipout}.
%
% \subsection{Summary}
%
% \subsubsection{With \eTeX}
%
% Putting the pieces together we get for \eTeX:
%\begin{quote}
%\begin{verbatim}
%\newbox\mybox
%\let\original@shipout\shipout
%
%\shipout :=
%  \edef\saved@grouplevel{\number\currentgrouplevel}
%  \afterassignment\@test
%  \setbox\mybox=
%
%\@test :=
%  \ifnum\saved@grouplevel<\currentgrouplevel
%    \expandafter\aftergroup
%  \fi
%  \@output
%
%\@output :=
%  \ifvoid\mybox
%    % cancel output of void box
%  \else
%    \@hook
%    \ifvoid\mybox
%      % user code in \@hook could have voided the box
%    \else
%      \original@shipout\box\mybox
%    \fi
%  \fi
%\end{verbatim}
%\end{quote}
%
% \subsubsection{Without \eTeX, traditional way}
%
% And for \TeX\ without \eTeX:
%\begin{quote}
%\begin{verbatim}
%\newbox\mybox
%\begingroup
%   \setbox\mybox=\box\mybox % ensure \mybox is void
%\endgroup
%\let\original@shipout\shipout
%
%\shipout :=
%  % trick to get a void box \mybox
%  \begingroup
%    \setbox\mybox=\box\mybox
%  \endgroup
%  \afterassignment\@test
%  \setbox\mybox=
%
%\@test :=
%  \ifvoid\mybox
%    \expandafter\aftergroup
%  \fi
%  \@output
%
%\@output :=
%  \ifvoid\mybox
%    % cancel output of void box
%  \else
%    \@hook
%    \ifvoid\mybox
%      % user code in \@hook could have voided the box
%    \else
%      \original@shipout\box\mybox
%    \fi
%  \fi
%\end{verbatim}
%\end{quote}
%
% \subsubsection{\cs{lastkern} method}
%
% And for \TeX\ without \eTeX\ using the \cs{lastkern} method:
%\begin{quote}
%\begin{verbatim}
%\newbox\mybox
%\let\original@shipout\shipout
%
%\shipout :=
%  \begingroup
%  \setbox\mybox=\hbox\bgroup
%  \kern1pt
%  \afterassignment\@test
%  \setbox\mybox=
%
%\@test :=
%  \ifdim\lastkern=0pt
%    \expandafter\aftergroup
%  \fi
%  \@output
%
%\@output :=
%  \egroup
%  \endgroup
%  \ifvoid\mybox
%    % cancel output of void box
%  \else
%    \@hook
%    \ifvoid\mybox
%      % user code in \@hook could have voided the box
%    \else
%      \original@shipout\box\mybox
%    \fi
%  \fi
%\end{verbatim}
%\end{quote}
%
%
% \StopEventually{
% }
%
% \section{Implementation}
%
% Package \xpackage{atbegshi} uses \eTeX's \cs{currentgrouplevel},
% if it is available. Otherwise the \cs{lastkern} method is used.
%
%    \begin{macrocode}
%<*package>
%    \end{macrocode}
%
% \subsection{Reload check and package identification}
%    Reload check, especially if the package is not used with \LaTeX.
%    \begin{macrocode}
\begingroup\catcode61\catcode48\catcode32=10\relax%
  \catcode13=5 % ^^M
  \endlinechar=13 %
  \catcode35=6 % #
  \catcode39=12 % '
  \catcode44=12 % ,
  \catcode45=12 % -
  \catcode46=12 % .
  \catcode58=12 % :
  \catcode64=11 % @
  \catcode123=1 % {
  \catcode125=2 % }
  \expandafter\let\expandafter\x\csname ver@atbegshi.sty\endcsname
  \ifx\x\relax % plain-TeX, first loading
  \else
    \def\empty{}%
    \ifx\x\empty % LaTeX, first loading,
      % variable is initialized, but \ProvidesPackage not yet seen
    \else
      \expandafter\ifx\csname PackageInfo\endcsname\relax
        \def\x#1#2{%
          \immediate\write-1{Package #1 Info: #2.}%
        }%
      \else
        \def\x#1#2{\PackageInfo{#1}{#2, stopped}}%
      \fi
      \x{atbegshi}{The package is already loaded}%
      \aftergroup\endinput
    \fi
  \fi
\endgroup%
%    \end{macrocode}
%    Package identification:
%    \begin{macrocode}
\begingroup\catcode61\catcode48\catcode32=10\relax%
  \catcode13=5 % ^^M
  \endlinechar=13 %
  \catcode35=6 % #
  \catcode39=12 % '
  \catcode40=12 % (
  \catcode41=12 % )
  \catcode44=12 % ,
  \catcode45=12 % -
  \catcode46=12 % .
  \catcode47=12 % /
  \catcode58=12 % :
  \catcode64=11 % @
  \catcode91=12 % [
  \catcode93=12 % ]
  \catcode123=1 % {
  \catcode125=2 % }
  \expandafter\ifx\csname ProvidesPackage\endcsname\relax
    \def\x#1#2#3[#4]{\endgroup
      \immediate\write-1{Package: #3 #4}%
      \xdef#1{#4}%
    }%
  \else
    \def\x#1#2[#3]{\endgroup
      #2[{#3}]%
      \ifx#1\@undefined
        \xdef#1{#3}%
      \fi
      \ifx#1\relax
        \xdef#1{#3}%
      \fi
    }%
  \fi
\expandafter\x\csname ver@atbegshi.sty\endcsname
\ProvidesPackage{atbegshi}%
  [2016/06/09 v1.18 At begin shipout hook (HO)]%
%    \end{macrocode}
%
% \subsection{Catcodes}
%
%    \begin{macrocode}
\begingroup\catcode61\catcode48\catcode32=10\relax%
  \catcode13=5 % ^^M
  \endlinechar=13 %
  \catcode123=1 % {
  \catcode125=2 % }
  \catcode64=11 % @
  \def\x{\endgroup
    \expandafter\edef\csname AtBegShi@AtEnd\endcsname{%
      \endlinechar=\the\endlinechar\relax
      \catcode13=\the\catcode13\relax
      \catcode32=\the\catcode32\relax
      \catcode35=\the\catcode35\relax
      \catcode61=\the\catcode61\relax
      \catcode64=\the\catcode64\relax
      \catcode123=\the\catcode123\relax
      \catcode125=\the\catcode125\relax
    }%
  }%
\x\catcode61\catcode48\catcode32=10\relax%
\catcode13=5 % ^^M
\endlinechar=13 %
\catcode35=6 % #
\catcode64=11 % @
\catcode123=1 % {
\catcode125=2 % }
\def\TMP@EnsureCode#1#2{%
  \edef\AtBegShi@AtEnd{%
    \AtBegShi@AtEnd
    \catcode#1=\the\catcode#1\relax
  }%
  \catcode#1=#2\relax
}
\TMP@EnsureCode{40}{12}% (
\TMP@EnsureCode{41}{12}% )
\TMP@EnsureCode{44}{12}% ,
\TMP@EnsureCode{45}{12}% -
\TMP@EnsureCode{47}{12}% /
\TMP@EnsureCode{46}{12}% .
\TMP@EnsureCode{58}{12}% :
\TMP@EnsureCode{91}{12}% [
\TMP@EnsureCode{93}{12}% ]
\TMP@EnsureCode{94}{7}% ^(superscript)
\TMP@EnsureCode{96}{12}% `
\edef\AtBegShi@AtEnd{\AtBegShi@AtEnd\noexpand\endinput}
%    \end{macrocode}
%
% \subsection{Preparations}
%
%    \begin{macrocode}
\begingroup\expandafter\expandafter\expandafter\endgroup
\expandafter\ifx\csname RequirePackage\endcsname\relax
  \def\TMP@RequirePackage#1[#2]{%
    \begingroup\expandafter\expandafter\expandafter\endgroup
    \expandafter\ifx\csname ver@#1.sty\endcsname\relax
      \input #1.sty\relax
    \fi
  }%
  \TMP@RequirePackage{infwarerr}[2007/09/09]%
  \TMP@RequirePackage{ltxcmds}[2010/03/01]%
\else
  \RequirePackage{infwarerr}[2007/09/09]%
  \RequirePackage{ltxcmds}[2010/03/01]%
\fi
%    \end{macrocode}
%
%    \begin{macro}{\AtBegShi@CheckDefinable}
%    \begin{macrocode}
\begingroup\expandafter\expandafter\expandafter\endgroup
\expandafter\ifx\csname @ifdefinable\endcsname\relax
  \def\AtBegShi@CheckDefinable#1{%
    \ifcase\ifx#1\relax
             \ltx@one
           \else
             \ifx#1\@undefined
               \ltx@one
             \else
               \ltx@zero
             \fi
           \fi
      \@PackageError{atbegshi}{%
        \string#1\space is already defined%
      }\@ehd
    \fi
  }%
\else
  \def\AtBegShi@CheckDefinable#1{%
    \@ifdefinable{#1}{}%
  }%
\fi
%    \end{macrocode}
%    \end{macro}
%
%    \begin{macro}{\ifAtBegShi@Discarded}
%    \begin{macrocode}
\ltx@newif\ifAtBegShi@Discarded
%    \end{macrocode}
%    \end{macro}
%    \begin{macro}{\AtBeginShipoutDiscard}
%    \begin{macrocode}
\AtBegShi@CheckDefinable\AtBeginShipoutDiscard
\def\AtBeginShipoutDiscard{%
  \deadcycles=\ltx@zero
  \global\AtBegShi@Discardedtrue
}
%    \end{macrocode}
%    \end{macro}
%
%    \begin{macrocode}
\begingroup\expandafter\expandafter\expandafter\endgroup
\expandafter\ifx\csname currentgrouplevel\endcsname\relax
  \catcode`X=9  % ignore
  \catcode`E=14 % comment
\else
  \catcode`X=14 % comment
  \catcode`E=9  % ignore
\fi
%    \end{macrocode}
%
%    \begin{macro}{\AtBegShi@Shipout}
%    \begin{macrocode}
\def\AtBegShi@Shipout{%
X \begingroup
X \setbox\AtBeginShipoutBox=\hbox\bgroup
X \kern\p@
E \edef\AtBegShi@GroupLevel{\number\currentgrouplevel}%
  \afterassignment\AtBegShi@Test
X \global
  \setbox\AtBeginShipoutBox=%
}
%    \end{macrocode}
%    \end{macro}
%
%    \begin{macro}{\AtBegShi@Test}
%    \begin{macrocode}
\def\AtBegShi@Test{%
X \ifdim\lastkern=0pt %
E \ifnum\AtBegShi@GroupLevel<\currentgrouplevel
    \expandafter\aftergroup
  \fi
  \AtBegShi@Output
}
%    \end{macrocode}
%    \end{macro}
%
%    \begin{macro}{\AtBegShi@Output}
%    \begin{macrocode}
\def\AtBegShi@Output{%
X \egroup
X \endgroup
  \ifvoid\AtBeginShipoutBox
    \@PackageWarning{atbegshi}{Ignoring void shipout box}%
  \else
    \let\AtBegShi@OrgProtect\protect
    \csname set@typeset@protect\endcsname
    \global\AtBegShi@Discardedfalse
    \AtBegShi@Hook
    \expandafter\gdef\expandafter\AtBegShi@HookNext
    \expandafter{\expandafter}%
    \AtBegShi@HookNext
    \ifAtBegShi@Discarded
      \@PackageInfoNoLine{atbegshi}{Shipout page discarded}%
      \global\AtBegShi@Discardedfalse
      \begingroup
        \setbox\AtBeginShipoutBox\box\AtBeginShipoutBox
      \endgroup
      \let\protect\AtBegShi@OrgProtect
    \else
      \AtBegShi@First
      \let\protect\AtBegShi@OrgProtect
      \AtBegShi@GetBoxSize\AtBeginShipoutBox
      \ltx@ifundefined{AtNextShipout}{%
      }{%
        \AtNextShipout{\AtBegShi@GetBoxSize\@cclv}%
      }%
      \AtBeginShipoutOriginalShipout\box\AtBeginShipoutBox
    \fi
  \fi
}
%    \end{macrocode}
%    \end{macro}
%
%    \begin{macro}{\AtBegShi@GetBoxSize}
%    \begin{macrocode}
\def\AtBegShi@GetBoxSize#1{%
  \xdef\AtBeginShipoutBoxWidth{\the\wd#1}%
  \xdef\AtBeginShipoutBoxHeight{\the\ht#1}%
  \xdef\AtBeginShipoutBoxDepth{\the\dp#1}%
}
%    \end{macrocode}
%    \end{macro}
%    \begin{macro}{\AtBeginShipoutBoxWidth}
%    \begin{macrocode}
\def\AtBeginShipoutBoxWidth{0pt}
%    \end{macrocode}
%    \end{macro}
%    \begin{macro}{\AtBeginShipoutBoxHeight}
%    \begin{macrocode}
\def\AtBeginShipoutBoxHeight{0pt}
%    \end{macrocode}
%    \end{macro}
%    \begin{macro}{\AtBeginShipoutBoxDepth}
%    \begin{macrocode}
\def\AtBeginShipoutBoxDepth{0pt}
%    \end{macrocode}
%    \end{macro}
%
%    \begin{macrocode}
\catcode`\X=11 %
\catcode`\E=11 %
%    \end{macrocode}
%
%    \begin{macro}{\AtBegShi@First}
%    \begin{macrocode}
\def\AtBegShi@First{%
  \ifx\AtBegShi@HookFirst\ltx@empty
  \else
    \AtBeginShipoutAddToBox{\AtBegShi@HookFirst}%
  \fi
  \global\let\AtBegShi@First\ltx@empty
  \global\let\AtBeginShipoutFirst\AtBegShi@FirstDisabled
}
%    \end{macrocode}
%    \end{macro}
%
%    \begin{macro}{\AtBegShi@Hook}
%    \begin{macrocode}
\gdef\AtBegShi@Hook{}
%    \end{macrocode}
%    \end{macro}
%    \begin{macro}{\AtBegShi@HookNext}
%    \begin{macrocode}
\gdef\AtBegShi@HookNext{}
%    \end{macrocode}
%    \end{macro}
%    \begin{macro}{\AtBegShi@HookFirst}
%    \begin{macrocode}
\gdef\AtBegShi@HookFirst{}
%    \end{macrocode}
%    \end{macro}
%    \begin{macro}{\AtBeginShipout}
%    \begin{macrocode}
\AtBegShi@CheckDefinable\AtBeginShipout
\def\AtBeginShipout{%
  \AtBegShi@AddHook\AtBegShi@Hook
}
%    \end{macrocode}
%    \end{macro}
%    \begin{macro}{\AtBeginShipoutNext}
%    \begin{macrocode}
\AtBegShi@CheckDefinable\AtBeginShipoutNext
\def\AtBeginShipoutNext{%
  \AtBegShi@AddHook\AtBegShi@HookNext
}
%    \end{macrocode}
%    \end{macro}
%    \begin{macro}{\AtBeginShipoutFirst}
%    \begin{macrocode}
\AtBegShi@CheckDefinable\AtBeginShipoutFirst
\def\AtBeginShipoutFirst{%
  \AtBegShi@AddTo\AtBegShi@HookFirst
}
%    \end{macrocode}
%    \end{macro}
%    \begin{macro}{\AtBegShi@FirstDisabled}
%    \begin{macrocode}
\long\def\AtBegShi@FirstDisabled#1{%
  \@PackageWarning{atbegshi}{%
    First page is already shipped out, ignoring\MessageBreak
    \string\AtBeginShipoutFirst
  }%
}
%    \end{macrocode}
%    \end{macro}
%
%    \begin{macro}{\AtBegShi@AddTo}
%    \begin{macrocode}
\begingroup\expandafter\expandafter\expandafter\endgroup
\expandafter\ifx\csname g@addto@macro\endcsname\relax
  \long\def\AtBegShi@AddTo#1#2{%
    \begingroup
      \toks\ltx@zero\expandafter{#1#2}%
      \xdef#1{\the\toks\ltx@zero}%
    \endgroup
  }%
\else
  \let\AtBegShi@AddTo\g@addto@macro
\fi
%    \end{macrocode}
%    \end{macro}
%
%    \begin{macro}{\AtBegShi@AddHook}
%    \begin{macrocode}
\long\def\AtBegShi@AddHook#1#2{%
  \AtBegShi@AddTo#1{\AtBegShi@Item{#2}}%
}
%    \end{macrocode}
%    \end{macro}
%
%    \begin{macro}{\AtBegShi@Item}
%    \begin{macrocode}
\long\def\AtBegShi@Item#1{%
  \ifAtBegShi@Discarded
  \else
    #1%
    \ifAtBegShi@Discarded
    \else
      \ifvoid\AtBeginShipoutBox
        \@PackageWarning{atbegshi}{%
          Shipout box was voided by hook,\MessageBreak
          ignoring shipout box%
        }%
        \AtBeginShipoutDiscard
      \fi
    \fi
  \fi
}
%    \end{macrocode}
%    \end{macro}
%
%    \begin{macro}{\AtBeginShipoutInit}
%    \begin{macrocode}
\AtBegShi@CheckDefinable\AtBeginShipoutInit
\def\AtBeginShipoutInit{%
  \ltx@IfUndefined{newbox}{%
    \@PackageError{atbegshi}{%
      \string\AtBeginShipoutInit\space failed\MessageBreak
       because of missing \expandafter\string\csname newbox\endcsname
    }\@ehc
  }{%
    \csname newbox\endcsname\AtBeginShipoutBox
    \AtBegShi@CheckDefinable\AtBeginShipoutOriginalShipout
    \global\let\AtBeginShipoutOriginalShipout\shipout
    \global\let\shipout\AtBegShi@Shipout
  }%
  \gdef\AtBeginShipoutInit{}%
}
%    \end{macrocode}
%    \end{macro}
%    \begin{macrocode}
\begingroup\expandafter\expandafter\expandafter\endgroup
\expandafter\ifx\csname AtBeginDocument\endcsname\relax
  \AtBeginShipoutInit
\else
  \AtBeginDocument{\AtBeginShipoutInit}%
\fi
%    \end{macrocode}
%
% \subsection{Additions to the shipout box}
%
%    \begin{macro}{\AtBeginShipoutAddToBox}
%    \begin{macrocode}
\def\AtBeginShipoutAddToBox#1{%
  \ifhbox\AtBeginShipoutBox
    \edef\AtBegShi@restore{%
      \hfuzz=\the\hfuzz\relax
      \hbadness=\the\hbadness\relax
    }%
    \hfuzz=1073741823sp\relax
    \hbadness=2147483647\relax
    \setbox\AtBeginShipoutBox=\hbox to \wd\AtBeginShipoutBox{%
      \setbox\ltx@zero=\hbox{%
        \begingroup
          \AtBegShi@restore
          #1%
        \endgroup
      }%
      \wd\ltx@zero=0pt\relax
      \ht\ltx@zero=0pt\relax
      \dp\ltx@zero=0pt\relax
      \raise\ht\AtBeginShipoutBox\copy\ltx@zero
      \unhcopy\AtBeginShipoutBox
    }%
    \AtBegShi@restore
  \else
    \ifvbox\AtBeginShipoutBox
      \edef\AtBegShi@restore{%
        \vfuzz=\the\vfuzz\relax
        \vbadness=\the\vbadness\relax
        \dimen\ltx@zero=\the\dimen\ltx@zero\relax
      }%
      \edef\AtBegShi@restorebox{%
        \ht\AtBeginShipoutBox=\the\ht\AtBeginShipoutBox\relax
        \dp\AtBeginShipoutBox=\the\dp\AtBeginShipoutBox\relax
      }%
      \vfuzz=1073741823sp\relax
      \vbadness=2147483647\relax
      \dimen\ltx@zero=\ht\AtBeginShipoutBox
      \advance\dimen\ltx@zero by \dp\AtBeginShipoutBox
      \setbox\AtBeginShipoutBox=\vbox to \dimen\ltx@zero{%
        \setbox\ltx@zero=\hbox{%
          \begingroup
            \AtBegShi@restore
            #1%
          \endgroup
        }%
        \wd\ltx@zero=0pt\relax
        \ht\ltx@zero=0pt\relax
        \dp\ltx@zero=0pt\relax
        \baselineskip=0pt\relax
        \lineskip=0pt\relax
        \lineskiplimit=0pt\relax
        \copy\ltx@zero
        \unvbox\AtBeginShipoutBox
        \kern0pt%
      }%
      \AtBegShi@restore
      \AtBegShi@restorebox
    \fi
  \fi
}
%    \end{macrocode}
%    \end{macro}
%
%    \begin{macro}{\AtBeginShipoutAddToBoxForeground}
%    \begin{macrocode}
\def\AtBeginShipoutAddToBoxForeground#1{%
  \ifhbox\AtBeginShipoutBox
    \edef\AtBegShi@restore{%
      \hfuzz=\the\hfuzz\relax
      \hbadness=\the\hbadness\relax
    }%
    \hfuzz=1073741823sp\relax
    \hbadness=2147483647\relax
    \setbox\AtBeginShipoutBox=\hbox to \wd\AtBeginShipoutBox{%
      \unhcopy\AtBeginShipoutBox
      \kern-\wd\AtBeginShipoutBox
      \setbox\ltx@zero=\hbox{%
        \begingroup
          \AtBegShi@restore
          #1%
        \endgroup
      }%
      \wd\ltx@zero=0pt\relax
      \ht\ltx@zero=0pt\relax
      \dp\ltx@zero=0pt\relax
      \raise\ht\AtBeginShipoutBox\copy\ltx@zero
      \kern\wd\AtBeginShipoutBox
    }%
    \AtBegShi@restore
  \else
    \ifvbox\AtBeginShipoutBox
      \edef\AtBegShi@restore{%
        \vfuzz=\the\vfuzz\relax
        \vbadness=\the\vbadness\relax
        \dimen\ltx@zero=\the\dimen\ltx@zero\relax
      }%
      \edef\AtBegShi@restorebox{%
        \ht\AtBeginShipoutBox=\the\ht\AtBeginShipoutBox\relax
        \dp\AtBeginShipoutBox=\the\dp\AtBeginShipoutBox\relax
      }%
      \vfuzz=1073741823sp\relax
      \vbadness=2147483647\relax
      \dimen\ltx@zero=\ht\AtBeginShipoutBox
      \advance\dimen\ltx@zero by \dp\AtBeginShipoutBox
      \setbox\AtBeginShipoutBox=\vbox to \dimen\ltx@zero{%
        \setbox\ltx@zero=\hbox{%
          \begingroup
            \AtBegShi@restore
            #1%
          \endgroup
        }%
        \wd\ltx@zero=0pt\relax
        \ht\ltx@zero=0pt\relax
        \dp\ltx@zero=0pt\relax
        \baselineskip=0pt\relax
        \lineskip=0pt\relax
        \lineskiplimit=0pt\relax
        \unvbox\AtBeginShipoutBox
        \kern-\dimen\ltx@zero
        \copy\ltx@zero
        \kern\dimen\ltx@zero
      }%
      \AtBegShi@restore
      \AtBegShi@restorebox
    \fi
  \fi
}
%    \end{macrocode}
%    \end{macro}
%
% \subsection{Positioning}
%
%    \begin{macrocode}
\begingroup\expandafter\expandafter\expandafter\endgroup
\expandafter\ifx\csname RequirePackage\endcsname\relax
  \def\TMP@RequirePackage#1[#2]{%
    \begingroup\expandafter\expandafter\expandafter\endgroup
    \expandafter\ifx\csname ver@#1.sty\endcsname\relax
      \input #1.sty\relax
    \fi
  }%
  \TMP@RequirePackage{ifpdf}[2011/01/30]%
\else
  \RequirePackage{ifpdf}[2011/01/30]%
\fi
%    \end{macrocode}
%
%    \begin{macrocode}
\ifpdf
  \def\AtBegShi@horigin{%
    \ifx\pdfhorigin\@undefined\pdfvariable horigin\else\pdfhorigin\fi}%
  \def\AtBegShi@vorigin{%
    \ifx\pdfvorigin\@undefined\pdfvariable vorigin\else\pdfvorigin\fi}%
\else
  \def\AtBegShi@horigin{72.27pt}%
  \def\AtBegShi@vorigin{72.27pt}%
\fi
%    \end{macrocode}
%
%    \begin{macrocode}
\begingroup
\ifcase
    \expandafter\ifx\csname picture\endcsname\relax
      1%
    \else
      \expandafter\ifx\csname endpicture\endcsname\relax
        1%
      \else
        0%
      \fi
    \fi
  \endgroup
  \def\AtBegShi@BeginPicture{%
    \begingroup
    \picture(0,0)\relax
    \begingroup\expandafter\expandafter\expandafter\endgroup
    \expandafter\ifx\csname unitlength\endcsname\relax
    \else
      \unitlength=1pt\relax
    \fi
    \ignorespaces
  }%
  \def\AtBegShi@EndPicture{%
    \endpicture
    \endgroup
  }%
\else
  \endgroup
  \def\AtBegShi@BeginPicture{%
    \setbox\ltx@zero=\hbox\bgroup
    \begingroup
    \ignorespaces
  }%
  \def\AtBegShi@EndPicture{%
    \endgroup
    \egroup
    \ht\ltx@zero=0pt\relax
    \dp\ltx@zero=0pt\relax
    \copy\ltx@zero
  }%
\fi
%    \end{macrocode}
%
%    \begin{macro}{\AtBeginShipoutUpperLeft}
%    A surrounding \cs{rlap} is not necessary, because
%    the stuff is put in an \cs{hbox} with zero width.
%    \begin{macrocode}
\def\AtBeginShipoutUpperLeft#1{%
  \AtBeginShipoutAddToBox{%
    \kern-\AtBegShi@horigin\relax
    \vbox to 0pt{%
      \kern-\AtBegShi@vorigin\relax
      \AtBegShi@BeginPicture
      #1%
      \AtBegShi@EndPicture
      \vss
    }%
  }%
}
%    \end{macrocode}
%    \end{macro}
%
%    \begin{macro}{\AtBeginShipoutUpperLeftForeground}
%    \begin{macrocode}
\def\AtBeginShipoutUpperLeftForeground#1{%
  \AtBeginShipoutAddToBoxForeground{%
    \kern-\AtBegShi@horigin\relax
    \vbox to 0pt{%
      \kern-\AtBegShi@vorigin\relax
      \AtBegShi@BeginPicture
      #1%
      \AtBegShi@EndPicture
      \vss
    }%
  }%
}
%    \end{macrocode}
%    \end{macro}
%
% \subsection{Patches}
%
%    Patches for \LaTeX\ packages that redefine \cs{shipout}.
%    \LaTeX\ is now supposed to use \eTeX. Thus we do not patch,
%    without \LaTeX\ and \eTeX.
%    \begin{macrocode}
\def\AtBegShi@AbortIfUndefined#1{%
  \begingroup\expandafter\expandafter\expandafter\endgroup
  \expandafter\ifx\csname#1\endcsname\relax
    \expandafter\AtBegShi@AtEnd
  \fi
}
\AtBegShi@AbortIfUndefined{currentgrouplevel}%
\AtBegShi@AbortIfUndefined{AtBeginDocument}%
\AtBegShi@AbortIfUndefined{@ifpackageloaded}%
\AtBegShi@AbortIfUndefined{@ifclassloaded}%
%    \end{macrocode}
%
% \subsubsection{Package \xpackage{crop}}
%
% Fix of method and box.
%    \begin{macrocode}
\def\AtBegShi@PatchCrop{%
  \begingroup
    \def\AtBegShi@Crop@shipout{%
      \afterassignment\CROP@ship
      \setbox\@cclv=%
    }%
    \def\AtBegShi@Crop@ship{%
      \ifvoid\@cclv
        \expandafter\aftergroup
      \fi
      \CROP@@ship
    }%
    \def\AtBegShi@Crop@shiplist{%
      \lineskip\z@
      \lineskiplimit\z@
      \baselineskip\z@
      \CROP@kernel
      \box\@cclv
    }%
    \def\AtBegShi@Crop@@ship{%
      \CROP@shipout\vbox{%
        \CROP@shiplist
      }%
    }%
    \ifx\AtBegShi@Crop@ship\CROP@ship
      \ifx\AtBegShi@Crop@shiplist\CROP@shiplist
        \ifx\AtBegShi@Crop@@ship\CROP@@ship
          \let\AtBegShi@found\relax
          \ifx\shipout\AtBegShi@Crop@shipout
            \def\AtBegShi@found{\shipout}%
          \else\ifx\AtBeginShipoutOriginalShipout\AtBegShi@Crop@shipout
            \def\AtBegShi@found{\AtBeginShipoutOriginalShipout}%
          \else\ifx\@EveryShipout@Org@Shipout\AtBegShi@Crop@shipout
            \def\AtBegShi@found{\@EveryShipout@Org@Shipout}%
          \else\ifx\GPTorg@shipout\AtBegShi@Crop@shipout
            \def\AtBegShi@found{\GPTorg@shipout}%
          \else\ifx\THBorg@shipout\AtBegShi@Crop@shipout
            \def\AtBegShi@found{\THBorg@shipout}%
          \else\ifx\mem@oldshipout\AtBegShi@Crop@shipout
            \def\AtBegShi@found{\mem@oldshipout}%
          \fi\fi\fi\fi\fi\fi
          \ifx\AtBegShi@found\relax
          \else
            \expandafter\endgroup
            \expandafter\def\AtBegShi@found{%
              \edef\AtBegShi@GroupLevel{\number\currentgrouplevel}%
              \afterassignment\CROP@ship
              \setbox\AtBeginShipoutBox=%
            }%
            \def\CROP@ship{%
              \ifnum\AtBegShi@GroupLevel=\currentgrouplevel
              \else
                \expandafter\aftergroup
              \fi
              \CROP@@ship
            }%
            \def\CROP@shiplist{%
              \lineskip 0pt\relax
              \lineskiplimit 0pt\relax
              \baselineskip 0pt\relax
              \CROP@kernel
              \box\AtBeginShipoutBox
            }%
            \def\CROP@@ship{%
              \ifvoid\AtBeginShipoutBox
              \else
                \setbox\AtBeginShipoutBox=\vbox{%
                  \CROP@shiplist
                }%
                \AtBegShi@GetBoxSize\AtBeginShipoutBox
                \expandafter\CROP@shipout
                \expandafter\box
                \expandafter\AtBeginShipoutBox
              \fi
            }%
            \@PackageInfoNoLine{atbegshi}{Package `crop' patched}%
            \begingroup
          \fi
        \fi
      \fi
    \fi
  \endgroup
  \let\AtBegShi@PatchCrop\relax
}
\@ifpackageloaded{crop}{%
  \AtBegShi@PatchCrop
}{%
  \AtBeginDocument{\AtBegShi@PatchCrop}%
}
%    \end{macrocode}
%
% \subsubsection{Package \xpackage{everyshi}}
%
%     Fix of method. Use of box 255 is not changed.
%    \begin{macrocode}
\def\AtBegShi@PatchEveryshi{%
  \begingroup
    \long\def\AtBegShi@Everyshi@shipout{%
      \afterassignment\@EveryShipout@Test
      \global\setbox\@cclv= %
    }%
    \long\def\AtBegShi@Everyshi@Test{%
      \ifvoid\@cclv\relax
        \aftergroup\@EveryShipout@Output
      \else
        \@EveryShipout@Output
      \fi
    }%
    \ifx\AtBegShi@Everyshi@Test\@EveryShipout@Test
      \let\AtBegShi@found\relax
      \ifx\shipout\AtBegShi@Everyshi@shipout
        \def\AtBegShi@found{\shipout}%
      \else\ifx\AtBeginShipoutOriginalShipout\AtBegShi@Everyshi@shipout
        \def\AtBegShi@found{\AtBeginShipoutOriginalShipout}%
      \else\ifx\CROP@shipout\AtBegShi@Everyshi@shipout
        \def\AtBegShi@found{\CROP@shipout}%
      \else\ifx\GPTorg@shipout\AtBegShi@Everyshi@shipout
        \def\AtBegShi@found{\GPTorg@shipout}%
      \else\ifx\THBorg@shipout\AtBegShi@Everyshi@shipout
        \def\AtBegShi@found{\THBorg@shipout}%
      \else\ifx\mem@oldshipout\AtBegShi@Everyshi@shipout
        \def\AtBegShi@found{\mem@oldshipout}%
      \else
        \expandafter\ifx\csname @EveryShipout@Org@Shipout\endcsname
                        \relax
           \ifx\@EveryShipout@Shipout\AtBegShi@Everyshi@shipout
             \def\AtBegShi@found{\@EveryShipout@Shipout}%
           \fi
        \fi
      \fi\fi\fi\fi\fi\fi
      \ifx\AtBegShi@found\relax
      \else
        \expandafter\endgroup
        \expandafter\def\AtBegShi@found{%
          \edef\AtBegShi@GroupLevel{\number\currentgrouplevel}%
          \afterassignment\@EveryShipout@Test
          \setbox\AtBeginShipoutBox=%
        }%
        \def\@EveryShipout@Test{%
          \ifnum\AtBegShi@GroupLevel=\currentgrouplevel
          \else
            \expandafter\aftergroup
          \fi
          \AtBegShi@Everyshi@Output
        }%
        \def\AtBegShi@Everyshi@Output{%
          \ifvoid\AtBeginShipoutBox
          \else
            \global\setbox\ltx@cclv\box\AtBeginShipoutBox
            \expandafter\@EveryShipout@Output
          \fi
        }%
        \@PackageInfoNoLine{atbegshi}{Package `everyshi' patched}%
        \begingroup
      \fi
    \fi
  \endgroup
  \let\AtBegShi@PatchEveryshi\relax
}
\@ifpackageloaded{everyshi}{%
  \AtBegShi@PatchEveryshi
}{%
  \AtBeginDocument{\AtBegShi@PatchEveryshi}%
}
%    \end{macrocode}
%
% \subsubsection{Class \xpackage{memoir}}
%
%     Fix of method and box.
%    \begin{macrocode}
\def\AtBegShi@PatchMemoir{%
  \begingroup
    \def\AtBegShi@Memoir@shipout{%
      \afterassignment\mem@shipi
      \setbox\@cclv=%
    }%
    \def\AtBegShi@Memoir@shipi{%
      \ifvoid\@cclv
        \expandafter\aftergroup
      \fi
      \mem@shipii
    }%
    \def\AtBegShi@Memoir@shipiiA{%
      \mem@oldshipout\vbox{%
        \trimmarks
        \unvbox\@cclv
      }%
    }%
    \def\AtBegShi@Memoir@shipiiB{%
      \ifvoid\@cclv
        \mem@oldshipout\box\@cclv
      \else
        \mem@oldshipout\vbox{%
          \trimmarks
          \unvbox\@cclv
        }%
      \fi
    }%
    \def\AtBegShi@Memoir@PatchAB{%
      \ifvoid\AtBeginShipoutBox
      \else
        \setbox\AtBeginShipoutBox=\vbox{%
          \trimmarks
          \ifvbox\AtBeginShipoutBox
            \unvbox\AtBeginShipoutBox
          \else
            \box\AtBeginShipoutBox
          \fi
        }%
        \AtBegShi@GetBoxSize\AtBeginShipoutBox
        \expandafter\mem@oldshipout
        \expandafter\box
        \expandafter\AtBeginShipoutBox
      \fi
    }%
    \def\AtBegShi@Memoir@shipiiC{% 2008/08/07 v1.6180339a
      \ifvoid\@cclv
        \mem@oldshipout\box\@cclv
      \else
        \ifshowtrims
          \mem@oldshipout\vbox{\trimmarks\unvbox\@cclv}%
        \else
          \mem@oldshipout\box\@cclv
        \fi
      \fi
    }%
    \def\AtBegShi@Memoir@shipiiD{% 2011/03/06 v3.6j
      \ifvoid\@cclv
        \mem@oldshipout\box\@cclv
      \else
        \ifshowtrims
          \mem@oldshipout\vbox{%
            \trimmarks
            \nointerlineskip
            \box\@cclv
          }%
        \else
          \mem@oldshipout\box\@cclv
        \fi
      \fi
    }%
    \def\AtBegShi@Memoir@PatchCD{%
      \ifvoid\AtBeginShipoutBox
      \else
        \ifshowtrims
          \setbox\AtBeginShipoutBox=\vbox{%
            \trimmarks
            \nointerlineskip
            \box\AtBeginShipoutBox
          }%
        \fi
        \AtBegShi@GetBoxSize\AtBeginShipoutBox
        \expandafter\mem@oldshipout
        \expandafter\box
        \expandafter\AtBeginShipoutBox
      \fi
    }%
    \ifx\AtBegShi@Memoir@shipi\mem@shipi
      \let\AtBegShi@found\ltx@one
      \ifx\AtBegShi@Memoir@shipiiA\mem@shipii
        \let\AtBegShi@found\ltx@zero
        \global\let\AtBegShi@Memoir@PatchX\AtBegShi@Memoir@PatchAB
      \else\ifx\AtBegShi@Memoir@shipiiB\mem@shipii
        \let\AtBegShi@found\ltx@zero
        \global\let\AtBegShi@Memoir@PatchX\AtBegShi@Memoir@PatchAB
      \else\ifx\AtBegShi@Memoir@shipiiC\mem@shipii
        \let\AtBegShi@found\ltx@zero
        \global\let\AtBegShi@Memoir@PatchX\AtBegShi@Memoir@PatchCD
      \else\ifx\AtBegShi@Memoir@shipiiD\mem@shipii
        \let\AtBegShi@found\ltx@zero
        \global\let\AtBegShi@Memoir@PatchX\AtBegShi@Memoir@PatchCD
      \fi\fi\fi\fi
      \ifcase\AtBegShi@found
        \let\AtBegShi@found\relax
        \ifx\shipout\AtBegShi@Memoir@shipout
          \def\AtBegShi@found{\shipout}%
        \else\ifx\AtBeginShipoutOriginalShipout\AtBegShi@Memoir@shipout
          \def\AtBegShi@found{\AtBeginShipoutOriginalShipout}%
        \else\ifx\CROP@shipout\AtBegShi@Memoir@shipout
          \def\AtBegShi@found{\CROP@shipout}%
        \else\ifx\GPTorg@shipout\AtBegShi@Memoir@shipout
          \def\AtBegShi@found{\GPTorg@shipout}%
        \else\ifx\THBorg@shipout\AtBegShi@Memoir@shipout
          \def\AtBegShi@found{\THBorg@shipout}%
        \else\ifx\@EveryShipout@Org@Shipout\AtBegShi@Memoir@shipout
          \def\AtBegShi@found{\@EveryShipout@Org@Shipout}%
        \fi\fi\fi\fi\fi\fi
        \ifx\AtBegShi@found\relax
        \else
          \expandafter\endgroup
          \expandafter\def\AtBegShi@found{%
            \edef\AtBegShi@GroupLevel{\number\currentgrouplevel}%
            \afterassignment\mem@shipi
            \setbox\AtBeginShipoutBox=%
          }%
          \def\mem@shipi{%
            \ifnum\AtBegShi@GroupLevel=\currentgrouplevel
            \else
              \expandafter\aftergroup
            \fi
            \mem@shipii
          }%
          \let\mem@shipii\AtBegShi@Memoir@PatchX
          \@PackageInfoNoLine{atbegshi}{Class `memoir' patched}%
          \begingroup
        \fi
      \fi
    \fi
  \endgroup
  \let\AtBegShi@PatchMemoir\relax
}
\@ifclassloaded{memoir}{%
  \AtBegShi@PatchMemoir
}{%
  \AtBeginDocument{\AtBegShi@PatchMemoir}%
}
%    \end{macrocode}
%
%    \begin{macrocode}
\AtBegShi@AtEnd%
%</package>
%    \end{macrocode}
%
% \section{Test}
%
% \subsection{Catcode checks for loading}
%
%    \begin{macrocode}
%<*test1>
%    \end{macrocode}
%    \begin{macrocode}
\catcode`\{=1 %
\catcode`\}=2 %
\catcode`\#=6 %
\catcode`\@=11 %
\expandafter\ifx\csname count@\endcsname\relax
  \countdef\count@=255 %
\fi
\expandafter\ifx\csname @gobble\endcsname\relax
  \long\def\@gobble#1{}%
\fi
\expandafter\ifx\csname @firstofone\endcsname\relax
  \long\def\@firstofone#1{#1}%
\fi
\expandafter\ifx\csname loop\endcsname\relax
  \expandafter\@firstofone
\else
  \expandafter\@gobble
\fi
{%
  \def\loop#1\repeat{%
    \def\body{#1}%
    \iterate
  }%
  \def\iterate{%
    \body
      \let\next\iterate
    \else
      \let\next\relax
    \fi
    \next
  }%
  \let\repeat=\fi
}%
\def\RestoreCatcodes{}
\count@=0 %
\loop
  \edef\RestoreCatcodes{%
    \RestoreCatcodes
    \catcode\the\count@=\the\catcode\count@\relax
  }%
\ifnum\count@<255 %
  \advance\count@ 1 %
\repeat

\def\RangeCatcodeInvalid#1#2{%
  \count@=#1\relax
  \loop
    \catcode\count@=15 %
  \ifnum\count@<#2\relax
    \advance\count@ 1 %
  \repeat
}
\def\RangeCatcodeCheck#1#2#3{%
  \count@=#1\relax
  \loop
    \ifnum#3=\catcode\count@
    \else
      \errmessage{%
        Character \the\count@\space
        with wrong catcode \the\catcode\count@\space
        instead of \number#3%
      }%
    \fi
  \ifnum\count@<#2\relax
    \advance\count@ 1 %
  \repeat
}
\def\space{ }
\expandafter\ifx\csname LoadCommand\endcsname\relax
  \def\LoadCommand{\input atbegshi.sty\relax}%
\fi
\def\Test{%
  \RangeCatcodeInvalid{0}{47}%
  \RangeCatcodeInvalid{58}{64}%
  \RangeCatcodeInvalid{91}{96}%
  \RangeCatcodeInvalid{123}{255}%
  \catcode`\@=12 %
  \catcode`\\=0 %
  \catcode`\%=14 %
  \LoadCommand
  \RangeCatcodeCheck{0}{36}{15}%
  \RangeCatcodeCheck{37}{37}{14}%
  \RangeCatcodeCheck{38}{47}{15}%
  \RangeCatcodeCheck{48}{57}{12}%
  \RangeCatcodeCheck{58}{63}{15}%
  \RangeCatcodeCheck{64}{64}{12}%
  \RangeCatcodeCheck{65}{90}{11}%
  \RangeCatcodeCheck{91}{91}{15}%
  \RangeCatcodeCheck{92}{92}{0}%
  \RangeCatcodeCheck{93}{96}{15}%
  \RangeCatcodeCheck{97}{122}{11}%
  \RangeCatcodeCheck{123}{255}{15}%
  \RestoreCatcodes
}
\Test
\csname @@end\endcsname
\end
%    \end{macrocode}
%    \begin{macrocode}
%</test1>
%    \end{macrocode}
%
%    \begin{macrocode}
%<*test2>
\input atbegshi.sty\relax
\def\msg#{\immediate\write16}
\msg{File: atbegshi-test2.tex 2016/06/09 v1.18 Test file for plain-TeX}
\def\testmsg#1#2{%
  \msg{}%
  \msg{*** Test with box (#1), expected page output [#2]}% hash-ok
}

\newbox\voidbox
\def\void{\box\voidbox}
\begingroup
  \setbox\voidbox=\void
\endgroup

\count0=0\relax
\AtBeginShipout{%
  \global\advance\count0 by 1\relax
  \msg{* Inside \string\AtBeginShipout: [\the\count0]}%
}

\AtBeginShipoutFirst{%
  \msg{* Inside \string\AtBeginShipoutFirst}%
  Hello World%
}

\testmsg{\string\null}{1}
\shipout\null

\AtBeginShipoutFirst{%
  This is too late%
}

\testmsg{void}{}
\shipout\void

\testmsg{\string\copy255 (not void)}{2}
\setbox255\hbox{\vrule height 10bp width 10bp}
\shipout\copy255 %

\testmsg{\string\copy255 (again)}{3}
\shipout\copy255 %

\testmsg{\string\box255}{4}
\shipout\box255 %

\testmsg{\string\box255 (again)}{}
\shipout\box255 %

\testmsg{\string\hbox}{5}
\shipout\hbox{\vrule height 5bp width 20bp}

\testmsg{\string\vbox}{6}
\shipout\vbox{\hrule height 20bp width 5bp}

\testmsg{\string\null, voided by hook}{}
\def\VoidBox{%
  \begingroup
    \setbox\AtBeginShipoutBox=\box\AtBeginShipoutBox
  \endgroup
}
\AtBeginShipout{\VoidBox}
\shipout\null
\def\VoidBox{}

\msg{*** \string\begingroup}
\begingroup
  \testmsg{void}{}%
  \shipout\void
\msg{*** \string\endgroup}
\endgroup

\msg{*** \string\begingroup}
\begingroup
  \testmsg{void}{}%
  \shipout\void
  \testmsg{\string\null}{8}%
  \shipout\null
\msg{*** \string\endgroup}
\endgroup

\testmsg{output routine}{9}
Hello World
\vfill
\eject

\testmsg{\string\null\space(discarded)}{}
\AtBeginShipout{%
  \msg{* Inside \string\AtBeginShipout: DISCARD}%
  \AtBeginShipoutDiscard
}
\shipout\null

\end
%</test2>
%    \end{macrocode}
%
%    \begin{macrocode}
%<*test3>
\NeedsTeXFormat{LaTeX2e}
\ProvidesFile{atbegshi-test3.tex}[2016/06/09 v1.18 Test file for LaTeX]
\RequirePackage{color}
\pagecolor{yellow}
\documentclass[a5paper,showtrims]{memoir}
\usepackage{atbegshi}
\AtBeginShipout{%
  \setbox\AtBeginShipoutBox=\vbox{%
    \vbox to 0pt{%
      \kern-1.5in %
      \hbox to 0pt{%
        \kern-1.5in %
        \color{blue}%
        \rule{1in}{1in}%
        \hss
      }%
      \vss
    }%
    \hrule
    \hbox{\vrule\box\AtBeginShipoutBox\vrule}%
    \hrule
  }%
}
\usepackage{eso-pic}
\makeatletter
\@ifundefined{@EveryShipout@Init}{%
  \typeout{Test skipped}%
  \@@end
}{}
\@EveryShipout@Init
\let\@EveryShipout@Init\relax
\makeatother
\AddToShipoutPicture{%
  \hspace{.52\paperwidth}%
  \colorbox{cyan}{%
    \rule{0mm}{\paperheight}%
    \hspace{.48\paperwidth}%
  }%
}
%    \end{macrocode}
%    Newer versions of class \xclass{memoir} emulate package
%    \xpackage{crop} and prevents its loading. This is undone
%    in next line for this test file.
%    \begin{macrocode}
\expandafter\let\csname ver@crop.sty\endcsname\relax
\usepackage[color=red,cross,a4,center]{crop}
\begin{document}
\shipout\null
\shipout\box\csname voidb@x\endcsname
\section{Hello World}
\end{document}
%</test3>
%    \end{macrocode}
%
% \section{Installation}
%
% \subsection{Download}
%
% \paragraph{Package.} This package is available on
% CTAN\footnote{\url{http://ctan.org/pkg/atbegshi}}:
% \begin{description}
% \item[\CTAN{macros/latex/contrib/oberdiek/atbegshi.dtx}] The source file.
% \item[\CTAN{macros/latex/contrib/oberdiek/atbegshi.pdf}] Documentation.
% \end{description}
%
%
% \paragraph{Bundle.} All the packages of the bundle `oberdiek'
% are also available in a TDS compliant ZIP archive. There
% the packages are already unpacked and the documentation files
% are generated. The files and directories obey the TDS standard.
% \begin{description}
% \item[\CTAN{install/macros/latex/contrib/oberdiek.tds.zip}]
% \end{description}
% \emph{TDS} refers to the standard ``A Directory Structure
% for \TeX\ Files'' (\CTAN{tds/tds.pdf}). Directories
% with \xfile{texmf} in their name are usually organized this way.
%
% \subsection{Bundle installation}
%
% \paragraph{Unpacking.} Unpack the \xfile{oberdiek.tds.zip} in the
% TDS tree (also known as \xfile{texmf} tree) of your choice.
% Example (linux):
% \begin{quote}
%   |unzip oberdiek.tds.zip -d ~/texmf|
% \end{quote}
%
% \paragraph{Script installation.}
% Check the directory \xfile{TDS:scripts/oberdiek/} for
% scripts that need further installation steps.
% Package \xpackage{attachfile2} comes with the Perl script
% \xfile{pdfatfi.pl} that should be installed in such a way
% that it can be called as \texttt{pdfatfi}.
% Example (linux):
% \begin{quote}
%   |chmod +x scripts/oberdiek/pdfatfi.pl|\\
%   |cp scripts/oberdiek/pdfatfi.pl /usr/local/bin/|
% \end{quote}
%
% \subsection{Package installation}
%
% \paragraph{Unpacking.} The \xfile{.dtx} file is a self-extracting
% \docstrip\ archive. The files are extracted by running the
% \xfile{.dtx} through \plainTeX:
% \begin{quote}
%   \verb|tex atbegshi.dtx|
% \end{quote}
%
% \paragraph{TDS.} Now the different files must be moved into
% the different directories in your installation TDS tree
% (also known as \xfile{texmf} tree):
% \begin{quote}
% \def\t{^^A
% \begin{tabular}{@{}>{\ttfamily}l@{ $\rightarrow$ }>{\ttfamily}l@{}}
%   atbegshi.sty & tex/generic/oberdiek/atbegshi.sty\\
%   atbegshi.pdf & doc/latex/oberdiek/atbegshi.pdf\\
%   atbegshi-example1.tex & doc/latex/oberdiek/atbegshi-example1.tex\\
%   atbegshi-example2.tex & doc/latex/oberdiek/atbegshi-example2.tex\\
%   test/atbegshi-test1.tex & doc/latex/oberdiek/test/atbegshi-test1.tex\\
%   test/atbegshi-test2.tex & doc/latex/oberdiek/test/atbegshi-test2.tex\\
%   test/atbegshi-test3.tex & doc/latex/oberdiek/test/atbegshi-test3.tex\\
%   atbegshi.dtx & source/latex/oberdiek/atbegshi.dtx\\
% \end{tabular}^^A
% }^^A
% \sbox0{\t}^^A
% \ifdim\wd0>\linewidth
%   \begingroup
%     \advance\linewidth by\leftmargin
%     \advance\linewidth by\rightmargin
%   \edef\x{\endgroup
%     \def\noexpand\lw{\the\linewidth}^^A
%   }\x
%   \def\lwbox{^^A
%     \leavevmode
%     \hbox to \linewidth{^^A
%       \kern-\leftmargin\relax
%       \hss
%       \usebox0
%       \hss
%       \kern-\rightmargin\relax
%     }^^A
%   }^^A
%   \ifdim\wd0>\lw
%     \sbox0{\small\t}^^A
%     \ifdim\wd0>\linewidth
%       \ifdim\wd0>\lw
%         \sbox0{\footnotesize\t}^^A
%         \ifdim\wd0>\linewidth
%           \ifdim\wd0>\lw
%             \sbox0{\scriptsize\t}^^A
%             \ifdim\wd0>\linewidth
%               \ifdim\wd0>\lw
%                 \sbox0{\tiny\t}^^A
%                 \ifdim\wd0>\linewidth
%                   \lwbox
%                 \else
%                   \usebox0
%                 \fi
%               \else
%                 \lwbox
%               \fi
%             \else
%               \usebox0
%             \fi
%           \else
%             \lwbox
%           \fi
%         \else
%           \usebox0
%         \fi
%       \else
%         \lwbox
%       \fi
%     \else
%       \usebox0
%     \fi
%   \else
%     \lwbox
%   \fi
% \else
%   \usebox0
% \fi
% \end{quote}
% If you have a \xfile{docstrip.cfg} that configures and enables \docstrip's
% TDS installing feature, then some files can already be in the right
% place, see the documentation of \docstrip.
%
% \subsection{Refresh file name databases}
%
% If your \TeX~distribution
% (\teTeX, \mikTeX, \dots) relies on file name databases, you must refresh
% these. For example, \teTeX\ users run \verb|texhash| or
% \verb|mktexlsr|.
%
% \subsection{Some details for the interested}
%
% \paragraph{Attached source.}
%
% The PDF documentation on CTAN also includes the
% \xfile{.dtx} source file. It can be extracted by
% AcrobatReader 6 or higher. Another option is \textsf{pdftk},
% e.g. unpack the file into the current directory:
% \begin{quote}
%   \verb|pdftk atbegshi.pdf unpack_files output .|
% \end{quote}
%
% \paragraph{Unpacking with \LaTeX.}
% The \xfile{.dtx} chooses its action depending on the format:
% \begin{description}
% \item[\plainTeX:] Run \docstrip\ and extract the files.
% \item[\LaTeX:] Generate the documentation.
% \end{description}
% If you insist on using \LaTeX\ for \docstrip\ (really,
% \docstrip\ does not need \LaTeX), then inform the autodetect routine
% about your intention:
% \begin{quote}
%   \verb|latex \let\install=y% \iffalse meta-comment
%
% File: atbegshi.dtx
% Version: 2016/06/09 v1.18
% Info: At begin shipout hook
%
% Copyright (C) 2007-2011 by
%    Heiko Oberdiek <heiko.oberdiek at googlemail.com>
%    2016
%    https://github.com/ho-tex/oberdiek/issues
%
% This work may be distributed and/or modified under the
% conditions of the LaTeX Project Public License, either
% version 1.3c of this license or (at your option) any later
% version. This version of this license is in
%    http://www.latex-project.org/lppl/lppl-1-3c.txt
% and the latest version of this license is in
%    http://www.latex-project.org/lppl.txt
% and version 1.3 or later is part of all distributions of
% LaTeX version 2005/12/01 or later.
%
% This work has the LPPL maintenance status "maintained".
%
% This Current Maintainer of this work is Heiko Oberdiek.
%
% The Base Interpreter refers to any `TeX-Format',
% because some files are installed in TDS:tex/generic//.
%
% This work consists of the main source file atbegshi.dtx
% and the derived files
%    atbegshi.sty, atbegshi.pdf, atbegshi.ins, atbegshi.drv,
%    atbegshi-example1.tex, atbegshi-example2.tex,
%    atbegshi-test1.tex, atbegshi-test2.tex, atbegshi-test3.tex.
%
% Distribution:
%    CTAN:macros/latex/contrib/oberdiek/atbegshi.dtx
%    CTAN:macros/latex/contrib/oberdiek/atbegshi.pdf
%
% Unpacking:
%    (a) If atbegshi.ins is present:
%           tex atbegshi.ins
%    (b) Without atbegshi.ins:
%           tex atbegshi.dtx
%    (c) If you insist on using LaTeX
%           latex \let\install=y\input{atbegshi.dtx}
%        (quote the arguments according to the demands of your shell)
%
% Documentation:
%    (a) If atbegshi.drv is present:
%           latex atbegshi.drv
%    (b) Without atbegshi.drv:
%           latex atbegshi.dtx; ...
%    The class ltxdoc loads the configuration file ltxdoc.cfg
%    if available. Here you can specify further options, e.g.
%    use A4 as paper format:
%       \PassOptionsToClass{a4paper}{article}
%
%    Programm calls to get the documentation (example):
%       pdflatex atbegshi.dtx
%       makeindex -s gind.ist atbegshi.idx
%       pdflatex atbegshi.dtx
%       makeindex -s gind.ist atbegshi.idx
%       pdflatex atbegshi.dtx
%
% Installation:
%    TDS:tex/generic/oberdiek/atbegshi.sty
%    TDS:doc/latex/oberdiek/atbegshi.pdf
%    TDS:doc/latex/oberdiek/atbegshi-example1.tex
%    TDS:doc/latex/oberdiek/atbegshi-example2.tex
%    TDS:doc/latex/oberdiek/test/atbegshi-test1.tex
%    TDS:doc/latex/oberdiek/test/atbegshi-test2.tex
%    TDS:doc/latex/oberdiek/test/atbegshi-test3.tex
%    TDS:source/latex/oberdiek/atbegshi.dtx
%
%<*ignore>
\begingroup
  \catcode123=1 %
  \catcode125=2 %
  \def\x{LaTeX2e}%
\expandafter\endgroup
\ifcase 0\ifx\install y1\fi\expandafter
         \ifx\csname processbatchFile\endcsname\relax\else1\fi
         \ifx\fmtname\x\else 1\fi\relax
\else\csname fi\endcsname
%</ignore>
%<*install>
\input docstrip.tex
\Msg{************************************************************************}
\Msg{* Installation}
\Msg{* Package: atbegshi 2016/06/09 v1.18 At begin shipout hook (HO)}
\Msg{************************************************************************}

\keepsilent
\askforoverwritefalse

\let\MetaPrefix\relax
\preamble

This is a generated file.

Project: atbegshi
Version: 2016/06/09 v1.18

Copyright (C) 2007-2011 by
   Heiko Oberdiek <heiko.oberdiek at googlemail.com>

This work may be distributed and/or modified under the
conditions of the LaTeX Project Public License, either
version 1.3c of this license or (at your option) any later
version. This version of this license is in
   http://www.latex-project.org/lppl/lppl-1-3c.txt
and the latest version of this license is in
   http://www.latex-project.org/lppl.txt
and version 1.3 or later is part of all distributions of
LaTeX version 2005/12/01 or later.

This work has the LPPL maintenance status "maintained".

This Current Maintainer of this work is Heiko Oberdiek.

The Base Interpreter refers to any `TeX-Format',
because some files are installed in TDS:tex/generic//.

This work consists of the main source file atbegshi.dtx
and the derived files
   atbegshi.sty, atbegshi.pdf, atbegshi.ins, atbegshi.drv,
   atbegshi-example1.tex, atbegshi-example2.tex,
   atbegshi-test1.tex, atbegshi-test2.tex, atbegshi-test3.tex.

\endpreamble
\let\MetaPrefix\DoubleperCent

\generate{%
  \file{atbegshi.ins}{\from{atbegshi.dtx}{install}}%
  \file{atbegshi.drv}{\from{atbegshi.dtx}{driver}}%
  \usedir{tex/generic/oberdiek}%
  \file{atbegshi.sty}{\from{atbegshi.dtx}{package}}%
  \usedir{doc/latex/oberdiek}%
  \file{atbegshi-example1.tex}{\from{atbegshi.dtx}{example1}}%
  \file{atbegshi-example2.tex}{\from{atbegshi.dtx}{example2}}%
  \usedir{doc/latex/oberdiek/test}%
  \file{atbegshi-test1.tex}{\from{atbegshi.dtx}{test1}}%
  \file{atbegshi-test2.tex}{\from{atbegshi.dtx}{test2}}%
  \file{atbegshi-test3.tex}{\from{atbegshi.dtx}{test3}}%
  \nopreamble
  \nopostamble
  \usedir{source/latex/oberdiek/catalogue}%
  \file{atbegshi.xml}{\from{atbegshi.dtx}{catalogue}}%
}

\catcode32=13\relax% active space
\let =\space%
\Msg{************************************************************************}
\Msg{*}
\Msg{* To finish the installation you have to move the following}
\Msg{* file into a directory searched by TeX:}
\Msg{*}
\Msg{*     atbegshi.sty}
\Msg{*}
\Msg{* To produce the documentation run the file `atbegshi.drv'}
\Msg{* through LaTeX.}
\Msg{*}
\Msg{* Happy TeXing!}
\Msg{*}
\Msg{************************************************************************}

\endbatchfile
%</install>
%<*ignore>
\fi
%</ignore>
%<*driver>
\NeedsTeXFormat{LaTeX2e}
\ProvidesFile{atbegshi.drv}%
  [2016/06/09 v1.18 At begin shipout hook (HO)]%
\documentclass{ltxdoc}
\usepackage{holtxdoc}[2011/11/22]
\begin{document}
  \DocInput{atbegshi.dtx}%
\end{document}
%</driver>
% \fi
%
%
% \CharacterTable
%  {Upper-case    \A\B\C\D\E\F\G\H\I\J\K\L\M\N\O\P\Q\R\S\T\U\V\W\X\Y\Z
%   Lower-case    \a\b\c\d\e\f\g\h\i\j\k\l\m\n\o\p\q\r\s\t\u\v\w\x\y\z
%   Digits        \0\1\2\3\4\5\6\7\8\9
%   Exclamation   \!     Double quote  \"     Hash (number) \#
%   Dollar        \$     Percent       \%     Ampersand     \&
%   Acute accent  \'     Left paren    \(     Right paren   \)
%   Asterisk      \*     Plus          \+     Comma         \,
%   Minus         \-     Point         \.     Solidus       \/
%   Colon         \:     Semicolon     \;     Less than     \<
%   Equals        \=     Greater than  \>     Question mark \?
%   Commercial at \@     Left bracket  \[     Backslash     \\
%   Right bracket \]     Circumflex    \^     Underscore    \_
%   Grave accent  \`     Left brace    \{     Vertical bar  \|
%   Right brace   \}     Tilde         \~}
%
% \GetFileInfo{atbegshi.drv}
%
% \title{The \xpackage{atbegshi} package}
% \date{2016/06/09 v1.18}
% \author{Heiko Oberdiek\thanks
% {Please report any issues at https://github.com/ho-tex/oberdiek/issues}\\
% \xemail{heiko.oberdiek at googlemail.com}}
%
% \maketitle
%
% \begin{abstract}
% This package is a modern reimplementation of package \xpackage{everyshi}
% without the burden of compatibility. It makes use of \eTeX's if available.
% Both \LaTeX\ and \plainTeX\ are supported.
% \end{abstract}
%
% \tableofcontents
%
% \section{Documentation}
%
% Package \xpackage{atbegshi} redefines \cs{shipout} to insert
% hooks for user code that is executed before the page is shipped out.
% The code may modify or even discard the output page.
% Three hooks are implemented:
% \begin{enumerate}
% \item A hook that is executed for every page, see\\
%       \cs{AtBeginShipout}
% \item A hook that is executed for the next page only, see\\
%       \cs{AtBeginShipoutNext}
% \item A hook that is only executed for the first page, see\\
%       \cs{AtBeginShipoutFirst}
% \end{enumerate}
% The hooks are executed in this order. The following three macros
% provide the user interface for adding code to these hooks:
%
% \begin{declcs}{AtBeginShipout} \M{code}\\
%   \cs{AtBeginShipoutBox}
% \end{declcs}
% Execute the \meta{code} for every page.
% The page contents is held in box register \cs{AtBeginShipoutBox}
% and may be modified. Use \cs{AtBeginShipoutDiscard} if you want to
% discard the page.
%
% \emph{Note:} Package \xpackage{everyshi} uses box register 255. With
% package \xpackage{atbegshi} you must use \cs{AtBeginShipoutBox} instead.
%
% If \LaTeX\ calls \cs{shipout} in \cs{@outputpage} (part of its output
% routine),
% the meaning of \cs{protect} is \cs{noexpand}. \LaTeX\ sets
% \cs{protect} to the appropriate \cs{@typeset@protect} in the box
% that is shipped out. This is too late for the hooks, they
% are called earlier in the redefined \cs{shipout}. Therefore
% package \xpackage{atbegshi} sets \cs{protect} to \cs{@typeset@protect}
% before it calls the hooks.
% (In \cs{EveryShipout} of package \xpackage{everyshi} the
% user is responsible for the correct setting of \cs{protect}.)
%
% \begin{declcs}{AtBeginShipoutNext} \M{code}
% \end{declcs}
% This reimplements package \xpackage{everyshi}'s \cs{AtNextShipout}.
% The \meta{code} is executed at shipout time of the next page only.
% It is just a convenience macro, it can be easily replaced by something
% like:
%\begin{quote}
%\begin{verbatim}
%\newcommand{\MyShipoutHook}{}%
%\AtBeginShipout{\MyShipoutHook}
%\gdef\MyShipoutHook{%
%  ... do something with next page ...
%  \gdef\MyShipoutHook{}%
% }
%\end{verbatim}
%\end{quote}
% (This can be necessary, if hook order does matter).
%
% \begin{declcs}{AtBeginShipoutFirst} \M{code}
% \end{declcs}
% This reimplements \LaTeX's \cs{AtBeginDvi}. This hook is
% usually used for \cs{special} commands that include PostScript
% header files. The \cs{code} is directly executed in a \cs{vbox}
% that is put at the beginning of the output page. Dealing with
% the output box \cs{AtBeginShipoutBox} is not necessary and
% not permitted here.
%
% \begin{declcs}{AtBeginShipoutDiscard}
% \end{declcs}
% This macro notifies package \xpackage{atbegshi} that
% the output page is discarded. The remaining hook code
% and the remaining hooks are not executed and the page is thrown away.
% Also \cs{deadcycles} is cleared to zero like an ordinary \cs{shipout}
% would do.
%
% \begin{declcs}{AtBeginShipoutInit}
% \end{declcs}
% Usually the redefinition of \cs{shipout} is delayed by
% \cs{AtBeginDocument} (if this macro exists). This can be
% too late, if other packages also redefines \cs{shipout}
% and the order does matter.
% \cs{AtBeginShipoutInit} forces the immediate redefinition
% of \cs{shipout}.
%
% \begin{declcs}{AtBeginShipoutAddToBox} \M{stuff}\\
%   \cs{AtBeginShipoutAddToBoxForeground} \M{stuff}
% \end{declcs}
% A quite common use case is the addition of \cs{special}
% or other whatsits to the page output box. Macro
% \cs{AtBeginShipoutAddToBox} puts \meta{stuff} in a
% box with zeroed dimensions. The box with the \meta{stuff}
% is put in the upper left corner of the shipout box
% \cs{AtBeginShipoutBox}. Macro \cs{AtBeginShipoutAddToBox}
% puts the \meta{stuff} in the background, the other macro
% \cs{AtBeginShipoutAddToBoxForeground} in the foreground
% after the original shipout box contents is set.
%
% A void shipout box (that means a discarded page)
% remains void that means \meta{stuff} is ignored in this case.
% The box type of \cs{AtBeginShipoutBox} is preserved.
% Also the box nesting level for the original contents
% of \cs{AtBeginShipoutBox} remains, for example, to avoid trouble
% with links across pages in case of \hologo{pdfTeX}.
%
% \begin{declcs}{AtBeginShipoutUpperLeft} \M{background material}
% \end{declcs}
% This is a macro that puts material in the background of
% box \cs{AtBeginShipoutBox}.
% The \meta{background material} is set in an \cs{hbox}, the
% reference point is the upper left corner of the output page.
% In case of \pdfTeX\ in PDF mode, the settings of
% \cs{pdfhorigin} and \cs{pdfvorigin} are respected.
%
% The macro \cs{AtBeginShipoutUpperLeft} is intended to be used
% in one of the hook setting macros, such as
% \cs{AtBeginShipout}, \cs{AtBeginShipoutFirst}, or
% \cs{AtBeginShipoutNext}.
%
% For \LaTeX\ users the \meta{background material} is set
% inside a \texttt{picture} environment:
% \begin{quote}
%   |\begin{picture}(0,0)|\\
%   \mbox{}\quad |\setlength{\unitlength}{1pt}%|\\
%   \mbox{}\quad \meta{background material}\\
%   |\end{picture}|
% \end{quote}
%
% \begin{declcs}{AtBeginShipoutUpperLeftForeground} \M{foreground material}
% \end{declcs}
% See \cs{AtBeginShipoutUpperLeft}. The difference is that the material
% is put in the foreground.
%
% \begin{declcs}{AtBeginShipoutOriginalShipout} \meta{box}
% \end{declcs}
% It stores the meaning of \cs{shipout} at the time this package
% is loaded.
%
% \begin{declcs}{AtBeginShipoutBoxWidth}\\
%   \cs{AtBeginShipoutBoxHeight}\\
%   \cs{AtBeginShipoutBoxDepth}\\
% \end{declcs}
% These macros store the dimensions of the output box \cs{AtBeginShipoutBox}
% before the original shipout is called. If \cs{shipout} is not
% redefined before the package loading or the box dimensions
% are not changed by the redefined \cs{shipout}, these macros
% contain the dimensions of the shipout box. These values can be
% remembered by \cs{label} and \cs{ref}. For example, this is done
% by the package module \xpackage{zref-pagelayout} of project \xpackage{zref}.
% The dimensions of the shipout page can be used in some \hologo{TeX}
% engines (\hologo{pdfTeX} in PDF mode, \hologo{XeTeX}) to calculate
% the media size of the shipout page if \cs{pdfpagewidth} and
% \cs{pdfpageheight} are not set.
%
% \subsection{Examples}
%
% \subsubsection{Example: circle in background}
%
% In this example we put a circle in the background in the middle
% of the paper.
%
%    \begin{macrocode}
%<*example1>
%    \end{macrocode}
%    \begin{macrocode}
\documentclass[a4paper]{article}
\usepackage{color}
\usepackage{atbegshi}
%    \end{macrocode}
%    Package \xpackage{picture} makes life a little easier, because
%    we can now also use length specifications in |picture|'s commands.
%    \begin{macrocode}
\usepackage{picture}
%    \end{macrocode}
%    Now we draw the circle in the middle of the paper.
%    \cs{put} moves downwards, because the origin is at the top
%    of the page, not at its bottom.
%    \begin{macrocode}
\AtBeginShipout{%
  \AtBeginShipoutUpperLeft{%
    \put(0.5\paperwidth,-0.5\paperheight){\circle{10}}%
  }%
}
\begin{document}
\section{Hello World}
\newpage
\AtBeginShipoutNext{%
  \AtBeginShipoutUpperLeft{%
    \color{red}%
    \put(0,-0.5\paperheight){\line(1,0){\paperwidth}}%
    \put(0.5\paperwidth, 0){\line(0,-1){\paperheight}}%
  }%
}
Only on this page we add a red cross.
\newpage
This page has the circle only.
\par
\vspace{\fill}
The next page will be discarded.
\newpage
\AtBeginShipoutNext{%
  \AtBeginShipoutDiscard
}
This page is discarded.
\newpage
The last page.
\end{document}
%    \end{macrocode}
%    \begin{macrocode}
%</example1>
%    \end{macrocode}
%
% \subsubsection{Example: adding TrimBox for dvipdfmx}
%
% Now an example from ``real life'' follows. Someone from the mailing  list
% for dvipdfmx wants to put a TrimBox on every page. If we use
% \cs{AtBeginShipout}, we have to put the \cs{special} inside the
% box \cs{AtBeginShipoutBox} that gets shipped out.
%    \begin{macrocode}
%<*example2>
\documentclass{minimal}
\usepackage{atbegshi}
\usepackage[
  dvipdfm,
  paperwidth=630bp,
  paperheight=810bp
]{geometry}
\AtBeginShipout{%
  \setbox\AtBeginShipoutBox=\hbox{%
    \special{pdf: put @thispage <</TrimBox[9 9 621 801]>>}%
    \box\AtBeginShipoutBox
  }%
}
\begin{document}
  First page
  \newpage
  Second page
\end{document}
%</example2>
%    \end{macrocode}
% Remember, in \cs{AtBeginShipoutBoxFirst} the \cs{setbox} wrapper code
% is implicitly given and the \cs{special} is used directly.
%
% \section{Method of \cs{shipout} overloading}
%
% \subsection{\cs{shipout}}
%
% The \TeX\ primitive command \cs{shipout} takes a box specification
% and puts the box as a new page in the output file.
% There are two kinds of box specifications:
% \begin{description}
% \item[Direct boxes:]
%   They are given by \cs{hbox}, \cs{vbox}, or \cs{vtop},\\
%   e.g. |\shipout\hbox{Hello World}|.
% \item[Indirect boxes:]
%   \cs{box} or \cs{copy} references a box register by number.
%   The box register contains the contents of the box.
%
%   \emph{Note:} \cs{box} also clears the box register globally.
% \end{description}
% Then we have to differentiate between void and empty boxes:
% \begin{description}
% \item[Void:]
%   Initially or after \cs{box} there is no box in the box register.
%   In this cases the box register is not empty, but \emph{void}.
% \item[Empty:]
%   A box with empty contents, such as
%   |\hbox{}| (= \cs{null}) or |\vbox{}| is an \emph{empty \cs{hbox}}
%   or \emph{empty \cs{vbox}}. If a box register holds such a
%   box, the box still exists, therefore the box register is
%   \emph{not void}.
% \end{description}
%
% \subsection{\cs{afterassignment}}
%
% We want to overload \cs{shipout} to do something with the box.
% It is quite impossible to do this reliable by catching the box
% using macro arguments. The
% variety of box specifications is too large, Examples:
% \begin{quote}
%   |\shipout\null|\\
%   |\shipout\vbox{...}|\\
%   |\shipout\vtop\bgroup ...\egroup|\\
%   |\shipout\box255|
% \end{quote}
% Even worse, the braces don't need to be balanced:
% \begin{quote}
%   |\shipout\hbox\bgroup}|\\
%   |\shipout\vbox{\egroup|
% \end{quote}
% Happily TeX provides a reliable way via \cs{afterassignment}.
% It takes a macro name and executes it just after the assignment.
%
% Now we can redefine \cs{shipout}. The box specification that
% follows \cs{shipout} is catched by \cs{setbox}. This is an
% assignment to a box register. \cs{afterassignment}
% notifies \TeX, that we want to call \cs{@test} right after
% the assignment:
%\begin{quote}
%\begin{verbatim}
%\shipout :=
%  \afterassignment\@test
%  \setbox\mybox=
%\end{verbatim}
%\end{quote}
% We have seen different box specifications. Indirect boxes are easy
% to understand:
% \begin{quote}
%   |\shipout\box0| $\Rightarrow$ |\setbox\mybox=\box0 |\fbox{\cs{@test}}
% \end{quote}
% However direct boxes can have arbitrary contents with lots of
% other assignments. It would be quite inpredictable if \TeX\ would
% put |\@test| after the first of such an assignment or after the
% box specification if the box lacks of assignments.
% Therefore \TeX\ puts |\@test| right at the beginning of the
% box specification, e.g:
% \begin{quote}
%   |\shipout\hbox{Hello World}|\\
%   $\Rightarrow$
%   |\setbox\mybox=\hbox{|\fbox{\cs{@test}}| Hello World}|
% \end{quote}
%
% \subsection{Test for direct or indirect boxes}
%
% Now we want to execute \cs{@test}, but where are we?
% We can be after the completed box assignment, if \cs{shipout}
% was called with an indirect box. Or we are right at the beginning
% of a direct box.
%
% \subsubsection{With \eTeX}
%
% With the \eTeX's extensions the answer is very easy:
% Being inside the direct box means that we are inside a new group.
% The new primitive command \cs{currentgrouplevel} tells how
% deeply the groups are currently nested. Macro \cs{@test} just
% compares the previously stored group level with the current one:
%\begin{quote}
%\begin{verbatim}
%\shipout :=
%  \edef\saved@grouplevel{\number\currentgrouplevel}
%  \afterassignment\@test
%  \setbox\mybox=
%
%\@test :=
%  \ifnum\saved@grouplevel=\currentgrouplevel
%    % case: indirect box, the assignment is completed
%    \@output
%  \else
%    % case: direct box, we are inside the box
%    \aftergroup\@outbox
%  \fi
%\end{verbatim}
%\end{quote}
%
% \subsubsection{Without \eTeX}
%
% Life becomes complicate without \eTeX. We cannot ask the
% group level. However, if we are inside a direct box,
% the box register \cs{mybox} is not yet changed by \cs{setbox}.
% Thus we need a special initial value and compare
% it in \cs{@test} with the current value of the box.
%
% What can be used as initial value? Arbitrary box contents cannot
% be compared. \TeX\ only tells us a few properties:
% \begin{itemize}
% \item Box type:\cs{ifhbox}, \cs{ifvbox}
% \item Dimensions: \cs{wd}, \cs{ht}, \cs{dp}
% \item Voidness: \cs{ifvoid}
% \end{itemize}
% Unhappily all these qualities even combined are not sufficient
% for constructing an initial box value, because \cs{shipout} can be
% called with a box that is accidently just the same as the choosen
% initial value.
%
% Nevertheless we have two alternatives for an initial value:
% \begin{itemize}
% \item A box of some type with some funny settings that are
% unlikely to occur in real life, e.g a height of |4911sp-\maxdimen|.
% \item A void box.
% \end{itemize}
% A collision between this initial value and an indirect \cs{shipout}
% box with just the same value is possible.
% Then \cs{@test} will make a wrong decision that it is executed
% inside a direct box and delays \cs{@output} by \cs{aftergroup}.
% Thus \cs{@output} is not called at the place we want.
% In contrary, the result is an uncertainty about the place:
% \begin{itemize}
% \item \cs{shipout} is used in a group that perhaps closes some
%   pages later. A bad place for \cs{@output}.
% \item Without a surrounding group \cs{aftergroup} effectively kills
%   its argument.
% \end{itemize}
% In the first case of a box with special dimensions we can even loose
% the page. However in the case of the void box, this effect is even
% desired, because the original \cs{shipout} does not output void boxes.
% All we have to do is to ensure that our box \cs{mybox} is always
% void except for the phase when the overloaded \cs{shipout} is executed.
% And secondly we must keep this semantics of \cs{shipout} for the void case
% in our macros, namely \cs{@output}.
%\begin{quote}
%\begin{verbatim}
%\shipout :=
%  % trick to get a void box \mybox
%  \begingroup
%    \setbox\mybox=\box\mybox
%  \endgroup
%  \afterassignment\@test
%  \setbox\mybox=
%
%\@test :=
%  \ifvoid\mybox
%    \aftergroup\@output
%  \else
%    \@output
%  \fi
%\end{verbatim}
%\end{quote}
% The nasty case is |\shipout\box\voidb@x| where the indirect box
% is void and that must not generate an output page.
% If a surrounding group is missing the output is ignored
% because of \cs{aftergroup}.
% Otherwise \@output is called some time
% later when the surrounding group closes.
% But \cs{mybox} is void outside the execution phase
% of the redefined \cs{shipout}. Also \cs{@output} checks for a
% void box and cancels the page output.
% The disadvantage remains that the hook in \cs{@output} is called
% for a page that will not be output.
%
% \subsubsection{\cs{lastkern} method}
%
% At the beginning of a new box, there is no \cs{kern}, the contents
% of the box is still empty and \cs{lastkern} returns 0\,pt.
% This can be used to distinguish between direct and indirect boxes:
% We execute \cs{setbox} in a box with a preceding non-zero kern.
% After an indirect box, \cs{lastkern} sees this kern, otherwise
% it returns 0\,pt.
%\begin{quote}
%\begin{verbatim}
%\shipout :=
%  \begingroup
%    \setbox\mybox=\hbox\bgroup
%      \kern1pt
%      \afterassignment\shipout@test
%      \global\setbox\mybox=
%\@test :=
%  \ifdim\lastkern=0pt
%    % direct box
%    \aftergroup\egroup
%    \aftergroup\endgroup
%    \aftergroup\@output
%  \else
%    \egroup
%    \endgroup
%    \@output
%  \fi
%\end{verbatim}
%\end{quote}
% We have two \cs{setbox} commands. The first creates a controlled
% context box where we can safely insert a \cs{kern}. We get rid of
% this temporarily used context box by putting the local \cs{setbox}
% in a group.
%
% After the group we want to have our shipout box in \cs{mybox}.
% Therefore we use a global assignment here.
%
% \subsection{Output}
%
% With or without \eTeX\ we ensure the original behaviour of
% \cs{shipout} that void boxes do not generate output pages.
%
% Now we can place the hook \cs{@hook} for the user code that
% wants to manipulate the output box.
%\begin{quote}
%\begin{verbatim}
%\@output :=
%  \ifvoid\mybox
%    % cancel output of void box
%  \else
%    \@hook
%    \ifvoid\mybox
%      % user code in \@hook could has voided the box
%    \else
%      \original@shipout\box\mybox
%    \fi
%  \fi
%\end{verbatim}
%\end{quote}
%
% \subsection{Separate box register}
%
% So far we have said nothing about the box number of \cs{mybox}.
% The following case that outputs the same page twice
% shows that we are not free in the use
% of the box register:
% \begin{quote}
%   |\shipout\copy<num> \shipout\box<num>|
% \end{quote}
% We manipulate the box by the hook and without \eTeX\ the box
% must even be voided. However, the use case above requires that
% the box contents does not change at all. Therefore we must
% reserve a separate box register to avoid collisions with
% user box registers.
%
% \emph{Note:} Box register number 255 is special for the output
% routine, because TeX complains if this box is not voided
% by the output routine.
% However, this requirement does not apply to \cs{shipout} at all.
% In fact \cs{shipout} does not change any box register. This is
% usually done by a call of \cs{box}, but the output routine
% can do it later \emph{after} invoking of \cs{shipout}.
%
% \subsection{Summary}
%
% \subsubsection{With \eTeX}
%
% Putting the pieces together we get for \eTeX:
%\begin{quote}
%\begin{verbatim}
%\newbox\mybox
%\let\original@shipout\shipout
%
%\shipout :=
%  \edef\saved@grouplevel{\number\currentgrouplevel}
%  \afterassignment\@test
%  \setbox\mybox=
%
%\@test :=
%  \ifnum\saved@grouplevel<\currentgrouplevel
%    \expandafter\aftergroup
%  \fi
%  \@output
%
%\@output :=
%  \ifvoid\mybox
%    % cancel output of void box
%  \else
%    \@hook
%    \ifvoid\mybox
%      % user code in \@hook could have voided the box
%    \else
%      \original@shipout\box\mybox
%    \fi
%  \fi
%\end{verbatim}
%\end{quote}
%
% \subsubsection{Without \eTeX, traditional way}
%
% And for \TeX\ without \eTeX:
%\begin{quote}
%\begin{verbatim}
%\newbox\mybox
%\begingroup
%   \setbox\mybox=\box\mybox % ensure \mybox is void
%\endgroup
%\let\original@shipout\shipout
%
%\shipout :=
%  % trick to get a void box \mybox
%  \begingroup
%    \setbox\mybox=\box\mybox
%  \endgroup
%  \afterassignment\@test
%  \setbox\mybox=
%
%\@test :=
%  \ifvoid\mybox
%    \expandafter\aftergroup
%  \fi
%  \@output
%
%\@output :=
%  \ifvoid\mybox
%    % cancel output of void box
%  \else
%    \@hook
%    \ifvoid\mybox
%      % user code in \@hook could have voided the box
%    \else
%      \original@shipout\box\mybox
%    \fi
%  \fi
%\end{verbatim}
%\end{quote}
%
% \subsubsection{\cs{lastkern} method}
%
% And for \TeX\ without \eTeX\ using the \cs{lastkern} method:
%\begin{quote}
%\begin{verbatim}
%\newbox\mybox
%\let\original@shipout\shipout
%
%\shipout :=
%  \begingroup
%  \setbox\mybox=\hbox\bgroup
%  \kern1pt
%  \afterassignment\@test
%  \setbox\mybox=
%
%\@test :=
%  \ifdim\lastkern=0pt
%    \expandafter\aftergroup
%  \fi
%  \@output
%
%\@output :=
%  \egroup
%  \endgroup
%  \ifvoid\mybox
%    % cancel output of void box
%  \else
%    \@hook
%    \ifvoid\mybox
%      % user code in \@hook could have voided the box
%    \else
%      \original@shipout\box\mybox
%    \fi
%  \fi
%\end{verbatim}
%\end{quote}
%
%
% \StopEventually{
% }
%
% \section{Implementation}
%
% Package \xpackage{atbegshi} uses \eTeX's \cs{currentgrouplevel},
% if it is available. Otherwise the \cs{lastkern} method is used.
%
%    \begin{macrocode}
%<*package>
%    \end{macrocode}
%
% \subsection{Reload check and package identification}
%    Reload check, especially if the package is not used with \LaTeX.
%    \begin{macrocode}
\begingroup\catcode61\catcode48\catcode32=10\relax%
  \catcode13=5 % ^^M
  \endlinechar=13 %
  \catcode35=6 % #
  \catcode39=12 % '
  \catcode44=12 % ,
  \catcode45=12 % -
  \catcode46=12 % .
  \catcode58=12 % :
  \catcode64=11 % @
  \catcode123=1 % {
  \catcode125=2 % }
  \expandafter\let\expandafter\x\csname ver@atbegshi.sty\endcsname
  \ifx\x\relax % plain-TeX, first loading
  \else
    \def\empty{}%
    \ifx\x\empty % LaTeX, first loading,
      % variable is initialized, but \ProvidesPackage not yet seen
    \else
      \expandafter\ifx\csname PackageInfo\endcsname\relax
        \def\x#1#2{%
          \immediate\write-1{Package #1 Info: #2.}%
        }%
      \else
        \def\x#1#2{\PackageInfo{#1}{#2, stopped}}%
      \fi
      \x{atbegshi}{The package is already loaded}%
      \aftergroup\endinput
    \fi
  \fi
\endgroup%
%    \end{macrocode}
%    Package identification:
%    \begin{macrocode}
\begingroup\catcode61\catcode48\catcode32=10\relax%
  \catcode13=5 % ^^M
  \endlinechar=13 %
  \catcode35=6 % #
  \catcode39=12 % '
  \catcode40=12 % (
  \catcode41=12 % )
  \catcode44=12 % ,
  \catcode45=12 % -
  \catcode46=12 % .
  \catcode47=12 % /
  \catcode58=12 % :
  \catcode64=11 % @
  \catcode91=12 % [
  \catcode93=12 % ]
  \catcode123=1 % {
  \catcode125=2 % }
  \expandafter\ifx\csname ProvidesPackage\endcsname\relax
    \def\x#1#2#3[#4]{\endgroup
      \immediate\write-1{Package: #3 #4}%
      \xdef#1{#4}%
    }%
  \else
    \def\x#1#2[#3]{\endgroup
      #2[{#3}]%
      \ifx#1\@undefined
        \xdef#1{#3}%
      \fi
      \ifx#1\relax
        \xdef#1{#3}%
      \fi
    }%
  \fi
\expandafter\x\csname ver@atbegshi.sty\endcsname
\ProvidesPackage{atbegshi}%
  [2016/06/09 v1.18 At begin shipout hook (HO)]%
%    \end{macrocode}
%
% \subsection{Catcodes}
%
%    \begin{macrocode}
\begingroup\catcode61\catcode48\catcode32=10\relax%
  \catcode13=5 % ^^M
  \endlinechar=13 %
  \catcode123=1 % {
  \catcode125=2 % }
  \catcode64=11 % @
  \def\x{\endgroup
    \expandafter\edef\csname AtBegShi@AtEnd\endcsname{%
      \endlinechar=\the\endlinechar\relax
      \catcode13=\the\catcode13\relax
      \catcode32=\the\catcode32\relax
      \catcode35=\the\catcode35\relax
      \catcode61=\the\catcode61\relax
      \catcode64=\the\catcode64\relax
      \catcode123=\the\catcode123\relax
      \catcode125=\the\catcode125\relax
    }%
  }%
\x\catcode61\catcode48\catcode32=10\relax%
\catcode13=5 % ^^M
\endlinechar=13 %
\catcode35=6 % #
\catcode64=11 % @
\catcode123=1 % {
\catcode125=2 % }
\def\TMP@EnsureCode#1#2{%
  \edef\AtBegShi@AtEnd{%
    \AtBegShi@AtEnd
    \catcode#1=\the\catcode#1\relax
  }%
  \catcode#1=#2\relax
}
\TMP@EnsureCode{40}{12}% (
\TMP@EnsureCode{41}{12}% )
\TMP@EnsureCode{44}{12}% ,
\TMP@EnsureCode{45}{12}% -
\TMP@EnsureCode{47}{12}% /
\TMP@EnsureCode{46}{12}% .
\TMP@EnsureCode{58}{12}% :
\TMP@EnsureCode{91}{12}% [
\TMP@EnsureCode{93}{12}% ]
\TMP@EnsureCode{94}{7}% ^(superscript)
\TMP@EnsureCode{96}{12}% `
\edef\AtBegShi@AtEnd{\AtBegShi@AtEnd\noexpand\endinput}
%    \end{macrocode}
%
% \subsection{Preparations}
%
%    \begin{macrocode}
\begingroup\expandafter\expandafter\expandafter\endgroup
\expandafter\ifx\csname RequirePackage\endcsname\relax
  \def\TMP@RequirePackage#1[#2]{%
    \begingroup\expandafter\expandafter\expandafter\endgroup
    \expandafter\ifx\csname ver@#1.sty\endcsname\relax
      \input #1.sty\relax
    \fi
  }%
  \TMP@RequirePackage{infwarerr}[2007/09/09]%
  \TMP@RequirePackage{ltxcmds}[2010/03/01]%
\else
  \RequirePackage{infwarerr}[2007/09/09]%
  \RequirePackage{ltxcmds}[2010/03/01]%
\fi
%    \end{macrocode}
%
%    \begin{macro}{\AtBegShi@CheckDefinable}
%    \begin{macrocode}
\begingroup\expandafter\expandafter\expandafter\endgroup
\expandafter\ifx\csname @ifdefinable\endcsname\relax
  \def\AtBegShi@CheckDefinable#1{%
    \ifcase\ifx#1\relax
             \ltx@one
           \else
             \ifx#1\@undefined
               \ltx@one
             \else
               \ltx@zero
             \fi
           \fi
      \@PackageError{atbegshi}{%
        \string#1\space is already defined%
      }\@ehd
    \fi
  }%
\else
  \def\AtBegShi@CheckDefinable#1{%
    \@ifdefinable{#1}{}%
  }%
\fi
%    \end{macrocode}
%    \end{macro}
%
%    \begin{macro}{\ifAtBegShi@Discarded}
%    \begin{macrocode}
\ltx@newif\ifAtBegShi@Discarded
%    \end{macrocode}
%    \end{macro}
%    \begin{macro}{\AtBeginShipoutDiscard}
%    \begin{macrocode}
\AtBegShi@CheckDefinable\AtBeginShipoutDiscard
\def\AtBeginShipoutDiscard{%
  \deadcycles=\ltx@zero
  \global\AtBegShi@Discardedtrue
}
%    \end{macrocode}
%    \end{macro}
%
%    \begin{macrocode}
\begingroup\expandafter\expandafter\expandafter\endgroup
\expandafter\ifx\csname currentgrouplevel\endcsname\relax
  \catcode`X=9  % ignore
  \catcode`E=14 % comment
\else
  \catcode`X=14 % comment
  \catcode`E=9  % ignore
\fi
%    \end{macrocode}
%
%    \begin{macro}{\AtBegShi@Shipout}
%    \begin{macrocode}
\def\AtBegShi@Shipout{%
X \begingroup
X \setbox\AtBeginShipoutBox=\hbox\bgroup
X \kern\p@
E \edef\AtBegShi@GroupLevel{\number\currentgrouplevel}%
  \afterassignment\AtBegShi@Test
X \global
  \setbox\AtBeginShipoutBox=%
}
%    \end{macrocode}
%    \end{macro}
%
%    \begin{macro}{\AtBegShi@Test}
%    \begin{macrocode}
\def\AtBegShi@Test{%
X \ifdim\lastkern=0pt %
E \ifnum\AtBegShi@GroupLevel<\currentgrouplevel
    \expandafter\aftergroup
  \fi
  \AtBegShi@Output
}
%    \end{macrocode}
%    \end{macro}
%
%    \begin{macro}{\AtBegShi@Output}
%    \begin{macrocode}
\def\AtBegShi@Output{%
X \egroup
X \endgroup
  \ifvoid\AtBeginShipoutBox
    \@PackageWarning{atbegshi}{Ignoring void shipout box}%
  \else
    \let\AtBegShi@OrgProtect\protect
    \csname set@typeset@protect\endcsname
    \global\AtBegShi@Discardedfalse
    \AtBegShi@Hook
    \expandafter\gdef\expandafter\AtBegShi@HookNext
    \expandafter{\expandafter}%
    \AtBegShi@HookNext
    \ifAtBegShi@Discarded
      \@PackageInfoNoLine{atbegshi}{Shipout page discarded}%
      \global\AtBegShi@Discardedfalse
      \begingroup
        \setbox\AtBeginShipoutBox\box\AtBeginShipoutBox
      \endgroup
      \let\protect\AtBegShi@OrgProtect
    \else
      \AtBegShi@First
      \let\protect\AtBegShi@OrgProtect
      \AtBegShi@GetBoxSize\AtBeginShipoutBox
      \ltx@ifundefined{AtNextShipout}{%
      }{%
        \AtNextShipout{\AtBegShi@GetBoxSize\@cclv}%
      }%
      \AtBeginShipoutOriginalShipout\box\AtBeginShipoutBox
    \fi
  \fi
}
%    \end{macrocode}
%    \end{macro}
%
%    \begin{macro}{\AtBegShi@GetBoxSize}
%    \begin{macrocode}
\def\AtBegShi@GetBoxSize#1{%
  \xdef\AtBeginShipoutBoxWidth{\the\wd#1}%
  \xdef\AtBeginShipoutBoxHeight{\the\ht#1}%
  \xdef\AtBeginShipoutBoxDepth{\the\dp#1}%
}
%    \end{macrocode}
%    \end{macro}
%    \begin{macro}{\AtBeginShipoutBoxWidth}
%    \begin{macrocode}
\def\AtBeginShipoutBoxWidth{0pt}
%    \end{macrocode}
%    \end{macro}
%    \begin{macro}{\AtBeginShipoutBoxHeight}
%    \begin{macrocode}
\def\AtBeginShipoutBoxHeight{0pt}
%    \end{macrocode}
%    \end{macro}
%    \begin{macro}{\AtBeginShipoutBoxDepth}
%    \begin{macrocode}
\def\AtBeginShipoutBoxDepth{0pt}
%    \end{macrocode}
%    \end{macro}
%
%    \begin{macrocode}
\catcode`\X=11 %
\catcode`\E=11 %
%    \end{macrocode}
%
%    \begin{macro}{\AtBegShi@First}
%    \begin{macrocode}
\def\AtBegShi@First{%
  \ifx\AtBegShi@HookFirst\ltx@empty
  \else
    \AtBeginShipoutAddToBox{\AtBegShi@HookFirst}%
  \fi
  \global\let\AtBegShi@First\ltx@empty
  \global\let\AtBeginShipoutFirst\AtBegShi@FirstDisabled
}
%    \end{macrocode}
%    \end{macro}
%
%    \begin{macro}{\AtBegShi@Hook}
%    \begin{macrocode}
\gdef\AtBegShi@Hook{}
%    \end{macrocode}
%    \end{macro}
%    \begin{macro}{\AtBegShi@HookNext}
%    \begin{macrocode}
\gdef\AtBegShi@HookNext{}
%    \end{macrocode}
%    \end{macro}
%    \begin{macro}{\AtBegShi@HookFirst}
%    \begin{macrocode}
\gdef\AtBegShi@HookFirst{}
%    \end{macrocode}
%    \end{macro}
%    \begin{macro}{\AtBeginShipout}
%    \begin{macrocode}
\AtBegShi@CheckDefinable\AtBeginShipout
\def\AtBeginShipout{%
  \AtBegShi@AddHook\AtBegShi@Hook
}
%    \end{macrocode}
%    \end{macro}
%    \begin{macro}{\AtBeginShipoutNext}
%    \begin{macrocode}
\AtBegShi@CheckDefinable\AtBeginShipoutNext
\def\AtBeginShipoutNext{%
  \AtBegShi@AddHook\AtBegShi@HookNext
}
%    \end{macrocode}
%    \end{macro}
%    \begin{macro}{\AtBeginShipoutFirst}
%    \begin{macrocode}
\AtBegShi@CheckDefinable\AtBeginShipoutFirst
\def\AtBeginShipoutFirst{%
  \AtBegShi@AddTo\AtBegShi@HookFirst
}
%    \end{macrocode}
%    \end{macro}
%    \begin{macro}{\AtBegShi@FirstDisabled}
%    \begin{macrocode}
\long\def\AtBegShi@FirstDisabled#1{%
  \@PackageWarning{atbegshi}{%
    First page is already shipped out, ignoring\MessageBreak
    \string\AtBeginShipoutFirst
  }%
}
%    \end{macrocode}
%    \end{macro}
%
%    \begin{macro}{\AtBegShi@AddTo}
%    \begin{macrocode}
\begingroup\expandafter\expandafter\expandafter\endgroup
\expandafter\ifx\csname g@addto@macro\endcsname\relax
  \long\def\AtBegShi@AddTo#1#2{%
    \begingroup
      \toks\ltx@zero\expandafter{#1#2}%
      \xdef#1{\the\toks\ltx@zero}%
    \endgroup
  }%
\else
  \let\AtBegShi@AddTo\g@addto@macro
\fi
%    \end{macrocode}
%    \end{macro}
%
%    \begin{macro}{\AtBegShi@AddHook}
%    \begin{macrocode}
\long\def\AtBegShi@AddHook#1#2{%
  \AtBegShi@AddTo#1{\AtBegShi@Item{#2}}%
}
%    \end{macrocode}
%    \end{macro}
%
%    \begin{macro}{\AtBegShi@Item}
%    \begin{macrocode}
\long\def\AtBegShi@Item#1{%
  \ifAtBegShi@Discarded
  \else
    #1%
    \ifAtBegShi@Discarded
    \else
      \ifvoid\AtBeginShipoutBox
        \@PackageWarning{atbegshi}{%
          Shipout box was voided by hook,\MessageBreak
          ignoring shipout box%
        }%
        \AtBeginShipoutDiscard
      \fi
    \fi
  \fi
}
%    \end{macrocode}
%    \end{macro}
%
%    \begin{macro}{\AtBeginShipoutInit}
%    \begin{macrocode}
\AtBegShi@CheckDefinable\AtBeginShipoutInit
\def\AtBeginShipoutInit{%
  \ltx@IfUndefined{newbox}{%
    \@PackageError{atbegshi}{%
      \string\AtBeginShipoutInit\space failed\MessageBreak
       because of missing \expandafter\string\csname newbox\endcsname
    }\@ehc
  }{%
    \csname newbox\endcsname\AtBeginShipoutBox
    \AtBegShi@CheckDefinable\AtBeginShipoutOriginalShipout
    \global\let\AtBeginShipoutOriginalShipout\shipout
    \global\let\shipout\AtBegShi@Shipout
  }%
  \gdef\AtBeginShipoutInit{}%
}
%    \end{macrocode}
%    \end{macro}
%    \begin{macrocode}
\begingroup\expandafter\expandafter\expandafter\endgroup
\expandafter\ifx\csname AtBeginDocument\endcsname\relax
  \AtBeginShipoutInit
\else
  \AtBeginDocument{\AtBeginShipoutInit}%
\fi
%    \end{macrocode}
%
% \subsection{Additions to the shipout box}
%
%    \begin{macro}{\AtBeginShipoutAddToBox}
%    \begin{macrocode}
\def\AtBeginShipoutAddToBox#1{%
  \ifhbox\AtBeginShipoutBox
    \edef\AtBegShi@restore{%
      \hfuzz=\the\hfuzz\relax
      \hbadness=\the\hbadness\relax
    }%
    \hfuzz=1073741823sp\relax
    \hbadness=2147483647\relax
    \setbox\AtBeginShipoutBox=\hbox to \wd\AtBeginShipoutBox{%
      \setbox\ltx@zero=\hbox{%
        \begingroup
          \AtBegShi@restore
          #1%
        \endgroup
      }%
      \wd\ltx@zero=0pt\relax
      \ht\ltx@zero=0pt\relax
      \dp\ltx@zero=0pt\relax
      \raise\ht\AtBeginShipoutBox\copy\ltx@zero
      \unhcopy\AtBeginShipoutBox
    }%
    \AtBegShi@restore
  \else
    \ifvbox\AtBeginShipoutBox
      \edef\AtBegShi@restore{%
        \vfuzz=\the\vfuzz\relax
        \vbadness=\the\vbadness\relax
        \dimen\ltx@zero=\the\dimen\ltx@zero\relax
      }%
      \edef\AtBegShi@restorebox{%
        \ht\AtBeginShipoutBox=\the\ht\AtBeginShipoutBox\relax
        \dp\AtBeginShipoutBox=\the\dp\AtBeginShipoutBox\relax
      }%
      \vfuzz=1073741823sp\relax
      \vbadness=2147483647\relax
      \dimen\ltx@zero=\ht\AtBeginShipoutBox
      \advance\dimen\ltx@zero by \dp\AtBeginShipoutBox
      \setbox\AtBeginShipoutBox=\vbox to \dimen\ltx@zero{%
        \setbox\ltx@zero=\hbox{%
          \begingroup
            \AtBegShi@restore
            #1%
          \endgroup
        }%
        \wd\ltx@zero=0pt\relax
        \ht\ltx@zero=0pt\relax
        \dp\ltx@zero=0pt\relax
        \baselineskip=0pt\relax
        \lineskip=0pt\relax
        \lineskiplimit=0pt\relax
        \copy\ltx@zero
        \unvbox\AtBeginShipoutBox
        \kern0pt%
      }%
      \AtBegShi@restore
      \AtBegShi@restorebox
    \fi
  \fi
}
%    \end{macrocode}
%    \end{macro}
%
%    \begin{macro}{\AtBeginShipoutAddToBoxForeground}
%    \begin{macrocode}
\def\AtBeginShipoutAddToBoxForeground#1{%
  \ifhbox\AtBeginShipoutBox
    \edef\AtBegShi@restore{%
      \hfuzz=\the\hfuzz\relax
      \hbadness=\the\hbadness\relax
    }%
    \hfuzz=1073741823sp\relax
    \hbadness=2147483647\relax
    \setbox\AtBeginShipoutBox=\hbox to \wd\AtBeginShipoutBox{%
      \unhcopy\AtBeginShipoutBox
      \kern-\wd\AtBeginShipoutBox
      \setbox\ltx@zero=\hbox{%
        \begingroup
          \AtBegShi@restore
          #1%
        \endgroup
      }%
      \wd\ltx@zero=0pt\relax
      \ht\ltx@zero=0pt\relax
      \dp\ltx@zero=0pt\relax
      \raise\ht\AtBeginShipoutBox\copy\ltx@zero
      \kern\wd\AtBeginShipoutBox
    }%
    \AtBegShi@restore
  \else
    \ifvbox\AtBeginShipoutBox
      \edef\AtBegShi@restore{%
        \vfuzz=\the\vfuzz\relax
        \vbadness=\the\vbadness\relax
        \dimen\ltx@zero=\the\dimen\ltx@zero\relax
      }%
      \edef\AtBegShi@restorebox{%
        \ht\AtBeginShipoutBox=\the\ht\AtBeginShipoutBox\relax
        \dp\AtBeginShipoutBox=\the\dp\AtBeginShipoutBox\relax
      }%
      \vfuzz=1073741823sp\relax
      \vbadness=2147483647\relax
      \dimen\ltx@zero=\ht\AtBeginShipoutBox
      \advance\dimen\ltx@zero by \dp\AtBeginShipoutBox
      \setbox\AtBeginShipoutBox=\vbox to \dimen\ltx@zero{%
        \setbox\ltx@zero=\hbox{%
          \begingroup
            \AtBegShi@restore
            #1%
          \endgroup
        }%
        \wd\ltx@zero=0pt\relax
        \ht\ltx@zero=0pt\relax
        \dp\ltx@zero=0pt\relax
        \baselineskip=0pt\relax
        \lineskip=0pt\relax
        \lineskiplimit=0pt\relax
        \unvbox\AtBeginShipoutBox
        \kern-\dimen\ltx@zero
        \copy\ltx@zero
        \kern\dimen\ltx@zero
      }%
      \AtBegShi@restore
      \AtBegShi@restorebox
    \fi
  \fi
}
%    \end{macrocode}
%    \end{macro}
%
% \subsection{Positioning}
%
%    \begin{macrocode}
\begingroup\expandafter\expandafter\expandafter\endgroup
\expandafter\ifx\csname RequirePackage\endcsname\relax
  \def\TMP@RequirePackage#1[#2]{%
    \begingroup\expandafter\expandafter\expandafter\endgroup
    \expandafter\ifx\csname ver@#1.sty\endcsname\relax
      \input #1.sty\relax
    \fi
  }%
  \TMP@RequirePackage{ifpdf}[2011/01/30]%
\else
  \RequirePackage{ifpdf}[2011/01/30]%
\fi
%    \end{macrocode}
%
%    \begin{macrocode}
\ifpdf
  \def\AtBegShi@horigin{%
    \ifx\pdfhorigin\@undefined\pdfvariable horigin\else\pdfhorigin\fi}%
  \def\AtBegShi@vorigin{%
    \ifx\pdfvorigin\@undefined\pdfvariable vorigin\else\pdfvorigin\fi}%
\else
  \def\AtBegShi@horigin{72.27pt}%
  \def\AtBegShi@vorigin{72.27pt}%
\fi
%    \end{macrocode}
%
%    \begin{macrocode}
\begingroup
\ifcase
    \expandafter\ifx\csname picture\endcsname\relax
      1%
    \else
      \expandafter\ifx\csname endpicture\endcsname\relax
        1%
      \else
        0%
      \fi
    \fi
  \endgroup
  \def\AtBegShi@BeginPicture{%
    \begingroup
    \picture(0,0)\relax
    \begingroup\expandafter\expandafter\expandafter\endgroup
    \expandafter\ifx\csname unitlength\endcsname\relax
    \else
      \unitlength=1pt\relax
    \fi
    \ignorespaces
  }%
  \def\AtBegShi@EndPicture{%
    \endpicture
    \endgroup
  }%
\else
  \endgroup
  \def\AtBegShi@BeginPicture{%
    \setbox\ltx@zero=\hbox\bgroup
    \begingroup
    \ignorespaces
  }%
  \def\AtBegShi@EndPicture{%
    \endgroup
    \egroup
    \ht\ltx@zero=0pt\relax
    \dp\ltx@zero=0pt\relax
    \copy\ltx@zero
  }%
\fi
%    \end{macrocode}
%
%    \begin{macro}{\AtBeginShipoutUpperLeft}
%    A surrounding \cs{rlap} is not necessary, because
%    the stuff is put in an \cs{hbox} with zero width.
%    \begin{macrocode}
\def\AtBeginShipoutUpperLeft#1{%
  \AtBeginShipoutAddToBox{%
    \kern-\AtBegShi@horigin\relax
    \vbox to 0pt{%
      \kern-\AtBegShi@vorigin\relax
      \AtBegShi@BeginPicture
      #1%
      \AtBegShi@EndPicture
      \vss
    }%
  }%
}
%    \end{macrocode}
%    \end{macro}
%
%    \begin{macro}{\AtBeginShipoutUpperLeftForeground}
%    \begin{macrocode}
\def\AtBeginShipoutUpperLeftForeground#1{%
  \AtBeginShipoutAddToBoxForeground{%
    \kern-\AtBegShi@horigin\relax
    \vbox to 0pt{%
      \kern-\AtBegShi@vorigin\relax
      \AtBegShi@BeginPicture
      #1%
      \AtBegShi@EndPicture
      \vss
    }%
  }%
}
%    \end{macrocode}
%    \end{macro}
%
% \subsection{Patches}
%
%    Patches for \LaTeX\ packages that redefine \cs{shipout}.
%    \LaTeX\ is now supposed to use \eTeX. Thus we do not patch,
%    without \LaTeX\ and \eTeX.
%    \begin{macrocode}
\def\AtBegShi@AbortIfUndefined#1{%
  \begingroup\expandafter\expandafter\expandafter\endgroup
  \expandafter\ifx\csname#1\endcsname\relax
    \expandafter\AtBegShi@AtEnd
  \fi
}
\AtBegShi@AbortIfUndefined{currentgrouplevel}%
\AtBegShi@AbortIfUndefined{AtBeginDocument}%
\AtBegShi@AbortIfUndefined{@ifpackageloaded}%
\AtBegShi@AbortIfUndefined{@ifclassloaded}%
%    \end{macrocode}
%
% \subsubsection{Package \xpackage{crop}}
%
% Fix of method and box.
%    \begin{macrocode}
\def\AtBegShi@PatchCrop{%
  \begingroup
    \def\AtBegShi@Crop@shipout{%
      \afterassignment\CROP@ship
      \setbox\@cclv=%
    }%
    \def\AtBegShi@Crop@ship{%
      \ifvoid\@cclv
        \expandafter\aftergroup
      \fi
      \CROP@@ship
    }%
    \def\AtBegShi@Crop@shiplist{%
      \lineskip\z@
      \lineskiplimit\z@
      \baselineskip\z@
      \CROP@kernel
      \box\@cclv
    }%
    \def\AtBegShi@Crop@@ship{%
      \CROP@shipout\vbox{%
        \CROP@shiplist
      }%
    }%
    \ifx\AtBegShi@Crop@ship\CROP@ship
      \ifx\AtBegShi@Crop@shiplist\CROP@shiplist
        \ifx\AtBegShi@Crop@@ship\CROP@@ship
          \let\AtBegShi@found\relax
          \ifx\shipout\AtBegShi@Crop@shipout
            \def\AtBegShi@found{\shipout}%
          \else\ifx\AtBeginShipoutOriginalShipout\AtBegShi@Crop@shipout
            \def\AtBegShi@found{\AtBeginShipoutOriginalShipout}%
          \else\ifx\@EveryShipout@Org@Shipout\AtBegShi@Crop@shipout
            \def\AtBegShi@found{\@EveryShipout@Org@Shipout}%
          \else\ifx\GPTorg@shipout\AtBegShi@Crop@shipout
            \def\AtBegShi@found{\GPTorg@shipout}%
          \else\ifx\THBorg@shipout\AtBegShi@Crop@shipout
            \def\AtBegShi@found{\THBorg@shipout}%
          \else\ifx\mem@oldshipout\AtBegShi@Crop@shipout
            \def\AtBegShi@found{\mem@oldshipout}%
          \fi\fi\fi\fi\fi\fi
          \ifx\AtBegShi@found\relax
          \else
            \expandafter\endgroup
            \expandafter\def\AtBegShi@found{%
              \edef\AtBegShi@GroupLevel{\number\currentgrouplevel}%
              \afterassignment\CROP@ship
              \setbox\AtBeginShipoutBox=%
            }%
            \def\CROP@ship{%
              \ifnum\AtBegShi@GroupLevel=\currentgrouplevel
              \else
                \expandafter\aftergroup
              \fi
              \CROP@@ship
            }%
            \def\CROP@shiplist{%
              \lineskip 0pt\relax
              \lineskiplimit 0pt\relax
              \baselineskip 0pt\relax
              \CROP@kernel
              \box\AtBeginShipoutBox
            }%
            \def\CROP@@ship{%
              \ifvoid\AtBeginShipoutBox
              \else
                \setbox\AtBeginShipoutBox=\vbox{%
                  \CROP@shiplist
                }%
                \AtBegShi@GetBoxSize\AtBeginShipoutBox
                \expandafter\CROP@shipout
                \expandafter\box
                \expandafter\AtBeginShipoutBox
              \fi
            }%
            \@PackageInfoNoLine{atbegshi}{Package `crop' patched}%
            \begingroup
          \fi
        \fi
      \fi
    \fi
  \endgroup
  \let\AtBegShi@PatchCrop\relax
}
\@ifpackageloaded{crop}{%
  \AtBegShi@PatchCrop
}{%
  \AtBeginDocument{\AtBegShi@PatchCrop}%
}
%    \end{macrocode}
%
% \subsubsection{Package \xpackage{everyshi}}
%
%     Fix of method. Use of box 255 is not changed.
%    \begin{macrocode}
\def\AtBegShi@PatchEveryshi{%
  \begingroup
    \long\def\AtBegShi@Everyshi@shipout{%
      \afterassignment\@EveryShipout@Test
      \global\setbox\@cclv= %
    }%
    \long\def\AtBegShi@Everyshi@Test{%
      \ifvoid\@cclv\relax
        \aftergroup\@EveryShipout@Output
      \else
        \@EveryShipout@Output
      \fi
    }%
    \ifx\AtBegShi@Everyshi@Test\@EveryShipout@Test
      \let\AtBegShi@found\relax
      \ifx\shipout\AtBegShi@Everyshi@shipout
        \def\AtBegShi@found{\shipout}%
      \else\ifx\AtBeginShipoutOriginalShipout\AtBegShi@Everyshi@shipout
        \def\AtBegShi@found{\AtBeginShipoutOriginalShipout}%
      \else\ifx\CROP@shipout\AtBegShi@Everyshi@shipout
        \def\AtBegShi@found{\CROP@shipout}%
      \else\ifx\GPTorg@shipout\AtBegShi@Everyshi@shipout
        \def\AtBegShi@found{\GPTorg@shipout}%
      \else\ifx\THBorg@shipout\AtBegShi@Everyshi@shipout
        \def\AtBegShi@found{\THBorg@shipout}%
      \else\ifx\mem@oldshipout\AtBegShi@Everyshi@shipout
        \def\AtBegShi@found{\mem@oldshipout}%
      \else
        \expandafter\ifx\csname @EveryShipout@Org@Shipout\endcsname
                        \relax
           \ifx\@EveryShipout@Shipout\AtBegShi@Everyshi@shipout
             \def\AtBegShi@found{\@EveryShipout@Shipout}%
           \fi
        \fi
      \fi\fi\fi\fi\fi\fi
      \ifx\AtBegShi@found\relax
      \else
        \expandafter\endgroup
        \expandafter\def\AtBegShi@found{%
          \edef\AtBegShi@GroupLevel{\number\currentgrouplevel}%
          \afterassignment\@EveryShipout@Test
          \setbox\AtBeginShipoutBox=%
        }%
        \def\@EveryShipout@Test{%
          \ifnum\AtBegShi@GroupLevel=\currentgrouplevel
          \else
            \expandafter\aftergroup
          \fi
          \AtBegShi@Everyshi@Output
        }%
        \def\AtBegShi@Everyshi@Output{%
          \ifvoid\AtBeginShipoutBox
          \else
            \global\setbox\ltx@cclv\box\AtBeginShipoutBox
            \expandafter\@EveryShipout@Output
          \fi
        }%
        \@PackageInfoNoLine{atbegshi}{Package `everyshi' patched}%
        \begingroup
      \fi
    \fi
  \endgroup
  \let\AtBegShi@PatchEveryshi\relax
}
\@ifpackageloaded{everyshi}{%
  \AtBegShi@PatchEveryshi
}{%
  \AtBeginDocument{\AtBegShi@PatchEveryshi}%
}
%    \end{macrocode}
%
% \subsubsection{Class \xpackage{memoir}}
%
%     Fix of method and box.
%    \begin{macrocode}
\def\AtBegShi@PatchMemoir{%
  \begingroup
    \def\AtBegShi@Memoir@shipout{%
      \afterassignment\mem@shipi
      \setbox\@cclv=%
    }%
    \def\AtBegShi@Memoir@shipi{%
      \ifvoid\@cclv
        \expandafter\aftergroup
      \fi
      \mem@shipii
    }%
    \def\AtBegShi@Memoir@shipiiA{%
      \mem@oldshipout\vbox{%
        \trimmarks
        \unvbox\@cclv
      }%
    }%
    \def\AtBegShi@Memoir@shipiiB{%
      \ifvoid\@cclv
        \mem@oldshipout\box\@cclv
      \else
        \mem@oldshipout\vbox{%
          \trimmarks
          \unvbox\@cclv
        }%
      \fi
    }%
    \def\AtBegShi@Memoir@PatchAB{%
      \ifvoid\AtBeginShipoutBox
      \else
        \setbox\AtBeginShipoutBox=\vbox{%
          \trimmarks
          \ifvbox\AtBeginShipoutBox
            \unvbox\AtBeginShipoutBox
          \else
            \box\AtBeginShipoutBox
          \fi
        }%
        \AtBegShi@GetBoxSize\AtBeginShipoutBox
        \expandafter\mem@oldshipout
        \expandafter\box
        \expandafter\AtBeginShipoutBox
      \fi
    }%
    \def\AtBegShi@Memoir@shipiiC{% 2008/08/07 v1.6180339a
      \ifvoid\@cclv
        \mem@oldshipout\box\@cclv
      \else
        \ifshowtrims
          \mem@oldshipout\vbox{\trimmarks\unvbox\@cclv}%
        \else
          \mem@oldshipout\box\@cclv
        \fi
      \fi
    }%
    \def\AtBegShi@Memoir@shipiiD{% 2011/03/06 v3.6j
      \ifvoid\@cclv
        \mem@oldshipout\box\@cclv
      \else
        \ifshowtrims
          \mem@oldshipout\vbox{%
            \trimmarks
            \nointerlineskip
            \box\@cclv
          }%
        \else
          \mem@oldshipout\box\@cclv
        \fi
      \fi
    }%
    \def\AtBegShi@Memoir@PatchCD{%
      \ifvoid\AtBeginShipoutBox
      \else
        \ifshowtrims
          \setbox\AtBeginShipoutBox=\vbox{%
            \trimmarks
            \nointerlineskip
            \box\AtBeginShipoutBox
          }%
        \fi
        \AtBegShi@GetBoxSize\AtBeginShipoutBox
        \expandafter\mem@oldshipout
        \expandafter\box
        \expandafter\AtBeginShipoutBox
      \fi
    }%
    \ifx\AtBegShi@Memoir@shipi\mem@shipi
      \let\AtBegShi@found\ltx@one
      \ifx\AtBegShi@Memoir@shipiiA\mem@shipii
        \let\AtBegShi@found\ltx@zero
        \global\let\AtBegShi@Memoir@PatchX\AtBegShi@Memoir@PatchAB
      \else\ifx\AtBegShi@Memoir@shipiiB\mem@shipii
        \let\AtBegShi@found\ltx@zero
        \global\let\AtBegShi@Memoir@PatchX\AtBegShi@Memoir@PatchAB
      \else\ifx\AtBegShi@Memoir@shipiiC\mem@shipii
        \let\AtBegShi@found\ltx@zero
        \global\let\AtBegShi@Memoir@PatchX\AtBegShi@Memoir@PatchCD
      \else\ifx\AtBegShi@Memoir@shipiiD\mem@shipii
        \let\AtBegShi@found\ltx@zero
        \global\let\AtBegShi@Memoir@PatchX\AtBegShi@Memoir@PatchCD
      \fi\fi\fi\fi
      \ifcase\AtBegShi@found
        \let\AtBegShi@found\relax
        \ifx\shipout\AtBegShi@Memoir@shipout
          \def\AtBegShi@found{\shipout}%
        \else\ifx\AtBeginShipoutOriginalShipout\AtBegShi@Memoir@shipout
          \def\AtBegShi@found{\AtBeginShipoutOriginalShipout}%
        \else\ifx\CROP@shipout\AtBegShi@Memoir@shipout
          \def\AtBegShi@found{\CROP@shipout}%
        \else\ifx\GPTorg@shipout\AtBegShi@Memoir@shipout
          \def\AtBegShi@found{\GPTorg@shipout}%
        \else\ifx\THBorg@shipout\AtBegShi@Memoir@shipout
          \def\AtBegShi@found{\THBorg@shipout}%
        \else\ifx\@EveryShipout@Org@Shipout\AtBegShi@Memoir@shipout
          \def\AtBegShi@found{\@EveryShipout@Org@Shipout}%
        \fi\fi\fi\fi\fi\fi
        \ifx\AtBegShi@found\relax
        \else
          \expandafter\endgroup
          \expandafter\def\AtBegShi@found{%
            \edef\AtBegShi@GroupLevel{\number\currentgrouplevel}%
            \afterassignment\mem@shipi
            \setbox\AtBeginShipoutBox=%
          }%
          \def\mem@shipi{%
            \ifnum\AtBegShi@GroupLevel=\currentgrouplevel
            \else
              \expandafter\aftergroup
            \fi
            \mem@shipii
          }%
          \let\mem@shipii\AtBegShi@Memoir@PatchX
          \@PackageInfoNoLine{atbegshi}{Class `memoir' patched}%
          \begingroup
        \fi
      \fi
    \fi
  \endgroup
  \let\AtBegShi@PatchMemoir\relax
}
\@ifclassloaded{memoir}{%
  \AtBegShi@PatchMemoir
}{%
  \AtBeginDocument{\AtBegShi@PatchMemoir}%
}
%    \end{macrocode}
%
%    \begin{macrocode}
\AtBegShi@AtEnd%
%</package>
%    \end{macrocode}
%
% \section{Test}
%
% \subsection{Catcode checks for loading}
%
%    \begin{macrocode}
%<*test1>
%    \end{macrocode}
%    \begin{macrocode}
\catcode`\{=1 %
\catcode`\}=2 %
\catcode`\#=6 %
\catcode`\@=11 %
\expandafter\ifx\csname count@\endcsname\relax
  \countdef\count@=255 %
\fi
\expandafter\ifx\csname @gobble\endcsname\relax
  \long\def\@gobble#1{}%
\fi
\expandafter\ifx\csname @firstofone\endcsname\relax
  \long\def\@firstofone#1{#1}%
\fi
\expandafter\ifx\csname loop\endcsname\relax
  \expandafter\@firstofone
\else
  \expandafter\@gobble
\fi
{%
  \def\loop#1\repeat{%
    \def\body{#1}%
    \iterate
  }%
  \def\iterate{%
    \body
      \let\next\iterate
    \else
      \let\next\relax
    \fi
    \next
  }%
  \let\repeat=\fi
}%
\def\RestoreCatcodes{}
\count@=0 %
\loop
  \edef\RestoreCatcodes{%
    \RestoreCatcodes
    \catcode\the\count@=\the\catcode\count@\relax
  }%
\ifnum\count@<255 %
  \advance\count@ 1 %
\repeat

\def\RangeCatcodeInvalid#1#2{%
  \count@=#1\relax
  \loop
    \catcode\count@=15 %
  \ifnum\count@<#2\relax
    \advance\count@ 1 %
  \repeat
}
\def\RangeCatcodeCheck#1#2#3{%
  \count@=#1\relax
  \loop
    \ifnum#3=\catcode\count@
    \else
      \errmessage{%
        Character \the\count@\space
        with wrong catcode \the\catcode\count@\space
        instead of \number#3%
      }%
    \fi
  \ifnum\count@<#2\relax
    \advance\count@ 1 %
  \repeat
}
\def\space{ }
\expandafter\ifx\csname LoadCommand\endcsname\relax
  \def\LoadCommand{\input atbegshi.sty\relax}%
\fi
\def\Test{%
  \RangeCatcodeInvalid{0}{47}%
  \RangeCatcodeInvalid{58}{64}%
  \RangeCatcodeInvalid{91}{96}%
  \RangeCatcodeInvalid{123}{255}%
  \catcode`\@=12 %
  \catcode`\\=0 %
  \catcode`\%=14 %
  \LoadCommand
  \RangeCatcodeCheck{0}{36}{15}%
  \RangeCatcodeCheck{37}{37}{14}%
  \RangeCatcodeCheck{38}{47}{15}%
  \RangeCatcodeCheck{48}{57}{12}%
  \RangeCatcodeCheck{58}{63}{15}%
  \RangeCatcodeCheck{64}{64}{12}%
  \RangeCatcodeCheck{65}{90}{11}%
  \RangeCatcodeCheck{91}{91}{15}%
  \RangeCatcodeCheck{92}{92}{0}%
  \RangeCatcodeCheck{93}{96}{15}%
  \RangeCatcodeCheck{97}{122}{11}%
  \RangeCatcodeCheck{123}{255}{15}%
  \RestoreCatcodes
}
\Test
\csname @@end\endcsname
\end
%    \end{macrocode}
%    \begin{macrocode}
%</test1>
%    \end{macrocode}
%
%    \begin{macrocode}
%<*test2>
\input atbegshi.sty\relax
\def\msg#{\immediate\write16}
\msg{File: atbegshi-test2.tex 2016/06/09 v1.18 Test file for plain-TeX}
\def\testmsg#1#2{%
  \msg{}%
  \msg{*** Test with box (#1), expected page output [#2]}% hash-ok
}

\newbox\voidbox
\def\void{\box\voidbox}
\begingroup
  \setbox\voidbox=\void
\endgroup

\count0=0\relax
\AtBeginShipout{%
  \global\advance\count0 by 1\relax
  \msg{* Inside \string\AtBeginShipout: [\the\count0]}%
}

\AtBeginShipoutFirst{%
  \msg{* Inside \string\AtBeginShipoutFirst}%
  Hello World%
}

\testmsg{\string\null}{1}
\shipout\null

\AtBeginShipoutFirst{%
  This is too late%
}

\testmsg{void}{}
\shipout\void

\testmsg{\string\copy255 (not void)}{2}
\setbox255\hbox{\vrule height 10bp width 10bp}
\shipout\copy255 %

\testmsg{\string\copy255 (again)}{3}
\shipout\copy255 %

\testmsg{\string\box255}{4}
\shipout\box255 %

\testmsg{\string\box255 (again)}{}
\shipout\box255 %

\testmsg{\string\hbox}{5}
\shipout\hbox{\vrule height 5bp width 20bp}

\testmsg{\string\vbox}{6}
\shipout\vbox{\hrule height 20bp width 5bp}

\testmsg{\string\null, voided by hook}{}
\def\VoidBox{%
  \begingroup
    \setbox\AtBeginShipoutBox=\box\AtBeginShipoutBox
  \endgroup
}
\AtBeginShipout{\VoidBox}
\shipout\null
\def\VoidBox{}

\msg{*** \string\begingroup}
\begingroup
  \testmsg{void}{}%
  \shipout\void
\msg{*** \string\endgroup}
\endgroup

\msg{*** \string\begingroup}
\begingroup
  \testmsg{void}{}%
  \shipout\void
  \testmsg{\string\null}{8}%
  \shipout\null
\msg{*** \string\endgroup}
\endgroup

\testmsg{output routine}{9}
Hello World
\vfill
\eject

\testmsg{\string\null\space(discarded)}{}
\AtBeginShipout{%
  \msg{* Inside \string\AtBeginShipout: DISCARD}%
  \AtBeginShipoutDiscard
}
\shipout\null

\end
%</test2>
%    \end{macrocode}
%
%    \begin{macrocode}
%<*test3>
\NeedsTeXFormat{LaTeX2e}
\ProvidesFile{atbegshi-test3.tex}[2016/06/09 v1.18 Test file for LaTeX]
\RequirePackage{color}
\pagecolor{yellow}
\documentclass[a5paper,showtrims]{memoir}
\usepackage{atbegshi}
\AtBeginShipout{%
  \setbox\AtBeginShipoutBox=\vbox{%
    \vbox to 0pt{%
      \kern-1.5in %
      \hbox to 0pt{%
        \kern-1.5in %
        \color{blue}%
        \rule{1in}{1in}%
        \hss
      }%
      \vss
    }%
    \hrule
    \hbox{\vrule\box\AtBeginShipoutBox\vrule}%
    \hrule
  }%
}
\usepackage{eso-pic}
\makeatletter
\@ifundefined{@EveryShipout@Init}{%
  \typeout{Test skipped}%
  \@@end
}{}
\@EveryShipout@Init
\let\@EveryShipout@Init\relax
\makeatother
\AddToShipoutPicture{%
  \hspace{.52\paperwidth}%
  \colorbox{cyan}{%
    \rule{0mm}{\paperheight}%
    \hspace{.48\paperwidth}%
  }%
}
%    \end{macrocode}
%    Newer versions of class \xclass{memoir} emulate package
%    \xpackage{crop} and prevents its loading. This is undone
%    in next line for this test file.
%    \begin{macrocode}
\expandafter\let\csname ver@crop.sty\endcsname\relax
\usepackage[color=red,cross,a4,center]{crop}
\begin{document}
\shipout\null
\shipout\box\csname voidb@x\endcsname
\section{Hello World}
\end{document}
%</test3>
%    \end{macrocode}
%
% \section{Installation}
%
% \subsection{Download}
%
% \paragraph{Package.} This package is available on
% CTAN\footnote{\url{http://ctan.org/pkg/atbegshi}}:
% \begin{description}
% \item[\CTAN{macros/latex/contrib/oberdiek/atbegshi.dtx}] The source file.
% \item[\CTAN{macros/latex/contrib/oberdiek/atbegshi.pdf}] Documentation.
% \end{description}
%
%
% \paragraph{Bundle.} All the packages of the bundle `oberdiek'
% are also available in a TDS compliant ZIP archive. There
% the packages are already unpacked and the documentation files
% are generated. The files and directories obey the TDS standard.
% \begin{description}
% \item[\CTAN{install/macros/latex/contrib/oberdiek.tds.zip}]
% \end{description}
% \emph{TDS} refers to the standard ``A Directory Structure
% for \TeX\ Files'' (\CTAN{tds/tds.pdf}). Directories
% with \xfile{texmf} in their name are usually organized this way.
%
% \subsection{Bundle installation}
%
% \paragraph{Unpacking.} Unpack the \xfile{oberdiek.tds.zip} in the
% TDS tree (also known as \xfile{texmf} tree) of your choice.
% Example (linux):
% \begin{quote}
%   |unzip oberdiek.tds.zip -d ~/texmf|
% \end{quote}
%
% \paragraph{Script installation.}
% Check the directory \xfile{TDS:scripts/oberdiek/} for
% scripts that need further installation steps.
% Package \xpackage{attachfile2} comes with the Perl script
% \xfile{pdfatfi.pl} that should be installed in such a way
% that it can be called as \texttt{pdfatfi}.
% Example (linux):
% \begin{quote}
%   |chmod +x scripts/oberdiek/pdfatfi.pl|\\
%   |cp scripts/oberdiek/pdfatfi.pl /usr/local/bin/|
% \end{quote}
%
% \subsection{Package installation}
%
% \paragraph{Unpacking.} The \xfile{.dtx} file is a self-extracting
% \docstrip\ archive. The files are extracted by running the
% \xfile{.dtx} through \plainTeX:
% \begin{quote}
%   \verb|tex atbegshi.dtx|
% \end{quote}
%
% \paragraph{TDS.} Now the different files must be moved into
% the different directories in your installation TDS tree
% (also known as \xfile{texmf} tree):
% \begin{quote}
% \def\t{^^A
% \begin{tabular}{@{}>{\ttfamily}l@{ $\rightarrow$ }>{\ttfamily}l@{}}
%   atbegshi.sty & tex/generic/oberdiek/atbegshi.sty\\
%   atbegshi.pdf & doc/latex/oberdiek/atbegshi.pdf\\
%   atbegshi-example1.tex & doc/latex/oberdiek/atbegshi-example1.tex\\
%   atbegshi-example2.tex & doc/latex/oberdiek/atbegshi-example2.tex\\
%   test/atbegshi-test1.tex & doc/latex/oberdiek/test/atbegshi-test1.tex\\
%   test/atbegshi-test2.tex & doc/latex/oberdiek/test/atbegshi-test2.tex\\
%   test/atbegshi-test3.tex & doc/latex/oberdiek/test/atbegshi-test3.tex\\
%   atbegshi.dtx & source/latex/oberdiek/atbegshi.dtx\\
% \end{tabular}^^A
% }^^A
% \sbox0{\t}^^A
% \ifdim\wd0>\linewidth
%   \begingroup
%     \advance\linewidth by\leftmargin
%     \advance\linewidth by\rightmargin
%   \edef\x{\endgroup
%     \def\noexpand\lw{\the\linewidth}^^A
%   }\x
%   \def\lwbox{^^A
%     \leavevmode
%     \hbox to \linewidth{^^A
%       \kern-\leftmargin\relax
%       \hss
%       \usebox0
%       \hss
%       \kern-\rightmargin\relax
%     }^^A
%   }^^A
%   \ifdim\wd0>\lw
%     \sbox0{\small\t}^^A
%     \ifdim\wd0>\linewidth
%       \ifdim\wd0>\lw
%         \sbox0{\footnotesize\t}^^A
%         \ifdim\wd0>\linewidth
%           \ifdim\wd0>\lw
%             \sbox0{\scriptsize\t}^^A
%             \ifdim\wd0>\linewidth
%               \ifdim\wd0>\lw
%                 \sbox0{\tiny\t}^^A
%                 \ifdim\wd0>\linewidth
%                   \lwbox
%                 \else
%                   \usebox0
%                 \fi
%               \else
%                 \lwbox
%               \fi
%             \else
%               \usebox0
%             \fi
%           \else
%             \lwbox
%           \fi
%         \else
%           \usebox0
%         \fi
%       \else
%         \lwbox
%       \fi
%     \else
%       \usebox0
%     \fi
%   \else
%     \lwbox
%   \fi
% \else
%   \usebox0
% \fi
% \end{quote}
% If you have a \xfile{docstrip.cfg} that configures and enables \docstrip's
% TDS installing feature, then some files can already be in the right
% place, see the documentation of \docstrip.
%
% \subsection{Refresh file name databases}
%
% If your \TeX~distribution
% (\teTeX, \mikTeX, \dots) relies on file name databases, you must refresh
% these. For example, \teTeX\ users run \verb|texhash| or
% \verb|mktexlsr|.
%
% \subsection{Some details for the interested}
%
% \paragraph{Attached source.}
%
% The PDF documentation on CTAN also includes the
% \xfile{.dtx} source file. It can be extracted by
% AcrobatReader 6 or higher. Another option is \textsf{pdftk},
% e.g. unpack the file into the current directory:
% \begin{quote}
%   \verb|pdftk atbegshi.pdf unpack_files output .|
% \end{quote}
%
% \paragraph{Unpacking with \LaTeX.}
% The \xfile{.dtx} chooses its action depending on the format:
% \begin{description}
% \item[\plainTeX:] Run \docstrip\ and extract the files.
% \item[\LaTeX:] Generate the documentation.
% \end{description}
% If you insist on using \LaTeX\ for \docstrip\ (really,
% \docstrip\ does not need \LaTeX), then inform the autodetect routine
% about your intention:
% \begin{quote}
%   \verb|latex \let\install=y\input{atbegshi.dtx}|
% \end{quote}
% Do not forget to quote the argument according to the demands
% of your shell.
%
% \paragraph{Generating the documentation.}
% You can use both the \xfile{.dtx} or the \xfile{.drv} to generate
% the documentation. The process can be configured by the
% configuration file \xfile{ltxdoc.cfg}. For instance, put this
% line into this file, if you want to have A4 as paper format:
% \begin{quote}
%   \verb|\PassOptionsToClass{a4paper}{article}|
% \end{quote}
% An example follows how to generate the
% documentation with pdf\LaTeX:
% \begin{quote}
%\begin{verbatim}
%pdflatex atbegshi.dtx
%makeindex -s gind.ist atbegshi.idx
%pdflatex atbegshi.dtx
%makeindex -s gind.ist atbegshi.idx
%pdflatex atbegshi.dtx
%\end{verbatim}
% \end{quote}
%
% \section{Catalogue}
%
% The following XML file can be used as source for the
% \href{http://mirror.ctan.org/help/Catalogue/catalogue.html}{\TeX\ Catalogue}.
% The elements \texttt{caption} and \texttt{description} are imported
% from the original XML file from the Catalogue.
% The name of the XML file in the Catalogue is \xfile{atbegshi.xml}.
%    \begin{macrocode}
%<*catalogue>
<?xml version='1.0' encoding='us-ascii'?>
<!DOCTYPE entry SYSTEM 'catalogue.dtd'>
<entry datestamp='$Date$' modifier='$Author$' id='atbegshi'>
  <name>atbegshi</name>
  <caption>Execute stuff at \shipout time.</caption>
  <authorref id='auth:oberdiek'/>
  <copyright owner='Heiko Oberdiek' year='2007-2011'/>
  <license type='lppl1.3'/>
  <version number='1.18'/>
  <description>
    This package is a modern reimplementation of package
    <xref refid='everyshi'>everyshi</xref>, providing various commands
    to be executed before a <tt>\shipout</tt> command.  It makes use of
    e-TeX&#x2019;s facilities if they are available.  The package may
    be used either with LaTeX or with plain TeX.
    <p/>
    The package is part of the <xref refid='oberdiek'>oberdiek</xref> bundle.
  </description>
  <documentation details='Package documentation'
      href='ctan:/macros/latex/contrib/oberdiek/atbegshi.pdf'/>
  <ctan file='true' path='/macros/latex/contrib/oberdiek/atbegshi.dtx'/>
  <miktex location='oberdiek'/>
  <texlive location='oberdiek'/>
  <install path='/macros/latex/contrib/oberdiek/oberdiek.tds.zip'/>
</entry>
%</catalogue>
%    \end{macrocode}
%
% \begin{History}
%   \begin{Version}{2007/04/17 v1.0}
%   \item
%     First version.
%   \end{Version}
%   \begin{Version}{2007/04/18 v1.1}
%   \item
%     New method based on \cs{lastkern} is used if \eTeX\ is missing.
%   \item
%     \cs{AtBeginShipoutDiscard} also resets \cs{deadcycles}.
%   \end{Version}
%   \begin{Version}{2007/04/19 v1.2}
%   \item
%     \cs{AtBeginShipoutEarly} removed for simplification reasons.
%   \item
%     Forgotten definition of \cs{AtBegShi@Info} added.
%   \item
%     Patches for packages \xpackage{crop} and \xpackage{everyshi}
%     and class \xclass{memoir} added.
%   \end{Version}
%   \begin{Version}{2007/04/26 v1.3}
%   \item
%     Use of package \xpackage{infwarerr}.
%   \item
%     Catcode section after generic header.
%   \end{Version}
%   \begin{Version}{2007/04/27 v1.4}
%   \item
%     Small optimizations.
%   \end{Version}
%   \begin{Version}{2007/06/06 v1.5}
%   \item
%     \cs{AtBeginShipoutUpperLeft} added.
%   \item
%     Example added.
%   \item
%     Fix in second test file for newer version of \xclass{memoir}.
%   \end{Version}
%   \begin{Version}{2007/09/09 v1.6}
%   \item
%     Catcode section rewritten.
%   \end{Version}
%   \begin{Version}{2008/07/18 v1.7}
%   \item
%     Documentation of \cs{AtBeginShipoutUpperLeft} fixed and extended.
%   \end{Version}
%   \begin{Version}{2008/07/19 v1.8}
%   \item
%     \cs{AtBeginShipoutUpperLeftForeground} added.
%   \end{Version}
%   \begin{Version}{2008/07/31 v1.9}
%   \item
%     Second example (TrimBox for dvipdfmx) added.
%   \item
%     No changes in package code.
%   \end{Version}
%   \begin{Version}{2009/12/02 v1.10}
%   \item
%     \cs{AtBeginShipoutOriginalShipout} added.
%   \item
%     Test file fixed.
%   \end{Version}
%   \begin{Version}{2010/03/01 v1.11}
%   \item
%     Compatibility with \iniTeX\ except for \cs{newbox}.
%   \end{Version}
%   \begin{Version}{2010/03/25 v1.12}
%   \item
%     \cs{AtBeginShipoutNext} can now be used inside
%     \cs{AtBeginShipoutNext}.
%   \end{Version}
%   \begin{Version}{2010/08/18 v1.13}
%   \item
%     Fixes for \cs{AtBegShi@CheckDefinable}.
%   \end{Version}
%   \begin{Version}{2010/12/02 v1.14}
%   \item
%     Remove the warning because of void box if the hook
%     calls \AtBeginShipoutDiscard.
%   \end{Version}
%   \begin{Version}{2011/01/30 v1.15}
%   \item
%     Already loaded package files are not input in \hologo{plainTeX}.
%   \end{Version}
%   \begin{Version}{2011/10/05 v1.16}
%   \item
%     \cs{AtBeginShipoutAddToBox}, \cs{AtBeginShipoutAddToBoxForeground} added.
%   \item
%     \cs{AtBeginShipoutBoxWidth}, \cs{AtBeginShipoutBoxHeight},
%     \cs{AtBeginShipoutBoxDepth} added.
%   \item
%     Updates for patches of class \xclass{memoir}.
%   \end{Version}
%   \begin{Version}{2016/05/16 v1.17}
%   \item
%     Documentation updates.
%   \end{Version}
%   \begin{Version}{2016/06/09 v1.18}
%   \item
%     Update for \cs{pdfhorign} in new LuaTeX.
%   \end{Version}
% \end{History}
%
% \PrintIndex
%
% \Finale
\endinput
|
% \end{quote}
% Do not forget to quote the argument according to the demands
% of your shell.
%
% \paragraph{Generating the documentation.}
% You can use both the \xfile{.dtx} or the \xfile{.drv} to generate
% the documentation. The process can be configured by the
% configuration file \xfile{ltxdoc.cfg}. For instance, put this
% line into this file, if you want to have A4 as paper format:
% \begin{quote}
%   \verb|\PassOptionsToClass{a4paper}{article}|
% \end{quote}
% An example follows how to generate the
% documentation with pdf\LaTeX:
% \begin{quote}
%\begin{verbatim}
%pdflatex atbegshi.dtx
%makeindex -s gind.ist atbegshi.idx
%pdflatex atbegshi.dtx
%makeindex -s gind.ist atbegshi.idx
%pdflatex atbegshi.dtx
%\end{verbatim}
% \end{quote}
%
% \section{Catalogue}
%
% The following XML file can be used as source for the
% \href{http://mirror.ctan.org/help/Catalogue/catalogue.html}{\TeX\ Catalogue}.
% The elements \texttt{caption} and \texttt{description} are imported
% from the original XML file from the Catalogue.
% The name of the XML file in the Catalogue is \xfile{atbegshi.xml}.
%    \begin{macrocode}
%<*catalogue>
<?xml version='1.0' encoding='us-ascii'?>
<!DOCTYPE entry SYSTEM 'catalogue.dtd'>
<entry datestamp='$Date$' modifier='$Author$' id='atbegshi'>
  <name>atbegshi</name>
  <caption>Execute stuff at \shipout time.</caption>
  <authorref id='auth:oberdiek'/>
  <copyright owner='Heiko Oberdiek' year='2007-2011'/>
  <license type='lppl1.3'/>
  <version number='1.18'/>
  <description>
    This package is a modern reimplementation of package
    <xref refid='everyshi'>everyshi</xref>, providing various commands
    to be executed before a <tt>\shipout</tt> command.  It makes use of
    e-TeX&#x2019;s facilities if they are available.  The package may
    be used either with LaTeX or with plain TeX.
    <p/>
    The package is part of the <xref refid='oberdiek'>oberdiek</xref> bundle.
  </description>
  <documentation details='Package documentation'
      href='ctan:/macros/latex/contrib/oberdiek/atbegshi.pdf'/>
  <ctan file='true' path='/macros/latex/contrib/oberdiek/atbegshi.dtx'/>
  <miktex location='oberdiek'/>
  <texlive location='oberdiek'/>
  <install path='/macros/latex/contrib/oberdiek/oberdiek.tds.zip'/>
</entry>
%</catalogue>
%    \end{macrocode}
%
% \begin{History}
%   \begin{Version}{2007/04/17 v1.0}
%   \item
%     First version.
%   \end{Version}
%   \begin{Version}{2007/04/18 v1.1}
%   \item
%     New method based on \cs{lastkern} is used if \eTeX\ is missing.
%   \item
%     \cs{AtBeginShipoutDiscard} also resets \cs{deadcycles}.
%   \end{Version}
%   \begin{Version}{2007/04/19 v1.2}
%   \item
%     \cs{AtBeginShipoutEarly} removed for simplification reasons.
%   \item
%     Forgotten definition of \cs{AtBegShi@Info} added.
%   \item
%     Patches for packages \xpackage{crop} and \xpackage{everyshi}
%     and class \xclass{memoir} added.
%   \end{Version}
%   \begin{Version}{2007/04/26 v1.3}
%   \item
%     Use of package \xpackage{infwarerr}.
%   \item
%     Catcode section after generic header.
%   \end{Version}
%   \begin{Version}{2007/04/27 v1.4}
%   \item
%     Small optimizations.
%   \end{Version}
%   \begin{Version}{2007/06/06 v1.5}
%   \item
%     \cs{AtBeginShipoutUpperLeft} added.
%   \item
%     Example added.
%   \item
%     Fix in second test file for newer version of \xclass{memoir}.
%   \end{Version}
%   \begin{Version}{2007/09/09 v1.6}
%   \item
%     Catcode section rewritten.
%   \end{Version}
%   \begin{Version}{2008/07/18 v1.7}
%   \item
%     Documentation of \cs{AtBeginShipoutUpperLeft} fixed and extended.
%   \end{Version}
%   \begin{Version}{2008/07/19 v1.8}
%   \item
%     \cs{AtBeginShipoutUpperLeftForeground} added.
%   \end{Version}
%   \begin{Version}{2008/07/31 v1.9}
%   \item
%     Second example (TrimBox for dvipdfmx) added.
%   \item
%     No changes in package code.
%   \end{Version}
%   \begin{Version}{2009/12/02 v1.10}
%   \item
%     \cs{AtBeginShipoutOriginalShipout} added.
%   \item
%     Test file fixed.
%   \end{Version}
%   \begin{Version}{2010/03/01 v1.11}
%   \item
%     Compatibility with \iniTeX\ except for \cs{newbox}.
%   \end{Version}
%   \begin{Version}{2010/03/25 v1.12}
%   \item
%     \cs{AtBeginShipoutNext} can now be used inside
%     \cs{AtBeginShipoutNext}.
%   \end{Version}
%   \begin{Version}{2010/08/18 v1.13}
%   \item
%     Fixes for \cs{AtBegShi@CheckDefinable}.
%   \end{Version}
%   \begin{Version}{2010/12/02 v1.14}
%   \item
%     Remove the warning because of void box if the hook
%     calls \AtBeginShipoutDiscard.
%   \end{Version}
%   \begin{Version}{2011/01/30 v1.15}
%   \item
%     Already loaded package files are not input in \hologo{plainTeX}.
%   \end{Version}
%   \begin{Version}{2011/10/05 v1.16}
%   \item
%     \cs{AtBeginShipoutAddToBox}, \cs{AtBeginShipoutAddToBoxForeground} added.
%   \item
%     \cs{AtBeginShipoutBoxWidth}, \cs{AtBeginShipoutBoxHeight},
%     \cs{AtBeginShipoutBoxDepth} added.
%   \item
%     Updates for patches of class \xclass{memoir}.
%   \end{Version}
%   \begin{Version}{2016/05/16 v1.17}
%   \item
%     Documentation updates.
%   \end{Version}
%   \begin{Version}{2016/06/09 v1.18}
%   \item
%     Update for \cs{pdfhorign} in new LuaTeX.
%   \end{Version}
% \end{History}
%
% \PrintIndex
%
% \Finale
\endinput
|
% \end{quote}
% Do not forget to quote the argument according to the demands
% of your shell.
%
% \paragraph{Generating the documentation.}
% You can use both the \xfile{.dtx} or the \xfile{.drv} to generate
% the documentation. The process can be configured by the
% configuration file \xfile{ltxdoc.cfg}. For instance, put this
% line into this file, if you want to have A4 as paper format:
% \begin{quote}
%   \verb|\PassOptionsToClass{a4paper}{article}|
% \end{quote}
% An example follows how to generate the
% documentation with pdf\LaTeX:
% \begin{quote}
%\begin{verbatim}
%pdflatex atbegshi.dtx
%makeindex -s gind.ist atbegshi.idx
%pdflatex atbegshi.dtx
%makeindex -s gind.ist atbegshi.idx
%pdflatex atbegshi.dtx
%\end{verbatim}
% \end{quote}
%
% \section{Catalogue}
%
% The following XML file can be used as source for the
% \href{http://mirror.ctan.org/help/Catalogue/catalogue.html}{\TeX\ Catalogue}.
% The elements \texttt{caption} and \texttt{description} are imported
% from the original XML file from the Catalogue.
% The name of the XML file in the Catalogue is \xfile{atbegshi.xml}.
%    \begin{macrocode}
%<*catalogue>
<?xml version='1.0' encoding='us-ascii'?>
<!DOCTYPE entry SYSTEM 'catalogue.dtd'>
<entry datestamp='$Date$' modifier='$Author$' id='atbegshi'>
  <name>atbegshi</name>
  <caption>Execute stuff at \shipout time.</caption>
  <authorref id='auth:oberdiek'/>
  <copyright owner='Heiko Oberdiek' year='2007-2011'/>
  <license type='lppl1.3'/>
  <version number='1.18'/>
  <description>
    This package is a modern reimplementation of package
    <xref refid='everyshi'>everyshi</xref>, providing various commands
    to be executed before a <tt>\shipout</tt> command.  It makes use of
    e-TeX&#x2019;s facilities if they are available.  The package may
    be used either with LaTeX or with plain TeX.
    <p/>
    The package is part of the <xref refid='oberdiek'>oberdiek</xref> bundle.
  </description>
  <documentation details='Package documentation'
      href='ctan:/macros/latex/contrib/oberdiek/atbegshi.pdf'/>
  <ctan file='true' path='/macros/latex/contrib/oberdiek/atbegshi.dtx'/>
  <miktex location='oberdiek'/>
  <texlive location='oberdiek'/>
  <install path='/macros/latex/contrib/oberdiek/oberdiek.tds.zip'/>
</entry>
%</catalogue>
%    \end{macrocode}
%
% \begin{History}
%   \begin{Version}{2007/04/17 v1.0}
%   \item
%     First version.
%   \end{Version}
%   \begin{Version}{2007/04/18 v1.1}
%   \item
%     New method based on \cs{lastkern} is used if \eTeX\ is missing.
%   \item
%     \cs{AtBeginShipoutDiscard} also resets \cs{deadcycles}.
%   \end{Version}
%   \begin{Version}{2007/04/19 v1.2}
%   \item
%     \cs{AtBeginShipoutEarly} removed for simplification reasons.
%   \item
%     Forgotten definition of \cs{AtBegShi@Info} added.
%   \item
%     Patches for packages \xpackage{crop} and \xpackage{everyshi}
%     and class \xclass{memoir} added.
%   \end{Version}
%   \begin{Version}{2007/04/26 v1.3}
%   \item
%     Use of package \xpackage{infwarerr}.
%   \item
%     Catcode section after generic header.
%   \end{Version}
%   \begin{Version}{2007/04/27 v1.4}
%   \item
%     Small optimizations.
%   \end{Version}
%   \begin{Version}{2007/06/06 v1.5}
%   \item
%     \cs{AtBeginShipoutUpperLeft} added.
%   \item
%     Example added.
%   \item
%     Fix in second test file for newer version of \xclass{memoir}.
%   \end{Version}
%   \begin{Version}{2007/09/09 v1.6}
%   \item
%     Catcode section rewritten.
%   \end{Version}
%   \begin{Version}{2008/07/18 v1.7}
%   \item
%     Documentation of \cs{AtBeginShipoutUpperLeft} fixed and extended.
%   \end{Version}
%   \begin{Version}{2008/07/19 v1.8}
%   \item
%     \cs{AtBeginShipoutUpperLeftForeground} added.
%   \end{Version}
%   \begin{Version}{2008/07/31 v1.9}
%   \item
%     Second example (TrimBox for dvipdfmx) added.
%   \item
%     No changes in package code.
%   \end{Version}
%   \begin{Version}{2009/12/02 v1.10}
%   \item
%     \cs{AtBeginShipoutOriginalShipout} added.
%   \item
%     Test file fixed.
%   \end{Version}
%   \begin{Version}{2010/03/01 v1.11}
%   \item
%     Compatibility with \iniTeX\ except for \cs{newbox}.
%   \end{Version}
%   \begin{Version}{2010/03/25 v1.12}
%   \item
%     \cs{AtBeginShipoutNext} can now be used inside
%     \cs{AtBeginShipoutNext}.
%   \end{Version}
%   \begin{Version}{2010/08/18 v1.13}
%   \item
%     Fixes for \cs{AtBegShi@CheckDefinable}.
%   \end{Version}
%   \begin{Version}{2010/12/02 v1.14}
%   \item
%     Remove the warning because of void box if the hook
%     calls \AtBeginShipoutDiscard.
%   \end{Version}
%   \begin{Version}{2011/01/30 v1.15}
%   \item
%     Already loaded package files are not input in \hologo{plainTeX}.
%   \end{Version}
%   \begin{Version}{2011/10/05 v1.16}
%   \item
%     \cs{AtBeginShipoutAddToBox}, \cs{AtBeginShipoutAddToBoxForeground} added.
%   \item
%     \cs{AtBeginShipoutBoxWidth}, \cs{AtBeginShipoutBoxHeight},
%     \cs{AtBeginShipoutBoxDepth} added.
%   \item
%     Updates for patches of class \xclass{memoir}.
%   \end{Version}
%   \begin{Version}{2016/05/16 v1.17}
%   \item
%     Documentation updates.
%   \end{Version}
%   \begin{Version}{2016/06/09 v1.18}
%   \item
%     Update for \cs{pdfhorign} in new LuaTeX.
%   \end{Version}
% \end{History}
%
% \PrintIndex
%
% \Finale
\endinput

%        (quote the arguments according to the demands of your shell)
%
% Documentation:
%    (a) If atbegshi.drv is present:
%           latex atbegshi.drv
%    (b) Without atbegshi.drv:
%           latex atbegshi.dtx; ...
%    The class ltxdoc loads the configuration file ltxdoc.cfg
%    if available. Here you can specify further options, e.g.
%    use A4 as paper format:
%       \PassOptionsToClass{a4paper}{article}
%
%    Programm calls to get the documentation (example):
%       pdflatex atbegshi.dtx
%       makeindex -s gind.ist atbegshi.idx
%       pdflatex atbegshi.dtx
%       makeindex -s gind.ist atbegshi.idx
%       pdflatex atbegshi.dtx
%
% Installation:
%    TDS:tex/generic/oberdiek/atbegshi.sty
%    TDS:doc/latex/oberdiek/atbegshi.pdf
%    TDS:doc/latex/oberdiek/atbegshi-example1.tex
%    TDS:doc/latex/oberdiek/atbegshi-example2.tex
%    TDS:doc/latex/oberdiek/test/atbegshi-test1.tex
%    TDS:doc/latex/oberdiek/test/atbegshi-test2.tex
%    TDS:doc/latex/oberdiek/test/atbegshi-test3.tex
%    TDS:source/latex/oberdiek/atbegshi.dtx
%
%<*ignore>
\begingroup
  \catcode123=1 %
  \catcode125=2 %
  \def\x{LaTeX2e}%
\expandafter\endgroup
\ifcase 0\ifx\install y1\fi\expandafter
         \ifx\csname processbatchFile\endcsname\relax\else1\fi
         \ifx\fmtname\x\else 1\fi\relax
\else\csname fi\endcsname
%</ignore>
%<*install>
\input docstrip.tex
\Msg{************************************************************************}
\Msg{* Installation}
\Msg{* Package: atbegshi 2016/06/09 v1.18 At begin shipout hook (HO)}
\Msg{************************************************************************}

\keepsilent
\askforoverwritefalse

\let\MetaPrefix\relax
\preamble

This is a generated file.

Project: atbegshi
Version: 2016/06/09 v1.18

Copyright (C) 2007-2011 by
   Heiko Oberdiek <heiko.oberdiek at googlemail.com>

This work may be distributed and/or modified under the
conditions of the LaTeX Project Public License, either
version 1.3c of this license or (at your option) any later
version. This version of this license is in
   http://www.latex-project.org/lppl/lppl-1-3c.txt
and the latest version of this license is in
   http://www.latex-project.org/lppl.txt
and version 1.3 or later is part of all distributions of
LaTeX version 2005/12/01 or later.

This work has the LPPL maintenance status "maintained".

This Current Maintainer of this work is Heiko Oberdiek.

The Base Interpreter refers to any `TeX-Format',
because some files are installed in TDS:tex/generic//.

This work consists of the main source file atbegshi.dtx
and the derived files
   atbegshi.sty, atbegshi.pdf, atbegshi.ins, atbegshi.drv,
   atbegshi-example1.tex, atbegshi-example2.tex,
   atbegshi-test1.tex, atbegshi-test2.tex, atbegshi-test3.tex.

\endpreamble
\let\MetaPrefix\DoubleperCent

\generate{%
  \file{atbegshi.ins}{\from{atbegshi.dtx}{install}}%
  \file{atbegshi.drv}{\from{atbegshi.dtx}{driver}}%
  \usedir{tex/generic/oberdiek}%
  \file{atbegshi.sty}{\from{atbegshi.dtx}{package}}%
  \usedir{doc/latex/oberdiek}%
  \file{atbegshi-example1.tex}{\from{atbegshi.dtx}{example1}}%
  \file{atbegshi-example2.tex}{\from{atbegshi.dtx}{example2}}%
  \usedir{doc/latex/oberdiek/test}%
  \file{atbegshi-test1.tex}{\from{atbegshi.dtx}{test1}}%
  \file{atbegshi-test2.tex}{\from{atbegshi.dtx}{test2}}%
  \file{atbegshi-test3.tex}{\from{atbegshi.dtx}{test3}}%
  \nopreamble
  \nopostamble
  \usedir{source/latex/oberdiek/catalogue}%
  \file{atbegshi.xml}{\from{atbegshi.dtx}{catalogue}}%
}

\catcode32=13\relax% active space
\let =\space%
\Msg{************************************************************************}
\Msg{*}
\Msg{* To finish the installation you have to move the following}
\Msg{* file into a directory searched by TeX:}
\Msg{*}
\Msg{*     atbegshi.sty}
\Msg{*}
\Msg{* To produce the documentation run the file `atbegshi.drv'}
\Msg{* through LaTeX.}
\Msg{*}
\Msg{* Happy TeXing!}
\Msg{*}
\Msg{************************************************************************}

\endbatchfile
%</install>
%<*ignore>
\fi
%</ignore>
%<*driver>
\NeedsTeXFormat{LaTeX2e}
\ProvidesFile{atbegshi.drv}%
  [2016/06/09 v1.18 At begin shipout hook (HO)]%
\documentclass{ltxdoc}
\usepackage{holtxdoc}[2011/11/22]
\begin{document}
  \DocInput{atbegshi.dtx}%
\end{document}
%</driver>
% \fi
%
%
% \CharacterTable
%  {Upper-case    \A\B\C\D\E\F\G\H\I\J\K\L\M\N\O\P\Q\R\S\T\U\V\W\X\Y\Z
%   Lower-case    \a\b\c\d\e\f\g\h\i\j\k\l\m\n\o\p\q\r\s\t\u\v\w\x\y\z
%   Digits        \0\1\2\3\4\5\6\7\8\9
%   Exclamation   \!     Double quote  \"     Hash (number) \#
%   Dollar        \$     Percent       \%     Ampersand     \&
%   Acute accent  \'     Left paren    \(     Right paren   \)
%   Asterisk      \*     Plus          \+     Comma         \,
%   Minus         \-     Point         \.     Solidus       \/
%   Colon         \:     Semicolon     \;     Less than     \<
%   Equals        \=     Greater than  \>     Question mark \?
%   Commercial at \@     Left bracket  \[     Backslash     \\
%   Right bracket \]     Circumflex    \^     Underscore    \_
%   Grave accent  \`     Left brace    \{     Vertical bar  \|
%   Right brace   \}     Tilde         \~}
%
% \GetFileInfo{atbegshi.drv}
%
% \title{The \xpackage{atbegshi} package}
% \date{2016/06/09 v1.18}
% \author{Heiko Oberdiek\thanks
% {Please report any issues at https://github.com/ho-tex/oberdiek/issues}\\
% \xemail{heiko.oberdiek at googlemail.com}}
%
% \maketitle
%
% \begin{abstract}
% This package is a modern reimplementation of package \xpackage{everyshi}
% without the burden of compatibility. It makes use of \eTeX's if available.
% Both \LaTeX\ and \plainTeX\ are supported.
% \end{abstract}
%
% \tableofcontents
%
% \section{Documentation}
%
% Package \xpackage{atbegshi} redefines \cs{shipout} to insert
% hooks for user code that is executed before the page is shipped out.
% The code may modify or even discard the output page.
% Three hooks are implemented:
% \begin{enumerate}
% \item A hook that is executed for every page, see\\
%       \cs{AtBeginShipout}
% \item A hook that is executed for the next page only, see\\
%       \cs{AtBeginShipoutNext}
% \item A hook that is only executed for the first page, see\\
%       \cs{AtBeginShipoutFirst}
% \end{enumerate}
% The hooks are executed in this order. The following three macros
% provide the user interface for adding code to these hooks:
%
% \begin{declcs}{AtBeginShipout} \M{code}\\
%   \cs{AtBeginShipoutBox}
% \end{declcs}
% Execute the \meta{code} for every page.
% The page contents is held in box register \cs{AtBeginShipoutBox}
% and may be modified. Use \cs{AtBeginShipoutDiscard} if you want to
% discard the page.
%
% \emph{Note:} Package \xpackage{everyshi} uses box register 255. With
% package \xpackage{atbegshi} you must use \cs{AtBeginShipoutBox} instead.
%
% If \LaTeX\ calls \cs{shipout} in \cs{@outputpage} (part of its output
% routine),
% the meaning of \cs{protect} is \cs{noexpand}. \LaTeX\ sets
% \cs{protect} to the appropriate \cs{@typeset@protect} in the box
% that is shipped out. This is too late for the hooks, they
% are called earlier in the redefined \cs{shipout}. Therefore
% package \xpackage{atbegshi} sets \cs{protect} to \cs{@typeset@protect}
% before it calls the hooks.
% (In \cs{EveryShipout} of package \xpackage{everyshi} the
% user is responsible for the correct setting of \cs{protect}.)
%
% \begin{declcs}{AtBeginShipoutNext} \M{code}
% \end{declcs}
% This reimplements package \xpackage{everyshi}'s \cs{AtNextShipout}.
% The \meta{code} is executed at shipout time of the next page only.
% It is just a convenience macro, it can be easily replaced by something
% like:
%\begin{quote}
%\begin{verbatim}
%\newcommand{\MyShipoutHook}{}%
%\AtBeginShipout{\MyShipoutHook}
%\gdef\MyShipoutHook{%
%  ... do something with next page ...
%  \gdef\MyShipoutHook{}%
% }
%\end{verbatim}
%\end{quote}
% (This can be necessary, if hook order does matter).
%
% \begin{declcs}{AtBeginShipoutFirst} \M{code}
% \end{declcs}
% This reimplements \LaTeX's \cs{AtBeginDvi}. This hook is
% usually used for \cs{special} commands that include PostScript
% header files. The \cs{code} is directly executed in a \cs{vbox}
% that is put at the beginning of the output page. Dealing with
% the output box \cs{AtBeginShipoutBox} is not necessary and
% not permitted here.
%
% \begin{declcs}{AtBeginShipoutDiscard}
% \end{declcs}
% This macro notifies package \xpackage{atbegshi} that
% the output page is discarded. The remaining hook code
% and the remaining hooks are not executed and the page is thrown away.
% Also \cs{deadcycles} is cleared to zero like an ordinary \cs{shipout}
% would do.
%
% \begin{declcs}{AtBeginShipoutInit}
% \end{declcs}
% Usually the redefinition of \cs{shipout} is delayed by
% \cs{AtBeginDocument} (if this macro exists). This can be
% too late, if other packages also redefines \cs{shipout}
% and the order does matter.
% \cs{AtBeginShipoutInit} forces the immediate redefinition
% of \cs{shipout}.
%
% \begin{declcs}{AtBeginShipoutAddToBox} \M{stuff}\\
%   \cs{AtBeginShipoutAddToBoxForeground} \M{stuff}
% \end{declcs}
% A quite common use case is the addition of \cs{special}
% or other whatsits to the page output box. Macro
% \cs{AtBeginShipoutAddToBox} puts \meta{stuff} in a
% box with zeroed dimensions. The box with the \meta{stuff}
% is put in the upper left corner of the shipout box
% \cs{AtBeginShipoutBox}. Macro \cs{AtBeginShipoutAddToBox}
% puts the \meta{stuff} in the background, the other macro
% \cs{AtBeginShipoutAddToBoxForeground} in the foreground
% after the original shipout box contents is set.
%
% A void shipout box (that means a discarded page)
% remains void that means \meta{stuff} is ignored in this case.
% The box type of \cs{AtBeginShipoutBox} is preserved.
% Also the box nesting level for the original contents
% of \cs{AtBeginShipoutBox} remains, for example, to avoid trouble
% with links across pages in case of \hologo{pdfTeX}.
%
% \begin{declcs}{AtBeginShipoutUpperLeft} \M{background material}
% \end{declcs}
% This is a macro that puts material in the background of
% box \cs{AtBeginShipoutBox}.
% The \meta{background material} is set in an \cs{hbox}, the
% reference point is the upper left corner of the output page.
% In case of \pdfTeX\ in PDF mode, the settings of
% \cs{pdfhorigin} and \cs{pdfvorigin} are respected.
%
% The macro \cs{AtBeginShipoutUpperLeft} is intended to be used
% in one of the hook setting macros, such as
% \cs{AtBeginShipout}, \cs{AtBeginShipoutFirst}, or
% \cs{AtBeginShipoutNext}.
%
% For \LaTeX\ users the \meta{background material} is set
% inside a \texttt{picture} environment:
% \begin{quote}
%   |\begin{picture}(0,0)|\\
%   \mbox{}\quad |\setlength{\unitlength}{1pt}%|\\
%   \mbox{}\quad \meta{background material}\\
%   |\end{picture}|
% \end{quote}
%
% \begin{declcs}{AtBeginShipoutUpperLeftForeground} \M{foreground material}
% \end{declcs}
% See \cs{AtBeginShipoutUpperLeft}. The difference is that the material
% is put in the foreground.
%
% \begin{declcs}{AtBeginShipoutOriginalShipout} \meta{box}
% \end{declcs}
% It stores the meaning of \cs{shipout} at the time this package
% is loaded.
%
% \begin{declcs}{AtBeginShipoutBoxWidth}\\
%   \cs{AtBeginShipoutBoxHeight}\\
%   \cs{AtBeginShipoutBoxDepth}\\
% \end{declcs}
% These macros store the dimensions of the output box \cs{AtBeginShipoutBox}
% before the original shipout is called. If \cs{shipout} is not
% redefined before the package loading or the box dimensions
% are not changed by the redefined \cs{shipout}, these macros
% contain the dimensions of the shipout box. These values can be
% remembered by \cs{label} and \cs{ref}. For example, this is done
% by the package module \xpackage{zref-pagelayout} of project \xpackage{zref}.
% The dimensions of the shipout page can be used in some \hologo{TeX}
% engines (\hologo{pdfTeX} in PDF mode, \hologo{XeTeX}) to calculate
% the media size of the shipout page if \cs{pdfpagewidth} and
% \cs{pdfpageheight} are not set.
%
% \subsection{Examples}
%
% \subsubsection{Example: circle in background}
%
% In this example we put a circle in the background in the middle
% of the paper.
%
%    \begin{macrocode}
%<*example1>
%    \end{macrocode}
%    \begin{macrocode}
\documentclass[a4paper]{article}
\usepackage{color}
\usepackage{atbegshi}
%    \end{macrocode}
%    Package \xpackage{picture} makes life a little easier, because
%    we can now also use length specifications in |picture|'s commands.
%    \begin{macrocode}
\usepackage{picture}
%    \end{macrocode}
%    Now we draw the circle in the middle of the paper.
%    \cs{put} moves downwards, because the origin is at the top
%    of the page, not at its bottom.
%    \begin{macrocode}
\AtBeginShipout{%
  \AtBeginShipoutUpperLeft{%
    \put(0.5\paperwidth,-0.5\paperheight){\circle{10}}%
  }%
}
\begin{document}
\section{Hello World}
\newpage
\AtBeginShipoutNext{%
  \AtBeginShipoutUpperLeft{%
    \color{red}%
    \put(0,-0.5\paperheight){\line(1,0){\paperwidth}}%
    \put(0.5\paperwidth, 0){\line(0,-1){\paperheight}}%
  }%
}
Only on this page we add a red cross.
\newpage
This page has the circle only.
\par
\vspace{\fill}
The next page will be discarded.
\newpage
\AtBeginShipoutNext{%
  \AtBeginShipoutDiscard
}
This page is discarded.
\newpage
The last page.
\end{document}
%    \end{macrocode}
%    \begin{macrocode}
%</example1>
%    \end{macrocode}
%
% \subsubsection{Example: adding TrimBox for dvipdfmx}
%
% Now an example from ``real life'' follows. Someone from the mailing  list
% for dvipdfmx wants to put a TrimBox on every page. If we use
% \cs{AtBeginShipout}, we have to put the \cs{special} inside the
% box \cs{AtBeginShipoutBox} that gets shipped out.
%    \begin{macrocode}
%<*example2>
\documentclass{minimal}
\usepackage{atbegshi}
\usepackage[
  dvipdfm,
  paperwidth=630bp,
  paperheight=810bp
]{geometry}
\AtBeginShipout{%
  \setbox\AtBeginShipoutBox=\hbox{%
    \special{pdf: put @thispage <</TrimBox[9 9 621 801]>>}%
    \box\AtBeginShipoutBox
  }%
}
\begin{document}
  First page
  \newpage
  Second page
\end{document}
%</example2>
%    \end{macrocode}
% Remember, in \cs{AtBeginShipoutBoxFirst} the \cs{setbox} wrapper code
% is implicitly given and the \cs{special} is used directly.
%
% \section{Method of \cs{shipout} overloading}
%
% \subsection{\cs{shipout}}
%
% The \TeX\ primitive command \cs{shipout} takes a box specification
% and puts the box as a new page in the output file.
% There are two kinds of box specifications:
% \begin{description}
% \item[Direct boxes:]
%   They are given by \cs{hbox}, \cs{vbox}, or \cs{vtop},\\
%   e.g. |\shipout\hbox{Hello World}|.
% \item[Indirect boxes:]
%   \cs{box} or \cs{copy} references a box register by number.
%   The box register contains the contents of the box.
%
%   \emph{Note:} \cs{box} also clears the box register globally.
% \end{description}
% Then we have to differentiate between void and empty boxes:
% \begin{description}
% \item[Void:]
%   Initially or after \cs{box} there is no box in the box register.
%   In this cases the box register is not empty, but \emph{void}.
% \item[Empty:]
%   A box with empty contents, such as
%   |\hbox{}| (= \cs{null}) or |\vbox{}| is an \emph{empty \cs{hbox}}
%   or \emph{empty \cs{vbox}}. If a box register holds such a
%   box, the box still exists, therefore the box register is
%   \emph{not void}.
% \end{description}
%
% \subsection{\cs{afterassignment}}
%
% We want to overload \cs{shipout} to do something with the box.
% It is quite impossible to do this reliable by catching the box
% using macro arguments. The
% variety of box specifications is too large, Examples:
% \begin{quote}
%   |\shipout\null|\\
%   |\shipout\vbox{...}|\\
%   |\shipout\vtop\bgroup ...\egroup|\\
%   |\shipout\box255|
% \end{quote}
% Even worse, the braces don't need to be balanced:
% \begin{quote}
%   |\shipout\hbox\bgroup}|\\
%   |\shipout\vbox{\egroup|
% \end{quote}
% Happily TeX provides a reliable way via \cs{afterassignment}.
% It takes a macro name and executes it just after the assignment.
%
% Now we can redefine \cs{shipout}. The box specification that
% follows \cs{shipout} is catched by \cs{setbox}. This is an
% assignment to a box register. \cs{afterassignment}
% notifies \TeX, that we want to call \cs{@test} right after
% the assignment:
%\begin{quote}
%\begin{verbatim}
%\shipout :=
%  \afterassignment\@test
%  \setbox\mybox=
%\end{verbatim}
%\end{quote}
% We have seen different box specifications. Indirect boxes are easy
% to understand:
% \begin{quote}
%   |\shipout\box0| $\Rightarrow$ |\setbox\mybox=\box0 |\fbox{\cs{@test}}
% \end{quote}
% However direct boxes can have arbitrary contents with lots of
% other assignments. It would be quite inpredictable if \TeX\ would
% put |\@test| after the first of such an assignment or after the
% box specification if the box lacks of assignments.
% Therefore \TeX\ puts |\@test| right at the beginning of the
% box specification, e.g:
% \begin{quote}
%   |\shipout\hbox{Hello World}|\\
%   $\Rightarrow$
%   |\setbox\mybox=\hbox{|\fbox{\cs{@test}}| Hello World}|
% \end{quote}
%
% \subsection{Test for direct or indirect boxes}
%
% Now we want to execute \cs{@test}, but where are we?
% We can be after the completed box assignment, if \cs{shipout}
% was called with an indirect box. Or we are right at the beginning
% of a direct box.
%
% \subsubsection{With \eTeX}
%
% With the \eTeX's extensions the answer is very easy:
% Being inside the direct box means that we are inside a new group.
% The new primitive command \cs{currentgrouplevel} tells how
% deeply the groups are currently nested. Macro \cs{@test} just
% compares the previously stored group level with the current one:
%\begin{quote}
%\begin{verbatim}
%\shipout :=
%  \edef\saved@grouplevel{\number\currentgrouplevel}
%  \afterassignment\@test
%  \setbox\mybox=
%
%\@test :=
%  \ifnum\saved@grouplevel=\currentgrouplevel
%    % case: indirect box, the assignment is completed
%    \@output
%  \else
%    % case: direct box, we are inside the box
%    \aftergroup\@outbox
%  \fi
%\end{verbatim}
%\end{quote}
%
% \subsubsection{Without \eTeX}
%
% Life becomes complicate without \eTeX. We cannot ask the
% group level. However, if we are inside a direct box,
% the box register \cs{mybox} is not yet changed by \cs{setbox}.
% Thus we need a special initial value and compare
% it in \cs{@test} with the current value of the box.
%
% What can be used as initial value? Arbitrary box contents cannot
% be compared. \TeX\ only tells us a few properties:
% \begin{itemize}
% \item Box type:\cs{ifhbox}, \cs{ifvbox}
% \item Dimensions: \cs{wd}, \cs{ht}, \cs{dp}
% \item Voidness: \cs{ifvoid}
% \end{itemize}
% Unhappily all these qualities even combined are not sufficient
% for constructing an initial box value, because \cs{shipout} can be
% called with a box that is accidently just the same as the choosen
% initial value.
%
% Nevertheless we have two alternatives for an initial value:
% \begin{itemize}
% \item A box of some type with some funny settings that are
% unlikely to occur in real life, e.g a height of |4911sp-\maxdimen|.
% \item A void box.
% \end{itemize}
% A collision between this initial value and an indirect \cs{shipout}
% box with just the same value is possible.
% Then \cs{@test} will make a wrong decision that it is executed
% inside a direct box and delays \cs{@output} by \cs{aftergroup}.
% Thus \cs{@output} is not called at the place we want.
% In contrary, the result is an uncertainty about the place:
% \begin{itemize}
% \item \cs{shipout} is used in a group that perhaps closes some
%   pages later. A bad place for \cs{@output}.
% \item Without a surrounding group \cs{aftergroup} effectively kills
%   its argument.
% \end{itemize}
% In the first case of a box with special dimensions we can even loose
% the page. However in the case of the void box, this effect is even
% desired, because the original \cs{shipout} does not output void boxes.
% All we have to do is to ensure that our box \cs{mybox} is always
% void except for the phase when the overloaded \cs{shipout} is executed.
% And secondly we must keep this semantics of \cs{shipout} for the void case
% in our macros, namely \cs{@output}.
%\begin{quote}
%\begin{verbatim}
%\shipout :=
%  % trick to get a void box \mybox
%  \begingroup
%    \setbox\mybox=\box\mybox
%  \endgroup
%  \afterassignment\@test
%  \setbox\mybox=
%
%\@test :=
%  \ifvoid\mybox
%    \aftergroup\@output
%  \else
%    \@output
%  \fi
%\end{verbatim}
%\end{quote}
% The nasty case is |\shipout\box\voidb@x| where the indirect box
% is void and that must not generate an output page.
% If a surrounding group is missing the output is ignored
% because of \cs{aftergroup}.
% Otherwise \@output is called some time
% later when the surrounding group closes.
% But \cs{mybox} is void outside the execution phase
% of the redefined \cs{shipout}. Also \cs{@output} checks for a
% void box and cancels the page output.
% The disadvantage remains that the hook in \cs{@output} is called
% for a page that will not be output.
%
% \subsubsection{\cs{lastkern} method}
%
% At the beginning of a new box, there is no \cs{kern}, the contents
% of the box is still empty and \cs{lastkern} returns 0\,pt.
% This can be used to distinguish between direct and indirect boxes:
% We execute \cs{setbox} in a box with a preceding non-zero kern.
% After an indirect box, \cs{lastkern} sees this kern, otherwise
% it returns 0\,pt.
%\begin{quote}
%\begin{verbatim}
%\shipout :=
%  \begingroup
%    \setbox\mybox=\hbox\bgroup
%      \kern1pt
%      \afterassignment\shipout@test
%      \global\setbox\mybox=
%\@test :=
%  \ifdim\lastkern=0pt
%    % direct box
%    \aftergroup\egroup
%    \aftergroup\endgroup
%    \aftergroup\@output
%  \else
%    \egroup
%    \endgroup
%    \@output
%  \fi
%\end{verbatim}
%\end{quote}
% We have two \cs{setbox} commands. The first creates a controlled
% context box where we can safely insert a \cs{kern}. We get rid of
% this temporarily used context box by putting the local \cs{setbox}
% in a group.
%
% After the group we want to have our shipout box in \cs{mybox}.
% Therefore we use a global assignment here.
%
% \subsection{Output}
%
% With or without \eTeX\ we ensure the original behaviour of
% \cs{shipout} that void boxes do not generate output pages.
%
% Now we can place the hook \cs{@hook} for the user code that
% wants to manipulate the output box.
%\begin{quote}
%\begin{verbatim}
%\@output :=
%  \ifvoid\mybox
%    % cancel output of void box
%  \else
%    \@hook
%    \ifvoid\mybox
%      % user code in \@hook could has voided the box
%    \else
%      \original@shipout\box\mybox
%    \fi
%  \fi
%\end{verbatim}
%\end{quote}
%
% \subsection{Separate box register}
%
% So far we have said nothing about the box number of \cs{mybox}.
% The following case that outputs the same page twice
% shows that we are not free in the use
% of the box register:
% \begin{quote}
%   |\shipout\copy<num> \shipout\box<num>|
% \end{quote}
% We manipulate the box by the hook and without \eTeX\ the box
% must even be voided. However, the use case above requires that
% the box contents does not change at all. Therefore we must
% reserve a separate box register to avoid collisions with
% user box registers.
%
% \emph{Note:} Box register number 255 is special for the output
% routine, because TeX complains if this box is not voided
% by the output routine.
% However, this requirement does not apply to \cs{shipout} at all.
% In fact \cs{shipout} does not change any box register. This is
% usually done by a call of \cs{box}, but the output routine
% can do it later \emph{after} invoking of \cs{shipout}.
%
% \subsection{Summary}
%
% \subsubsection{With \eTeX}
%
% Putting the pieces together we get for \eTeX:
%\begin{quote}
%\begin{verbatim}
%\newbox\mybox
%\let\original@shipout\shipout
%
%\shipout :=
%  \edef\saved@grouplevel{\number\currentgrouplevel}
%  \afterassignment\@test
%  \setbox\mybox=
%
%\@test :=
%  \ifnum\saved@grouplevel<\currentgrouplevel
%    \expandafter\aftergroup
%  \fi
%  \@output
%
%\@output :=
%  \ifvoid\mybox
%    % cancel output of void box
%  \else
%    \@hook
%    \ifvoid\mybox
%      % user code in \@hook could have voided the box
%    \else
%      \original@shipout\box\mybox
%    \fi
%  \fi
%\end{verbatim}
%\end{quote}
%
% \subsubsection{Without \eTeX, traditional way}
%
% And for \TeX\ without \eTeX:
%\begin{quote}
%\begin{verbatim}
%\newbox\mybox
%\begingroup
%   \setbox\mybox=\box\mybox % ensure \mybox is void
%\endgroup
%\let\original@shipout\shipout
%
%\shipout :=
%  % trick to get a void box \mybox
%  \begingroup
%    \setbox\mybox=\box\mybox
%  \endgroup
%  \afterassignment\@test
%  \setbox\mybox=
%
%\@test :=
%  \ifvoid\mybox
%    \expandafter\aftergroup
%  \fi
%  \@output
%
%\@output :=
%  \ifvoid\mybox
%    % cancel output of void box
%  \else
%    \@hook
%    \ifvoid\mybox
%      % user code in \@hook could have voided the box
%    \else
%      \original@shipout\box\mybox
%    \fi
%  \fi
%\end{verbatim}
%\end{quote}
%
% \subsubsection{\cs{lastkern} method}
%
% And for \TeX\ without \eTeX\ using the \cs{lastkern} method:
%\begin{quote}
%\begin{verbatim}
%\newbox\mybox
%\let\original@shipout\shipout
%
%\shipout :=
%  \begingroup
%  \setbox\mybox=\hbox\bgroup
%  \kern1pt
%  \afterassignment\@test
%  \setbox\mybox=
%
%\@test :=
%  \ifdim\lastkern=0pt
%    \expandafter\aftergroup
%  \fi
%  \@output
%
%\@output :=
%  \egroup
%  \endgroup
%  \ifvoid\mybox
%    % cancel output of void box
%  \else
%    \@hook
%    \ifvoid\mybox
%      % user code in \@hook could have voided the box
%    \else
%      \original@shipout\box\mybox
%    \fi
%  \fi
%\end{verbatim}
%\end{quote}
%
%
% \StopEventually{
% }
%
% \section{Implementation}
%
% Package \xpackage{atbegshi} uses \eTeX's \cs{currentgrouplevel},
% if it is available. Otherwise the \cs{lastkern} method is used.
%
%    \begin{macrocode}
%<*package>
%    \end{macrocode}
%
% \subsection{Reload check and package identification}
%    Reload check, especially if the package is not used with \LaTeX.
%    \begin{macrocode}
\begingroup\catcode61\catcode48\catcode32=10\relax%
  \catcode13=5 % ^^M
  \endlinechar=13 %
  \catcode35=6 % #
  \catcode39=12 % '
  \catcode44=12 % ,
  \catcode45=12 % -
  \catcode46=12 % .
  \catcode58=12 % :
  \catcode64=11 % @
  \catcode123=1 % {
  \catcode125=2 % }
  \expandafter\let\expandafter\x\csname ver@atbegshi.sty\endcsname
  \ifx\x\relax % plain-TeX, first loading
  \else
    \def\empty{}%
    \ifx\x\empty % LaTeX, first loading,
      % variable is initialized, but \ProvidesPackage not yet seen
    \else
      \expandafter\ifx\csname PackageInfo\endcsname\relax
        \def\x#1#2{%
          \immediate\write-1{Package #1 Info: #2.}%
        }%
      \else
        \def\x#1#2{\PackageInfo{#1}{#2, stopped}}%
      \fi
      \x{atbegshi}{The package is already loaded}%
      \aftergroup\endinput
    \fi
  \fi
\endgroup%
%    \end{macrocode}
%    Package identification:
%    \begin{macrocode}
\begingroup\catcode61\catcode48\catcode32=10\relax%
  \catcode13=5 % ^^M
  \endlinechar=13 %
  \catcode35=6 % #
  \catcode39=12 % '
  \catcode40=12 % (
  \catcode41=12 % )
  \catcode44=12 % ,
  \catcode45=12 % -
  \catcode46=12 % .
  \catcode47=12 % /
  \catcode58=12 % :
  \catcode64=11 % @
  \catcode91=12 % [
  \catcode93=12 % ]
  \catcode123=1 % {
  \catcode125=2 % }
  \expandafter\ifx\csname ProvidesPackage\endcsname\relax
    \def\x#1#2#3[#4]{\endgroup
      \immediate\write-1{Package: #3 #4}%
      \xdef#1{#4}%
    }%
  \else
    \def\x#1#2[#3]{\endgroup
      #2[{#3}]%
      \ifx#1\@undefined
        \xdef#1{#3}%
      \fi
      \ifx#1\relax
        \xdef#1{#3}%
      \fi
    }%
  \fi
\expandafter\x\csname ver@atbegshi.sty\endcsname
\ProvidesPackage{atbegshi}%
  [2016/06/09 v1.18 At begin shipout hook (HO)]%
%    \end{macrocode}
%
% \subsection{Catcodes}
%
%    \begin{macrocode}
\begingroup\catcode61\catcode48\catcode32=10\relax%
  \catcode13=5 % ^^M
  \endlinechar=13 %
  \catcode123=1 % {
  \catcode125=2 % }
  \catcode64=11 % @
  \def\x{\endgroup
    \expandafter\edef\csname AtBegShi@AtEnd\endcsname{%
      \endlinechar=\the\endlinechar\relax
      \catcode13=\the\catcode13\relax
      \catcode32=\the\catcode32\relax
      \catcode35=\the\catcode35\relax
      \catcode61=\the\catcode61\relax
      \catcode64=\the\catcode64\relax
      \catcode123=\the\catcode123\relax
      \catcode125=\the\catcode125\relax
    }%
  }%
\x\catcode61\catcode48\catcode32=10\relax%
\catcode13=5 % ^^M
\endlinechar=13 %
\catcode35=6 % #
\catcode64=11 % @
\catcode123=1 % {
\catcode125=2 % }
\def\TMP@EnsureCode#1#2{%
  \edef\AtBegShi@AtEnd{%
    \AtBegShi@AtEnd
    \catcode#1=\the\catcode#1\relax
  }%
  \catcode#1=#2\relax
}
\TMP@EnsureCode{40}{12}% (
\TMP@EnsureCode{41}{12}% )
\TMP@EnsureCode{44}{12}% ,
\TMP@EnsureCode{45}{12}% -
\TMP@EnsureCode{47}{12}% /
\TMP@EnsureCode{46}{12}% .
\TMP@EnsureCode{58}{12}% :
\TMP@EnsureCode{91}{12}% [
\TMP@EnsureCode{93}{12}% ]
\TMP@EnsureCode{94}{7}% ^(superscript)
\TMP@EnsureCode{96}{12}% `
\edef\AtBegShi@AtEnd{\AtBegShi@AtEnd\noexpand\endinput}
%    \end{macrocode}
%
% \subsection{Preparations}
%
%    \begin{macrocode}
\begingroup\expandafter\expandafter\expandafter\endgroup
\expandafter\ifx\csname RequirePackage\endcsname\relax
  \def\TMP@RequirePackage#1[#2]{%
    \begingroup\expandafter\expandafter\expandafter\endgroup
    \expandafter\ifx\csname ver@#1.sty\endcsname\relax
      \input #1.sty\relax
    \fi
  }%
  \TMP@RequirePackage{infwarerr}[2007/09/09]%
  \TMP@RequirePackage{ltxcmds}[2010/03/01]%
\else
  \RequirePackage{infwarerr}[2007/09/09]%
  \RequirePackage{ltxcmds}[2010/03/01]%
\fi
%    \end{macrocode}
%
%    \begin{macro}{\AtBegShi@CheckDefinable}
%    \begin{macrocode}
\begingroup\expandafter\expandafter\expandafter\endgroup
\expandafter\ifx\csname @ifdefinable\endcsname\relax
  \def\AtBegShi@CheckDefinable#1{%
    \ifcase\ifx#1\relax
             \ltx@one
           \else
             \ifx#1\@undefined
               \ltx@one
             \else
               \ltx@zero
             \fi
           \fi
      \@PackageError{atbegshi}{%
        \string#1\space is already defined%
      }\@ehd
    \fi
  }%
\else
  \def\AtBegShi@CheckDefinable#1{%
    \@ifdefinable{#1}{}%
  }%
\fi
%    \end{macrocode}
%    \end{macro}
%
%    \begin{macro}{\ifAtBegShi@Discarded}
%    \begin{macrocode}
\ltx@newif\ifAtBegShi@Discarded
%    \end{macrocode}
%    \end{macro}
%    \begin{macro}{\AtBeginShipoutDiscard}
%    \begin{macrocode}
\AtBegShi@CheckDefinable\AtBeginShipoutDiscard
\def\AtBeginShipoutDiscard{%
  \deadcycles=\ltx@zero
  \global\AtBegShi@Discardedtrue
}
%    \end{macrocode}
%    \end{macro}
%
%    \begin{macrocode}
\begingroup\expandafter\expandafter\expandafter\endgroup
\expandafter\ifx\csname currentgrouplevel\endcsname\relax
  \catcode`X=9  % ignore
  \catcode`E=14 % comment
\else
  \catcode`X=14 % comment
  \catcode`E=9  % ignore
\fi
%    \end{macrocode}
%
%    \begin{macro}{\AtBegShi@Shipout}
%    \begin{macrocode}
\def\AtBegShi@Shipout{%
X \begingroup
X \setbox\AtBeginShipoutBox=\hbox\bgroup
X \kern\p@
E \edef\AtBegShi@GroupLevel{\number\currentgrouplevel}%
  \afterassignment\AtBegShi@Test
X \global
  \setbox\AtBeginShipoutBox=%
}
%    \end{macrocode}
%    \end{macro}
%
%    \begin{macro}{\AtBegShi@Test}
%    \begin{macrocode}
\def\AtBegShi@Test{%
X \ifdim\lastkern=0pt %
E \ifnum\AtBegShi@GroupLevel<\currentgrouplevel
    \expandafter\aftergroup
  \fi
  \AtBegShi@Output
}
%    \end{macrocode}
%    \end{macro}
%
%    \begin{macro}{\AtBegShi@Output}
%    \begin{macrocode}
\def\AtBegShi@Output{%
X \egroup
X \endgroup
  \ifvoid\AtBeginShipoutBox
    \@PackageWarning{atbegshi}{Ignoring void shipout box}%
  \else
    \let\AtBegShi@OrgProtect\protect
    \csname set@typeset@protect\endcsname
    \global\AtBegShi@Discardedfalse
    \AtBegShi@Hook
    \expandafter\gdef\expandafter\AtBegShi@HookNext
    \expandafter{\expandafter}%
    \AtBegShi@HookNext
    \ifAtBegShi@Discarded
      \@PackageInfoNoLine{atbegshi}{Shipout page discarded}%
      \global\AtBegShi@Discardedfalse
      \begingroup
        \setbox\AtBeginShipoutBox\box\AtBeginShipoutBox
      \endgroup
      \let\protect\AtBegShi@OrgProtect
    \else
      \AtBegShi@First
      \let\protect\AtBegShi@OrgProtect
      \AtBegShi@GetBoxSize\AtBeginShipoutBox
      \ltx@ifundefined{AtNextShipout}{%
      }{%
        \AtNextShipout{\AtBegShi@GetBoxSize\@cclv}%
      }%
      \AtBeginShipoutOriginalShipout\box\AtBeginShipoutBox
    \fi
  \fi
}
%    \end{macrocode}
%    \end{macro}
%
%    \begin{macro}{\AtBegShi@GetBoxSize}
%    \begin{macrocode}
\def\AtBegShi@GetBoxSize#1{%
  \xdef\AtBeginShipoutBoxWidth{\the\wd#1}%
  \xdef\AtBeginShipoutBoxHeight{\the\ht#1}%
  \xdef\AtBeginShipoutBoxDepth{\the\dp#1}%
}
%    \end{macrocode}
%    \end{macro}
%    \begin{macro}{\AtBeginShipoutBoxWidth}
%    \begin{macrocode}
\def\AtBeginShipoutBoxWidth{0pt}
%    \end{macrocode}
%    \end{macro}
%    \begin{macro}{\AtBeginShipoutBoxHeight}
%    \begin{macrocode}
\def\AtBeginShipoutBoxHeight{0pt}
%    \end{macrocode}
%    \end{macro}
%    \begin{macro}{\AtBeginShipoutBoxDepth}
%    \begin{macrocode}
\def\AtBeginShipoutBoxDepth{0pt}
%    \end{macrocode}
%    \end{macro}
%
%    \begin{macrocode}
\catcode`\X=11 %
\catcode`\E=11 %
%    \end{macrocode}
%
%    \begin{macro}{\AtBegShi@First}
%    \begin{macrocode}
\def\AtBegShi@First{%
  \ifx\AtBegShi@HookFirst\ltx@empty
  \else
    \AtBeginShipoutAddToBox{\AtBegShi@HookFirst}%
  \fi
  \global\let\AtBegShi@First\ltx@empty
  \global\let\AtBeginShipoutFirst\AtBegShi@FirstDisabled
}
%    \end{macrocode}
%    \end{macro}
%
%    \begin{macro}{\AtBegShi@Hook}
%    \begin{macrocode}
\gdef\AtBegShi@Hook{}
%    \end{macrocode}
%    \end{macro}
%    \begin{macro}{\AtBegShi@HookNext}
%    \begin{macrocode}
\gdef\AtBegShi@HookNext{}
%    \end{macrocode}
%    \end{macro}
%    \begin{macro}{\AtBegShi@HookFirst}
%    \begin{macrocode}
\gdef\AtBegShi@HookFirst{}
%    \end{macrocode}
%    \end{macro}
%    \begin{macro}{\AtBeginShipout}
%    \begin{macrocode}
\AtBegShi@CheckDefinable\AtBeginShipout
\def\AtBeginShipout{%
  \AtBegShi@AddHook\AtBegShi@Hook
}
%    \end{macrocode}
%    \end{macro}
%    \begin{macro}{\AtBeginShipoutNext}
%    \begin{macrocode}
\AtBegShi@CheckDefinable\AtBeginShipoutNext
\def\AtBeginShipoutNext{%
  \AtBegShi@AddHook\AtBegShi@HookNext
}
%    \end{macrocode}
%    \end{macro}
%    \begin{macro}{\AtBeginShipoutFirst}
%    \begin{macrocode}
\AtBegShi@CheckDefinable\AtBeginShipoutFirst
\def\AtBeginShipoutFirst{%
  \AtBegShi@AddTo\AtBegShi@HookFirst
}
%    \end{macrocode}
%    \end{macro}
%    \begin{macro}{\AtBegShi@FirstDisabled}
%    \begin{macrocode}
\long\def\AtBegShi@FirstDisabled#1{%
  \@PackageWarning{atbegshi}{%
    First page is already shipped out, ignoring\MessageBreak
    \string\AtBeginShipoutFirst
  }%
}
%    \end{macrocode}
%    \end{macro}
%
%    \begin{macro}{\AtBegShi@AddTo}
%    \begin{macrocode}
\begingroup\expandafter\expandafter\expandafter\endgroup
\expandafter\ifx\csname g@addto@macro\endcsname\relax
  \long\def\AtBegShi@AddTo#1#2{%
    \begingroup
      \toks\ltx@zero\expandafter{#1#2}%
      \xdef#1{\the\toks\ltx@zero}%
    \endgroup
  }%
\else
  \let\AtBegShi@AddTo\g@addto@macro
\fi
%    \end{macrocode}
%    \end{macro}
%
%    \begin{macro}{\AtBegShi@AddHook}
%    \begin{macrocode}
\long\def\AtBegShi@AddHook#1#2{%
  \AtBegShi@AddTo#1{\AtBegShi@Item{#2}}%
}
%    \end{macrocode}
%    \end{macro}
%
%    \begin{macro}{\AtBegShi@Item}
%    \begin{macrocode}
\long\def\AtBegShi@Item#1{%
  \ifAtBegShi@Discarded
  \else
    #1%
    \ifAtBegShi@Discarded
    \else
      \ifvoid\AtBeginShipoutBox
        \@PackageWarning{atbegshi}{%
          Shipout box was voided by hook,\MessageBreak
          ignoring shipout box%
        }%
        \AtBeginShipoutDiscard
      \fi
    \fi
  \fi
}
%    \end{macrocode}
%    \end{macro}
%
%    \begin{macro}{\AtBeginShipoutInit}
%    \begin{macrocode}
\AtBegShi@CheckDefinable\AtBeginShipoutInit
\def\AtBeginShipoutInit{%
  \ltx@IfUndefined{newbox}{%
    \@PackageError{atbegshi}{%
      \string\AtBeginShipoutInit\space failed\MessageBreak
       because of missing \expandafter\string\csname newbox\endcsname
    }\@ehc
  }{%
    \csname newbox\endcsname\AtBeginShipoutBox
    \AtBegShi@CheckDefinable\AtBeginShipoutOriginalShipout
    \global\let\AtBeginShipoutOriginalShipout\shipout
    \global\let\shipout\AtBegShi@Shipout
  }%
  \gdef\AtBeginShipoutInit{}%
}
%    \end{macrocode}
%    \end{macro}
%    \begin{macrocode}
\begingroup\expandafter\expandafter\expandafter\endgroup
\expandafter\ifx\csname AtBeginDocument\endcsname\relax
  \AtBeginShipoutInit
\else
  \AtBeginDocument{\AtBeginShipoutInit}%
\fi
%    \end{macrocode}
%
% \subsection{Additions to the shipout box}
%
%    \begin{macro}{\AtBeginShipoutAddToBox}
%    \begin{macrocode}
\def\AtBeginShipoutAddToBox#1{%
  \ifhbox\AtBeginShipoutBox
    \edef\AtBegShi@restore{%
      \hfuzz=\the\hfuzz\relax
      \hbadness=\the\hbadness\relax
    }%
    \hfuzz=1073741823sp\relax
    \hbadness=2147483647\relax
    \setbox\AtBeginShipoutBox=\hbox to \wd\AtBeginShipoutBox{%
      \setbox\ltx@zero=\hbox{%
        \begingroup
          \AtBegShi@restore
          #1%
        \endgroup
      }%
      \wd\ltx@zero=0pt\relax
      \ht\ltx@zero=0pt\relax
      \dp\ltx@zero=0pt\relax
      \raise\ht\AtBeginShipoutBox\copy\ltx@zero
      \unhcopy\AtBeginShipoutBox
    }%
    \AtBegShi@restore
  \else
    \ifvbox\AtBeginShipoutBox
      \edef\AtBegShi@restore{%
        \vfuzz=\the\vfuzz\relax
        \vbadness=\the\vbadness\relax
        \dimen\ltx@zero=\the\dimen\ltx@zero\relax
      }%
      \edef\AtBegShi@restorebox{%
        \ht\AtBeginShipoutBox=\the\ht\AtBeginShipoutBox\relax
        \dp\AtBeginShipoutBox=\the\dp\AtBeginShipoutBox\relax
      }%
      \vfuzz=1073741823sp\relax
      \vbadness=2147483647\relax
      \dimen\ltx@zero=\ht\AtBeginShipoutBox
      \advance\dimen\ltx@zero by \dp\AtBeginShipoutBox
      \setbox\AtBeginShipoutBox=\vbox to \dimen\ltx@zero{%
        \setbox\ltx@zero=\hbox{%
          \begingroup
            \AtBegShi@restore
            #1%
          \endgroup
        }%
        \wd\ltx@zero=0pt\relax
        \ht\ltx@zero=0pt\relax
        \dp\ltx@zero=0pt\relax
        \baselineskip=0pt\relax
        \lineskip=0pt\relax
        \lineskiplimit=0pt\relax
        \copy\ltx@zero
        \unvbox\AtBeginShipoutBox
        \kern0pt%
      }%
      \AtBegShi@restore
      \AtBegShi@restorebox
    \fi
  \fi
}
%    \end{macrocode}
%    \end{macro}
%
%    \begin{macro}{\AtBeginShipoutAddToBoxForeground}
%    \begin{macrocode}
\def\AtBeginShipoutAddToBoxForeground#1{%
  \ifhbox\AtBeginShipoutBox
    \edef\AtBegShi@restore{%
      \hfuzz=\the\hfuzz\relax
      \hbadness=\the\hbadness\relax
    }%
    \hfuzz=1073741823sp\relax
    \hbadness=2147483647\relax
    \setbox\AtBeginShipoutBox=\hbox to \wd\AtBeginShipoutBox{%
      \unhcopy\AtBeginShipoutBox
      \kern-\wd\AtBeginShipoutBox
      \setbox\ltx@zero=\hbox{%
        \begingroup
          \AtBegShi@restore
          #1%
        \endgroup
      }%
      \wd\ltx@zero=0pt\relax
      \ht\ltx@zero=0pt\relax
      \dp\ltx@zero=0pt\relax
      \raise\ht\AtBeginShipoutBox\copy\ltx@zero
      \kern\wd\AtBeginShipoutBox
    }%
    \AtBegShi@restore
  \else
    \ifvbox\AtBeginShipoutBox
      \edef\AtBegShi@restore{%
        \vfuzz=\the\vfuzz\relax
        \vbadness=\the\vbadness\relax
        \dimen\ltx@zero=\the\dimen\ltx@zero\relax
      }%
      \edef\AtBegShi@restorebox{%
        \ht\AtBeginShipoutBox=\the\ht\AtBeginShipoutBox\relax
        \dp\AtBeginShipoutBox=\the\dp\AtBeginShipoutBox\relax
      }%
      \vfuzz=1073741823sp\relax
      \vbadness=2147483647\relax
      \dimen\ltx@zero=\ht\AtBeginShipoutBox
      \advance\dimen\ltx@zero by \dp\AtBeginShipoutBox
      \setbox\AtBeginShipoutBox=\vbox to \dimen\ltx@zero{%
        \setbox\ltx@zero=\hbox{%
          \begingroup
            \AtBegShi@restore
            #1%
          \endgroup
        }%
        \wd\ltx@zero=0pt\relax
        \ht\ltx@zero=0pt\relax
        \dp\ltx@zero=0pt\relax
        \baselineskip=0pt\relax
        \lineskip=0pt\relax
        \lineskiplimit=0pt\relax
        \unvbox\AtBeginShipoutBox
        \kern-\dimen\ltx@zero
        \copy\ltx@zero
        \kern\dimen\ltx@zero
      }%
      \AtBegShi@restore
      \AtBegShi@restorebox
    \fi
  \fi
}
%    \end{macrocode}
%    \end{macro}
%
% \subsection{Positioning}
%
%    \begin{macrocode}
\begingroup\expandafter\expandafter\expandafter\endgroup
\expandafter\ifx\csname RequirePackage\endcsname\relax
  \def\TMP@RequirePackage#1[#2]{%
    \begingroup\expandafter\expandafter\expandafter\endgroup
    \expandafter\ifx\csname ver@#1.sty\endcsname\relax
      \input #1.sty\relax
    \fi
  }%
  \TMP@RequirePackage{ifpdf}[2011/01/30]%
\else
  \RequirePackage{ifpdf}[2011/01/30]%
\fi
%    \end{macrocode}
%
%    \begin{macrocode}
\ifpdf
  \def\AtBegShi@horigin{%
    \ifx\pdfhorigin\@undefined\pdfvariable horigin\else\pdfhorigin\fi}%
  \def\AtBegShi@vorigin{%
    \ifx\pdfvorigin\@undefined\pdfvariable vorigin\else\pdfvorigin\fi}%
\else
  \def\AtBegShi@horigin{72.27pt}%
  \def\AtBegShi@vorigin{72.27pt}%
\fi
%    \end{macrocode}
%
%    \begin{macrocode}
\begingroup
\ifcase
    \expandafter\ifx\csname picture\endcsname\relax
      1%
    \else
      \expandafter\ifx\csname endpicture\endcsname\relax
        1%
      \else
        0%
      \fi
    \fi
  \endgroup
  \def\AtBegShi@BeginPicture{%
    \begingroup
    \picture(0,0)\relax
    \begingroup\expandafter\expandafter\expandafter\endgroup
    \expandafter\ifx\csname unitlength\endcsname\relax
    \else
      \unitlength=1pt\relax
    \fi
    \ignorespaces
  }%
  \def\AtBegShi@EndPicture{%
    \endpicture
    \endgroup
  }%
\else
  \endgroup
  \def\AtBegShi@BeginPicture{%
    \setbox\ltx@zero=\hbox\bgroup
    \begingroup
    \ignorespaces
  }%
  \def\AtBegShi@EndPicture{%
    \endgroup
    \egroup
    \ht\ltx@zero=0pt\relax
    \dp\ltx@zero=0pt\relax
    \copy\ltx@zero
  }%
\fi
%    \end{macrocode}
%
%    \begin{macro}{\AtBeginShipoutUpperLeft}
%    A surrounding \cs{rlap} is not necessary, because
%    the stuff is put in an \cs{hbox} with zero width.
%    \begin{macrocode}
\def\AtBeginShipoutUpperLeft#1{%
  \AtBeginShipoutAddToBox{%
    \kern-\AtBegShi@horigin\relax
    \vbox to 0pt{%
      \kern-\AtBegShi@vorigin\relax
      \AtBegShi@BeginPicture
      #1%
      \AtBegShi@EndPicture
      \vss
    }%
  }%
}
%    \end{macrocode}
%    \end{macro}
%
%    \begin{macro}{\AtBeginShipoutUpperLeftForeground}
%    \begin{macrocode}
\def\AtBeginShipoutUpperLeftForeground#1{%
  \AtBeginShipoutAddToBoxForeground{%
    \kern-\AtBegShi@horigin\relax
    \vbox to 0pt{%
      \kern-\AtBegShi@vorigin\relax
      \AtBegShi@BeginPicture
      #1%
      \AtBegShi@EndPicture
      \vss
    }%
  }%
}
%    \end{macrocode}
%    \end{macro}
%
% \subsection{Patches}
%
%    Patches for \LaTeX\ packages that redefine \cs{shipout}.
%    \LaTeX\ is now supposed to use \eTeX. Thus we do not patch,
%    without \LaTeX\ and \eTeX.
%    \begin{macrocode}
\def\AtBegShi@AbortIfUndefined#1{%
  \begingroup\expandafter\expandafter\expandafter\endgroup
  \expandafter\ifx\csname#1\endcsname\relax
    \expandafter\AtBegShi@AtEnd
  \fi
}
\AtBegShi@AbortIfUndefined{currentgrouplevel}%
\AtBegShi@AbortIfUndefined{AtBeginDocument}%
\AtBegShi@AbortIfUndefined{@ifpackageloaded}%
\AtBegShi@AbortIfUndefined{@ifclassloaded}%
%    \end{macrocode}
%
% \subsubsection{Package \xpackage{crop}}
%
% Fix of method and box.
%    \begin{macrocode}
\def\AtBegShi@PatchCrop{%
  \begingroup
    \def\AtBegShi@Crop@shipout{%
      \afterassignment\CROP@ship
      \setbox\@cclv=%
    }%
    \def\AtBegShi@Crop@ship{%
      \ifvoid\@cclv
        \expandafter\aftergroup
      \fi
      \CROP@@ship
    }%
    \def\AtBegShi@Crop@shiplist{%
      \lineskip\z@
      \lineskiplimit\z@
      \baselineskip\z@
      \CROP@kernel
      \box\@cclv
    }%
    \def\AtBegShi@Crop@@ship{%
      \CROP@shipout\vbox{%
        \CROP@shiplist
      }%
    }%
    \ifx\AtBegShi@Crop@ship\CROP@ship
      \ifx\AtBegShi@Crop@shiplist\CROP@shiplist
        \ifx\AtBegShi@Crop@@ship\CROP@@ship
          \let\AtBegShi@found\relax
          \ifx\shipout\AtBegShi@Crop@shipout
            \def\AtBegShi@found{\shipout}%
          \else\ifx\AtBeginShipoutOriginalShipout\AtBegShi@Crop@shipout
            \def\AtBegShi@found{\AtBeginShipoutOriginalShipout}%
          \else\ifx\@EveryShipout@Org@Shipout\AtBegShi@Crop@shipout
            \def\AtBegShi@found{\@EveryShipout@Org@Shipout}%
          \else\ifx\GPTorg@shipout\AtBegShi@Crop@shipout
            \def\AtBegShi@found{\GPTorg@shipout}%
          \else\ifx\THBorg@shipout\AtBegShi@Crop@shipout
            \def\AtBegShi@found{\THBorg@shipout}%
          \else\ifx\mem@oldshipout\AtBegShi@Crop@shipout
            \def\AtBegShi@found{\mem@oldshipout}%
          \fi\fi\fi\fi\fi\fi
          \ifx\AtBegShi@found\relax
          \else
            \expandafter\endgroup
            \expandafter\def\AtBegShi@found{%
              \edef\AtBegShi@GroupLevel{\number\currentgrouplevel}%
              \afterassignment\CROP@ship
              \setbox\AtBeginShipoutBox=%
            }%
            \def\CROP@ship{%
              \ifnum\AtBegShi@GroupLevel=\currentgrouplevel
              \else
                \expandafter\aftergroup
              \fi
              \CROP@@ship
            }%
            \def\CROP@shiplist{%
              \lineskip 0pt\relax
              \lineskiplimit 0pt\relax
              \baselineskip 0pt\relax
              \CROP@kernel
              \box\AtBeginShipoutBox
            }%
            \def\CROP@@ship{%
              \ifvoid\AtBeginShipoutBox
              \else
                \setbox\AtBeginShipoutBox=\vbox{%
                  \CROP@shiplist
                }%
                \AtBegShi@GetBoxSize\AtBeginShipoutBox
                \expandafter\CROP@shipout
                \expandafter\box
                \expandafter\AtBeginShipoutBox
              \fi
            }%
            \@PackageInfoNoLine{atbegshi}{Package `crop' patched}%
            \begingroup
          \fi
        \fi
      \fi
    \fi
  \endgroup
  \let\AtBegShi@PatchCrop\relax
}
\@ifpackageloaded{crop}{%
  \AtBegShi@PatchCrop
}{%
  \AtBeginDocument{\AtBegShi@PatchCrop}%
}
%    \end{macrocode}
%
% \subsubsection{Package \xpackage{everyshi}}
%
%     Fix of method. Use of box 255 is not changed.
%    \begin{macrocode}
\def\AtBegShi@PatchEveryshi{%
  \begingroup
    \long\def\AtBegShi@Everyshi@shipout{%
      \afterassignment\@EveryShipout@Test
      \global\setbox\@cclv= %
    }%
    \long\def\AtBegShi@Everyshi@Test{%
      \ifvoid\@cclv\relax
        \aftergroup\@EveryShipout@Output
      \else
        \@EveryShipout@Output
      \fi
    }%
    \ifx\AtBegShi@Everyshi@Test\@EveryShipout@Test
      \let\AtBegShi@found\relax
      \ifx\shipout\AtBegShi@Everyshi@shipout
        \def\AtBegShi@found{\shipout}%
      \else\ifx\AtBeginShipoutOriginalShipout\AtBegShi@Everyshi@shipout
        \def\AtBegShi@found{\AtBeginShipoutOriginalShipout}%
      \else\ifx\CROP@shipout\AtBegShi@Everyshi@shipout
        \def\AtBegShi@found{\CROP@shipout}%
      \else\ifx\GPTorg@shipout\AtBegShi@Everyshi@shipout
        \def\AtBegShi@found{\GPTorg@shipout}%
      \else\ifx\THBorg@shipout\AtBegShi@Everyshi@shipout
        \def\AtBegShi@found{\THBorg@shipout}%
      \else\ifx\mem@oldshipout\AtBegShi@Everyshi@shipout
        \def\AtBegShi@found{\mem@oldshipout}%
      \else
        \expandafter\ifx\csname @EveryShipout@Org@Shipout\endcsname
                        \relax
           \ifx\@EveryShipout@Shipout\AtBegShi@Everyshi@shipout
             \def\AtBegShi@found{\@EveryShipout@Shipout}%
           \fi
        \fi
      \fi\fi\fi\fi\fi\fi
      \ifx\AtBegShi@found\relax
      \else
        \expandafter\endgroup
        \expandafter\def\AtBegShi@found{%
          \edef\AtBegShi@GroupLevel{\number\currentgrouplevel}%
          \afterassignment\@EveryShipout@Test
          \setbox\AtBeginShipoutBox=%
        }%
        \def\@EveryShipout@Test{%
          \ifnum\AtBegShi@GroupLevel=\currentgrouplevel
          \else
            \expandafter\aftergroup
          \fi
          \AtBegShi@Everyshi@Output
        }%
        \def\AtBegShi@Everyshi@Output{%
          \ifvoid\AtBeginShipoutBox
          \else
            \global\setbox\ltx@cclv\box\AtBeginShipoutBox
            \expandafter\@EveryShipout@Output
          \fi
        }%
        \@PackageInfoNoLine{atbegshi}{Package `everyshi' patched}%
        \begingroup
      \fi
    \fi
  \endgroup
  \let\AtBegShi@PatchEveryshi\relax
}
\@ifpackageloaded{everyshi}{%
  \AtBegShi@PatchEveryshi
}{%
  \AtBeginDocument{\AtBegShi@PatchEveryshi}%
}
%    \end{macrocode}
%
% \subsubsection{Class \xpackage{memoir}}
%
%     Fix of method and box.
%    \begin{macrocode}
\def\AtBegShi@PatchMemoir{%
  \begingroup
    \def\AtBegShi@Memoir@shipout{%
      \afterassignment\mem@shipi
      \setbox\@cclv=%
    }%
    \def\AtBegShi@Memoir@shipi{%
      \ifvoid\@cclv
        \expandafter\aftergroup
      \fi
      \mem@shipii
    }%
    \def\AtBegShi@Memoir@shipiiA{%
      \mem@oldshipout\vbox{%
        \trimmarks
        \unvbox\@cclv
      }%
    }%
    \def\AtBegShi@Memoir@shipiiB{%
      \ifvoid\@cclv
        \mem@oldshipout\box\@cclv
      \else
        \mem@oldshipout\vbox{%
          \trimmarks
          \unvbox\@cclv
        }%
      \fi
    }%
    \def\AtBegShi@Memoir@PatchAB{%
      \ifvoid\AtBeginShipoutBox
      \else
        \setbox\AtBeginShipoutBox=\vbox{%
          \trimmarks
          \ifvbox\AtBeginShipoutBox
            \unvbox\AtBeginShipoutBox
          \else
            \box\AtBeginShipoutBox
          \fi
        }%
        \AtBegShi@GetBoxSize\AtBeginShipoutBox
        \expandafter\mem@oldshipout
        \expandafter\box
        \expandafter\AtBeginShipoutBox
      \fi
    }%
    \def\AtBegShi@Memoir@shipiiC{% 2008/08/07 v1.6180339a
      \ifvoid\@cclv
        \mem@oldshipout\box\@cclv
      \else
        \ifshowtrims
          \mem@oldshipout\vbox{\trimmarks\unvbox\@cclv}%
        \else
          \mem@oldshipout\box\@cclv
        \fi
      \fi
    }%
    \def\AtBegShi@Memoir@shipiiD{% 2011/03/06 v3.6j
      \ifvoid\@cclv
        \mem@oldshipout\box\@cclv
      \else
        \ifshowtrims
          \mem@oldshipout\vbox{%
            \trimmarks
            \nointerlineskip
            \box\@cclv
          }%
        \else
          \mem@oldshipout\box\@cclv
        \fi
      \fi
    }%
    \def\AtBegShi@Memoir@PatchCD{%
      \ifvoid\AtBeginShipoutBox
      \else
        \ifshowtrims
          \setbox\AtBeginShipoutBox=\vbox{%
            \trimmarks
            \nointerlineskip
            \box\AtBeginShipoutBox
          }%
        \fi
        \AtBegShi@GetBoxSize\AtBeginShipoutBox
        \expandafter\mem@oldshipout
        \expandafter\box
        \expandafter\AtBeginShipoutBox
      \fi
    }%
    \ifx\AtBegShi@Memoir@shipi\mem@shipi
      \let\AtBegShi@found\ltx@one
      \ifx\AtBegShi@Memoir@shipiiA\mem@shipii
        \let\AtBegShi@found\ltx@zero
        \global\let\AtBegShi@Memoir@PatchX\AtBegShi@Memoir@PatchAB
      \else\ifx\AtBegShi@Memoir@shipiiB\mem@shipii
        \let\AtBegShi@found\ltx@zero
        \global\let\AtBegShi@Memoir@PatchX\AtBegShi@Memoir@PatchAB
      \else\ifx\AtBegShi@Memoir@shipiiC\mem@shipii
        \let\AtBegShi@found\ltx@zero
        \global\let\AtBegShi@Memoir@PatchX\AtBegShi@Memoir@PatchCD
      \else\ifx\AtBegShi@Memoir@shipiiD\mem@shipii
        \let\AtBegShi@found\ltx@zero
        \global\let\AtBegShi@Memoir@PatchX\AtBegShi@Memoir@PatchCD
      \fi\fi\fi\fi
      \ifcase\AtBegShi@found
        \let\AtBegShi@found\relax
        \ifx\shipout\AtBegShi@Memoir@shipout
          \def\AtBegShi@found{\shipout}%
        \else\ifx\AtBeginShipoutOriginalShipout\AtBegShi@Memoir@shipout
          \def\AtBegShi@found{\AtBeginShipoutOriginalShipout}%
        \else\ifx\CROP@shipout\AtBegShi@Memoir@shipout
          \def\AtBegShi@found{\CROP@shipout}%
        \else\ifx\GPTorg@shipout\AtBegShi@Memoir@shipout
          \def\AtBegShi@found{\GPTorg@shipout}%
        \else\ifx\THBorg@shipout\AtBegShi@Memoir@shipout
          \def\AtBegShi@found{\THBorg@shipout}%
        \else\ifx\@EveryShipout@Org@Shipout\AtBegShi@Memoir@shipout
          \def\AtBegShi@found{\@EveryShipout@Org@Shipout}%
        \fi\fi\fi\fi\fi\fi
        \ifx\AtBegShi@found\relax
        \else
          \expandafter\endgroup
          \expandafter\def\AtBegShi@found{%
            \edef\AtBegShi@GroupLevel{\number\currentgrouplevel}%
            \afterassignment\mem@shipi
            \setbox\AtBeginShipoutBox=%
          }%
          \def\mem@shipi{%
            \ifnum\AtBegShi@GroupLevel=\currentgrouplevel
            \else
              \expandafter\aftergroup
            \fi
            \mem@shipii
          }%
          \let\mem@shipii\AtBegShi@Memoir@PatchX
          \@PackageInfoNoLine{atbegshi}{Class `memoir' patched}%
          \begingroup
        \fi
      \fi
    \fi
  \endgroup
  \let\AtBegShi@PatchMemoir\relax
}
\@ifclassloaded{memoir}{%
  \AtBegShi@PatchMemoir
}{%
  \AtBeginDocument{\AtBegShi@PatchMemoir}%
}
%    \end{macrocode}
%
%    \begin{macrocode}
\AtBegShi@AtEnd%
%</package>
%    \end{macrocode}
%
% \section{Test}
%
% \subsection{Catcode checks for loading}
%
%    \begin{macrocode}
%<*test1>
%    \end{macrocode}
%    \begin{macrocode}
\catcode`\{=1 %
\catcode`\}=2 %
\catcode`\#=6 %
\catcode`\@=11 %
\expandafter\ifx\csname count@\endcsname\relax
  \countdef\count@=255 %
\fi
\expandafter\ifx\csname @gobble\endcsname\relax
  \long\def\@gobble#1{}%
\fi
\expandafter\ifx\csname @firstofone\endcsname\relax
  \long\def\@firstofone#1{#1}%
\fi
\expandafter\ifx\csname loop\endcsname\relax
  \expandafter\@firstofone
\else
  \expandafter\@gobble
\fi
{%
  \def\loop#1\repeat{%
    \def\body{#1}%
    \iterate
  }%
  \def\iterate{%
    \body
      \let\next\iterate
    \else
      \let\next\relax
    \fi
    \next
  }%
  \let\repeat=\fi
}%
\def\RestoreCatcodes{}
\count@=0 %
\loop
  \edef\RestoreCatcodes{%
    \RestoreCatcodes
    \catcode\the\count@=\the\catcode\count@\relax
  }%
\ifnum\count@<255 %
  \advance\count@ 1 %
\repeat

\def\RangeCatcodeInvalid#1#2{%
  \count@=#1\relax
  \loop
    \catcode\count@=15 %
  \ifnum\count@<#2\relax
    \advance\count@ 1 %
  \repeat
}
\def\RangeCatcodeCheck#1#2#3{%
  \count@=#1\relax
  \loop
    \ifnum#3=\catcode\count@
    \else
      \errmessage{%
        Character \the\count@\space
        with wrong catcode \the\catcode\count@\space
        instead of \number#3%
      }%
    \fi
  \ifnum\count@<#2\relax
    \advance\count@ 1 %
  \repeat
}
\def\space{ }
\expandafter\ifx\csname LoadCommand\endcsname\relax
  \def\LoadCommand{\input atbegshi.sty\relax}%
\fi
\def\Test{%
  \RangeCatcodeInvalid{0}{47}%
  \RangeCatcodeInvalid{58}{64}%
  \RangeCatcodeInvalid{91}{96}%
  \RangeCatcodeInvalid{123}{255}%
  \catcode`\@=12 %
  \catcode`\\=0 %
  \catcode`\%=14 %
  \LoadCommand
  \RangeCatcodeCheck{0}{36}{15}%
  \RangeCatcodeCheck{37}{37}{14}%
  \RangeCatcodeCheck{38}{47}{15}%
  \RangeCatcodeCheck{48}{57}{12}%
  \RangeCatcodeCheck{58}{63}{15}%
  \RangeCatcodeCheck{64}{64}{12}%
  \RangeCatcodeCheck{65}{90}{11}%
  \RangeCatcodeCheck{91}{91}{15}%
  \RangeCatcodeCheck{92}{92}{0}%
  \RangeCatcodeCheck{93}{96}{15}%
  \RangeCatcodeCheck{97}{122}{11}%
  \RangeCatcodeCheck{123}{255}{15}%
  \RestoreCatcodes
}
\Test
\csname @@end\endcsname
\end
%    \end{macrocode}
%    \begin{macrocode}
%</test1>
%    \end{macrocode}
%
%    \begin{macrocode}
%<*test2>
\input atbegshi.sty\relax
\def\msg#{\immediate\write16}
\msg{File: atbegshi-test2.tex 2016/06/09 v1.18 Test file for plain-TeX}
\def\testmsg#1#2{%
  \msg{}%
  \msg{*** Test with box (#1), expected page output [#2]}% hash-ok
}

\newbox\voidbox
\def\void{\box\voidbox}
\begingroup
  \setbox\voidbox=\void
\endgroup

\count0=0\relax
\AtBeginShipout{%
  \global\advance\count0 by 1\relax
  \msg{* Inside \string\AtBeginShipout: [\the\count0]}%
}

\AtBeginShipoutFirst{%
  \msg{* Inside \string\AtBeginShipoutFirst}%
  Hello World%
}

\testmsg{\string\null}{1}
\shipout\null

\AtBeginShipoutFirst{%
  This is too late%
}

\testmsg{void}{}
\shipout\void

\testmsg{\string\copy255 (not void)}{2}
\setbox255\hbox{\vrule height 10bp width 10bp}
\shipout\copy255 %

\testmsg{\string\copy255 (again)}{3}
\shipout\copy255 %

\testmsg{\string\box255}{4}
\shipout\box255 %

\testmsg{\string\box255 (again)}{}
\shipout\box255 %

\testmsg{\string\hbox}{5}
\shipout\hbox{\vrule height 5bp width 20bp}

\testmsg{\string\vbox}{6}
\shipout\vbox{\hrule height 20bp width 5bp}

\testmsg{\string\null, voided by hook}{}
\def\VoidBox{%
  \begingroup
    \setbox\AtBeginShipoutBox=\box\AtBeginShipoutBox
  \endgroup
}
\AtBeginShipout{\VoidBox}
\shipout\null
\def\VoidBox{}

\msg{*** \string\begingroup}
\begingroup
  \testmsg{void}{}%
  \shipout\void
\msg{*** \string\endgroup}
\endgroup

\msg{*** \string\begingroup}
\begingroup
  \testmsg{void}{}%
  \shipout\void
  \testmsg{\string\null}{8}%
  \shipout\null
\msg{*** \string\endgroup}
\endgroup

\testmsg{output routine}{9}
Hello World
\vfill
\eject

\testmsg{\string\null\space(discarded)}{}
\AtBeginShipout{%
  \msg{* Inside \string\AtBeginShipout: DISCARD}%
  \AtBeginShipoutDiscard
}
\shipout\null

\end
%</test2>
%    \end{macrocode}
%
%    \begin{macrocode}
%<*test3>
\NeedsTeXFormat{LaTeX2e}
\ProvidesFile{atbegshi-test3.tex}[2016/06/09 v1.18 Test file for LaTeX]
\RequirePackage{color}
\pagecolor{yellow}
\documentclass[a5paper,showtrims]{memoir}
\usepackage{atbegshi}
\AtBeginShipout{%
  \setbox\AtBeginShipoutBox=\vbox{%
    \vbox to 0pt{%
      \kern-1.5in %
      \hbox to 0pt{%
        \kern-1.5in %
        \color{blue}%
        \rule{1in}{1in}%
        \hss
      }%
      \vss
    }%
    \hrule
    \hbox{\vrule\box\AtBeginShipoutBox\vrule}%
    \hrule
  }%
}
\usepackage{eso-pic}
\makeatletter
\@ifundefined{@EveryShipout@Init}{%
  \typeout{Test skipped}%
  \@@end
}{}
\@EveryShipout@Init
\let\@EveryShipout@Init\relax
\makeatother
\AddToShipoutPicture{%
  \hspace{.52\paperwidth}%
  \colorbox{cyan}{%
    \rule{0mm}{\paperheight}%
    \hspace{.48\paperwidth}%
  }%
}
%    \end{macrocode}
%    Newer versions of class \xclass{memoir} emulate package
%    \xpackage{crop} and prevents its loading. This is undone
%    in next line for this test file.
%    \begin{macrocode}
\expandafter\let\csname ver@crop.sty\endcsname\relax
\usepackage[color=red,cross,a4,center]{crop}
\begin{document}
\shipout\null
\shipout\box\csname voidb@x\endcsname
\section{Hello World}
\end{document}
%</test3>
%    \end{macrocode}
%
% \section{Installation}
%
% \subsection{Download}
%
% \paragraph{Package.} This package is available on
% CTAN\footnote{\url{http://ctan.org/pkg/atbegshi}}:
% \begin{description}
% \item[\CTAN{macros/latex/contrib/oberdiek/atbegshi.dtx}] The source file.
% \item[\CTAN{macros/latex/contrib/oberdiek/atbegshi.pdf}] Documentation.
% \end{description}
%
%
% \paragraph{Bundle.} All the packages of the bundle `oberdiek'
% are also available in a TDS compliant ZIP archive. There
% the packages are already unpacked and the documentation files
% are generated. The files and directories obey the TDS standard.
% \begin{description}
% \item[\CTAN{install/macros/latex/contrib/oberdiek.tds.zip}]
% \end{description}
% \emph{TDS} refers to the standard ``A Directory Structure
% for \TeX\ Files'' (\CTAN{tds/tds.pdf}). Directories
% with \xfile{texmf} in their name are usually organized this way.
%
% \subsection{Bundle installation}
%
% \paragraph{Unpacking.} Unpack the \xfile{oberdiek.tds.zip} in the
% TDS tree (also known as \xfile{texmf} tree) of your choice.
% Example (linux):
% \begin{quote}
%   |unzip oberdiek.tds.zip -d ~/texmf|
% \end{quote}
%
% \paragraph{Script installation.}
% Check the directory \xfile{TDS:scripts/oberdiek/} for
% scripts that need further installation steps.
% Package \xpackage{attachfile2} comes with the Perl script
% \xfile{pdfatfi.pl} that should be installed in such a way
% that it can be called as \texttt{pdfatfi}.
% Example (linux):
% \begin{quote}
%   |chmod +x scripts/oberdiek/pdfatfi.pl|\\
%   |cp scripts/oberdiek/pdfatfi.pl /usr/local/bin/|
% \end{quote}
%
% \subsection{Package installation}
%
% \paragraph{Unpacking.} The \xfile{.dtx} file is a self-extracting
% \docstrip\ archive. The files are extracted by running the
% \xfile{.dtx} through \plainTeX:
% \begin{quote}
%   \verb|tex atbegshi.dtx|
% \end{quote}
%
% \paragraph{TDS.} Now the different files must be moved into
% the different directories in your installation TDS tree
% (also known as \xfile{texmf} tree):
% \begin{quote}
% \def\t{^^A
% \begin{tabular}{@{}>{\ttfamily}l@{ $\rightarrow$ }>{\ttfamily}l@{}}
%   atbegshi.sty & tex/generic/oberdiek/atbegshi.sty\\
%   atbegshi.pdf & doc/latex/oberdiek/atbegshi.pdf\\
%   atbegshi-example1.tex & doc/latex/oberdiek/atbegshi-example1.tex\\
%   atbegshi-example2.tex & doc/latex/oberdiek/atbegshi-example2.tex\\
%   test/atbegshi-test1.tex & doc/latex/oberdiek/test/atbegshi-test1.tex\\
%   test/atbegshi-test2.tex & doc/latex/oberdiek/test/atbegshi-test2.tex\\
%   test/atbegshi-test3.tex & doc/latex/oberdiek/test/atbegshi-test3.tex\\
%   atbegshi.dtx & source/latex/oberdiek/atbegshi.dtx\\
% \end{tabular}^^A
% }^^A
% \sbox0{\t}^^A
% \ifdim\wd0>\linewidth
%   \begingroup
%     \advance\linewidth by\leftmargin
%     \advance\linewidth by\rightmargin
%   \edef\x{\endgroup
%     \def\noexpand\lw{\the\linewidth}^^A
%   }\x
%   \def\lwbox{^^A
%     \leavevmode
%     \hbox to \linewidth{^^A
%       \kern-\leftmargin\relax
%       \hss
%       \usebox0
%       \hss
%       \kern-\rightmargin\relax
%     }^^A
%   }^^A
%   \ifdim\wd0>\lw
%     \sbox0{\small\t}^^A
%     \ifdim\wd0>\linewidth
%       \ifdim\wd0>\lw
%         \sbox0{\footnotesize\t}^^A
%         \ifdim\wd0>\linewidth
%           \ifdim\wd0>\lw
%             \sbox0{\scriptsize\t}^^A
%             \ifdim\wd0>\linewidth
%               \ifdim\wd0>\lw
%                 \sbox0{\tiny\t}^^A
%                 \ifdim\wd0>\linewidth
%                   \lwbox
%                 \else
%                   \usebox0
%                 \fi
%               \else
%                 \lwbox
%               \fi
%             \else
%               \usebox0
%             \fi
%           \else
%             \lwbox
%           \fi
%         \else
%           \usebox0
%         \fi
%       \else
%         \lwbox
%       \fi
%     \else
%       \usebox0
%     \fi
%   \else
%     \lwbox
%   \fi
% \else
%   \usebox0
% \fi
% \end{quote}
% If you have a \xfile{docstrip.cfg} that configures and enables \docstrip's
% TDS installing feature, then some files can already be in the right
% place, see the documentation of \docstrip.
%
% \subsection{Refresh file name databases}
%
% If your \TeX~distribution
% (\teTeX, \mikTeX, \dots) relies on file name databases, you must refresh
% these. For example, \teTeX\ users run \verb|texhash| or
% \verb|mktexlsr|.
%
% \subsection{Some details for the interested}
%
% \paragraph{Attached source.}
%
% The PDF documentation on CTAN also includes the
% \xfile{.dtx} source file. It can be extracted by
% AcrobatReader 6 or higher. Another option is \textsf{pdftk},
% e.g. unpack the file into the current directory:
% \begin{quote}
%   \verb|pdftk atbegshi.pdf unpack_files output .|
% \end{quote}
%
% \paragraph{Unpacking with \LaTeX.}
% The \xfile{.dtx} chooses its action depending on the format:
% \begin{description}
% \item[\plainTeX:] Run \docstrip\ and extract the files.
% \item[\LaTeX:] Generate the documentation.
% \end{description}
% If you insist on using \LaTeX\ for \docstrip\ (really,
% \docstrip\ does not need \LaTeX), then inform the autodetect routine
% about your intention:
% \begin{quote}
%   \verb|latex \let\install=y% \iffalse meta-comment
%
% File: atbegshi.dtx
% Version: 2016/06/09 v1.18
% Info: At begin shipout hook
%
% Copyright (C) 2007-2011 by
%    Heiko Oberdiek <heiko.oberdiek at googlemail.com>
%    2016
%    https://github.com/ho-tex/oberdiek/issues
%
% This work may be distributed and/or modified under the
% conditions of the LaTeX Project Public License, either
% version 1.3c of this license or (at your option) any later
% version. This version of this license is in
%    http://www.latex-project.org/lppl/lppl-1-3c.txt
% and the latest version of this license is in
%    http://www.latex-project.org/lppl.txt
% and version 1.3 or later is part of all distributions of
% LaTeX version 2005/12/01 or later.
%
% This work has the LPPL maintenance status "maintained".
%
% This Current Maintainer of this work is Heiko Oberdiek.
%
% The Base Interpreter refers to any `TeX-Format',
% because some files are installed in TDS:tex/generic//.
%
% This work consists of the main source file atbegshi.dtx
% and the derived files
%    atbegshi.sty, atbegshi.pdf, atbegshi.ins, atbegshi.drv,
%    atbegshi-example1.tex, atbegshi-example2.tex,
%    atbegshi-test1.tex, atbegshi-test2.tex, atbegshi-test3.tex.
%
% Distribution:
%    CTAN:macros/latex/contrib/oberdiek/atbegshi.dtx
%    CTAN:macros/latex/contrib/oberdiek/atbegshi.pdf
%
% Unpacking:
%    (a) If atbegshi.ins is present:
%           tex atbegshi.ins
%    (b) Without atbegshi.ins:
%           tex atbegshi.dtx
%    (c) If you insist on using LaTeX
%           latex \let\install=y% \iffalse meta-comment
%
% File: atbegshi.dtx
% Version: 2016/06/09 v1.18
% Info: At begin shipout hook
%
% Copyright (C) 2007-2011 by
%    Heiko Oberdiek <heiko.oberdiek at googlemail.com>
%    2016
%    https://github.com/ho-tex/oberdiek/issues
%
% This work may be distributed and/or modified under the
% conditions of the LaTeX Project Public License, either
% version 1.3c of this license or (at your option) any later
% version. This version of this license is in
%    http://www.latex-project.org/lppl/lppl-1-3c.txt
% and the latest version of this license is in
%    http://www.latex-project.org/lppl.txt
% and version 1.3 or later is part of all distributions of
% LaTeX version 2005/12/01 or later.
%
% This work has the LPPL maintenance status "maintained".
%
% This Current Maintainer of this work is Heiko Oberdiek.
%
% The Base Interpreter refers to any `TeX-Format',
% because some files are installed in TDS:tex/generic//.
%
% This work consists of the main source file atbegshi.dtx
% and the derived files
%    atbegshi.sty, atbegshi.pdf, atbegshi.ins, atbegshi.drv,
%    atbegshi-example1.tex, atbegshi-example2.tex,
%    atbegshi-test1.tex, atbegshi-test2.tex, atbegshi-test3.tex.
%
% Distribution:
%    CTAN:macros/latex/contrib/oberdiek/atbegshi.dtx
%    CTAN:macros/latex/contrib/oberdiek/atbegshi.pdf
%
% Unpacking:
%    (a) If atbegshi.ins is present:
%           tex atbegshi.ins
%    (b) Without atbegshi.ins:
%           tex atbegshi.dtx
%    (c) If you insist on using LaTeX
%           latex \let\install=y% \iffalse meta-comment
%
% File: atbegshi.dtx
% Version: 2016/06/09 v1.18
% Info: At begin shipout hook
%
% Copyright (C) 2007-2011 by
%    Heiko Oberdiek <heiko.oberdiek at googlemail.com>
%    2016
%    https://github.com/ho-tex/oberdiek/issues
%
% This work may be distributed and/or modified under the
% conditions of the LaTeX Project Public License, either
% version 1.3c of this license or (at your option) any later
% version. This version of this license is in
%    http://www.latex-project.org/lppl/lppl-1-3c.txt
% and the latest version of this license is in
%    http://www.latex-project.org/lppl.txt
% and version 1.3 or later is part of all distributions of
% LaTeX version 2005/12/01 or later.
%
% This work has the LPPL maintenance status "maintained".
%
% This Current Maintainer of this work is Heiko Oberdiek.
%
% The Base Interpreter refers to any `TeX-Format',
% because some files are installed in TDS:tex/generic//.
%
% This work consists of the main source file atbegshi.dtx
% and the derived files
%    atbegshi.sty, atbegshi.pdf, atbegshi.ins, atbegshi.drv,
%    atbegshi-example1.tex, atbegshi-example2.tex,
%    atbegshi-test1.tex, atbegshi-test2.tex, atbegshi-test3.tex.
%
% Distribution:
%    CTAN:macros/latex/contrib/oberdiek/atbegshi.dtx
%    CTAN:macros/latex/contrib/oberdiek/atbegshi.pdf
%
% Unpacking:
%    (a) If atbegshi.ins is present:
%           tex atbegshi.ins
%    (b) Without atbegshi.ins:
%           tex atbegshi.dtx
%    (c) If you insist on using LaTeX
%           latex \let\install=y\input{atbegshi.dtx}
%        (quote the arguments according to the demands of your shell)
%
% Documentation:
%    (a) If atbegshi.drv is present:
%           latex atbegshi.drv
%    (b) Without atbegshi.drv:
%           latex atbegshi.dtx; ...
%    The class ltxdoc loads the configuration file ltxdoc.cfg
%    if available. Here you can specify further options, e.g.
%    use A4 as paper format:
%       \PassOptionsToClass{a4paper}{article}
%
%    Programm calls to get the documentation (example):
%       pdflatex atbegshi.dtx
%       makeindex -s gind.ist atbegshi.idx
%       pdflatex atbegshi.dtx
%       makeindex -s gind.ist atbegshi.idx
%       pdflatex atbegshi.dtx
%
% Installation:
%    TDS:tex/generic/oberdiek/atbegshi.sty
%    TDS:doc/latex/oberdiek/atbegshi.pdf
%    TDS:doc/latex/oberdiek/atbegshi-example1.tex
%    TDS:doc/latex/oberdiek/atbegshi-example2.tex
%    TDS:doc/latex/oberdiek/test/atbegshi-test1.tex
%    TDS:doc/latex/oberdiek/test/atbegshi-test2.tex
%    TDS:doc/latex/oberdiek/test/atbegshi-test3.tex
%    TDS:source/latex/oberdiek/atbegshi.dtx
%
%<*ignore>
\begingroup
  \catcode123=1 %
  \catcode125=2 %
  \def\x{LaTeX2e}%
\expandafter\endgroup
\ifcase 0\ifx\install y1\fi\expandafter
         \ifx\csname processbatchFile\endcsname\relax\else1\fi
         \ifx\fmtname\x\else 1\fi\relax
\else\csname fi\endcsname
%</ignore>
%<*install>
\input docstrip.tex
\Msg{************************************************************************}
\Msg{* Installation}
\Msg{* Package: atbegshi 2016/06/09 v1.18 At begin shipout hook (HO)}
\Msg{************************************************************************}

\keepsilent
\askforoverwritefalse

\let\MetaPrefix\relax
\preamble

This is a generated file.

Project: atbegshi
Version: 2016/06/09 v1.18

Copyright (C) 2007-2011 by
   Heiko Oberdiek <heiko.oberdiek at googlemail.com>

This work may be distributed and/or modified under the
conditions of the LaTeX Project Public License, either
version 1.3c of this license or (at your option) any later
version. This version of this license is in
   http://www.latex-project.org/lppl/lppl-1-3c.txt
and the latest version of this license is in
   http://www.latex-project.org/lppl.txt
and version 1.3 or later is part of all distributions of
LaTeX version 2005/12/01 or later.

This work has the LPPL maintenance status "maintained".

This Current Maintainer of this work is Heiko Oberdiek.

The Base Interpreter refers to any `TeX-Format',
because some files are installed in TDS:tex/generic//.

This work consists of the main source file atbegshi.dtx
and the derived files
   atbegshi.sty, atbegshi.pdf, atbegshi.ins, atbegshi.drv,
   atbegshi-example1.tex, atbegshi-example2.tex,
   atbegshi-test1.tex, atbegshi-test2.tex, atbegshi-test3.tex.

\endpreamble
\let\MetaPrefix\DoubleperCent

\generate{%
  \file{atbegshi.ins}{\from{atbegshi.dtx}{install}}%
  \file{atbegshi.drv}{\from{atbegshi.dtx}{driver}}%
  \usedir{tex/generic/oberdiek}%
  \file{atbegshi.sty}{\from{atbegshi.dtx}{package}}%
  \usedir{doc/latex/oberdiek}%
  \file{atbegshi-example1.tex}{\from{atbegshi.dtx}{example1}}%
  \file{atbegshi-example2.tex}{\from{atbegshi.dtx}{example2}}%
  \usedir{doc/latex/oberdiek/test}%
  \file{atbegshi-test1.tex}{\from{atbegshi.dtx}{test1}}%
  \file{atbegshi-test2.tex}{\from{atbegshi.dtx}{test2}}%
  \file{atbegshi-test3.tex}{\from{atbegshi.dtx}{test3}}%
  \nopreamble
  \nopostamble
  \usedir{source/latex/oberdiek/catalogue}%
  \file{atbegshi.xml}{\from{atbegshi.dtx}{catalogue}}%
}

\catcode32=13\relax% active space
\let =\space%
\Msg{************************************************************************}
\Msg{*}
\Msg{* To finish the installation you have to move the following}
\Msg{* file into a directory searched by TeX:}
\Msg{*}
\Msg{*     atbegshi.sty}
\Msg{*}
\Msg{* To produce the documentation run the file `atbegshi.drv'}
\Msg{* through LaTeX.}
\Msg{*}
\Msg{* Happy TeXing!}
\Msg{*}
\Msg{************************************************************************}

\endbatchfile
%</install>
%<*ignore>
\fi
%</ignore>
%<*driver>
\NeedsTeXFormat{LaTeX2e}
\ProvidesFile{atbegshi.drv}%
  [2016/06/09 v1.18 At begin shipout hook (HO)]%
\documentclass{ltxdoc}
\usepackage{holtxdoc}[2011/11/22]
\begin{document}
  \DocInput{atbegshi.dtx}%
\end{document}
%</driver>
% \fi
%
%
% \CharacterTable
%  {Upper-case    \A\B\C\D\E\F\G\H\I\J\K\L\M\N\O\P\Q\R\S\T\U\V\W\X\Y\Z
%   Lower-case    \a\b\c\d\e\f\g\h\i\j\k\l\m\n\o\p\q\r\s\t\u\v\w\x\y\z
%   Digits        \0\1\2\3\4\5\6\7\8\9
%   Exclamation   \!     Double quote  \"     Hash (number) \#
%   Dollar        \$     Percent       \%     Ampersand     \&
%   Acute accent  \'     Left paren    \(     Right paren   \)
%   Asterisk      \*     Plus          \+     Comma         \,
%   Minus         \-     Point         \.     Solidus       \/
%   Colon         \:     Semicolon     \;     Less than     \<
%   Equals        \=     Greater than  \>     Question mark \?
%   Commercial at \@     Left bracket  \[     Backslash     \\
%   Right bracket \]     Circumflex    \^     Underscore    \_
%   Grave accent  \`     Left brace    \{     Vertical bar  \|
%   Right brace   \}     Tilde         \~}
%
% \GetFileInfo{atbegshi.drv}
%
% \title{The \xpackage{atbegshi} package}
% \date{2016/06/09 v1.18}
% \author{Heiko Oberdiek\thanks
% {Please report any issues at https://github.com/ho-tex/oberdiek/issues}\\
% \xemail{heiko.oberdiek at googlemail.com}}
%
% \maketitle
%
% \begin{abstract}
% This package is a modern reimplementation of package \xpackage{everyshi}
% without the burden of compatibility. It makes use of \eTeX's if available.
% Both \LaTeX\ and \plainTeX\ are supported.
% \end{abstract}
%
% \tableofcontents
%
% \section{Documentation}
%
% Package \xpackage{atbegshi} redefines \cs{shipout} to insert
% hooks for user code that is executed before the page is shipped out.
% The code may modify or even discard the output page.
% Three hooks are implemented:
% \begin{enumerate}
% \item A hook that is executed for every page, see\\
%       \cs{AtBeginShipout}
% \item A hook that is executed for the next page only, see\\
%       \cs{AtBeginShipoutNext}
% \item A hook that is only executed for the first page, see\\
%       \cs{AtBeginShipoutFirst}
% \end{enumerate}
% The hooks are executed in this order. The following three macros
% provide the user interface for adding code to these hooks:
%
% \begin{declcs}{AtBeginShipout} \M{code}\\
%   \cs{AtBeginShipoutBox}
% \end{declcs}
% Execute the \meta{code} for every page.
% The page contents is held in box register \cs{AtBeginShipoutBox}
% and may be modified. Use \cs{AtBeginShipoutDiscard} if you want to
% discard the page.
%
% \emph{Note:} Package \xpackage{everyshi} uses box register 255. With
% package \xpackage{atbegshi} you must use \cs{AtBeginShipoutBox} instead.
%
% If \LaTeX\ calls \cs{shipout} in \cs{@outputpage} (part of its output
% routine),
% the meaning of \cs{protect} is \cs{noexpand}. \LaTeX\ sets
% \cs{protect} to the appropriate \cs{@typeset@protect} in the box
% that is shipped out. This is too late for the hooks, they
% are called earlier in the redefined \cs{shipout}. Therefore
% package \xpackage{atbegshi} sets \cs{protect} to \cs{@typeset@protect}
% before it calls the hooks.
% (In \cs{EveryShipout} of package \xpackage{everyshi} the
% user is responsible for the correct setting of \cs{protect}.)
%
% \begin{declcs}{AtBeginShipoutNext} \M{code}
% \end{declcs}
% This reimplements package \xpackage{everyshi}'s \cs{AtNextShipout}.
% The \meta{code} is executed at shipout time of the next page only.
% It is just a convenience macro, it can be easily replaced by something
% like:
%\begin{quote}
%\begin{verbatim}
%\newcommand{\MyShipoutHook}{}%
%\AtBeginShipout{\MyShipoutHook}
%\gdef\MyShipoutHook{%
%  ... do something with next page ...
%  \gdef\MyShipoutHook{}%
% }
%\end{verbatim}
%\end{quote}
% (This can be necessary, if hook order does matter).
%
% \begin{declcs}{AtBeginShipoutFirst} \M{code}
% \end{declcs}
% This reimplements \LaTeX's \cs{AtBeginDvi}. This hook is
% usually used for \cs{special} commands that include PostScript
% header files. The \cs{code} is directly executed in a \cs{vbox}
% that is put at the beginning of the output page. Dealing with
% the output box \cs{AtBeginShipoutBox} is not necessary and
% not permitted here.
%
% \begin{declcs}{AtBeginShipoutDiscard}
% \end{declcs}
% This macro notifies package \xpackage{atbegshi} that
% the output page is discarded. The remaining hook code
% and the remaining hooks are not executed and the page is thrown away.
% Also \cs{deadcycles} is cleared to zero like an ordinary \cs{shipout}
% would do.
%
% \begin{declcs}{AtBeginShipoutInit}
% \end{declcs}
% Usually the redefinition of \cs{shipout} is delayed by
% \cs{AtBeginDocument} (if this macro exists). This can be
% too late, if other packages also redefines \cs{shipout}
% and the order does matter.
% \cs{AtBeginShipoutInit} forces the immediate redefinition
% of \cs{shipout}.
%
% \begin{declcs}{AtBeginShipoutAddToBox} \M{stuff}\\
%   \cs{AtBeginShipoutAddToBoxForeground} \M{stuff}
% \end{declcs}
% A quite common use case is the addition of \cs{special}
% or other whatsits to the page output box. Macro
% \cs{AtBeginShipoutAddToBox} puts \meta{stuff} in a
% box with zeroed dimensions. The box with the \meta{stuff}
% is put in the upper left corner of the shipout box
% \cs{AtBeginShipoutBox}. Macro \cs{AtBeginShipoutAddToBox}
% puts the \meta{stuff} in the background, the other macro
% \cs{AtBeginShipoutAddToBoxForeground} in the foreground
% after the original shipout box contents is set.
%
% A void shipout box (that means a discarded page)
% remains void that means \meta{stuff} is ignored in this case.
% The box type of \cs{AtBeginShipoutBox} is preserved.
% Also the box nesting level for the original contents
% of \cs{AtBeginShipoutBox} remains, for example, to avoid trouble
% with links across pages in case of \hologo{pdfTeX}.
%
% \begin{declcs}{AtBeginShipoutUpperLeft} \M{background material}
% \end{declcs}
% This is a macro that puts material in the background of
% box \cs{AtBeginShipoutBox}.
% The \meta{background material} is set in an \cs{hbox}, the
% reference point is the upper left corner of the output page.
% In case of \pdfTeX\ in PDF mode, the settings of
% \cs{pdfhorigin} and \cs{pdfvorigin} are respected.
%
% The macro \cs{AtBeginShipoutUpperLeft} is intended to be used
% in one of the hook setting macros, such as
% \cs{AtBeginShipout}, \cs{AtBeginShipoutFirst}, or
% \cs{AtBeginShipoutNext}.
%
% For \LaTeX\ users the \meta{background material} is set
% inside a \texttt{picture} environment:
% \begin{quote}
%   |\begin{picture}(0,0)|\\
%   \mbox{}\quad |\setlength{\unitlength}{1pt}%|\\
%   \mbox{}\quad \meta{background material}\\
%   |\end{picture}|
% \end{quote}
%
% \begin{declcs}{AtBeginShipoutUpperLeftForeground} \M{foreground material}
% \end{declcs}
% See \cs{AtBeginShipoutUpperLeft}. The difference is that the material
% is put in the foreground.
%
% \begin{declcs}{AtBeginShipoutOriginalShipout} \meta{box}
% \end{declcs}
% It stores the meaning of \cs{shipout} at the time this package
% is loaded.
%
% \begin{declcs}{AtBeginShipoutBoxWidth}\\
%   \cs{AtBeginShipoutBoxHeight}\\
%   \cs{AtBeginShipoutBoxDepth}\\
% \end{declcs}
% These macros store the dimensions of the output box \cs{AtBeginShipoutBox}
% before the original shipout is called. If \cs{shipout} is not
% redefined before the package loading or the box dimensions
% are not changed by the redefined \cs{shipout}, these macros
% contain the dimensions of the shipout box. These values can be
% remembered by \cs{label} and \cs{ref}. For example, this is done
% by the package module \xpackage{zref-pagelayout} of project \xpackage{zref}.
% The dimensions of the shipout page can be used in some \hologo{TeX}
% engines (\hologo{pdfTeX} in PDF mode, \hologo{XeTeX}) to calculate
% the media size of the shipout page if \cs{pdfpagewidth} and
% \cs{pdfpageheight} are not set.
%
% \subsection{Examples}
%
% \subsubsection{Example: circle in background}
%
% In this example we put a circle in the background in the middle
% of the paper.
%
%    \begin{macrocode}
%<*example1>
%    \end{macrocode}
%    \begin{macrocode}
\documentclass[a4paper]{article}
\usepackage{color}
\usepackage{atbegshi}
%    \end{macrocode}
%    Package \xpackage{picture} makes life a little easier, because
%    we can now also use length specifications in |picture|'s commands.
%    \begin{macrocode}
\usepackage{picture}
%    \end{macrocode}
%    Now we draw the circle in the middle of the paper.
%    \cs{put} moves downwards, because the origin is at the top
%    of the page, not at its bottom.
%    \begin{macrocode}
\AtBeginShipout{%
  \AtBeginShipoutUpperLeft{%
    \put(0.5\paperwidth,-0.5\paperheight){\circle{10}}%
  }%
}
\begin{document}
\section{Hello World}
\newpage
\AtBeginShipoutNext{%
  \AtBeginShipoutUpperLeft{%
    \color{red}%
    \put(0,-0.5\paperheight){\line(1,0){\paperwidth}}%
    \put(0.5\paperwidth, 0){\line(0,-1){\paperheight}}%
  }%
}
Only on this page we add a red cross.
\newpage
This page has the circle only.
\par
\vspace{\fill}
The next page will be discarded.
\newpage
\AtBeginShipoutNext{%
  \AtBeginShipoutDiscard
}
This page is discarded.
\newpage
The last page.
\end{document}
%    \end{macrocode}
%    \begin{macrocode}
%</example1>
%    \end{macrocode}
%
% \subsubsection{Example: adding TrimBox for dvipdfmx}
%
% Now an example from ``real life'' follows. Someone from the mailing  list
% for dvipdfmx wants to put a TrimBox on every page. If we use
% \cs{AtBeginShipout}, we have to put the \cs{special} inside the
% box \cs{AtBeginShipoutBox} that gets shipped out.
%    \begin{macrocode}
%<*example2>
\documentclass{minimal}
\usepackage{atbegshi}
\usepackage[
  dvipdfm,
  paperwidth=630bp,
  paperheight=810bp
]{geometry}
\AtBeginShipout{%
  \setbox\AtBeginShipoutBox=\hbox{%
    \special{pdf: put @thispage <</TrimBox[9 9 621 801]>>}%
    \box\AtBeginShipoutBox
  }%
}
\begin{document}
  First page
  \newpage
  Second page
\end{document}
%</example2>
%    \end{macrocode}
% Remember, in \cs{AtBeginShipoutBoxFirst} the \cs{setbox} wrapper code
% is implicitly given and the \cs{special} is used directly.
%
% \section{Method of \cs{shipout} overloading}
%
% \subsection{\cs{shipout}}
%
% The \TeX\ primitive command \cs{shipout} takes a box specification
% and puts the box as a new page in the output file.
% There are two kinds of box specifications:
% \begin{description}
% \item[Direct boxes:]
%   They are given by \cs{hbox}, \cs{vbox}, or \cs{vtop},\\
%   e.g. |\shipout\hbox{Hello World}|.
% \item[Indirect boxes:]
%   \cs{box} or \cs{copy} references a box register by number.
%   The box register contains the contents of the box.
%
%   \emph{Note:} \cs{box} also clears the box register globally.
% \end{description}
% Then we have to differentiate between void and empty boxes:
% \begin{description}
% \item[Void:]
%   Initially or after \cs{box} there is no box in the box register.
%   In this cases the box register is not empty, but \emph{void}.
% \item[Empty:]
%   A box with empty contents, such as
%   |\hbox{}| (= \cs{null}) or |\vbox{}| is an \emph{empty \cs{hbox}}
%   or \emph{empty \cs{vbox}}. If a box register holds such a
%   box, the box still exists, therefore the box register is
%   \emph{not void}.
% \end{description}
%
% \subsection{\cs{afterassignment}}
%
% We want to overload \cs{shipout} to do something with the box.
% It is quite impossible to do this reliable by catching the box
% using macro arguments. The
% variety of box specifications is too large, Examples:
% \begin{quote}
%   |\shipout\null|\\
%   |\shipout\vbox{...}|\\
%   |\shipout\vtop\bgroup ...\egroup|\\
%   |\shipout\box255|
% \end{quote}
% Even worse, the braces don't need to be balanced:
% \begin{quote}
%   |\shipout\hbox\bgroup}|\\
%   |\shipout\vbox{\egroup|
% \end{quote}
% Happily TeX provides a reliable way via \cs{afterassignment}.
% It takes a macro name and executes it just after the assignment.
%
% Now we can redefine \cs{shipout}. The box specification that
% follows \cs{shipout} is catched by \cs{setbox}. This is an
% assignment to a box register. \cs{afterassignment}
% notifies \TeX, that we want to call \cs{@test} right after
% the assignment:
%\begin{quote}
%\begin{verbatim}
%\shipout :=
%  \afterassignment\@test
%  \setbox\mybox=
%\end{verbatim}
%\end{quote}
% We have seen different box specifications. Indirect boxes are easy
% to understand:
% \begin{quote}
%   |\shipout\box0| $\Rightarrow$ |\setbox\mybox=\box0 |\fbox{\cs{@test}}
% \end{quote}
% However direct boxes can have arbitrary contents with lots of
% other assignments. It would be quite inpredictable if \TeX\ would
% put |\@test| after the first of such an assignment or after the
% box specification if the box lacks of assignments.
% Therefore \TeX\ puts |\@test| right at the beginning of the
% box specification, e.g:
% \begin{quote}
%   |\shipout\hbox{Hello World}|\\
%   $\Rightarrow$
%   |\setbox\mybox=\hbox{|\fbox{\cs{@test}}| Hello World}|
% \end{quote}
%
% \subsection{Test for direct or indirect boxes}
%
% Now we want to execute \cs{@test}, but where are we?
% We can be after the completed box assignment, if \cs{shipout}
% was called with an indirect box. Or we are right at the beginning
% of a direct box.
%
% \subsubsection{With \eTeX}
%
% With the \eTeX's extensions the answer is very easy:
% Being inside the direct box means that we are inside a new group.
% The new primitive command \cs{currentgrouplevel} tells how
% deeply the groups are currently nested. Macro \cs{@test} just
% compares the previously stored group level with the current one:
%\begin{quote}
%\begin{verbatim}
%\shipout :=
%  \edef\saved@grouplevel{\number\currentgrouplevel}
%  \afterassignment\@test
%  \setbox\mybox=
%
%\@test :=
%  \ifnum\saved@grouplevel=\currentgrouplevel
%    % case: indirect box, the assignment is completed
%    \@output
%  \else
%    % case: direct box, we are inside the box
%    \aftergroup\@outbox
%  \fi
%\end{verbatim}
%\end{quote}
%
% \subsubsection{Without \eTeX}
%
% Life becomes complicate without \eTeX. We cannot ask the
% group level. However, if we are inside a direct box,
% the box register \cs{mybox} is not yet changed by \cs{setbox}.
% Thus we need a special initial value and compare
% it in \cs{@test} with the current value of the box.
%
% What can be used as initial value? Arbitrary box contents cannot
% be compared. \TeX\ only tells us a few properties:
% \begin{itemize}
% \item Box type:\cs{ifhbox}, \cs{ifvbox}
% \item Dimensions: \cs{wd}, \cs{ht}, \cs{dp}
% \item Voidness: \cs{ifvoid}
% \end{itemize}
% Unhappily all these qualities even combined are not sufficient
% for constructing an initial box value, because \cs{shipout} can be
% called with a box that is accidently just the same as the choosen
% initial value.
%
% Nevertheless we have two alternatives for an initial value:
% \begin{itemize}
% \item A box of some type with some funny settings that are
% unlikely to occur in real life, e.g a height of |4911sp-\maxdimen|.
% \item A void box.
% \end{itemize}
% A collision between this initial value and an indirect \cs{shipout}
% box with just the same value is possible.
% Then \cs{@test} will make a wrong decision that it is executed
% inside a direct box and delays \cs{@output} by \cs{aftergroup}.
% Thus \cs{@output} is not called at the place we want.
% In contrary, the result is an uncertainty about the place:
% \begin{itemize}
% \item \cs{shipout} is used in a group that perhaps closes some
%   pages later. A bad place for \cs{@output}.
% \item Without a surrounding group \cs{aftergroup} effectively kills
%   its argument.
% \end{itemize}
% In the first case of a box with special dimensions we can even loose
% the page. However in the case of the void box, this effect is even
% desired, because the original \cs{shipout} does not output void boxes.
% All we have to do is to ensure that our box \cs{mybox} is always
% void except for the phase when the overloaded \cs{shipout} is executed.
% And secondly we must keep this semantics of \cs{shipout} for the void case
% in our macros, namely \cs{@output}.
%\begin{quote}
%\begin{verbatim}
%\shipout :=
%  % trick to get a void box \mybox
%  \begingroup
%    \setbox\mybox=\box\mybox
%  \endgroup
%  \afterassignment\@test
%  \setbox\mybox=
%
%\@test :=
%  \ifvoid\mybox
%    \aftergroup\@output
%  \else
%    \@output
%  \fi
%\end{verbatim}
%\end{quote}
% The nasty case is |\shipout\box\voidb@x| where the indirect box
% is void and that must not generate an output page.
% If a surrounding group is missing the output is ignored
% because of \cs{aftergroup}.
% Otherwise \@output is called some time
% later when the surrounding group closes.
% But \cs{mybox} is void outside the execution phase
% of the redefined \cs{shipout}. Also \cs{@output} checks for a
% void box and cancels the page output.
% The disadvantage remains that the hook in \cs{@output} is called
% for a page that will not be output.
%
% \subsubsection{\cs{lastkern} method}
%
% At the beginning of a new box, there is no \cs{kern}, the contents
% of the box is still empty and \cs{lastkern} returns 0\,pt.
% This can be used to distinguish between direct and indirect boxes:
% We execute \cs{setbox} in a box with a preceding non-zero kern.
% After an indirect box, \cs{lastkern} sees this kern, otherwise
% it returns 0\,pt.
%\begin{quote}
%\begin{verbatim}
%\shipout :=
%  \begingroup
%    \setbox\mybox=\hbox\bgroup
%      \kern1pt
%      \afterassignment\shipout@test
%      \global\setbox\mybox=
%\@test :=
%  \ifdim\lastkern=0pt
%    % direct box
%    \aftergroup\egroup
%    \aftergroup\endgroup
%    \aftergroup\@output
%  \else
%    \egroup
%    \endgroup
%    \@output
%  \fi
%\end{verbatim}
%\end{quote}
% We have two \cs{setbox} commands. The first creates a controlled
% context box where we can safely insert a \cs{kern}. We get rid of
% this temporarily used context box by putting the local \cs{setbox}
% in a group.
%
% After the group we want to have our shipout box in \cs{mybox}.
% Therefore we use a global assignment here.
%
% \subsection{Output}
%
% With or without \eTeX\ we ensure the original behaviour of
% \cs{shipout} that void boxes do not generate output pages.
%
% Now we can place the hook \cs{@hook} for the user code that
% wants to manipulate the output box.
%\begin{quote}
%\begin{verbatim}
%\@output :=
%  \ifvoid\mybox
%    % cancel output of void box
%  \else
%    \@hook
%    \ifvoid\mybox
%      % user code in \@hook could has voided the box
%    \else
%      \original@shipout\box\mybox
%    \fi
%  \fi
%\end{verbatim}
%\end{quote}
%
% \subsection{Separate box register}
%
% So far we have said nothing about the box number of \cs{mybox}.
% The following case that outputs the same page twice
% shows that we are not free in the use
% of the box register:
% \begin{quote}
%   |\shipout\copy<num> \shipout\box<num>|
% \end{quote}
% We manipulate the box by the hook and without \eTeX\ the box
% must even be voided. However, the use case above requires that
% the box contents does not change at all. Therefore we must
% reserve a separate box register to avoid collisions with
% user box registers.
%
% \emph{Note:} Box register number 255 is special for the output
% routine, because TeX complains if this box is not voided
% by the output routine.
% However, this requirement does not apply to \cs{shipout} at all.
% In fact \cs{shipout} does not change any box register. This is
% usually done by a call of \cs{box}, but the output routine
% can do it later \emph{after} invoking of \cs{shipout}.
%
% \subsection{Summary}
%
% \subsubsection{With \eTeX}
%
% Putting the pieces together we get for \eTeX:
%\begin{quote}
%\begin{verbatim}
%\newbox\mybox
%\let\original@shipout\shipout
%
%\shipout :=
%  \edef\saved@grouplevel{\number\currentgrouplevel}
%  \afterassignment\@test
%  \setbox\mybox=
%
%\@test :=
%  \ifnum\saved@grouplevel<\currentgrouplevel
%    \expandafter\aftergroup
%  \fi
%  \@output
%
%\@output :=
%  \ifvoid\mybox
%    % cancel output of void box
%  \else
%    \@hook
%    \ifvoid\mybox
%      % user code in \@hook could have voided the box
%    \else
%      \original@shipout\box\mybox
%    \fi
%  \fi
%\end{verbatim}
%\end{quote}
%
% \subsubsection{Without \eTeX, traditional way}
%
% And for \TeX\ without \eTeX:
%\begin{quote}
%\begin{verbatim}
%\newbox\mybox
%\begingroup
%   \setbox\mybox=\box\mybox % ensure \mybox is void
%\endgroup
%\let\original@shipout\shipout
%
%\shipout :=
%  % trick to get a void box \mybox
%  \begingroup
%    \setbox\mybox=\box\mybox
%  \endgroup
%  \afterassignment\@test
%  \setbox\mybox=
%
%\@test :=
%  \ifvoid\mybox
%    \expandafter\aftergroup
%  \fi
%  \@output
%
%\@output :=
%  \ifvoid\mybox
%    % cancel output of void box
%  \else
%    \@hook
%    \ifvoid\mybox
%      % user code in \@hook could have voided the box
%    \else
%      \original@shipout\box\mybox
%    \fi
%  \fi
%\end{verbatim}
%\end{quote}
%
% \subsubsection{\cs{lastkern} method}
%
% And for \TeX\ without \eTeX\ using the \cs{lastkern} method:
%\begin{quote}
%\begin{verbatim}
%\newbox\mybox
%\let\original@shipout\shipout
%
%\shipout :=
%  \begingroup
%  \setbox\mybox=\hbox\bgroup
%  \kern1pt
%  \afterassignment\@test
%  \setbox\mybox=
%
%\@test :=
%  \ifdim\lastkern=0pt
%    \expandafter\aftergroup
%  \fi
%  \@output
%
%\@output :=
%  \egroup
%  \endgroup
%  \ifvoid\mybox
%    % cancel output of void box
%  \else
%    \@hook
%    \ifvoid\mybox
%      % user code in \@hook could have voided the box
%    \else
%      \original@shipout\box\mybox
%    \fi
%  \fi
%\end{verbatim}
%\end{quote}
%
%
% \StopEventually{
% }
%
% \section{Implementation}
%
% Package \xpackage{atbegshi} uses \eTeX's \cs{currentgrouplevel},
% if it is available. Otherwise the \cs{lastkern} method is used.
%
%    \begin{macrocode}
%<*package>
%    \end{macrocode}
%
% \subsection{Reload check and package identification}
%    Reload check, especially if the package is not used with \LaTeX.
%    \begin{macrocode}
\begingroup\catcode61\catcode48\catcode32=10\relax%
  \catcode13=5 % ^^M
  \endlinechar=13 %
  \catcode35=6 % #
  \catcode39=12 % '
  \catcode44=12 % ,
  \catcode45=12 % -
  \catcode46=12 % .
  \catcode58=12 % :
  \catcode64=11 % @
  \catcode123=1 % {
  \catcode125=2 % }
  \expandafter\let\expandafter\x\csname ver@atbegshi.sty\endcsname
  \ifx\x\relax % plain-TeX, first loading
  \else
    \def\empty{}%
    \ifx\x\empty % LaTeX, first loading,
      % variable is initialized, but \ProvidesPackage not yet seen
    \else
      \expandafter\ifx\csname PackageInfo\endcsname\relax
        \def\x#1#2{%
          \immediate\write-1{Package #1 Info: #2.}%
        }%
      \else
        \def\x#1#2{\PackageInfo{#1}{#2, stopped}}%
      \fi
      \x{atbegshi}{The package is already loaded}%
      \aftergroup\endinput
    \fi
  \fi
\endgroup%
%    \end{macrocode}
%    Package identification:
%    \begin{macrocode}
\begingroup\catcode61\catcode48\catcode32=10\relax%
  \catcode13=5 % ^^M
  \endlinechar=13 %
  \catcode35=6 % #
  \catcode39=12 % '
  \catcode40=12 % (
  \catcode41=12 % )
  \catcode44=12 % ,
  \catcode45=12 % -
  \catcode46=12 % .
  \catcode47=12 % /
  \catcode58=12 % :
  \catcode64=11 % @
  \catcode91=12 % [
  \catcode93=12 % ]
  \catcode123=1 % {
  \catcode125=2 % }
  \expandafter\ifx\csname ProvidesPackage\endcsname\relax
    \def\x#1#2#3[#4]{\endgroup
      \immediate\write-1{Package: #3 #4}%
      \xdef#1{#4}%
    }%
  \else
    \def\x#1#2[#3]{\endgroup
      #2[{#3}]%
      \ifx#1\@undefined
        \xdef#1{#3}%
      \fi
      \ifx#1\relax
        \xdef#1{#3}%
      \fi
    }%
  \fi
\expandafter\x\csname ver@atbegshi.sty\endcsname
\ProvidesPackage{atbegshi}%
  [2016/06/09 v1.18 At begin shipout hook (HO)]%
%    \end{macrocode}
%
% \subsection{Catcodes}
%
%    \begin{macrocode}
\begingroup\catcode61\catcode48\catcode32=10\relax%
  \catcode13=5 % ^^M
  \endlinechar=13 %
  \catcode123=1 % {
  \catcode125=2 % }
  \catcode64=11 % @
  \def\x{\endgroup
    \expandafter\edef\csname AtBegShi@AtEnd\endcsname{%
      \endlinechar=\the\endlinechar\relax
      \catcode13=\the\catcode13\relax
      \catcode32=\the\catcode32\relax
      \catcode35=\the\catcode35\relax
      \catcode61=\the\catcode61\relax
      \catcode64=\the\catcode64\relax
      \catcode123=\the\catcode123\relax
      \catcode125=\the\catcode125\relax
    }%
  }%
\x\catcode61\catcode48\catcode32=10\relax%
\catcode13=5 % ^^M
\endlinechar=13 %
\catcode35=6 % #
\catcode64=11 % @
\catcode123=1 % {
\catcode125=2 % }
\def\TMP@EnsureCode#1#2{%
  \edef\AtBegShi@AtEnd{%
    \AtBegShi@AtEnd
    \catcode#1=\the\catcode#1\relax
  }%
  \catcode#1=#2\relax
}
\TMP@EnsureCode{40}{12}% (
\TMP@EnsureCode{41}{12}% )
\TMP@EnsureCode{44}{12}% ,
\TMP@EnsureCode{45}{12}% -
\TMP@EnsureCode{47}{12}% /
\TMP@EnsureCode{46}{12}% .
\TMP@EnsureCode{58}{12}% :
\TMP@EnsureCode{91}{12}% [
\TMP@EnsureCode{93}{12}% ]
\TMP@EnsureCode{94}{7}% ^(superscript)
\TMP@EnsureCode{96}{12}% `
\edef\AtBegShi@AtEnd{\AtBegShi@AtEnd\noexpand\endinput}
%    \end{macrocode}
%
% \subsection{Preparations}
%
%    \begin{macrocode}
\begingroup\expandafter\expandafter\expandafter\endgroup
\expandafter\ifx\csname RequirePackage\endcsname\relax
  \def\TMP@RequirePackage#1[#2]{%
    \begingroup\expandafter\expandafter\expandafter\endgroup
    \expandafter\ifx\csname ver@#1.sty\endcsname\relax
      \input #1.sty\relax
    \fi
  }%
  \TMP@RequirePackage{infwarerr}[2007/09/09]%
  \TMP@RequirePackage{ltxcmds}[2010/03/01]%
\else
  \RequirePackage{infwarerr}[2007/09/09]%
  \RequirePackage{ltxcmds}[2010/03/01]%
\fi
%    \end{macrocode}
%
%    \begin{macro}{\AtBegShi@CheckDefinable}
%    \begin{macrocode}
\begingroup\expandafter\expandafter\expandafter\endgroup
\expandafter\ifx\csname @ifdefinable\endcsname\relax
  \def\AtBegShi@CheckDefinable#1{%
    \ifcase\ifx#1\relax
             \ltx@one
           \else
             \ifx#1\@undefined
               \ltx@one
             \else
               \ltx@zero
             \fi
           \fi
      \@PackageError{atbegshi}{%
        \string#1\space is already defined%
      }\@ehd
    \fi
  }%
\else
  \def\AtBegShi@CheckDefinable#1{%
    \@ifdefinable{#1}{}%
  }%
\fi
%    \end{macrocode}
%    \end{macro}
%
%    \begin{macro}{\ifAtBegShi@Discarded}
%    \begin{macrocode}
\ltx@newif\ifAtBegShi@Discarded
%    \end{macrocode}
%    \end{macro}
%    \begin{macro}{\AtBeginShipoutDiscard}
%    \begin{macrocode}
\AtBegShi@CheckDefinable\AtBeginShipoutDiscard
\def\AtBeginShipoutDiscard{%
  \deadcycles=\ltx@zero
  \global\AtBegShi@Discardedtrue
}
%    \end{macrocode}
%    \end{macro}
%
%    \begin{macrocode}
\begingroup\expandafter\expandafter\expandafter\endgroup
\expandafter\ifx\csname currentgrouplevel\endcsname\relax
  \catcode`X=9  % ignore
  \catcode`E=14 % comment
\else
  \catcode`X=14 % comment
  \catcode`E=9  % ignore
\fi
%    \end{macrocode}
%
%    \begin{macro}{\AtBegShi@Shipout}
%    \begin{macrocode}
\def\AtBegShi@Shipout{%
X \begingroup
X \setbox\AtBeginShipoutBox=\hbox\bgroup
X \kern\p@
E \edef\AtBegShi@GroupLevel{\number\currentgrouplevel}%
  \afterassignment\AtBegShi@Test
X \global
  \setbox\AtBeginShipoutBox=%
}
%    \end{macrocode}
%    \end{macro}
%
%    \begin{macro}{\AtBegShi@Test}
%    \begin{macrocode}
\def\AtBegShi@Test{%
X \ifdim\lastkern=0pt %
E \ifnum\AtBegShi@GroupLevel<\currentgrouplevel
    \expandafter\aftergroup
  \fi
  \AtBegShi@Output
}
%    \end{macrocode}
%    \end{macro}
%
%    \begin{macro}{\AtBegShi@Output}
%    \begin{macrocode}
\def\AtBegShi@Output{%
X \egroup
X \endgroup
  \ifvoid\AtBeginShipoutBox
    \@PackageWarning{atbegshi}{Ignoring void shipout box}%
  \else
    \let\AtBegShi@OrgProtect\protect
    \csname set@typeset@protect\endcsname
    \global\AtBegShi@Discardedfalse
    \AtBegShi@Hook
    \expandafter\gdef\expandafter\AtBegShi@HookNext
    \expandafter{\expandafter}%
    \AtBegShi@HookNext
    \ifAtBegShi@Discarded
      \@PackageInfoNoLine{atbegshi}{Shipout page discarded}%
      \global\AtBegShi@Discardedfalse
      \begingroup
        \setbox\AtBeginShipoutBox\box\AtBeginShipoutBox
      \endgroup
      \let\protect\AtBegShi@OrgProtect
    \else
      \AtBegShi@First
      \let\protect\AtBegShi@OrgProtect
      \AtBegShi@GetBoxSize\AtBeginShipoutBox
      \ltx@ifundefined{AtNextShipout}{%
      }{%
        \AtNextShipout{\AtBegShi@GetBoxSize\@cclv}%
      }%
      \AtBeginShipoutOriginalShipout\box\AtBeginShipoutBox
    \fi
  \fi
}
%    \end{macrocode}
%    \end{macro}
%
%    \begin{macro}{\AtBegShi@GetBoxSize}
%    \begin{macrocode}
\def\AtBegShi@GetBoxSize#1{%
  \xdef\AtBeginShipoutBoxWidth{\the\wd#1}%
  \xdef\AtBeginShipoutBoxHeight{\the\ht#1}%
  \xdef\AtBeginShipoutBoxDepth{\the\dp#1}%
}
%    \end{macrocode}
%    \end{macro}
%    \begin{macro}{\AtBeginShipoutBoxWidth}
%    \begin{macrocode}
\def\AtBeginShipoutBoxWidth{0pt}
%    \end{macrocode}
%    \end{macro}
%    \begin{macro}{\AtBeginShipoutBoxHeight}
%    \begin{macrocode}
\def\AtBeginShipoutBoxHeight{0pt}
%    \end{macrocode}
%    \end{macro}
%    \begin{macro}{\AtBeginShipoutBoxDepth}
%    \begin{macrocode}
\def\AtBeginShipoutBoxDepth{0pt}
%    \end{macrocode}
%    \end{macro}
%
%    \begin{macrocode}
\catcode`\X=11 %
\catcode`\E=11 %
%    \end{macrocode}
%
%    \begin{macro}{\AtBegShi@First}
%    \begin{macrocode}
\def\AtBegShi@First{%
  \ifx\AtBegShi@HookFirst\ltx@empty
  \else
    \AtBeginShipoutAddToBox{\AtBegShi@HookFirst}%
  \fi
  \global\let\AtBegShi@First\ltx@empty
  \global\let\AtBeginShipoutFirst\AtBegShi@FirstDisabled
}
%    \end{macrocode}
%    \end{macro}
%
%    \begin{macro}{\AtBegShi@Hook}
%    \begin{macrocode}
\gdef\AtBegShi@Hook{}
%    \end{macrocode}
%    \end{macro}
%    \begin{macro}{\AtBegShi@HookNext}
%    \begin{macrocode}
\gdef\AtBegShi@HookNext{}
%    \end{macrocode}
%    \end{macro}
%    \begin{macro}{\AtBegShi@HookFirst}
%    \begin{macrocode}
\gdef\AtBegShi@HookFirst{}
%    \end{macrocode}
%    \end{macro}
%    \begin{macro}{\AtBeginShipout}
%    \begin{macrocode}
\AtBegShi@CheckDefinable\AtBeginShipout
\def\AtBeginShipout{%
  \AtBegShi@AddHook\AtBegShi@Hook
}
%    \end{macrocode}
%    \end{macro}
%    \begin{macro}{\AtBeginShipoutNext}
%    \begin{macrocode}
\AtBegShi@CheckDefinable\AtBeginShipoutNext
\def\AtBeginShipoutNext{%
  \AtBegShi@AddHook\AtBegShi@HookNext
}
%    \end{macrocode}
%    \end{macro}
%    \begin{macro}{\AtBeginShipoutFirst}
%    \begin{macrocode}
\AtBegShi@CheckDefinable\AtBeginShipoutFirst
\def\AtBeginShipoutFirst{%
  \AtBegShi@AddTo\AtBegShi@HookFirst
}
%    \end{macrocode}
%    \end{macro}
%    \begin{macro}{\AtBegShi@FirstDisabled}
%    \begin{macrocode}
\long\def\AtBegShi@FirstDisabled#1{%
  \@PackageWarning{atbegshi}{%
    First page is already shipped out, ignoring\MessageBreak
    \string\AtBeginShipoutFirst
  }%
}
%    \end{macrocode}
%    \end{macro}
%
%    \begin{macro}{\AtBegShi@AddTo}
%    \begin{macrocode}
\begingroup\expandafter\expandafter\expandafter\endgroup
\expandafter\ifx\csname g@addto@macro\endcsname\relax
  \long\def\AtBegShi@AddTo#1#2{%
    \begingroup
      \toks\ltx@zero\expandafter{#1#2}%
      \xdef#1{\the\toks\ltx@zero}%
    \endgroup
  }%
\else
  \let\AtBegShi@AddTo\g@addto@macro
\fi
%    \end{macrocode}
%    \end{macro}
%
%    \begin{macro}{\AtBegShi@AddHook}
%    \begin{macrocode}
\long\def\AtBegShi@AddHook#1#2{%
  \AtBegShi@AddTo#1{\AtBegShi@Item{#2}}%
}
%    \end{macrocode}
%    \end{macro}
%
%    \begin{macro}{\AtBegShi@Item}
%    \begin{macrocode}
\long\def\AtBegShi@Item#1{%
  \ifAtBegShi@Discarded
  \else
    #1%
    \ifAtBegShi@Discarded
    \else
      \ifvoid\AtBeginShipoutBox
        \@PackageWarning{atbegshi}{%
          Shipout box was voided by hook,\MessageBreak
          ignoring shipout box%
        }%
        \AtBeginShipoutDiscard
      \fi
    \fi
  \fi
}
%    \end{macrocode}
%    \end{macro}
%
%    \begin{macro}{\AtBeginShipoutInit}
%    \begin{macrocode}
\AtBegShi@CheckDefinable\AtBeginShipoutInit
\def\AtBeginShipoutInit{%
  \ltx@IfUndefined{newbox}{%
    \@PackageError{atbegshi}{%
      \string\AtBeginShipoutInit\space failed\MessageBreak
       because of missing \expandafter\string\csname newbox\endcsname
    }\@ehc
  }{%
    \csname newbox\endcsname\AtBeginShipoutBox
    \AtBegShi@CheckDefinable\AtBeginShipoutOriginalShipout
    \global\let\AtBeginShipoutOriginalShipout\shipout
    \global\let\shipout\AtBegShi@Shipout
  }%
  \gdef\AtBeginShipoutInit{}%
}
%    \end{macrocode}
%    \end{macro}
%    \begin{macrocode}
\begingroup\expandafter\expandafter\expandafter\endgroup
\expandafter\ifx\csname AtBeginDocument\endcsname\relax
  \AtBeginShipoutInit
\else
  \AtBeginDocument{\AtBeginShipoutInit}%
\fi
%    \end{macrocode}
%
% \subsection{Additions to the shipout box}
%
%    \begin{macro}{\AtBeginShipoutAddToBox}
%    \begin{macrocode}
\def\AtBeginShipoutAddToBox#1{%
  \ifhbox\AtBeginShipoutBox
    \edef\AtBegShi@restore{%
      \hfuzz=\the\hfuzz\relax
      \hbadness=\the\hbadness\relax
    }%
    \hfuzz=1073741823sp\relax
    \hbadness=2147483647\relax
    \setbox\AtBeginShipoutBox=\hbox to \wd\AtBeginShipoutBox{%
      \setbox\ltx@zero=\hbox{%
        \begingroup
          \AtBegShi@restore
          #1%
        \endgroup
      }%
      \wd\ltx@zero=0pt\relax
      \ht\ltx@zero=0pt\relax
      \dp\ltx@zero=0pt\relax
      \raise\ht\AtBeginShipoutBox\copy\ltx@zero
      \unhcopy\AtBeginShipoutBox
    }%
    \AtBegShi@restore
  \else
    \ifvbox\AtBeginShipoutBox
      \edef\AtBegShi@restore{%
        \vfuzz=\the\vfuzz\relax
        \vbadness=\the\vbadness\relax
        \dimen\ltx@zero=\the\dimen\ltx@zero\relax
      }%
      \edef\AtBegShi@restorebox{%
        \ht\AtBeginShipoutBox=\the\ht\AtBeginShipoutBox\relax
        \dp\AtBeginShipoutBox=\the\dp\AtBeginShipoutBox\relax
      }%
      \vfuzz=1073741823sp\relax
      \vbadness=2147483647\relax
      \dimen\ltx@zero=\ht\AtBeginShipoutBox
      \advance\dimen\ltx@zero by \dp\AtBeginShipoutBox
      \setbox\AtBeginShipoutBox=\vbox to \dimen\ltx@zero{%
        \setbox\ltx@zero=\hbox{%
          \begingroup
            \AtBegShi@restore
            #1%
          \endgroup
        }%
        \wd\ltx@zero=0pt\relax
        \ht\ltx@zero=0pt\relax
        \dp\ltx@zero=0pt\relax
        \baselineskip=0pt\relax
        \lineskip=0pt\relax
        \lineskiplimit=0pt\relax
        \copy\ltx@zero
        \unvbox\AtBeginShipoutBox
        \kern0pt%
      }%
      \AtBegShi@restore
      \AtBegShi@restorebox
    \fi
  \fi
}
%    \end{macrocode}
%    \end{macro}
%
%    \begin{macro}{\AtBeginShipoutAddToBoxForeground}
%    \begin{macrocode}
\def\AtBeginShipoutAddToBoxForeground#1{%
  \ifhbox\AtBeginShipoutBox
    \edef\AtBegShi@restore{%
      \hfuzz=\the\hfuzz\relax
      \hbadness=\the\hbadness\relax
    }%
    \hfuzz=1073741823sp\relax
    \hbadness=2147483647\relax
    \setbox\AtBeginShipoutBox=\hbox to \wd\AtBeginShipoutBox{%
      \unhcopy\AtBeginShipoutBox
      \kern-\wd\AtBeginShipoutBox
      \setbox\ltx@zero=\hbox{%
        \begingroup
          \AtBegShi@restore
          #1%
        \endgroup
      }%
      \wd\ltx@zero=0pt\relax
      \ht\ltx@zero=0pt\relax
      \dp\ltx@zero=0pt\relax
      \raise\ht\AtBeginShipoutBox\copy\ltx@zero
      \kern\wd\AtBeginShipoutBox
    }%
    \AtBegShi@restore
  \else
    \ifvbox\AtBeginShipoutBox
      \edef\AtBegShi@restore{%
        \vfuzz=\the\vfuzz\relax
        \vbadness=\the\vbadness\relax
        \dimen\ltx@zero=\the\dimen\ltx@zero\relax
      }%
      \edef\AtBegShi@restorebox{%
        \ht\AtBeginShipoutBox=\the\ht\AtBeginShipoutBox\relax
        \dp\AtBeginShipoutBox=\the\dp\AtBeginShipoutBox\relax
      }%
      \vfuzz=1073741823sp\relax
      \vbadness=2147483647\relax
      \dimen\ltx@zero=\ht\AtBeginShipoutBox
      \advance\dimen\ltx@zero by \dp\AtBeginShipoutBox
      \setbox\AtBeginShipoutBox=\vbox to \dimen\ltx@zero{%
        \setbox\ltx@zero=\hbox{%
          \begingroup
            \AtBegShi@restore
            #1%
          \endgroup
        }%
        \wd\ltx@zero=0pt\relax
        \ht\ltx@zero=0pt\relax
        \dp\ltx@zero=0pt\relax
        \baselineskip=0pt\relax
        \lineskip=0pt\relax
        \lineskiplimit=0pt\relax
        \unvbox\AtBeginShipoutBox
        \kern-\dimen\ltx@zero
        \copy\ltx@zero
        \kern\dimen\ltx@zero
      }%
      \AtBegShi@restore
      \AtBegShi@restorebox
    \fi
  \fi
}
%    \end{macrocode}
%    \end{macro}
%
% \subsection{Positioning}
%
%    \begin{macrocode}
\begingroup\expandafter\expandafter\expandafter\endgroup
\expandafter\ifx\csname RequirePackage\endcsname\relax
  \def\TMP@RequirePackage#1[#2]{%
    \begingroup\expandafter\expandafter\expandafter\endgroup
    \expandafter\ifx\csname ver@#1.sty\endcsname\relax
      \input #1.sty\relax
    \fi
  }%
  \TMP@RequirePackage{ifpdf}[2011/01/30]%
\else
  \RequirePackage{ifpdf}[2011/01/30]%
\fi
%    \end{macrocode}
%
%    \begin{macrocode}
\ifpdf
  \def\AtBegShi@horigin{%
    \ifx\pdfhorigin\@undefined\pdfvariable horigin\else\pdfhorigin\fi}%
  \def\AtBegShi@vorigin{%
    \ifx\pdfvorigin\@undefined\pdfvariable vorigin\else\pdfvorigin\fi}%
\else
  \def\AtBegShi@horigin{72.27pt}%
  \def\AtBegShi@vorigin{72.27pt}%
\fi
%    \end{macrocode}
%
%    \begin{macrocode}
\begingroup
\ifcase
    \expandafter\ifx\csname picture\endcsname\relax
      1%
    \else
      \expandafter\ifx\csname endpicture\endcsname\relax
        1%
      \else
        0%
      \fi
    \fi
  \endgroup
  \def\AtBegShi@BeginPicture{%
    \begingroup
    \picture(0,0)\relax
    \begingroup\expandafter\expandafter\expandafter\endgroup
    \expandafter\ifx\csname unitlength\endcsname\relax
    \else
      \unitlength=1pt\relax
    \fi
    \ignorespaces
  }%
  \def\AtBegShi@EndPicture{%
    \endpicture
    \endgroup
  }%
\else
  \endgroup
  \def\AtBegShi@BeginPicture{%
    \setbox\ltx@zero=\hbox\bgroup
    \begingroup
    \ignorespaces
  }%
  \def\AtBegShi@EndPicture{%
    \endgroup
    \egroup
    \ht\ltx@zero=0pt\relax
    \dp\ltx@zero=0pt\relax
    \copy\ltx@zero
  }%
\fi
%    \end{macrocode}
%
%    \begin{macro}{\AtBeginShipoutUpperLeft}
%    A surrounding \cs{rlap} is not necessary, because
%    the stuff is put in an \cs{hbox} with zero width.
%    \begin{macrocode}
\def\AtBeginShipoutUpperLeft#1{%
  \AtBeginShipoutAddToBox{%
    \kern-\AtBegShi@horigin\relax
    \vbox to 0pt{%
      \kern-\AtBegShi@vorigin\relax
      \AtBegShi@BeginPicture
      #1%
      \AtBegShi@EndPicture
      \vss
    }%
  }%
}
%    \end{macrocode}
%    \end{macro}
%
%    \begin{macro}{\AtBeginShipoutUpperLeftForeground}
%    \begin{macrocode}
\def\AtBeginShipoutUpperLeftForeground#1{%
  \AtBeginShipoutAddToBoxForeground{%
    \kern-\AtBegShi@horigin\relax
    \vbox to 0pt{%
      \kern-\AtBegShi@vorigin\relax
      \AtBegShi@BeginPicture
      #1%
      \AtBegShi@EndPicture
      \vss
    }%
  }%
}
%    \end{macrocode}
%    \end{macro}
%
% \subsection{Patches}
%
%    Patches for \LaTeX\ packages that redefine \cs{shipout}.
%    \LaTeX\ is now supposed to use \eTeX. Thus we do not patch,
%    without \LaTeX\ and \eTeX.
%    \begin{macrocode}
\def\AtBegShi@AbortIfUndefined#1{%
  \begingroup\expandafter\expandafter\expandafter\endgroup
  \expandafter\ifx\csname#1\endcsname\relax
    \expandafter\AtBegShi@AtEnd
  \fi
}
\AtBegShi@AbortIfUndefined{currentgrouplevel}%
\AtBegShi@AbortIfUndefined{AtBeginDocument}%
\AtBegShi@AbortIfUndefined{@ifpackageloaded}%
\AtBegShi@AbortIfUndefined{@ifclassloaded}%
%    \end{macrocode}
%
% \subsubsection{Package \xpackage{crop}}
%
% Fix of method and box.
%    \begin{macrocode}
\def\AtBegShi@PatchCrop{%
  \begingroup
    \def\AtBegShi@Crop@shipout{%
      \afterassignment\CROP@ship
      \setbox\@cclv=%
    }%
    \def\AtBegShi@Crop@ship{%
      \ifvoid\@cclv
        \expandafter\aftergroup
      \fi
      \CROP@@ship
    }%
    \def\AtBegShi@Crop@shiplist{%
      \lineskip\z@
      \lineskiplimit\z@
      \baselineskip\z@
      \CROP@kernel
      \box\@cclv
    }%
    \def\AtBegShi@Crop@@ship{%
      \CROP@shipout\vbox{%
        \CROP@shiplist
      }%
    }%
    \ifx\AtBegShi@Crop@ship\CROP@ship
      \ifx\AtBegShi@Crop@shiplist\CROP@shiplist
        \ifx\AtBegShi@Crop@@ship\CROP@@ship
          \let\AtBegShi@found\relax
          \ifx\shipout\AtBegShi@Crop@shipout
            \def\AtBegShi@found{\shipout}%
          \else\ifx\AtBeginShipoutOriginalShipout\AtBegShi@Crop@shipout
            \def\AtBegShi@found{\AtBeginShipoutOriginalShipout}%
          \else\ifx\@EveryShipout@Org@Shipout\AtBegShi@Crop@shipout
            \def\AtBegShi@found{\@EveryShipout@Org@Shipout}%
          \else\ifx\GPTorg@shipout\AtBegShi@Crop@shipout
            \def\AtBegShi@found{\GPTorg@shipout}%
          \else\ifx\THBorg@shipout\AtBegShi@Crop@shipout
            \def\AtBegShi@found{\THBorg@shipout}%
          \else\ifx\mem@oldshipout\AtBegShi@Crop@shipout
            \def\AtBegShi@found{\mem@oldshipout}%
          \fi\fi\fi\fi\fi\fi
          \ifx\AtBegShi@found\relax
          \else
            \expandafter\endgroup
            \expandafter\def\AtBegShi@found{%
              \edef\AtBegShi@GroupLevel{\number\currentgrouplevel}%
              \afterassignment\CROP@ship
              \setbox\AtBeginShipoutBox=%
            }%
            \def\CROP@ship{%
              \ifnum\AtBegShi@GroupLevel=\currentgrouplevel
              \else
                \expandafter\aftergroup
              \fi
              \CROP@@ship
            }%
            \def\CROP@shiplist{%
              \lineskip 0pt\relax
              \lineskiplimit 0pt\relax
              \baselineskip 0pt\relax
              \CROP@kernel
              \box\AtBeginShipoutBox
            }%
            \def\CROP@@ship{%
              \ifvoid\AtBeginShipoutBox
              \else
                \setbox\AtBeginShipoutBox=\vbox{%
                  \CROP@shiplist
                }%
                \AtBegShi@GetBoxSize\AtBeginShipoutBox
                \expandafter\CROP@shipout
                \expandafter\box
                \expandafter\AtBeginShipoutBox
              \fi
            }%
            \@PackageInfoNoLine{atbegshi}{Package `crop' patched}%
            \begingroup
          \fi
        \fi
      \fi
    \fi
  \endgroup
  \let\AtBegShi@PatchCrop\relax
}
\@ifpackageloaded{crop}{%
  \AtBegShi@PatchCrop
}{%
  \AtBeginDocument{\AtBegShi@PatchCrop}%
}
%    \end{macrocode}
%
% \subsubsection{Package \xpackage{everyshi}}
%
%     Fix of method. Use of box 255 is not changed.
%    \begin{macrocode}
\def\AtBegShi@PatchEveryshi{%
  \begingroup
    \long\def\AtBegShi@Everyshi@shipout{%
      \afterassignment\@EveryShipout@Test
      \global\setbox\@cclv= %
    }%
    \long\def\AtBegShi@Everyshi@Test{%
      \ifvoid\@cclv\relax
        \aftergroup\@EveryShipout@Output
      \else
        \@EveryShipout@Output
      \fi
    }%
    \ifx\AtBegShi@Everyshi@Test\@EveryShipout@Test
      \let\AtBegShi@found\relax
      \ifx\shipout\AtBegShi@Everyshi@shipout
        \def\AtBegShi@found{\shipout}%
      \else\ifx\AtBeginShipoutOriginalShipout\AtBegShi@Everyshi@shipout
        \def\AtBegShi@found{\AtBeginShipoutOriginalShipout}%
      \else\ifx\CROP@shipout\AtBegShi@Everyshi@shipout
        \def\AtBegShi@found{\CROP@shipout}%
      \else\ifx\GPTorg@shipout\AtBegShi@Everyshi@shipout
        \def\AtBegShi@found{\GPTorg@shipout}%
      \else\ifx\THBorg@shipout\AtBegShi@Everyshi@shipout
        \def\AtBegShi@found{\THBorg@shipout}%
      \else\ifx\mem@oldshipout\AtBegShi@Everyshi@shipout
        \def\AtBegShi@found{\mem@oldshipout}%
      \else
        \expandafter\ifx\csname @EveryShipout@Org@Shipout\endcsname
                        \relax
           \ifx\@EveryShipout@Shipout\AtBegShi@Everyshi@shipout
             \def\AtBegShi@found{\@EveryShipout@Shipout}%
           \fi
        \fi
      \fi\fi\fi\fi\fi\fi
      \ifx\AtBegShi@found\relax
      \else
        \expandafter\endgroup
        \expandafter\def\AtBegShi@found{%
          \edef\AtBegShi@GroupLevel{\number\currentgrouplevel}%
          \afterassignment\@EveryShipout@Test
          \setbox\AtBeginShipoutBox=%
        }%
        \def\@EveryShipout@Test{%
          \ifnum\AtBegShi@GroupLevel=\currentgrouplevel
          \else
            \expandafter\aftergroup
          \fi
          \AtBegShi@Everyshi@Output
        }%
        \def\AtBegShi@Everyshi@Output{%
          \ifvoid\AtBeginShipoutBox
          \else
            \global\setbox\ltx@cclv\box\AtBeginShipoutBox
            \expandafter\@EveryShipout@Output
          \fi
        }%
        \@PackageInfoNoLine{atbegshi}{Package `everyshi' patched}%
        \begingroup
      \fi
    \fi
  \endgroup
  \let\AtBegShi@PatchEveryshi\relax
}
\@ifpackageloaded{everyshi}{%
  \AtBegShi@PatchEveryshi
}{%
  \AtBeginDocument{\AtBegShi@PatchEveryshi}%
}
%    \end{macrocode}
%
% \subsubsection{Class \xpackage{memoir}}
%
%     Fix of method and box.
%    \begin{macrocode}
\def\AtBegShi@PatchMemoir{%
  \begingroup
    \def\AtBegShi@Memoir@shipout{%
      \afterassignment\mem@shipi
      \setbox\@cclv=%
    }%
    \def\AtBegShi@Memoir@shipi{%
      \ifvoid\@cclv
        \expandafter\aftergroup
      \fi
      \mem@shipii
    }%
    \def\AtBegShi@Memoir@shipiiA{%
      \mem@oldshipout\vbox{%
        \trimmarks
        \unvbox\@cclv
      }%
    }%
    \def\AtBegShi@Memoir@shipiiB{%
      \ifvoid\@cclv
        \mem@oldshipout\box\@cclv
      \else
        \mem@oldshipout\vbox{%
          \trimmarks
          \unvbox\@cclv
        }%
      \fi
    }%
    \def\AtBegShi@Memoir@PatchAB{%
      \ifvoid\AtBeginShipoutBox
      \else
        \setbox\AtBeginShipoutBox=\vbox{%
          \trimmarks
          \ifvbox\AtBeginShipoutBox
            \unvbox\AtBeginShipoutBox
          \else
            \box\AtBeginShipoutBox
          \fi
        }%
        \AtBegShi@GetBoxSize\AtBeginShipoutBox
        \expandafter\mem@oldshipout
        \expandafter\box
        \expandafter\AtBeginShipoutBox
      \fi
    }%
    \def\AtBegShi@Memoir@shipiiC{% 2008/08/07 v1.6180339a
      \ifvoid\@cclv
        \mem@oldshipout\box\@cclv
      \else
        \ifshowtrims
          \mem@oldshipout\vbox{\trimmarks\unvbox\@cclv}%
        \else
          \mem@oldshipout\box\@cclv
        \fi
      \fi
    }%
    \def\AtBegShi@Memoir@shipiiD{% 2011/03/06 v3.6j
      \ifvoid\@cclv
        \mem@oldshipout\box\@cclv
      \else
        \ifshowtrims
          \mem@oldshipout\vbox{%
            \trimmarks
            \nointerlineskip
            \box\@cclv
          }%
        \else
          \mem@oldshipout\box\@cclv
        \fi
      \fi
    }%
    \def\AtBegShi@Memoir@PatchCD{%
      \ifvoid\AtBeginShipoutBox
      \else
        \ifshowtrims
          \setbox\AtBeginShipoutBox=\vbox{%
            \trimmarks
            \nointerlineskip
            \box\AtBeginShipoutBox
          }%
        \fi
        \AtBegShi@GetBoxSize\AtBeginShipoutBox
        \expandafter\mem@oldshipout
        \expandafter\box
        \expandafter\AtBeginShipoutBox
      \fi
    }%
    \ifx\AtBegShi@Memoir@shipi\mem@shipi
      \let\AtBegShi@found\ltx@one
      \ifx\AtBegShi@Memoir@shipiiA\mem@shipii
        \let\AtBegShi@found\ltx@zero
        \global\let\AtBegShi@Memoir@PatchX\AtBegShi@Memoir@PatchAB
      \else\ifx\AtBegShi@Memoir@shipiiB\mem@shipii
        \let\AtBegShi@found\ltx@zero
        \global\let\AtBegShi@Memoir@PatchX\AtBegShi@Memoir@PatchAB
      \else\ifx\AtBegShi@Memoir@shipiiC\mem@shipii
        \let\AtBegShi@found\ltx@zero
        \global\let\AtBegShi@Memoir@PatchX\AtBegShi@Memoir@PatchCD
      \else\ifx\AtBegShi@Memoir@shipiiD\mem@shipii
        \let\AtBegShi@found\ltx@zero
        \global\let\AtBegShi@Memoir@PatchX\AtBegShi@Memoir@PatchCD
      \fi\fi\fi\fi
      \ifcase\AtBegShi@found
        \let\AtBegShi@found\relax
        \ifx\shipout\AtBegShi@Memoir@shipout
          \def\AtBegShi@found{\shipout}%
        \else\ifx\AtBeginShipoutOriginalShipout\AtBegShi@Memoir@shipout
          \def\AtBegShi@found{\AtBeginShipoutOriginalShipout}%
        \else\ifx\CROP@shipout\AtBegShi@Memoir@shipout
          \def\AtBegShi@found{\CROP@shipout}%
        \else\ifx\GPTorg@shipout\AtBegShi@Memoir@shipout
          \def\AtBegShi@found{\GPTorg@shipout}%
        \else\ifx\THBorg@shipout\AtBegShi@Memoir@shipout
          \def\AtBegShi@found{\THBorg@shipout}%
        \else\ifx\@EveryShipout@Org@Shipout\AtBegShi@Memoir@shipout
          \def\AtBegShi@found{\@EveryShipout@Org@Shipout}%
        \fi\fi\fi\fi\fi\fi
        \ifx\AtBegShi@found\relax
        \else
          \expandafter\endgroup
          \expandafter\def\AtBegShi@found{%
            \edef\AtBegShi@GroupLevel{\number\currentgrouplevel}%
            \afterassignment\mem@shipi
            \setbox\AtBeginShipoutBox=%
          }%
          \def\mem@shipi{%
            \ifnum\AtBegShi@GroupLevel=\currentgrouplevel
            \else
              \expandafter\aftergroup
            \fi
            \mem@shipii
          }%
          \let\mem@shipii\AtBegShi@Memoir@PatchX
          \@PackageInfoNoLine{atbegshi}{Class `memoir' patched}%
          \begingroup
        \fi
      \fi
    \fi
  \endgroup
  \let\AtBegShi@PatchMemoir\relax
}
\@ifclassloaded{memoir}{%
  \AtBegShi@PatchMemoir
}{%
  \AtBeginDocument{\AtBegShi@PatchMemoir}%
}
%    \end{macrocode}
%
%    \begin{macrocode}
\AtBegShi@AtEnd%
%</package>
%    \end{macrocode}
%
% \section{Test}
%
% \subsection{Catcode checks for loading}
%
%    \begin{macrocode}
%<*test1>
%    \end{macrocode}
%    \begin{macrocode}
\catcode`\{=1 %
\catcode`\}=2 %
\catcode`\#=6 %
\catcode`\@=11 %
\expandafter\ifx\csname count@\endcsname\relax
  \countdef\count@=255 %
\fi
\expandafter\ifx\csname @gobble\endcsname\relax
  \long\def\@gobble#1{}%
\fi
\expandafter\ifx\csname @firstofone\endcsname\relax
  \long\def\@firstofone#1{#1}%
\fi
\expandafter\ifx\csname loop\endcsname\relax
  \expandafter\@firstofone
\else
  \expandafter\@gobble
\fi
{%
  \def\loop#1\repeat{%
    \def\body{#1}%
    \iterate
  }%
  \def\iterate{%
    \body
      \let\next\iterate
    \else
      \let\next\relax
    \fi
    \next
  }%
  \let\repeat=\fi
}%
\def\RestoreCatcodes{}
\count@=0 %
\loop
  \edef\RestoreCatcodes{%
    \RestoreCatcodes
    \catcode\the\count@=\the\catcode\count@\relax
  }%
\ifnum\count@<255 %
  \advance\count@ 1 %
\repeat

\def\RangeCatcodeInvalid#1#2{%
  \count@=#1\relax
  \loop
    \catcode\count@=15 %
  \ifnum\count@<#2\relax
    \advance\count@ 1 %
  \repeat
}
\def\RangeCatcodeCheck#1#2#3{%
  \count@=#1\relax
  \loop
    \ifnum#3=\catcode\count@
    \else
      \errmessage{%
        Character \the\count@\space
        with wrong catcode \the\catcode\count@\space
        instead of \number#3%
      }%
    \fi
  \ifnum\count@<#2\relax
    \advance\count@ 1 %
  \repeat
}
\def\space{ }
\expandafter\ifx\csname LoadCommand\endcsname\relax
  \def\LoadCommand{\input atbegshi.sty\relax}%
\fi
\def\Test{%
  \RangeCatcodeInvalid{0}{47}%
  \RangeCatcodeInvalid{58}{64}%
  \RangeCatcodeInvalid{91}{96}%
  \RangeCatcodeInvalid{123}{255}%
  \catcode`\@=12 %
  \catcode`\\=0 %
  \catcode`\%=14 %
  \LoadCommand
  \RangeCatcodeCheck{0}{36}{15}%
  \RangeCatcodeCheck{37}{37}{14}%
  \RangeCatcodeCheck{38}{47}{15}%
  \RangeCatcodeCheck{48}{57}{12}%
  \RangeCatcodeCheck{58}{63}{15}%
  \RangeCatcodeCheck{64}{64}{12}%
  \RangeCatcodeCheck{65}{90}{11}%
  \RangeCatcodeCheck{91}{91}{15}%
  \RangeCatcodeCheck{92}{92}{0}%
  \RangeCatcodeCheck{93}{96}{15}%
  \RangeCatcodeCheck{97}{122}{11}%
  \RangeCatcodeCheck{123}{255}{15}%
  \RestoreCatcodes
}
\Test
\csname @@end\endcsname
\end
%    \end{macrocode}
%    \begin{macrocode}
%</test1>
%    \end{macrocode}
%
%    \begin{macrocode}
%<*test2>
\input atbegshi.sty\relax
\def\msg#{\immediate\write16}
\msg{File: atbegshi-test2.tex 2016/06/09 v1.18 Test file for plain-TeX}
\def\testmsg#1#2{%
  \msg{}%
  \msg{*** Test with box (#1), expected page output [#2]}% hash-ok
}

\newbox\voidbox
\def\void{\box\voidbox}
\begingroup
  \setbox\voidbox=\void
\endgroup

\count0=0\relax
\AtBeginShipout{%
  \global\advance\count0 by 1\relax
  \msg{* Inside \string\AtBeginShipout: [\the\count0]}%
}

\AtBeginShipoutFirst{%
  \msg{* Inside \string\AtBeginShipoutFirst}%
  Hello World%
}

\testmsg{\string\null}{1}
\shipout\null

\AtBeginShipoutFirst{%
  This is too late%
}

\testmsg{void}{}
\shipout\void

\testmsg{\string\copy255 (not void)}{2}
\setbox255\hbox{\vrule height 10bp width 10bp}
\shipout\copy255 %

\testmsg{\string\copy255 (again)}{3}
\shipout\copy255 %

\testmsg{\string\box255}{4}
\shipout\box255 %

\testmsg{\string\box255 (again)}{}
\shipout\box255 %

\testmsg{\string\hbox}{5}
\shipout\hbox{\vrule height 5bp width 20bp}

\testmsg{\string\vbox}{6}
\shipout\vbox{\hrule height 20bp width 5bp}

\testmsg{\string\null, voided by hook}{}
\def\VoidBox{%
  \begingroup
    \setbox\AtBeginShipoutBox=\box\AtBeginShipoutBox
  \endgroup
}
\AtBeginShipout{\VoidBox}
\shipout\null
\def\VoidBox{}

\msg{*** \string\begingroup}
\begingroup
  \testmsg{void}{}%
  \shipout\void
\msg{*** \string\endgroup}
\endgroup

\msg{*** \string\begingroup}
\begingroup
  \testmsg{void}{}%
  \shipout\void
  \testmsg{\string\null}{8}%
  \shipout\null
\msg{*** \string\endgroup}
\endgroup

\testmsg{output routine}{9}
Hello World
\vfill
\eject

\testmsg{\string\null\space(discarded)}{}
\AtBeginShipout{%
  \msg{* Inside \string\AtBeginShipout: DISCARD}%
  \AtBeginShipoutDiscard
}
\shipout\null

\end
%</test2>
%    \end{macrocode}
%
%    \begin{macrocode}
%<*test3>
\NeedsTeXFormat{LaTeX2e}
\ProvidesFile{atbegshi-test3.tex}[2016/06/09 v1.18 Test file for LaTeX]
\RequirePackage{color}
\pagecolor{yellow}
\documentclass[a5paper,showtrims]{memoir}
\usepackage{atbegshi}
\AtBeginShipout{%
  \setbox\AtBeginShipoutBox=\vbox{%
    \vbox to 0pt{%
      \kern-1.5in %
      \hbox to 0pt{%
        \kern-1.5in %
        \color{blue}%
        \rule{1in}{1in}%
        \hss
      }%
      \vss
    }%
    \hrule
    \hbox{\vrule\box\AtBeginShipoutBox\vrule}%
    \hrule
  }%
}
\usepackage{eso-pic}
\makeatletter
\@ifundefined{@EveryShipout@Init}{%
  \typeout{Test skipped}%
  \@@end
}{}
\@EveryShipout@Init
\let\@EveryShipout@Init\relax
\makeatother
\AddToShipoutPicture{%
  \hspace{.52\paperwidth}%
  \colorbox{cyan}{%
    \rule{0mm}{\paperheight}%
    \hspace{.48\paperwidth}%
  }%
}
%    \end{macrocode}
%    Newer versions of class \xclass{memoir} emulate package
%    \xpackage{crop} and prevents its loading. This is undone
%    in next line for this test file.
%    \begin{macrocode}
\expandafter\let\csname ver@crop.sty\endcsname\relax
\usepackage[color=red,cross,a4,center]{crop}
\begin{document}
\shipout\null
\shipout\box\csname voidb@x\endcsname
\section{Hello World}
\end{document}
%</test3>
%    \end{macrocode}
%
% \section{Installation}
%
% \subsection{Download}
%
% \paragraph{Package.} This package is available on
% CTAN\footnote{\url{http://ctan.org/pkg/atbegshi}}:
% \begin{description}
% \item[\CTAN{macros/latex/contrib/oberdiek/atbegshi.dtx}] The source file.
% \item[\CTAN{macros/latex/contrib/oberdiek/atbegshi.pdf}] Documentation.
% \end{description}
%
%
% \paragraph{Bundle.} All the packages of the bundle `oberdiek'
% are also available in a TDS compliant ZIP archive. There
% the packages are already unpacked and the documentation files
% are generated. The files and directories obey the TDS standard.
% \begin{description}
% \item[\CTAN{install/macros/latex/contrib/oberdiek.tds.zip}]
% \end{description}
% \emph{TDS} refers to the standard ``A Directory Structure
% for \TeX\ Files'' (\CTAN{tds/tds.pdf}). Directories
% with \xfile{texmf} in their name are usually organized this way.
%
% \subsection{Bundle installation}
%
% \paragraph{Unpacking.} Unpack the \xfile{oberdiek.tds.zip} in the
% TDS tree (also known as \xfile{texmf} tree) of your choice.
% Example (linux):
% \begin{quote}
%   |unzip oberdiek.tds.zip -d ~/texmf|
% \end{quote}
%
% \paragraph{Script installation.}
% Check the directory \xfile{TDS:scripts/oberdiek/} for
% scripts that need further installation steps.
% Package \xpackage{attachfile2} comes with the Perl script
% \xfile{pdfatfi.pl} that should be installed in such a way
% that it can be called as \texttt{pdfatfi}.
% Example (linux):
% \begin{quote}
%   |chmod +x scripts/oberdiek/pdfatfi.pl|\\
%   |cp scripts/oberdiek/pdfatfi.pl /usr/local/bin/|
% \end{quote}
%
% \subsection{Package installation}
%
% \paragraph{Unpacking.} The \xfile{.dtx} file is a self-extracting
% \docstrip\ archive. The files are extracted by running the
% \xfile{.dtx} through \plainTeX:
% \begin{quote}
%   \verb|tex atbegshi.dtx|
% \end{quote}
%
% \paragraph{TDS.} Now the different files must be moved into
% the different directories in your installation TDS tree
% (also known as \xfile{texmf} tree):
% \begin{quote}
% \def\t{^^A
% \begin{tabular}{@{}>{\ttfamily}l@{ $\rightarrow$ }>{\ttfamily}l@{}}
%   atbegshi.sty & tex/generic/oberdiek/atbegshi.sty\\
%   atbegshi.pdf & doc/latex/oberdiek/atbegshi.pdf\\
%   atbegshi-example1.tex & doc/latex/oberdiek/atbegshi-example1.tex\\
%   atbegshi-example2.tex & doc/latex/oberdiek/atbegshi-example2.tex\\
%   test/atbegshi-test1.tex & doc/latex/oberdiek/test/atbegshi-test1.tex\\
%   test/atbegshi-test2.tex & doc/latex/oberdiek/test/atbegshi-test2.tex\\
%   test/atbegshi-test3.tex & doc/latex/oberdiek/test/atbegshi-test3.tex\\
%   atbegshi.dtx & source/latex/oberdiek/atbegshi.dtx\\
% \end{tabular}^^A
% }^^A
% \sbox0{\t}^^A
% \ifdim\wd0>\linewidth
%   \begingroup
%     \advance\linewidth by\leftmargin
%     \advance\linewidth by\rightmargin
%   \edef\x{\endgroup
%     \def\noexpand\lw{\the\linewidth}^^A
%   }\x
%   \def\lwbox{^^A
%     \leavevmode
%     \hbox to \linewidth{^^A
%       \kern-\leftmargin\relax
%       \hss
%       \usebox0
%       \hss
%       \kern-\rightmargin\relax
%     }^^A
%   }^^A
%   \ifdim\wd0>\lw
%     \sbox0{\small\t}^^A
%     \ifdim\wd0>\linewidth
%       \ifdim\wd0>\lw
%         \sbox0{\footnotesize\t}^^A
%         \ifdim\wd0>\linewidth
%           \ifdim\wd0>\lw
%             \sbox0{\scriptsize\t}^^A
%             \ifdim\wd0>\linewidth
%               \ifdim\wd0>\lw
%                 \sbox0{\tiny\t}^^A
%                 \ifdim\wd0>\linewidth
%                   \lwbox
%                 \else
%                   \usebox0
%                 \fi
%               \else
%                 \lwbox
%               \fi
%             \else
%               \usebox0
%             \fi
%           \else
%             \lwbox
%           \fi
%         \else
%           \usebox0
%         \fi
%       \else
%         \lwbox
%       \fi
%     \else
%       \usebox0
%     \fi
%   \else
%     \lwbox
%   \fi
% \else
%   \usebox0
% \fi
% \end{quote}
% If you have a \xfile{docstrip.cfg} that configures and enables \docstrip's
% TDS installing feature, then some files can already be in the right
% place, see the documentation of \docstrip.
%
% \subsection{Refresh file name databases}
%
% If your \TeX~distribution
% (\teTeX, \mikTeX, \dots) relies on file name databases, you must refresh
% these. For example, \teTeX\ users run \verb|texhash| or
% \verb|mktexlsr|.
%
% \subsection{Some details for the interested}
%
% \paragraph{Attached source.}
%
% The PDF documentation on CTAN also includes the
% \xfile{.dtx} source file. It can be extracted by
% AcrobatReader 6 or higher. Another option is \textsf{pdftk},
% e.g. unpack the file into the current directory:
% \begin{quote}
%   \verb|pdftk atbegshi.pdf unpack_files output .|
% \end{quote}
%
% \paragraph{Unpacking with \LaTeX.}
% The \xfile{.dtx} chooses its action depending on the format:
% \begin{description}
% \item[\plainTeX:] Run \docstrip\ and extract the files.
% \item[\LaTeX:] Generate the documentation.
% \end{description}
% If you insist on using \LaTeX\ for \docstrip\ (really,
% \docstrip\ does not need \LaTeX), then inform the autodetect routine
% about your intention:
% \begin{quote}
%   \verb|latex \let\install=y\input{atbegshi.dtx}|
% \end{quote}
% Do not forget to quote the argument according to the demands
% of your shell.
%
% \paragraph{Generating the documentation.}
% You can use both the \xfile{.dtx} or the \xfile{.drv} to generate
% the documentation. The process can be configured by the
% configuration file \xfile{ltxdoc.cfg}. For instance, put this
% line into this file, if you want to have A4 as paper format:
% \begin{quote}
%   \verb|\PassOptionsToClass{a4paper}{article}|
% \end{quote}
% An example follows how to generate the
% documentation with pdf\LaTeX:
% \begin{quote}
%\begin{verbatim}
%pdflatex atbegshi.dtx
%makeindex -s gind.ist atbegshi.idx
%pdflatex atbegshi.dtx
%makeindex -s gind.ist atbegshi.idx
%pdflatex atbegshi.dtx
%\end{verbatim}
% \end{quote}
%
% \section{Catalogue}
%
% The following XML file can be used as source for the
% \href{http://mirror.ctan.org/help/Catalogue/catalogue.html}{\TeX\ Catalogue}.
% The elements \texttt{caption} and \texttt{description} are imported
% from the original XML file from the Catalogue.
% The name of the XML file in the Catalogue is \xfile{atbegshi.xml}.
%    \begin{macrocode}
%<*catalogue>
<?xml version='1.0' encoding='us-ascii'?>
<!DOCTYPE entry SYSTEM 'catalogue.dtd'>
<entry datestamp='$Date$' modifier='$Author$' id='atbegshi'>
  <name>atbegshi</name>
  <caption>Execute stuff at \shipout time.</caption>
  <authorref id='auth:oberdiek'/>
  <copyright owner='Heiko Oberdiek' year='2007-2011'/>
  <license type='lppl1.3'/>
  <version number='1.18'/>
  <description>
    This package is a modern reimplementation of package
    <xref refid='everyshi'>everyshi</xref>, providing various commands
    to be executed before a <tt>\shipout</tt> command.  It makes use of
    e-TeX&#x2019;s facilities if they are available.  The package may
    be used either with LaTeX or with plain TeX.
    <p/>
    The package is part of the <xref refid='oberdiek'>oberdiek</xref> bundle.
  </description>
  <documentation details='Package documentation'
      href='ctan:/macros/latex/contrib/oberdiek/atbegshi.pdf'/>
  <ctan file='true' path='/macros/latex/contrib/oberdiek/atbegshi.dtx'/>
  <miktex location='oberdiek'/>
  <texlive location='oberdiek'/>
  <install path='/macros/latex/contrib/oberdiek/oberdiek.tds.zip'/>
</entry>
%</catalogue>
%    \end{macrocode}
%
% \begin{History}
%   \begin{Version}{2007/04/17 v1.0}
%   \item
%     First version.
%   \end{Version}
%   \begin{Version}{2007/04/18 v1.1}
%   \item
%     New method based on \cs{lastkern} is used if \eTeX\ is missing.
%   \item
%     \cs{AtBeginShipoutDiscard} also resets \cs{deadcycles}.
%   \end{Version}
%   \begin{Version}{2007/04/19 v1.2}
%   \item
%     \cs{AtBeginShipoutEarly} removed for simplification reasons.
%   \item
%     Forgotten definition of \cs{AtBegShi@Info} added.
%   \item
%     Patches for packages \xpackage{crop} and \xpackage{everyshi}
%     and class \xclass{memoir} added.
%   \end{Version}
%   \begin{Version}{2007/04/26 v1.3}
%   \item
%     Use of package \xpackage{infwarerr}.
%   \item
%     Catcode section after generic header.
%   \end{Version}
%   \begin{Version}{2007/04/27 v1.4}
%   \item
%     Small optimizations.
%   \end{Version}
%   \begin{Version}{2007/06/06 v1.5}
%   \item
%     \cs{AtBeginShipoutUpperLeft} added.
%   \item
%     Example added.
%   \item
%     Fix in second test file for newer version of \xclass{memoir}.
%   \end{Version}
%   \begin{Version}{2007/09/09 v1.6}
%   \item
%     Catcode section rewritten.
%   \end{Version}
%   \begin{Version}{2008/07/18 v1.7}
%   \item
%     Documentation of \cs{AtBeginShipoutUpperLeft} fixed and extended.
%   \end{Version}
%   \begin{Version}{2008/07/19 v1.8}
%   \item
%     \cs{AtBeginShipoutUpperLeftForeground} added.
%   \end{Version}
%   \begin{Version}{2008/07/31 v1.9}
%   \item
%     Second example (TrimBox for dvipdfmx) added.
%   \item
%     No changes in package code.
%   \end{Version}
%   \begin{Version}{2009/12/02 v1.10}
%   \item
%     \cs{AtBeginShipoutOriginalShipout} added.
%   \item
%     Test file fixed.
%   \end{Version}
%   \begin{Version}{2010/03/01 v1.11}
%   \item
%     Compatibility with \iniTeX\ except for \cs{newbox}.
%   \end{Version}
%   \begin{Version}{2010/03/25 v1.12}
%   \item
%     \cs{AtBeginShipoutNext} can now be used inside
%     \cs{AtBeginShipoutNext}.
%   \end{Version}
%   \begin{Version}{2010/08/18 v1.13}
%   \item
%     Fixes for \cs{AtBegShi@CheckDefinable}.
%   \end{Version}
%   \begin{Version}{2010/12/02 v1.14}
%   \item
%     Remove the warning because of void box if the hook
%     calls \AtBeginShipoutDiscard.
%   \end{Version}
%   \begin{Version}{2011/01/30 v1.15}
%   \item
%     Already loaded package files are not input in \hologo{plainTeX}.
%   \end{Version}
%   \begin{Version}{2011/10/05 v1.16}
%   \item
%     \cs{AtBeginShipoutAddToBox}, \cs{AtBeginShipoutAddToBoxForeground} added.
%   \item
%     \cs{AtBeginShipoutBoxWidth}, \cs{AtBeginShipoutBoxHeight},
%     \cs{AtBeginShipoutBoxDepth} added.
%   \item
%     Updates for patches of class \xclass{memoir}.
%   \end{Version}
%   \begin{Version}{2016/05/16 v1.17}
%   \item
%     Documentation updates.
%   \end{Version}
%   \begin{Version}{2016/06/09 v1.18}
%   \item
%     Update for \cs{pdfhorign} in new LuaTeX.
%   \end{Version}
% \end{History}
%
% \PrintIndex
%
% \Finale
\endinput

%        (quote the arguments according to the demands of your shell)
%
% Documentation:
%    (a) If atbegshi.drv is present:
%           latex atbegshi.drv
%    (b) Without atbegshi.drv:
%           latex atbegshi.dtx; ...
%    The class ltxdoc loads the configuration file ltxdoc.cfg
%    if available. Here you can specify further options, e.g.
%    use A4 as paper format:
%       \PassOptionsToClass{a4paper}{article}
%
%    Programm calls to get the documentation (example):
%       pdflatex atbegshi.dtx
%       makeindex -s gind.ist atbegshi.idx
%       pdflatex atbegshi.dtx
%       makeindex -s gind.ist atbegshi.idx
%       pdflatex atbegshi.dtx
%
% Installation:
%    TDS:tex/generic/oberdiek/atbegshi.sty
%    TDS:doc/latex/oberdiek/atbegshi.pdf
%    TDS:doc/latex/oberdiek/atbegshi-example1.tex
%    TDS:doc/latex/oberdiek/atbegshi-example2.tex
%    TDS:doc/latex/oberdiek/test/atbegshi-test1.tex
%    TDS:doc/latex/oberdiek/test/atbegshi-test2.tex
%    TDS:doc/latex/oberdiek/test/atbegshi-test3.tex
%    TDS:source/latex/oberdiek/atbegshi.dtx
%
%<*ignore>
\begingroup
  \catcode123=1 %
  \catcode125=2 %
  \def\x{LaTeX2e}%
\expandafter\endgroup
\ifcase 0\ifx\install y1\fi\expandafter
         \ifx\csname processbatchFile\endcsname\relax\else1\fi
         \ifx\fmtname\x\else 1\fi\relax
\else\csname fi\endcsname
%</ignore>
%<*install>
\input docstrip.tex
\Msg{************************************************************************}
\Msg{* Installation}
\Msg{* Package: atbegshi 2016/06/09 v1.18 At begin shipout hook (HO)}
\Msg{************************************************************************}

\keepsilent
\askforoverwritefalse

\let\MetaPrefix\relax
\preamble

This is a generated file.

Project: atbegshi
Version: 2016/06/09 v1.18

Copyright (C) 2007-2011 by
   Heiko Oberdiek <heiko.oberdiek at googlemail.com>

This work may be distributed and/or modified under the
conditions of the LaTeX Project Public License, either
version 1.3c of this license or (at your option) any later
version. This version of this license is in
   http://www.latex-project.org/lppl/lppl-1-3c.txt
and the latest version of this license is in
   http://www.latex-project.org/lppl.txt
and version 1.3 or later is part of all distributions of
LaTeX version 2005/12/01 or later.

This work has the LPPL maintenance status "maintained".

This Current Maintainer of this work is Heiko Oberdiek.

The Base Interpreter refers to any `TeX-Format',
because some files are installed in TDS:tex/generic//.

This work consists of the main source file atbegshi.dtx
and the derived files
   atbegshi.sty, atbegshi.pdf, atbegshi.ins, atbegshi.drv,
   atbegshi-example1.tex, atbegshi-example2.tex,
   atbegshi-test1.tex, atbegshi-test2.tex, atbegshi-test3.tex.

\endpreamble
\let\MetaPrefix\DoubleperCent

\generate{%
  \file{atbegshi.ins}{\from{atbegshi.dtx}{install}}%
  \file{atbegshi.drv}{\from{atbegshi.dtx}{driver}}%
  \usedir{tex/generic/oberdiek}%
  \file{atbegshi.sty}{\from{atbegshi.dtx}{package}}%
  \usedir{doc/latex/oberdiek}%
  \file{atbegshi-example1.tex}{\from{atbegshi.dtx}{example1}}%
  \file{atbegshi-example2.tex}{\from{atbegshi.dtx}{example2}}%
  \usedir{doc/latex/oberdiek/test}%
  \file{atbegshi-test1.tex}{\from{atbegshi.dtx}{test1}}%
  \file{atbegshi-test2.tex}{\from{atbegshi.dtx}{test2}}%
  \file{atbegshi-test3.tex}{\from{atbegshi.dtx}{test3}}%
  \nopreamble
  \nopostamble
  \usedir{source/latex/oberdiek/catalogue}%
  \file{atbegshi.xml}{\from{atbegshi.dtx}{catalogue}}%
}

\catcode32=13\relax% active space
\let =\space%
\Msg{************************************************************************}
\Msg{*}
\Msg{* To finish the installation you have to move the following}
\Msg{* file into a directory searched by TeX:}
\Msg{*}
\Msg{*     atbegshi.sty}
\Msg{*}
\Msg{* To produce the documentation run the file `atbegshi.drv'}
\Msg{* through LaTeX.}
\Msg{*}
\Msg{* Happy TeXing!}
\Msg{*}
\Msg{************************************************************************}

\endbatchfile
%</install>
%<*ignore>
\fi
%</ignore>
%<*driver>
\NeedsTeXFormat{LaTeX2e}
\ProvidesFile{atbegshi.drv}%
  [2016/06/09 v1.18 At begin shipout hook (HO)]%
\documentclass{ltxdoc}
\usepackage{holtxdoc}[2011/11/22]
\begin{document}
  \DocInput{atbegshi.dtx}%
\end{document}
%</driver>
% \fi
%
%
% \CharacterTable
%  {Upper-case    \A\B\C\D\E\F\G\H\I\J\K\L\M\N\O\P\Q\R\S\T\U\V\W\X\Y\Z
%   Lower-case    \a\b\c\d\e\f\g\h\i\j\k\l\m\n\o\p\q\r\s\t\u\v\w\x\y\z
%   Digits        \0\1\2\3\4\5\6\7\8\9
%   Exclamation   \!     Double quote  \"     Hash (number) \#
%   Dollar        \$     Percent       \%     Ampersand     \&
%   Acute accent  \'     Left paren    \(     Right paren   \)
%   Asterisk      \*     Plus          \+     Comma         \,
%   Minus         \-     Point         \.     Solidus       \/
%   Colon         \:     Semicolon     \;     Less than     \<
%   Equals        \=     Greater than  \>     Question mark \?
%   Commercial at \@     Left bracket  \[     Backslash     \\
%   Right bracket \]     Circumflex    \^     Underscore    \_
%   Grave accent  \`     Left brace    \{     Vertical bar  \|
%   Right brace   \}     Tilde         \~}
%
% \GetFileInfo{atbegshi.drv}
%
% \title{The \xpackage{atbegshi} package}
% \date{2016/06/09 v1.18}
% \author{Heiko Oberdiek\thanks
% {Please report any issues at https://github.com/ho-tex/oberdiek/issues}\\
% \xemail{heiko.oberdiek at googlemail.com}}
%
% \maketitle
%
% \begin{abstract}
% This package is a modern reimplementation of package \xpackage{everyshi}
% without the burden of compatibility. It makes use of \eTeX's if available.
% Both \LaTeX\ and \plainTeX\ are supported.
% \end{abstract}
%
% \tableofcontents
%
% \section{Documentation}
%
% Package \xpackage{atbegshi} redefines \cs{shipout} to insert
% hooks for user code that is executed before the page is shipped out.
% The code may modify or even discard the output page.
% Three hooks are implemented:
% \begin{enumerate}
% \item A hook that is executed for every page, see\\
%       \cs{AtBeginShipout}
% \item A hook that is executed for the next page only, see\\
%       \cs{AtBeginShipoutNext}
% \item A hook that is only executed for the first page, see\\
%       \cs{AtBeginShipoutFirst}
% \end{enumerate}
% The hooks are executed in this order. The following three macros
% provide the user interface for adding code to these hooks:
%
% \begin{declcs}{AtBeginShipout} \M{code}\\
%   \cs{AtBeginShipoutBox}
% \end{declcs}
% Execute the \meta{code} for every page.
% The page contents is held in box register \cs{AtBeginShipoutBox}
% and may be modified. Use \cs{AtBeginShipoutDiscard} if you want to
% discard the page.
%
% \emph{Note:} Package \xpackage{everyshi} uses box register 255. With
% package \xpackage{atbegshi} you must use \cs{AtBeginShipoutBox} instead.
%
% If \LaTeX\ calls \cs{shipout} in \cs{@outputpage} (part of its output
% routine),
% the meaning of \cs{protect} is \cs{noexpand}. \LaTeX\ sets
% \cs{protect} to the appropriate \cs{@typeset@protect} in the box
% that is shipped out. This is too late for the hooks, they
% are called earlier in the redefined \cs{shipout}. Therefore
% package \xpackage{atbegshi} sets \cs{protect} to \cs{@typeset@protect}
% before it calls the hooks.
% (In \cs{EveryShipout} of package \xpackage{everyshi} the
% user is responsible for the correct setting of \cs{protect}.)
%
% \begin{declcs}{AtBeginShipoutNext} \M{code}
% \end{declcs}
% This reimplements package \xpackage{everyshi}'s \cs{AtNextShipout}.
% The \meta{code} is executed at shipout time of the next page only.
% It is just a convenience macro, it can be easily replaced by something
% like:
%\begin{quote}
%\begin{verbatim}
%\newcommand{\MyShipoutHook}{}%
%\AtBeginShipout{\MyShipoutHook}
%\gdef\MyShipoutHook{%
%  ... do something with next page ...
%  \gdef\MyShipoutHook{}%
% }
%\end{verbatim}
%\end{quote}
% (This can be necessary, if hook order does matter).
%
% \begin{declcs}{AtBeginShipoutFirst} \M{code}
% \end{declcs}
% This reimplements \LaTeX's \cs{AtBeginDvi}. This hook is
% usually used for \cs{special} commands that include PostScript
% header files. The \cs{code} is directly executed in a \cs{vbox}
% that is put at the beginning of the output page. Dealing with
% the output box \cs{AtBeginShipoutBox} is not necessary and
% not permitted here.
%
% \begin{declcs}{AtBeginShipoutDiscard}
% \end{declcs}
% This macro notifies package \xpackage{atbegshi} that
% the output page is discarded. The remaining hook code
% and the remaining hooks are not executed and the page is thrown away.
% Also \cs{deadcycles} is cleared to zero like an ordinary \cs{shipout}
% would do.
%
% \begin{declcs}{AtBeginShipoutInit}
% \end{declcs}
% Usually the redefinition of \cs{shipout} is delayed by
% \cs{AtBeginDocument} (if this macro exists). This can be
% too late, if other packages also redefines \cs{shipout}
% and the order does matter.
% \cs{AtBeginShipoutInit} forces the immediate redefinition
% of \cs{shipout}.
%
% \begin{declcs}{AtBeginShipoutAddToBox} \M{stuff}\\
%   \cs{AtBeginShipoutAddToBoxForeground} \M{stuff}
% \end{declcs}
% A quite common use case is the addition of \cs{special}
% or other whatsits to the page output box. Macro
% \cs{AtBeginShipoutAddToBox} puts \meta{stuff} in a
% box with zeroed dimensions. The box with the \meta{stuff}
% is put in the upper left corner of the shipout box
% \cs{AtBeginShipoutBox}. Macro \cs{AtBeginShipoutAddToBox}
% puts the \meta{stuff} in the background, the other macro
% \cs{AtBeginShipoutAddToBoxForeground} in the foreground
% after the original shipout box contents is set.
%
% A void shipout box (that means a discarded page)
% remains void that means \meta{stuff} is ignored in this case.
% The box type of \cs{AtBeginShipoutBox} is preserved.
% Also the box nesting level for the original contents
% of \cs{AtBeginShipoutBox} remains, for example, to avoid trouble
% with links across pages in case of \hologo{pdfTeX}.
%
% \begin{declcs}{AtBeginShipoutUpperLeft} \M{background material}
% \end{declcs}
% This is a macro that puts material in the background of
% box \cs{AtBeginShipoutBox}.
% The \meta{background material} is set in an \cs{hbox}, the
% reference point is the upper left corner of the output page.
% In case of \pdfTeX\ in PDF mode, the settings of
% \cs{pdfhorigin} and \cs{pdfvorigin} are respected.
%
% The macro \cs{AtBeginShipoutUpperLeft} is intended to be used
% in one of the hook setting macros, such as
% \cs{AtBeginShipout}, \cs{AtBeginShipoutFirst}, or
% \cs{AtBeginShipoutNext}.
%
% For \LaTeX\ users the \meta{background material} is set
% inside a \texttt{picture} environment:
% \begin{quote}
%   |\begin{picture}(0,0)|\\
%   \mbox{}\quad |\setlength{\unitlength}{1pt}%|\\
%   \mbox{}\quad \meta{background material}\\
%   |\end{picture}|
% \end{quote}
%
% \begin{declcs}{AtBeginShipoutUpperLeftForeground} \M{foreground material}
% \end{declcs}
% See \cs{AtBeginShipoutUpperLeft}. The difference is that the material
% is put in the foreground.
%
% \begin{declcs}{AtBeginShipoutOriginalShipout} \meta{box}
% \end{declcs}
% It stores the meaning of \cs{shipout} at the time this package
% is loaded.
%
% \begin{declcs}{AtBeginShipoutBoxWidth}\\
%   \cs{AtBeginShipoutBoxHeight}\\
%   \cs{AtBeginShipoutBoxDepth}\\
% \end{declcs}
% These macros store the dimensions of the output box \cs{AtBeginShipoutBox}
% before the original shipout is called. If \cs{shipout} is not
% redefined before the package loading or the box dimensions
% are not changed by the redefined \cs{shipout}, these macros
% contain the dimensions of the shipout box. These values can be
% remembered by \cs{label} and \cs{ref}. For example, this is done
% by the package module \xpackage{zref-pagelayout} of project \xpackage{zref}.
% The dimensions of the shipout page can be used in some \hologo{TeX}
% engines (\hologo{pdfTeX} in PDF mode, \hologo{XeTeX}) to calculate
% the media size of the shipout page if \cs{pdfpagewidth} and
% \cs{pdfpageheight} are not set.
%
% \subsection{Examples}
%
% \subsubsection{Example: circle in background}
%
% In this example we put a circle in the background in the middle
% of the paper.
%
%    \begin{macrocode}
%<*example1>
%    \end{macrocode}
%    \begin{macrocode}
\documentclass[a4paper]{article}
\usepackage{color}
\usepackage{atbegshi}
%    \end{macrocode}
%    Package \xpackage{picture} makes life a little easier, because
%    we can now also use length specifications in |picture|'s commands.
%    \begin{macrocode}
\usepackage{picture}
%    \end{macrocode}
%    Now we draw the circle in the middle of the paper.
%    \cs{put} moves downwards, because the origin is at the top
%    of the page, not at its bottom.
%    \begin{macrocode}
\AtBeginShipout{%
  \AtBeginShipoutUpperLeft{%
    \put(0.5\paperwidth,-0.5\paperheight){\circle{10}}%
  }%
}
\begin{document}
\section{Hello World}
\newpage
\AtBeginShipoutNext{%
  \AtBeginShipoutUpperLeft{%
    \color{red}%
    \put(0,-0.5\paperheight){\line(1,0){\paperwidth}}%
    \put(0.5\paperwidth, 0){\line(0,-1){\paperheight}}%
  }%
}
Only on this page we add a red cross.
\newpage
This page has the circle only.
\par
\vspace{\fill}
The next page will be discarded.
\newpage
\AtBeginShipoutNext{%
  \AtBeginShipoutDiscard
}
This page is discarded.
\newpage
The last page.
\end{document}
%    \end{macrocode}
%    \begin{macrocode}
%</example1>
%    \end{macrocode}
%
% \subsubsection{Example: adding TrimBox for dvipdfmx}
%
% Now an example from ``real life'' follows. Someone from the mailing  list
% for dvipdfmx wants to put a TrimBox on every page. If we use
% \cs{AtBeginShipout}, we have to put the \cs{special} inside the
% box \cs{AtBeginShipoutBox} that gets shipped out.
%    \begin{macrocode}
%<*example2>
\documentclass{minimal}
\usepackage{atbegshi}
\usepackage[
  dvipdfm,
  paperwidth=630bp,
  paperheight=810bp
]{geometry}
\AtBeginShipout{%
  \setbox\AtBeginShipoutBox=\hbox{%
    \special{pdf: put @thispage <</TrimBox[9 9 621 801]>>}%
    \box\AtBeginShipoutBox
  }%
}
\begin{document}
  First page
  \newpage
  Second page
\end{document}
%</example2>
%    \end{macrocode}
% Remember, in \cs{AtBeginShipoutBoxFirst} the \cs{setbox} wrapper code
% is implicitly given and the \cs{special} is used directly.
%
% \section{Method of \cs{shipout} overloading}
%
% \subsection{\cs{shipout}}
%
% The \TeX\ primitive command \cs{shipout} takes a box specification
% and puts the box as a new page in the output file.
% There are two kinds of box specifications:
% \begin{description}
% \item[Direct boxes:]
%   They are given by \cs{hbox}, \cs{vbox}, or \cs{vtop},\\
%   e.g. |\shipout\hbox{Hello World}|.
% \item[Indirect boxes:]
%   \cs{box} or \cs{copy} references a box register by number.
%   The box register contains the contents of the box.
%
%   \emph{Note:} \cs{box} also clears the box register globally.
% \end{description}
% Then we have to differentiate between void and empty boxes:
% \begin{description}
% \item[Void:]
%   Initially or after \cs{box} there is no box in the box register.
%   In this cases the box register is not empty, but \emph{void}.
% \item[Empty:]
%   A box with empty contents, such as
%   |\hbox{}| (= \cs{null}) or |\vbox{}| is an \emph{empty \cs{hbox}}
%   or \emph{empty \cs{vbox}}. If a box register holds such a
%   box, the box still exists, therefore the box register is
%   \emph{not void}.
% \end{description}
%
% \subsection{\cs{afterassignment}}
%
% We want to overload \cs{shipout} to do something with the box.
% It is quite impossible to do this reliable by catching the box
% using macro arguments. The
% variety of box specifications is too large, Examples:
% \begin{quote}
%   |\shipout\null|\\
%   |\shipout\vbox{...}|\\
%   |\shipout\vtop\bgroup ...\egroup|\\
%   |\shipout\box255|
% \end{quote}
% Even worse, the braces don't need to be balanced:
% \begin{quote}
%   |\shipout\hbox\bgroup}|\\
%   |\shipout\vbox{\egroup|
% \end{quote}
% Happily TeX provides a reliable way via \cs{afterassignment}.
% It takes a macro name and executes it just after the assignment.
%
% Now we can redefine \cs{shipout}. The box specification that
% follows \cs{shipout} is catched by \cs{setbox}. This is an
% assignment to a box register. \cs{afterassignment}
% notifies \TeX, that we want to call \cs{@test} right after
% the assignment:
%\begin{quote}
%\begin{verbatim}
%\shipout :=
%  \afterassignment\@test
%  \setbox\mybox=
%\end{verbatim}
%\end{quote}
% We have seen different box specifications. Indirect boxes are easy
% to understand:
% \begin{quote}
%   |\shipout\box0| $\Rightarrow$ |\setbox\mybox=\box0 |\fbox{\cs{@test}}
% \end{quote}
% However direct boxes can have arbitrary contents with lots of
% other assignments. It would be quite inpredictable if \TeX\ would
% put |\@test| after the first of such an assignment or after the
% box specification if the box lacks of assignments.
% Therefore \TeX\ puts |\@test| right at the beginning of the
% box specification, e.g:
% \begin{quote}
%   |\shipout\hbox{Hello World}|\\
%   $\Rightarrow$
%   |\setbox\mybox=\hbox{|\fbox{\cs{@test}}| Hello World}|
% \end{quote}
%
% \subsection{Test for direct or indirect boxes}
%
% Now we want to execute \cs{@test}, but where are we?
% We can be after the completed box assignment, if \cs{shipout}
% was called with an indirect box. Or we are right at the beginning
% of a direct box.
%
% \subsubsection{With \eTeX}
%
% With the \eTeX's extensions the answer is very easy:
% Being inside the direct box means that we are inside a new group.
% The new primitive command \cs{currentgrouplevel} tells how
% deeply the groups are currently nested. Macro \cs{@test} just
% compares the previously stored group level with the current one:
%\begin{quote}
%\begin{verbatim}
%\shipout :=
%  \edef\saved@grouplevel{\number\currentgrouplevel}
%  \afterassignment\@test
%  \setbox\mybox=
%
%\@test :=
%  \ifnum\saved@grouplevel=\currentgrouplevel
%    % case: indirect box, the assignment is completed
%    \@output
%  \else
%    % case: direct box, we are inside the box
%    \aftergroup\@outbox
%  \fi
%\end{verbatim}
%\end{quote}
%
% \subsubsection{Without \eTeX}
%
% Life becomes complicate without \eTeX. We cannot ask the
% group level. However, if we are inside a direct box,
% the box register \cs{mybox} is not yet changed by \cs{setbox}.
% Thus we need a special initial value and compare
% it in \cs{@test} with the current value of the box.
%
% What can be used as initial value? Arbitrary box contents cannot
% be compared. \TeX\ only tells us a few properties:
% \begin{itemize}
% \item Box type:\cs{ifhbox}, \cs{ifvbox}
% \item Dimensions: \cs{wd}, \cs{ht}, \cs{dp}
% \item Voidness: \cs{ifvoid}
% \end{itemize}
% Unhappily all these qualities even combined are not sufficient
% for constructing an initial box value, because \cs{shipout} can be
% called with a box that is accidently just the same as the choosen
% initial value.
%
% Nevertheless we have two alternatives for an initial value:
% \begin{itemize}
% \item A box of some type with some funny settings that are
% unlikely to occur in real life, e.g a height of |4911sp-\maxdimen|.
% \item A void box.
% \end{itemize}
% A collision between this initial value and an indirect \cs{shipout}
% box with just the same value is possible.
% Then \cs{@test} will make a wrong decision that it is executed
% inside a direct box and delays \cs{@output} by \cs{aftergroup}.
% Thus \cs{@output} is not called at the place we want.
% In contrary, the result is an uncertainty about the place:
% \begin{itemize}
% \item \cs{shipout} is used in a group that perhaps closes some
%   pages later. A bad place for \cs{@output}.
% \item Without a surrounding group \cs{aftergroup} effectively kills
%   its argument.
% \end{itemize}
% In the first case of a box with special dimensions we can even loose
% the page. However in the case of the void box, this effect is even
% desired, because the original \cs{shipout} does not output void boxes.
% All we have to do is to ensure that our box \cs{mybox} is always
% void except for the phase when the overloaded \cs{shipout} is executed.
% And secondly we must keep this semantics of \cs{shipout} for the void case
% in our macros, namely \cs{@output}.
%\begin{quote}
%\begin{verbatim}
%\shipout :=
%  % trick to get a void box \mybox
%  \begingroup
%    \setbox\mybox=\box\mybox
%  \endgroup
%  \afterassignment\@test
%  \setbox\mybox=
%
%\@test :=
%  \ifvoid\mybox
%    \aftergroup\@output
%  \else
%    \@output
%  \fi
%\end{verbatim}
%\end{quote}
% The nasty case is |\shipout\box\voidb@x| where the indirect box
% is void and that must not generate an output page.
% If a surrounding group is missing the output is ignored
% because of \cs{aftergroup}.
% Otherwise \@output is called some time
% later when the surrounding group closes.
% But \cs{mybox} is void outside the execution phase
% of the redefined \cs{shipout}. Also \cs{@output} checks for a
% void box and cancels the page output.
% The disadvantage remains that the hook in \cs{@output} is called
% for a page that will not be output.
%
% \subsubsection{\cs{lastkern} method}
%
% At the beginning of a new box, there is no \cs{kern}, the contents
% of the box is still empty and \cs{lastkern} returns 0\,pt.
% This can be used to distinguish between direct and indirect boxes:
% We execute \cs{setbox} in a box with a preceding non-zero kern.
% After an indirect box, \cs{lastkern} sees this kern, otherwise
% it returns 0\,pt.
%\begin{quote}
%\begin{verbatim}
%\shipout :=
%  \begingroup
%    \setbox\mybox=\hbox\bgroup
%      \kern1pt
%      \afterassignment\shipout@test
%      \global\setbox\mybox=
%\@test :=
%  \ifdim\lastkern=0pt
%    % direct box
%    \aftergroup\egroup
%    \aftergroup\endgroup
%    \aftergroup\@output
%  \else
%    \egroup
%    \endgroup
%    \@output
%  \fi
%\end{verbatim}
%\end{quote}
% We have two \cs{setbox} commands. The first creates a controlled
% context box where we can safely insert a \cs{kern}. We get rid of
% this temporarily used context box by putting the local \cs{setbox}
% in a group.
%
% After the group we want to have our shipout box in \cs{mybox}.
% Therefore we use a global assignment here.
%
% \subsection{Output}
%
% With or without \eTeX\ we ensure the original behaviour of
% \cs{shipout} that void boxes do not generate output pages.
%
% Now we can place the hook \cs{@hook} for the user code that
% wants to manipulate the output box.
%\begin{quote}
%\begin{verbatim}
%\@output :=
%  \ifvoid\mybox
%    % cancel output of void box
%  \else
%    \@hook
%    \ifvoid\mybox
%      % user code in \@hook could has voided the box
%    \else
%      \original@shipout\box\mybox
%    \fi
%  \fi
%\end{verbatim}
%\end{quote}
%
% \subsection{Separate box register}
%
% So far we have said nothing about the box number of \cs{mybox}.
% The following case that outputs the same page twice
% shows that we are not free in the use
% of the box register:
% \begin{quote}
%   |\shipout\copy<num> \shipout\box<num>|
% \end{quote}
% We manipulate the box by the hook and without \eTeX\ the box
% must even be voided. However, the use case above requires that
% the box contents does not change at all. Therefore we must
% reserve a separate box register to avoid collisions with
% user box registers.
%
% \emph{Note:} Box register number 255 is special for the output
% routine, because TeX complains if this box is not voided
% by the output routine.
% However, this requirement does not apply to \cs{shipout} at all.
% In fact \cs{shipout} does not change any box register. This is
% usually done by a call of \cs{box}, but the output routine
% can do it later \emph{after} invoking of \cs{shipout}.
%
% \subsection{Summary}
%
% \subsubsection{With \eTeX}
%
% Putting the pieces together we get for \eTeX:
%\begin{quote}
%\begin{verbatim}
%\newbox\mybox
%\let\original@shipout\shipout
%
%\shipout :=
%  \edef\saved@grouplevel{\number\currentgrouplevel}
%  \afterassignment\@test
%  \setbox\mybox=
%
%\@test :=
%  \ifnum\saved@grouplevel<\currentgrouplevel
%    \expandafter\aftergroup
%  \fi
%  \@output
%
%\@output :=
%  \ifvoid\mybox
%    % cancel output of void box
%  \else
%    \@hook
%    \ifvoid\mybox
%      % user code in \@hook could have voided the box
%    \else
%      \original@shipout\box\mybox
%    \fi
%  \fi
%\end{verbatim}
%\end{quote}
%
% \subsubsection{Without \eTeX, traditional way}
%
% And for \TeX\ without \eTeX:
%\begin{quote}
%\begin{verbatim}
%\newbox\mybox
%\begingroup
%   \setbox\mybox=\box\mybox % ensure \mybox is void
%\endgroup
%\let\original@shipout\shipout
%
%\shipout :=
%  % trick to get a void box \mybox
%  \begingroup
%    \setbox\mybox=\box\mybox
%  \endgroup
%  \afterassignment\@test
%  \setbox\mybox=
%
%\@test :=
%  \ifvoid\mybox
%    \expandafter\aftergroup
%  \fi
%  \@output
%
%\@output :=
%  \ifvoid\mybox
%    % cancel output of void box
%  \else
%    \@hook
%    \ifvoid\mybox
%      % user code in \@hook could have voided the box
%    \else
%      \original@shipout\box\mybox
%    \fi
%  \fi
%\end{verbatim}
%\end{quote}
%
% \subsubsection{\cs{lastkern} method}
%
% And for \TeX\ without \eTeX\ using the \cs{lastkern} method:
%\begin{quote}
%\begin{verbatim}
%\newbox\mybox
%\let\original@shipout\shipout
%
%\shipout :=
%  \begingroup
%  \setbox\mybox=\hbox\bgroup
%  \kern1pt
%  \afterassignment\@test
%  \setbox\mybox=
%
%\@test :=
%  \ifdim\lastkern=0pt
%    \expandafter\aftergroup
%  \fi
%  \@output
%
%\@output :=
%  \egroup
%  \endgroup
%  \ifvoid\mybox
%    % cancel output of void box
%  \else
%    \@hook
%    \ifvoid\mybox
%      % user code in \@hook could have voided the box
%    \else
%      \original@shipout\box\mybox
%    \fi
%  \fi
%\end{verbatim}
%\end{quote}
%
%
% \StopEventually{
% }
%
% \section{Implementation}
%
% Package \xpackage{atbegshi} uses \eTeX's \cs{currentgrouplevel},
% if it is available. Otherwise the \cs{lastkern} method is used.
%
%    \begin{macrocode}
%<*package>
%    \end{macrocode}
%
% \subsection{Reload check and package identification}
%    Reload check, especially if the package is not used with \LaTeX.
%    \begin{macrocode}
\begingroup\catcode61\catcode48\catcode32=10\relax%
  \catcode13=5 % ^^M
  \endlinechar=13 %
  \catcode35=6 % #
  \catcode39=12 % '
  \catcode44=12 % ,
  \catcode45=12 % -
  \catcode46=12 % .
  \catcode58=12 % :
  \catcode64=11 % @
  \catcode123=1 % {
  \catcode125=2 % }
  \expandafter\let\expandafter\x\csname ver@atbegshi.sty\endcsname
  \ifx\x\relax % plain-TeX, first loading
  \else
    \def\empty{}%
    \ifx\x\empty % LaTeX, first loading,
      % variable is initialized, but \ProvidesPackage not yet seen
    \else
      \expandafter\ifx\csname PackageInfo\endcsname\relax
        \def\x#1#2{%
          \immediate\write-1{Package #1 Info: #2.}%
        }%
      \else
        \def\x#1#2{\PackageInfo{#1}{#2, stopped}}%
      \fi
      \x{atbegshi}{The package is already loaded}%
      \aftergroup\endinput
    \fi
  \fi
\endgroup%
%    \end{macrocode}
%    Package identification:
%    \begin{macrocode}
\begingroup\catcode61\catcode48\catcode32=10\relax%
  \catcode13=5 % ^^M
  \endlinechar=13 %
  \catcode35=6 % #
  \catcode39=12 % '
  \catcode40=12 % (
  \catcode41=12 % )
  \catcode44=12 % ,
  \catcode45=12 % -
  \catcode46=12 % .
  \catcode47=12 % /
  \catcode58=12 % :
  \catcode64=11 % @
  \catcode91=12 % [
  \catcode93=12 % ]
  \catcode123=1 % {
  \catcode125=2 % }
  \expandafter\ifx\csname ProvidesPackage\endcsname\relax
    \def\x#1#2#3[#4]{\endgroup
      \immediate\write-1{Package: #3 #4}%
      \xdef#1{#4}%
    }%
  \else
    \def\x#1#2[#3]{\endgroup
      #2[{#3}]%
      \ifx#1\@undefined
        \xdef#1{#3}%
      \fi
      \ifx#1\relax
        \xdef#1{#3}%
      \fi
    }%
  \fi
\expandafter\x\csname ver@atbegshi.sty\endcsname
\ProvidesPackage{atbegshi}%
  [2016/06/09 v1.18 At begin shipout hook (HO)]%
%    \end{macrocode}
%
% \subsection{Catcodes}
%
%    \begin{macrocode}
\begingroup\catcode61\catcode48\catcode32=10\relax%
  \catcode13=5 % ^^M
  \endlinechar=13 %
  \catcode123=1 % {
  \catcode125=2 % }
  \catcode64=11 % @
  \def\x{\endgroup
    \expandafter\edef\csname AtBegShi@AtEnd\endcsname{%
      \endlinechar=\the\endlinechar\relax
      \catcode13=\the\catcode13\relax
      \catcode32=\the\catcode32\relax
      \catcode35=\the\catcode35\relax
      \catcode61=\the\catcode61\relax
      \catcode64=\the\catcode64\relax
      \catcode123=\the\catcode123\relax
      \catcode125=\the\catcode125\relax
    }%
  }%
\x\catcode61\catcode48\catcode32=10\relax%
\catcode13=5 % ^^M
\endlinechar=13 %
\catcode35=6 % #
\catcode64=11 % @
\catcode123=1 % {
\catcode125=2 % }
\def\TMP@EnsureCode#1#2{%
  \edef\AtBegShi@AtEnd{%
    \AtBegShi@AtEnd
    \catcode#1=\the\catcode#1\relax
  }%
  \catcode#1=#2\relax
}
\TMP@EnsureCode{40}{12}% (
\TMP@EnsureCode{41}{12}% )
\TMP@EnsureCode{44}{12}% ,
\TMP@EnsureCode{45}{12}% -
\TMP@EnsureCode{47}{12}% /
\TMP@EnsureCode{46}{12}% .
\TMP@EnsureCode{58}{12}% :
\TMP@EnsureCode{91}{12}% [
\TMP@EnsureCode{93}{12}% ]
\TMP@EnsureCode{94}{7}% ^(superscript)
\TMP@EnsureCode{96}{12}% `
\edef\AtBegShi@AtEnd{\AtBegShi@AtEnd\noexpand\endinput}
%    \end{macrocode}
%
% \subsection{Preparations}
%
%    \begin{macrocode}
\begingroup\expandafter\expandafter\expandafter\endgroup
\expandafter\ifx\csname RequirePackage\endcsname\relax
  \def\TMP@RequirePackage#1[#2]{%
    \begingroup\expandafter\expandafter\expandafter\endgroup
    \expandafter\ifx\csname ver@#1.sty\endcsname\relax
      \input #1.sty\relax
    \fi
  }%
  \TMP@RequirePackage{infwarerr}[2007/09/09]%
  \TMP@RequirePackage{ltxcmds}[2010/03/01]%
\else
  \RequirePackage{infwarerr}[2007/09/09]%
  \RequirePackage{ltxcmds}[2010/03/01]%
\fi
%    \end{macrocode}
%
%    \begin{macro}{\AtBegShi@CheckDefinable}
%    \begin{macrocode}
\begingroup\expandafter\expandafter\expandafter\endgroup
\expandafter\ifx\csname @ifdefinable\endcsname\relax
  \def\AtBegShi@CheckDefinable#1{%
    \ifcase\ifx#1\relax
             \ltx@one
           \else
             \ifx#1\@undefined
               \ltx@one
             \else
               \ltx@zero
             \fi
           \fi
      \@PackageError{atbegshi}{%
        \string#1\space is already defined%
      }\@ehd
    \fi
  }%
\else
  \def\AtBegShi@CheckDefinable#1{%
    \@ifdefinable{#1}{}%
  }%
\fi
%    \end{macrocode}
%    \end{macro}
%
%    \begin{macro}{\ifAtBegShi@Discarded}
%    \begin{macrocode}
\ltx@newif\ifAtBegShi@Discarded
%    \end{macrocode}
%    \end{macro}
%    \begin{macro}{\AtBeginShipoutDiscard}
%    \begin{macrocode}
\AtBegShi@CheckDefinable\AtBeginShipoutDiscard
\def\AtBeginShipoutDiscard{%
  \deadcycles=\ltx@zero
  \global\AtBegShi@Discardedtrue
}
%    \end{macrocode}
%    \end{macro}
%
%    \begin{macrocode}
\begingroup\expandafter\expandafter\expandafter\endgroup
\expandafter\ifx\csname currentgrouplevel\endcsname\relax
  \catcode`X=9  % ignore
  \catcode`E=14 % comment
\else
  \catcode`X=14 % comment
  \catcode`E=9  % ignore
\fi
%    \end{macrocode}
%
%    \begin{macro}{\AtBegShi@Shipout}
%    \begin{macrocode}
\def\AtBegShi@Shipout{%
X \begingroup
X \setbox\AtBeginShipoutBox=\hbox\bgroup
X \kern\p@
E \edef\AtBegShi@GroupLevel{\number\currentgrouplevel}%
  \afterassignment\AtBegShi@Test
X \global
  \setbox\AtBeginShipoutBox=%
}
%    \end{macrocode}
%    \end{macro}
%
%    \begin{macro}{\AtBegShi@Test}
%    \begin{macrocode}
\def\AtBegShi@Test{%
X \ifdim\lastkern=0pt %
E \ifnum\AtBegShi@GroupLevel<\currentgrouplevel
    \expandafter\aftergroup
  \fi
  \AtBegShi@Output
}
%    \end{macrocode}
%    \end{macro}
%
%    \begin{macro}{\AtBegShi@Output}
%    \begin{macrocode}
\def\AtBegShi@Output{%
X \egroup
X \endgroup
  \ifvoid\AtBeginShipoutBox
    \@PackageWarning{atbegshi}{Ignoring void shipout box}%
  \else
    \let\AtBegShi@OrgProtect\protect
    \csname set@typeset@protect\endcsname
    \global\AtBegShi@Discardedfalse
    \AtBegShi@Hook
    \expandafter\gdef\expandafter\AtBegShi@HookNext
    \expandafter{\expandafter}%
    \AtBegShi@HookNext
    \ifAtBegShi@Discarded
      \@PackageInfoNoLine{atbegshi}{Shipout page discarded}%
      \global\AtBegShi@Discardedfalse
      \begingroup
        \setbox\AtBeginShipoutBox\box\AtBeginShipoutBox
      \endgroup
      \let\protect\AtBegShi@OrgProtect
    \else
      \AtBegShi@First
      \let\protect\AtBegShi@OrgProtect
      \AtBegShi@GetBoxSize\AtBeginShipoutBox
      \ltx@ifundefined{AtNextShipout}{%
      }{%
        \AtNextShipout{\AtBegShi@GetBoxSize\@cclv}%
      }%
      \AtBeginShipoutOriginalShipout\box\AtBeginShipoutBox
    \fi
  \fi
}
%    \end{macrocode}
%    \end{macro}
%
%    \begin{macro}{\AtBegShi@GetBoxSize}
%    \begin{macrocode}
\def\AtBegShi@GetBoxSize#1{%
  \xdef\AtBeginShipoutBoxWidth{\the\wd#1}%
  \xdef\AtBeginShipoutBoxHeight{\the\ht#1}%
  \xdef\AtBeginShipoutBoxDepth{\the\dp#1}%
}
%    \end{macrocode}
%    \end{macro}
%    \begin{macro}{\AtBeginShipoutBoxWidth}
%    \begin{macrocode}
\def\AtBeginShipoutBoxWidth{0pt}
%    \end{macrocode}
%    \end{macro}
%    \begin{macro}{\AtBeginShipoutBoxHeight}
%    \begin{macrocode}
\def\AtBeginShipoutBoxHeight{0pt}
%    \end{macrocode}
%    \end{macro}
%    \begin{macro}{\AtBeginShipoutBoxDepth}
%    \begin{macrocode}
\def\AtBeginShipoutBoxDepth{0pt}
%    \end{macrocode}
%    \end{macro}
%
%    \begin{macrocode}
\catcode`\X=11 %
\catcode`\E=11 %
%    \end{macrocode}
%
%    \begin{macro}{\AtBegShi@First}
%    \begin{macrocode}
\def\AtBegShi@First{%
  \ifx\AtBegShi@HookFirst\ltx@empty
  \else
    \AtBeginShipoutAddToBox{\AtBegShi@HookFirst}%
  \fi
  \global\let\AtBegShi@First\ltx@empty
  \global\let\AtBeginShipoutFirst\AtBegShi@FirstDisabled
}
%    \end{macrocode}
%    \end{macro}
%
%    \begin{macro}{\AtBegShi@Hook}
%    \begin{macrocode}
\gdef\AtBegShi@Hook{}
%    \end{macrocode}
%    \end{macro}
%    \begin{macro}{\AtBegShi@HookNext}
%    \begin{macrocode}
\gdef\AtBegShi@HookNext{}
%    \end{macrocode}
%    \end{macro}
%    \begin{macro}{\AtBegShi@HookFirst}
%    \begin{macrocode}
\gdef\AtBegShi@HookFirst{}
%    \end{macrocode}
%    \end{macro}
%    \begin{macro}{\AtBeginShipout}
%    \begin{macrocode}
\AtBegShi@CheckDefinable\AtBeginShipout
\def\AtBeginShipout{%
  \AtBegShi@AddHook\AtBegShi@Hook
}
%    \end{macrocode}
%    \end{macro}
%    \begin{macro}{\AtBeginShipoutNext}
%    \begin{macrocode}
\AtBegShi@CheckDefinable\AtBeginShipoutNext
\def\AtBeginShipoutNext{%
  \AtBegShi@AddHook\AtBegShi@HookNext
}
%    \end{macrocode}
%    \end{macro}
%    \begin{macro}{\AtBeginShipoutFirst}
%    \begin{macrocode}
\AtBegShi@CheckDefinable\AtBeginShipoutFirst
\def\AtBeginShipoutFirst{%
  \AtBegShi@AddTo\AtBegShi@HookFirst
}
%    \end{macrocode}
%    \end{macro}
%    \begin{macro}{\AtBegShi@FirstDisabled}
%    \begin{macrocode}
\long\def\AtBegShi@FirstDisabled#1{%
  \@PackageWarning{atbegshi}{%
    First page is already shipped out, ignoring\MessageBreak
    \string\AtBeginShipoutFirst
  }%
}
%    \end{macrocode}
%    \end{macro}
%
%    \begin{macro}{\AtBegShi@AddTo}
%    \begin{macrocode}
\begingroup\expandafter\expandafter\expandafter\endgroup
\expandafter\ifx\csname g@addto@macro\endcsname\relax
  \long\def\AtBegShi@AddTo#1#2{%
    \begingroup
      \toks\ltx@zero\expandafter{#1#2}%
      \xdef#1{\the\toks\ltx@zero}%
    \endgroup
  }%
\else
  \let\AtBegShi@AddTo\g@addto@macro
\fi
%    \end{macrocode}
%    \end{macro}
%
%    \begin{macro}{\AtBegShi@AddHook}
%    \begin{macrocode}
\long\def\AtBegShi@AddHook#1#2{%
  \AtBegShi@AddTo#1{\AtBegShi@Item{#2}}%
}
%    \end{macrocode}
%    \end{macro}
%
%    \begin{macro}{\AtBegShi@Item}
%    \begin{macrocode}
\long\def\AtBegShi@Item#1{%
  \ifAtBegShi@Discarded
  \else
    #1%
    \ifAtBegShi@Discarded
    \else
      \ifvoid\AtBeginShipoutBox
        \@PackageWarning{atbegshi}{%
          Shipout box was voided by hook,\MessageBreak
          ignoring shipout box%
        }%
        \AtBeginShipoutDiscard
      \fi
    \fi
  \fi
}
%    \end{macrocode}
%    \end{macro}
%
%    \begin{macro}{\AtBeginShipoutInit}
%    \begin{macrocode}
\AtBegShi@CheckDefinable\AtBeginShipoutInit
\def\AtBeginShipoutInit{%
  \ltx@IfUndefined{newbox}{%
    \@PackageError{atbegshi}{%
      \string\AtBeginShipoutInit\space failed\MessageBreak
       because of missing \expandafter\string\csname newbox\endcsname
    }\@ehc
  }{%
    \csname newbox\endcsname\AtBeginShipoutBox
    \AtBegShi@CheckDefinable\AtBeginShipoutOriginalShipout
    \global\let\AtBeginShipoutOriginalShipout\shipout
    \global\let\shipout\AtBegShi@Shipout
  }%
  \gdef\AtBeginShipoutInit{}%
}
%    \end{macrocode}
%    \end{macro}
%    \begin{macrocode}
\begingroup\expandafter\expandafter\expandafter\endgroup
\expandafter\ifx\csname AtBeginDocument\endcsname\relax
  \AtBeginShipoutInit
\else
  \AtBeginDocument{\AtBeginShipoutInit}%
\fi
%    \end{macrocode}
%
% \subsection{Additions to the shipout box}
%
%    \begin{macro}{\AtBeginShipoutAddToBox}
%    \begin{macrocode}
\def\AtBeginShipoutAddToBox#1{%
  \ifhbox\AtBeginShipoutBox
    \edef\AtBegShi@restore{%
      \hfuzz=\the\hfuzz\relax
      \hbadness=\the\hbadness\relax
    }%
    \hfuzz=1073741823sp\relax
    \hbadness=2147483647\relax
    \setbox\AtBeginShipoutBox=\hbox to \wd\AtBeginShipoutBox{%
      \setbox\ltx@zero=\hbox{%
        \begingroup
          \AtBegShi@restore
          #1%
        \endgroup
      }%
      \wd\ltx@zero=0pt\relax
      \ht\ltx@zero=0pt\relax
      \dp\ltx@zero=0pt\relax
      \raise\ht\AtBeginShipoutBox\copy\ltx@zero
      \unhcopy\AtBeginShipoutBox
    }%
    \AtBegShi@restore
  \else
    \ifvbox\AtBeginShipoutBox
      \edef\AtBegShi@restore{%
        \vfuzz=\the\vfuzz\relax
        \vbadness=\the\vbadness\relax
        \dimen\ltx@zero=\the\dimen\ltx@zero\relax
      }%
      \edef\AtBegShi@restorebox{%
        \ht\AtBeginShipoutBox=\the\ht\AtBeginShipoutBox\relax
        \dp\AtBeginShipoutBox=\the\dp\AtBeginShipoutBox\relax
      }%
      \vfuzz=1073741823sp\relax
      \vbadness=2147483647\relax
      \dimen\ltx@zero=\ht\AtBeginShipoutBox
      \advance\dimen\ltx@zero by \dp\AtBeginShipoutBox
      \setbox\AtBeginShipoutBox=\vbox to \dimen\ltx@zero{%
        \setbox\ltx@zero=\hbox{%
          \begingroup
            \AtBegShi@restore
            #1%
          \endgroup
        }%
        \wd\ltx@zero=0pt\relax
        \ht\ltx@zero=0pt\relax
        \dp\ltx@zero=0pt\relax
        \baselineskip=0pt\relax
        \lineskip=0pt\relax
        \lineskiplimit=0pt\relax
        \copy\ltx@zero
        \unvbox\AtBeginShipoutBox
        \kern0pt%
      }%
      \AtBegShi@restore
      \AtBegShi@restorebox
    \fi
  \fi
}
%    \end{macrocode}
%    \end{macro}
%
%    \begin{macro}{\AtBeginShipoutAddToBoxForeground}
%    \begin{macrocode}
\def\AtBeginShipoutAddToBoxForeground#1{%
  \ifhbox\AtBeginShipoutBox
    \edef\AtBegShi@restore{%
      \hfuzz=\the\hfuzz\relax
      \hbadness=\the\hbadness\relax
    }%
    \hfuzz=1073741823sp\relax
    \hbadness=2147483647\relax
    \setbox\AtBeginShipoutBox=\hbox to \wd\AtBeginShipoutBox{%
      \unhcopy\AtBeginShipoutBox
      \kern-\wd\AtBeginShipoutBox
      \setbox\ltx@zero=\hbox{%
        \begingroup
          \AtBegShi@restore
          #1%
        \endgroup
      }%
      \wd\ltx@zero=0pt\relax
      \ht\ltx@zero=0pt\relax
      \dp\ltx@zero=0pt\relax
      \raise\ht\AtBeginShipoutBox\copy\ltx@zero
      \kern\wd\AtBeginShipoutBox
    }%
    \AtBegShi@restore
  \else
    \ifvbox\AtBeginShipoutBox
      \edef\AtBegShi@restore{%
        \vfuzz=\the\vfuzz\relax
        \vbadness=\the\vbadness\relax
        \dimen\ltx@zero=\the\dimen\ltx@zero\relax
      }%
      \edef\AtBegShi@restorebox{%
        \ht\AtBeginShipoutBox=\the\ht\AtBeginShipoutBox\relax
        \dp\AtBeginShipoutBox=\the\dp\AtBeginShipoutBox\relax
      }%
      \vfuzz=1073741823sp\relax
      \vbadness=2147483647\relax
      \dimen\ltx@zero=\ht\AtBeginShipoutBox
      \advance\dimen\ltx@zero by \dp\AtBeginShipoutBox
      \setbox\AtBeginShipoutBox=\vbox to \dimen\ltx@zero{%
        \setbox\ltx@zero=\hbox{%
          \begingroup
            \AtBegShi@restore
            #1%
          \endgroup
        }%
        \wd\ltx@zero=0pt\relax
        \ht\ltx@zero=0pt\relax
        \dp\ltx@zero=0pt\relax
        \baselineskip=0pt\relax
        \lineskip=0pt\relax
        \lineskiplimit=0pt\relax
        \unvbox\AtBeginShipoutBox
        \kern-\dimen\ltx@zero
        \copy\ltx@zero
        \kern\dimen\ltx@zero
      }%
      \AtBegShi@restore
      \AtBegShi@restorebox
    \fi
  \fi
}
%    \end{macrocode}
%    \end{macro}
%
% \subsection{Positioning}
%
%    \begin{macrocode}
\begingroup\expandafter\expandafter\expandafter\endgroup
\expandafter\ifx\csname RequirePackage\endcsname\relax
  \def\TMP@RequirePackage#1[#2]{%
    \begingroup\expandafter\expandafter\expandafter\endgroup
    \expandafter\ifx\csname ver@#1.sty\endcsname\relax
      \input #1.sty\relax
    \fi
  }%
  \TMP@RequirePackage{ifpdf}[2011/01/30]%
\else
  \RequirePackage{ifpdf}[2011/01/30]%
\fi
%    \end{macrocode}
%
%    \begin{macrocode}
\ifpdf
  \def\AtBegShi@horigin{%
    \ifx\pdfhorigin\@undefined\pdfvariable horigin\else\pdfhorigin\fi}%
  \def\AtBegShi@vorigin{%
    \ifx\pdfvorigin\@undefined\pdfvariable vorigin\else\pdfvorigin\fi}%
\else
  \def\AtBegShi@horigin{72.27pt}%
  \def\AtBegShi@vorigin{72.27pt}%
\fi
%    \end{macrocode}
%
%    \begin{macrocode}
\begingroup
\ifcase
    \expandafter\ifx\csname picture\endcsname\relax
      1%
    \else
      \expandafter\ifx\csname endpicture\endcsname\relax
        1%
      \else
        0%
      \fi
    \fi
  \endgroup
  \def\AtBegShi@BeginPicture{%
    \begingroup
    \picture(0,0)\relax
    \begingroup\expandafter\expandafter\expandafter\endgroup
    \expandafter\ifx\csname unitlength\endcsname\relax
    \else
      \unitlength=1pt\relax
    \fi
    \ignorespaces
  }%
  \def\AtBegShi@EndPicture{%
    \endpicture
    \endgroup
  }%
\else
  \endgroup
  \def\AtBegShi@BeginPicture{%
    \setbox\ltx@zero=\hbox\bgroup
    \begingroup
    \ignorespaces
  }%
  \def\AtBegShi@EndPicture{%
    \endgroup
    \egroup
    \ht\ltx@zero=0pt\relax
    \dp\ltx@zero=0pt\relax
    \copy\ltx@zero
  }%
\fi
%    \end{macrocode}
%
%    \begin{macro}{\AtBeginShipoutUpperLeft}
%    A surrounding \cs{rlap} is not necessary, because
%    the stuff is put in an \cs{hbox} with zero width.
%    \begin{macrocode}
\def\AtBeginShipoutUpperLeft#1{%
  \AtBeginShipoutAddToBox{%
    \kern-\AtBegShi@horigin\relax
    \vbox to 0pt{%
      \kern-\AtBegShi@vorigin\relax
      \AtBegShi@BeginPicture
      #1%
      \AtBegShi@EndPicture
      \vss
    }%
  }%
}
%    \end{macrocode}
%    \end{macro}
%
%    \begin{macro}{\AtBeginShipoutUpperLeftForeground}
%    \begin{macrocode}
\def\AtBeginShipoutUpperLeftForeground#1{%
  \AtBeginShipoutAddToBoxForeground{%
    \kern-\AtBegShi@horigin\relax
    \vbox to 0pt{%
      \kern-\AtBegShi@vorigin\relax
      \AtBegShi@BeginPicture
      #1%
      \AtBegShi@EndPicture
      \vss
    }%
  }%
}
%    \end{macrocode}
%    \end{macro}
%
% \subsection{Patches}
%
%    Patches for \LaTeX\ packages that redefine \cs{shipout}.
%    \LaTeX\ is now supposed to use \eTeX. Thus we do not patch,
%    without \LaTeX\ and \eTeX.
%    \begin{macrocode}
\def\AtBegShi@AbortIfUndefined#1{%
  \begingroup\expandafter\expandafter\expandafter\endgroup
  \expandafter\ifx\csname#1\endcsname\relax
    \expandafter\AtBegShi@AtEnd
  \fi
}
\AtBegShi@AbortIfUndefined{currentgrouplevel}%
\AtBegShi@AbortIfUndefined{AtBeginDocument}%
\AtBegShi@AbortIfUndefined{@ifpackageloaded}%
\AtBegShi@AbortIfUndefined{@ifclassloaded}%
%    \end{macrocode}
%
% \subsubsection{Package \xpackage{crop}}
%
% Fix of method and box.
%    \begin{macrocode}
\def\AtBegShi@PatchCrop{%
  \begingroup
    \def\AtBegShi@Crop@shipout{%
      \afterassignment\CROP@ship
      \setbox\@cclv=%
    }%
    \def\AtBegShi@Crop@ship{%
      \ifvoid\@cclv
        \expandafter\aftergroup
      \fi
      \CROP@@ship
    }%
    \def\AtBegShi@Crop@shiplist{%
      \lineskip\z@
      \lineskiplimit\z@
      \baselineskip\z@
      \CROP@kernel
      \box\@cclv
    }%
    \def\AtBegShi@Crop@@ship{%
      \CROP@shipout\vbox{%
        \CROP@shiplist
      }%
    }%
    \ifx\AtBegShi@Crop@ship\CROP@ship
      \ifx\AtBegShi@Crop@shiplist\CROP@shiplist
        \ifx\AtBegShi@Crop@@ship\CROP@@ship
          \let\AtBegShi@found\relax
          \ifx\shipout\AtBegShi@Crop@shipout
            \def\AtBegShi@found{\shipout}%
          \else\ifx\AtBeginShipoutOriginalShipout\AtBegShi@Crop@shipout
            \def\AtBegShi@found{\AtBeginShipoutOriginalShipout}%
          \else\ifx\@EveryShipout@Org@Shipout\AtBegShi@Crop@shipout
            \def\AtBegShi@found{\@EveryShipout@Org@Shipout}%
          \else\ifx\GPTorg@shipout\AtBegShi@Crop@shipout
            \def\AtBegShi@found{\GPTorg@shipout}%
          \else\ifx\THBorg@shipout\AtBegShi@Crop@shipout
            \def\AtBegShi@found{\THBorg@shipout}%
          \else\ifx\mem@oldshipout\AtBegShi@Crop@shipout
            \def\AtBegShi@found{\mem@oldshipout}%
          \fi\fi\fi\fi\fi\fi
          \ifx\AtBegShi@found\relax
          \else
            \expandafter\endgroup
            \expandafter\def\AtBegShi@found{%
              \edef\AtBegShi@GroupLevel{\number\currentgrouplevel}%
              \afterassignment\CROP@ship
              \setbox\AtBeginShipoutBox=%
            }%
            \def\CROP@ship{%
              \ifnum\AtBegShi@GroupLevel=\currentgrouplevel
              \else
                \expandafter\aftergroup
              \fi
              \CROP@@ship
            }%
            \def\CROP@shiplist{%
              \lineskip 0pt\relax
              \lineskiplimit 0pt\relax
              \baselineskip 0pt\relax
              \CROP@kernel
              \box\AtBeginShipoutBox
            }%
            \def\CROP@@ship{%
              \ifvoid\AtBeginShipoutBox
              \else
                \setbox\AtBeginShipoutBox=\vbox{%
                  \CROP@shiplist
                }%
                \AtBegShi@GetBoxSize\AtBeginShipoutBox
                \expandafter\CROP@shipout
                \expandafter\box
                \expandafter\AtBeginShipoutBox
              \fi
            }%
            \@PackageInfoNoLine{atbegshi}{Package `crop' patched}%
            \begingroup
          \fi
        \fi
      \fi
    \fi
  \endgroup
  \let\AtBegShi@PatchCrop\relax
}
\@ifpackageloaded{crop}{%
  \AtBegShi@PatchCrop
}{%
  \AtBeginDocument{\AtBegShi@PatchCrop}%
}
%    \end{macrocode}
%
% \subsubsection{Package \xpackage{everyshi}}
%
%     Fix of method. Use of box 255 is not changed.
%    \begin{macrocode}
\def\AtBegShi@PatchEveryshi{%
  \begingroup
    \long\def\AtBegShi@Everyshi@shipout{%
      \afterassignment\@EveryShipout@Test
      \global\setbox\@cclv= %
    }%
    \long\def\AtBegShi@Everyshi@Test{%
      \ifvoid\@cclv\relax
        \aftergroup\@EveryShipout@Output
      \else
        \@EveryShipout@Output
      \fi
    }%
    \ifx\AtBegShi@Everyshi@Test\@EveryShipout@Test
      \let\AtBegShi@found\relax
      \ifx\shipout\AtBegShi@Everyshi@shipout
        \def\AtBegShi@found{\shipout}%
      \else\ifx\AtBeginShipoutOriginalShipout\AtBegShi@Everyshi@shipout
        \def\AtBegShi@found{\AtBeginShipoutOriginalShipout}%
      \else\ifx\CROP@shipout\AtBegShi@Everyshi@shipout
        \def\AtBegShi@found{\CROP@shipout}%
      \else\ifx\GPTorg@shipout\AtBegShi@Everyshi@shipout
        \def\AtBegShi@found{\GPTorg@shipout}%
      \else\ifx\THBorg@shipout\AtBegShi@Everyshi@shipout
        \def\AtBegShi@found{\THBorg@shipout}%
      \else\ifx\mem@oldshipout\AtBegShi@Everyshi@shipout
        \def\AtBegShi@found{\mem@oldshipout}%
      \else
        \expandafter\ifx\csname @EveryShipout@Org@Shipout\endcsname
                        \relax
           \ifx\@EveryShipout@Shipout\AtBegShi@Everyshi@shipout
             \def\AtBegShi@found{\@EveryShipout@Shipout}%
           \fi
        \fi
      \fi\fi\fi\fi\fi\fi
      \ifx\AtBegShi@found\relax
      \else
        \expandafter\endgroup
        \expandafter\def\AtBegShi@found{%
          \edef\AtBegShi@GroupLevel{\number\currentgrouplevel}%
          \afterassignment\@EveryShipout@Test
          \setbox\AtBeginShipoutBox=%
        }%
        \def\@EveryShipout@Test{%
          \ifnum\AtBegShi@GroupLevel=\currentgrouplevel
          \else
            \expandafter\aftergroup
          \fi
          \AtBegShi@Everyshi@Output
        }%
        \def\AtBegShi@Everyshi@Output{%
          \ifvoid\AtBeginShipoutBox
          \else
            \global\setbox\ltx@cclv\box\AtBeginShipoutBox
            \expandafter\@EveryShipout@Output
          \fi
        }%
        \@PackageInfoNoLine{atbegshi}{Package `everyshi' patched}%
        \begingroup
      \fi
    \fi
  \endgroup
  \let\AtBegShi@PatchEveryshi\relax
}
\@ifpackageloaded{everyshi}{%
  \AtBegShi@PatchEveryshi
}{%
  \AtBeginDocument{\AtBegShi@PatchEveryshi}%
}
%    \end{macrocode}
%
% \subsubsection{Class \xpackage{memoir}}
%
%     Fix of method and box.
%    \begin{macrocode}
\def\AtBegShi@PatchMemoir{%
  \begingroup
    \def\AtBegShi@Memoir@shipout{%
      \afterassignment\mem@shipi
      \setbox\@cclv=%
    }%
    \def\AtBegShi@Memoir@shipi{%
      \ifvoid\@cclv
        \expandafter\aftergroup
      \fi
      \mem@shipii
    }%
    \def\AtBegShi@Memoir@shipiiA{%
      \mem@oldshipout\vbox{%
        \trimmarks
        \unvbox\@cclv
      }%
    }%
    \def\AtBegShi@Memoir@shipiiB{%
      \ifvoid\@cclv
        \mem@oldshipout\box\@cclv
      \else
        \mem@oldshipout\vbox{%
          \trimmarks
          \unvbox\@cclv
        }%
      \fi
    }%
    \def\AtBegShi@Memoir@PatchAB{%
      \ifvoid\AtBeginShipoutBox
      \else
        \setbox\AtBeginShipoutBox=\vbox{%
          \trimmarks
          \ifvbox\AtBeginShipoutBox
            \unvbox\AtBeginShipoutBox
          \else
            \box\AtBeginShipoutBox
          \fi
        }%
        \AtBegShi@GetBoxSize\AtBeginShipoutBox
        \expandafter\mem@oldshipout
        \expandafter\box
        \expandafter\AtBeginShipoutBox
      \fi
    }%
    \def\AtBegShi@Memoir@shipiiC{% 2008/08/07 v1.6180339a
      \ifvoid\@cclv
        \mem@oldshipout\box\@cclv
      \else
        \ifshowtrims
          \mem@oldshipout\vbox{\trimmarks\unvbox\@cclv}%
        \else
          \mem@oldshipout\box\@cclv
        \fi
      \fi
    }%
    \def\AtBegShi@Memoir@shipiiD{% 2011/03/06 v3.6j
      \ifvoid\@cclv
        \mem@oldshipout\box\@cclv
      \else
        \ifshowtrims
          \mem@oldshipout\vbox{%
            \trimmarks
            \nointerlineskip
            \box\@cclv
          }%
        \else
          \mem@oldshipout\box\@cclv
        \fi
      \fi
    }%
    \def\AtBegShi@Memoir@PatchCD{%
      \ifvoid\AtBeginShipoutBox
      \else
        \ifshowtrims
          \setbox\AtBeginShipoutBox=\vbox{%
            \trimmarks
            \nointerlineskip
            \box\AtBeginShipoutBox
          }%
        \fi
        \AtBegShi@GetBoxSize\AtBeginShipoutBox
        \expandafter\mem@oldshipout
        \expandafter\box
        \expandafter\AtBeginShipoutBox
      \fi
    }%
    \ifx\AtBegShi@Memoir@shipi\mem@shipi
      \let\AtBegShi@found\ltx@one
      \ifx\AtBegShi@Memoir@shipiiA\mem@shipii
        \let\AtBegShi@found\ltx@zero
        \global\let\AtBegShi@Memoir@PatchX\AtBegShi@Memoir@PatchAB
      \else\ifx\AtBegShi@Memoir@shipiiB\mem@shipii
        \let\AtBegShi@found\ltx@zero
        \global\let\AtBegShi@Memoir@PatchX\AtBegShi@Memoir@PatchAB
      \else\ifx\AtBegShi@Memoir@shipiiC\mem@shipii
        \let\AtBegShi@found\ltx@zero
        \global\let\AtBegShi@Memoir@PatchX\AtBegShi@Memoir@PatchCD
      \else\ifx\AtBegShi@Memoir@shipiiD\mem@shipii
        \let\AtBegShi@found\ltx@zero
        \global\let\AtBegShi@Memoir@PatchX\AtBegShi@Memoir@PatchCD
      \fi\fi\fi\fi
      \ifcase\AtBegShi@found
        \let\AtBegShi@found\relax
        \ifx\shipout\AtBegShi@Memoir@shipout
          \def\AtBegShi@found{\shipout}%
        \else\ifx\AtBeginShipoutOriginalShipout\AtBegShi@Memoir@shipout
          \def\AtBegShi@found{\AtBeginShipoutOriginalShipout}%
        \else\ifx\CROP@shipout\AtBegShi@Memoir@shipout
          \def\AtBegShi@found{\CROP@shipout}%
        \else\ifx\GPTorg@shipout\AtBegShi@Memoir@shipout
          \def\AtBegShi@found{\GPTorg@shipout}%
        \else\ifx\THBorg@shipout\AtBegShi@Memoir@shipout
          \def\AtBegShi@found{\THBorg@shipout}%
        \else\ifx\@EveryShipout@Org@Shipout\AtBegShi@Memoir@shipout
          \def\AtBegShi@found{\@EveryShipout@Org@Shipout}%
        \fi\fi\fi\fi\fi\fi
        \ifx\AtBegShi@found\relax
        \else
          \expandafter\endgroup
          \expandafter\def\AtBegShi@found{%
            \edef\AtBegShi@GroupLevel{\number\currentgrouplevel}%
            \afterassignment\mem@shipi
            \setbox\AtBeginShipoutBox=%
          }%
          \def\mem@shipi{%
            \ifnum\AtBegShi@GroupLevel=\currentgrouplevel
            \else
              \expandafter\aftergroup
            \fi
            \mem@shipii
          }%
          \let\mem@shipii\AtBegShi@Memoir@PatchX
          \@PackageInfoNoLine{atbegshi}{Class `memoir' patched}%
          \begingroup
        \fi
      \fi
    \fi
  \endgroup
  \let\AtBegShi@PatchMemoir\relax
}
\@ifclassloaded{memoir}{%
  \AtBegShi@PatchMemoir
}{%
  \AtBeginDocument{\AtBegShi@PatchMemoir}%
}
%    \end{macrocode}
%
%    \begin{macrocode}
\AtBegShi@AtEnd%
%</package>
%    \end{macrocode}
%
% \section{Test}
%
% \subsection{Catcode checks for loading}
%
%    \begin{macrocode}
%<*test1>
%    \end{macrocode}
%    \begin{macrocode}
\catcode`\{=1 %
\catcode`\}=2 %
\catcode`\#=6 %
\catcode`\@=11 %
\expandafter\ifx\csname count@\endcsname\relax
  \countdef\count@=255 %
\fi
\expandafter\ifx\csname @gobble\endcsname\relax
  \long\def\@gobble#1{}%
\fi
\expandafter\ifx\csname @firstofone\endcsname\relax
  \long\def\@firstofone#1{#1}%
\fi
\expandafter\ifx\csname loop\endcsname\relax
  \expandafter\@firstofone
\else
  \expandafter\@gobble
\fi
{%
  \def\loop#1\repeat{%
    \def\body{#1}%
    \iterate
  }%
  \def\iterate{%
    \body
      \let\next\iterate
    \else
      \let\next\relax
    \fi
    \next
  }%
  \let\repeat=\fi
}%
\def\RestoreCatcodes{}
\count@=0 %
\loop
  \edef\RestoreCatcodes{%
    \RestoreCatcodes
    \catcode\the\count@=\the\catcode\count@\relax
  }%
\ifnum\count@<255 %
  \advance\count@ 1 %
\repeat

\def\RangeCatcodeInvalid#1#2{%
  \count@=#1\relax
  \loop
    \catcode\count@=15 %
  \ifnum\count@<#2\relax
    \advance\count@ 1 %
  \repeat
}
\def\RangeCatcodeCheck#1#2#3{%
  \count@=#1\relax
  \loop
    \ifnum#3=\catcode\count@
    \else
      \errmessage{%
        Character \the\count@\space
        with wrong catcode \the\catcode\count@\space
        instead of \number#3%
      }%
    \fi
  \ifnum\count@<#2\relax
    \advance\count@ 1 %
  \repeat
}
\def\space{ }
\expandafter\ifx\csname LoadCommand\endcsname\relax
  \def\LoadCommand{\input atbegshi.sty\relax}%
\fi
\def\Test{%
  \RangeCatcodeInvalid{0}{47}%
  \RangeCatcodeInvalid{58}{64}%
  \RangeCatcodeInvalid{91}{96}%
  \RangeCatcodeInvalid{123}{255}%
  \catcode`\@=12 %
  \catcode`\\=0 %
  \catcode`\%=14 %
  \LoadCommand
  \RangeCatcodeCheck{0}{36}{15}%
  \RangeCatcodeCheck{37}{37}{14}%
  \RangeCatcodeCheck{38}{47}{15}%
  \RangeCatcodeCheck{48}{57}{12}%
  \RangeCatcodeCheck{58}{63}{15}%
  \RangeCatcodeCheck{64}{64}{12}%
  \RangeCatcodeCheck{65}{90}{11}%
  \RangeCatcodeCheck{91}{91}{15}%
  \RangeCatcodeCheck{92}{92}{0}%
  \RangeCatcodeCheck{93}{96}{15}%
  \RangeCatcodeCheck{97}{122}{11}%
  \RangeCatcodeCheck{123}{255}{15}%
  \RestoreCatcodes
}
\Test
\csname @@end\endcsname
\end
%    \end{macrocode}
%    \begin{macrocode}
%</test1>
%    \end{macrocode}
%
%    \begin{macrocode}
%<*test2>
\input atbegshi.sty\relax
\def\msg#{\immediate\write16}
\msg{File: atbegshi-test2.tex 2016/06/09 v1.18 Test file for plain-TeX}
\def\testmsg#1#2{%
  \msg{}%
  \msg{*** Test with box (#1), expected page output [#2]}% hash-ok
}

\newbox\voidbox
\def\void{\box\voidbox}
\begingroup
  \setbox\voidbox=\void
\endgroup

\count0=0\relax
\AtBeginShipout{%
  \global\advance\count0 by 1\relax
  \msg{* Inside \string\AtBeginShipout: [\the\count0]}%
}

\AtBeginShipoutFirst{%
  \msg{* Inside \string\AtBeginShipoutFirst}%
  Hello World%
}

\testmsg{\string\null}{1}
\shipout\null

\AtBeginShipoutFirst{%
  This is too late%
}

\testmsg{void}{}
\shipout\void

\testmsg{\string\copy255 (not void)}{2}
\setbox255\hbox{\vrule height 10bp width 10bp}
\shipout\copy255 %

\testmsg{\string\copy255 (again)}{3}
\shipout\copy255 %

\testmsg{\string\box255}{4}
\shipout\box255 %

\testmsg{\string\box255 (again)}{}
\shipout\box255 %

\testmsg{\string\hbox}{5}
\shipout\hbox{\vrule height 5bp width 20bp}

\testmsg{\string\vbox}{6}
\shipout\vbox{\hrule height 20bp width 5bp}

\testmsg{\string\null, voided by hook}{}
\def\VoidBox{%
  \begingroup
    \setbox\AtBeginShipoutBox=\box\AtBeginShipoutBox
  \endgroup
}
\AtBeginShipout{\VoidBox}
\shipout\null
\def\VoidBox{}

\msg{*** \string\begingroup}
\begingroup
  \testmsg{void}{}%
  \shipout\void
\msg{*** \string\endgroup}
\endgroup

\msg{*** \string\begingroup}
\begingroup
  \testmsg{void}{}%
  \shipout\void
  \testmsg{\string\null}{8}%
  \shipout\null
\msg{*** \string\endgroup}
\endgroup

\testmsg{output routine}{9}
Hello World
\vfill
\eject

\testmsg{\string\null\space(discarded)}{}
\AtBeginShipout{%
  \msg{* Inside \string\AtBeginShipout: DISCARD}%
  \AtBeginShipoutDiscard
}
\shipout\null

\end
%</test2>
%    \end{macrocode}
%
%    \begin{macrocode}
%<*test3>
\NeedsTeXFormat{LaTeX2e}
\ProvidesFile{atbegshi-test3.tex}[2016/06/09 v1.18 Test file for LaTeX]
\RequirePackage{color}
\pagecolor{yellow}
\documentclass[a5paper,showtrims]{memoir}
\usepackage{atbegshi}
\AtBeginShipout{%
  \setbox\AtBeginShipoutBox=\vbox{%
    \vbox to 0pt{%
      \kern-1.5in %
      \hbox to 0pt{%
        \kern-1.5in %
        \color{blue}%
        \rule{1in}{1in}%
        \hss
      }%
      \vss
    }%
    \hrule
    \hbox{\vrule\box\AtBeginShipoutBox\vrule}%
    \hrule
  }%
}
\usepackage{eso-pic}
\makeatletter
\@ifundefined{@EveryShipout@Init}{%
  \typeout{Test skipped}%
  \@@end
}{}
\@EveryShipout@Init
\let\@EveryShipout@Init\relax
\makeatother
\AddToShipoutPicture{%
  \hspace{.52\paperwidth}%
  \colorbox{cyan}{%
    \rule{0mm}{\paperheight}%
    \hspace{.48\paperwidth}%
  }%
}
%    \end{macrocode}
%    Newer versions of class \xclass{memoir} emulate package
%    \xpackage{crop} and prevents its loading. This is undone
%    in next line for this test file.
%    \begin{macrocode}
\expandafter\let\csname ver@crop.sty\endcsname\relax
\usepackage[color=red,cross,a4,center]{crop}
\begin{document}
\shipout\null
\shipout\box\csname voidb@x\endcsname
\section{Hello World}
\end{document}
%</test3>
%    \end{macrocode}
%
% \section{Installation}
%
% \subsection{Download}
%
% \paragraph{Package.} This package is available on
% CTAN\footnote{\url{http://ctan.org/pkg/atbegshi}}:
% \begin{description}
% \item[\CTAN{macros/latex/contrib/oberdiek/atbegshi.dtx}] The source file.
% \item[\CTAN{macros/latex/contrib/oberdiek/atbegshi.pdf}] Documentation.
% \end{description}
%
%
% \paragraph{Bundle.} All the packages of the bundle `oberdiek'
% are also available in a TDS compliant ZIP archive. There
% the packages are already unpacked and the documentation files
% are generated. The files and directories obey the TDS standard.
% \begin{description}
% \item[\CTAN{install/macros/latex/contrib/oberdiek.tds.zip}]
% \end{description}
% \emph{TDS} refers to the standard ``A Directory Structure
% for \TeX\ Files'' (\CTAN{tds/tds.pdf}). Directories
% with \xfile{texmf} in their name are usually organized this way.
%
% \subsection{Bundle installation}
%
% \paragraph{Unpacking.} Unpack the \xfile{oberdiek.tds.zip} in the
% TDS tree (also known as \xfile{texmf} tree) of your choice.
% Example (linux):
% \begin{quote}
%   |unzip oberdiek.tds.zip -d ~/texmf|
% \end{quote}
%
% \paragraph{Script installation.}
% Check the directory \xfile{TDS:scripts/oberdiek/} for
% scripts that need further installation steps.
% Package \xpackage{attachfile2} comes with the Perl script
% \xfile{pdfatfi.pl} that should be installed in such a way
% that it can be called as \texttt{pdfatfi}.
% Example (linux):
% \begin{quote}
%   |chmod +x scripts/oberdiek/pdfatfi.pl|\\
%   |cp scripts/oberdiek/pdfatfi.pl /usr/local/bin/|
% \end{quote}
%
% \subsection{Package installation}
%
% \paragraph{Unpacking.} The \xfile{.dtx} file is a self-extracting
% \docstrip\ archive. The files are extracted by running the
% \xfile{.dtx} through \plainTeX:
% \begin{quote}
%   \verb|tex atbegshi.dtx|
% \end{quote}
%
% \paragraph{TDS.} Now the different files must be moved into
% the different directories in your installation TDS tree
% (also known as \xfile{texmf} tree):
% \begin{quote}
% \def\t{^^A
% \begin{tabular}{@{}>{\ttfamily}l@{ $\rightarrow$ }>{\ttfamily}l@{}}
%   atbegshi.sty & tex/generic/oberdiek/atbegshi.sty\\
%   atbegshi.pdf & doc/latex/oberdiek/atbegshi.pdf\\
%   atbegshi-example1.tex & doc/latex/oberdiek/atbegshi-example1.tex\\
%   atbegshi-example2.tex & doc/latex/oberdiek/atbegshi-example2.tex\\
%   test/atbegshi-test1.tex & doc/latex/oberdiek/test/atbegshi-test1.tex\\
%   test/atbegshi-test2.tex & doc/latex/oberdiek/test/atbegshi-test2.tex\\
%   test/atbegshi-test3.tex & doc/latex/oberdiek/test/atbegshi-test3.tex\\
%   atbegshi.dtx & source/latex/oberdiek/atbegshi.dtx\\
% \end{tabular}^^A
% }^^A
% \sbox0{\t}^^A
% \ifdim\wd0>\linewidth
%   \begingroup
%     \advance\linewidth by\leftmargin
%     \advance\linewidth by\rightmargin
%   \edef\x{\endgroup
%     \def\noexpand\lw{\the\linewidth}^^A
%   }\x
%   \def\lwbox{^^A
%     \leavevmode
%     \hbox to \linewidth{^^A
%       \kern-\leftmargin\relax
%       \hss
%       \usebox0
%       \hss
%       \kern-\rightmargin\relax
%     }^^A
%   }^^A
%   \ifdim\wd0>\lw
%     \sbox0{\small\t}^^A
%     \ifdim\wd0>\linewidth
%       \ifdim\wd0>\lw
%         \sbox0{\footnotesize\t}^^A
%         \ifdim\wd0>\linewidth
%           \ifdim\wd0>\lw
%             \sbox0{\scriptsize\t}^^A
%             \ifdim\wd0>\linewidth
%               \ifdim\wd0>\lw
%                 \sbox0{\tiny\t}^^A
%                 \ifdim\wd0>\linewidth
%                   \lwbox
%                 \else
%                   \usebox0
%                 \fi
%               \else
%                 \lwbox
%               \fi
%             \else
%               \usebox0
%             \fi
%           \else
%             \lwbox
%           \fi
%         \else
%           \usebox0
%         \fi
%       \else
%         \lwbox
%       \fi
%     \else
%       \usebox0
%     \fi
%   \else
%     \lwbox
%   \fi
% \else
%   \usebox0
% \fi
% \end{quote}
% If you have a \xfile{docstrip.cfg} that configures and enables \docstrip's
% TDS installing feature, then some files can already be in the right
% place, see the documentation of \docstrip.
%
% \subsection{Refresh file name databases}
%
% If your \TeX~distribution
% (\teTeX, \mikTeX, \dots) relies on file name databases, you must refresh
% these. For example, \teTeX\ users run \verb|texhash| or
% \verb|mktexlsr|.
%
% \subsection{Some details for the interested}
%
% \paragraph{Attached source.}
%
% The PDF documentation on CTAN also includes the
% \xfile{.dtx} source file. It can be extracted by
% AcrobatReader 6 or higher. Another option is \textsf{pdftk},
% e.g. unpack the file into the current directory:
% \begin{quote}
%   \verb|pdftk atbegshi.pdf unpack_files output .|
% \end{quote}
%
% \paragraph{Unpacking with \LaTeX.}
% The \xfile{.dtx} chooses its action depending on the format:
% \begin{description}
% \item[\plainTeX:] Run \docstrip\ and extract the files.
% \item[\LaTeX:] Generate the documentation.
% \end{description}
% If you insist on using \LaTeX\ for \docstrip\ (really,
% \docstrip\ does not need \LaTeX), then inform the autodetect routine
% about your intention:
% \begin{quote}
%   \verb|latex \let\install=y% \iffalse meta-comment
%
% File: atbegshi.dtx
% Version: 2016/06/09 v1.18
% Info: At begin shipout hook
%
% Copyright (C) 2007-2011 by
%    Heiko Oberdiek <heiko.oberdiek at googlemail.com>
%    2016
%    https://github.com/ho-tex/oberdiek/issues
%
% This work may be distributed and/or modified under the
% conditions of the LaTeX Project Public License, either
% version 1.3c of this license or (at your option) any later
% version. This version of this license is in
%    http://www.latex-project.org/lppl/lppl-1-3c.txt
% and the latest version of this license is in
%    http://www.latex-project.org/lppl.txt
% and version 1.3 or later is part of all distributions of
% LaTeX version 2005/12/01 or later.
%
% This work has the LPPL maintenance status "maintained".
%
% This Current Maintainer of this work is Heiko Oberdiek.
%
% The Base Interpreter refers to any `TeX-Format',
% because some files are installed in TDS:tex/generic//.
%
% This work consists of the main source file atbegshi.dtx
% and the derived files
%    atbegshi.sty, atbegshi.pdf, atbegshi.ins, atbegshi.drv,
%    atbegshi-example1.tex, atbegshi-example2.tex,
%    atbegshi-test1.tex, atbegshi-test2.tex, atbegshi-test3.tex.
%
% Distribution:
%    CTAN:macros/latex/contrib/oberdiek/atbegshi.dtx
%    CTAN:macros/latex/contrib/oberdiek/atbegshi.pdf
%
% Unpacking:
%    (a) If atbegshi.ins is present:
%           tex atbegshi.ins
%    (b) Without atbegshi.ins:
%           tex atbegshi.dtx
%    (c) If you insist on using LaTeX
%           latex \let\install=y\input{atbegshi.dtx}
%        (quote the arguments according to the demands of your shell)
%
% Documentation:
%    (a) If atbegshi.drv is present:
%           latex atbegshi.drv
%    (b) Without atbegshi.drv:
%           latex atbegshi.dtx; ...
%    The class ltxdoc loads the configuration file ltxdoc.cfg
%    if available. Here you can specify further options, e.g.
%    use A4 as paper format:
%       \PassOptionsToClass{a4paper}{article}
%
%    Programm calls to get the documentation (example):
%       pdflatex atbegshi.dtx
%       makeindex -s gind.ist atbegshi.idx
%       pdflatex atbegshi.dtx
%       makeindex -s gind.ist atbegshi.idx
%       pdflatex atbegshi.dtx
%
% Installation:
%    TDS:tex/generic/oberdiek/atbegshi.sty
%    TDS:doc/latex/oberdiek/atbegshi.pdf
%    TDS:doc/latex/oberdiek/atbegshi-example1.tex
%    TDS:doc/latex/oberdiek/atbegshi-example2.tex
%    TDS:doc/latex/oberdiek/test/atbegshi-test1.tex
%    TDS:doc/latex/oberdiek/test/atbegshi-test2.tex
%    TDS:doc/latex/oberdiek/test/atbegshi-test3.tex
%    TDS:source/latex/oberdiek/atbegshi.dtx
%
%<*ignore>
\begingroup
  \catcode123=1 %
  \catcode125=2 %
  \def\x{LaTeX2e}%
\expandafter\endgroup
\ifcase 0\ifx\install y1\fi\expandafter
         \ifx\csname processbatchFile\endcsname\relax\else1\fi
         \ifx\fmtname\x\else 1\fi\relax
\else\csname fi\endcsname
%</ignore>
%<*install>
\input docstrip.tex
\Msg{************************************************************************}
\Msg{* Installation}
\Msg{* Package: atbegshi 2016/06/09 v1.18 At begin shipout hook (HO)}
\Msg{************************************************************************}

\keepsilent
\askforoverwritefalse

\let\MetaPrefix\relax
\preamble

This is a generated file.

Project: atbegshi
Version: 2016/06/09 v1.18

Copyright (C) 2007-2011 by
   Heiko Oberdiek <heiko.oberdiek at googlemail.com>

This work may be distributed and/or modified under the
conditions of the LaTeX Project Public License, either
version 1.3c of this license or (at your option) any later
version. This version of this license is in
   http://www.latex-project.org/lppl/lppl-1-3c.txt
and the latest version of this license is in
   http://www.latex-project.org/lppl.txt
and version 1.3 or later is part of all distributions of
LaTeX version 2005/12/01 or later.

This work has the LPPL maintenance status "maintained".

This Current Maintainer of this work is Heiko Oberdiek.

The Base Interpreter refers to any `TeX-Format',
because some files are installed in TDS:tex/generic//.

This work consists of the main source file atbegshi.dtx
and the derived files
   atbegshi.sty, atbegshi.pdf, atbegshi.ins, atbegshi.drv,
   atbegshi-example1.tex, atbegshi-example2.tex,
   atbegshi-test1.tex, atbegshi-test2.tex, atbegshi-test3.tex.

\endpreamble
\let\MetaPrefix\DoubleperCent

\generate{%
  \file{atbegshi.ins}{\from{atbegshi.dtx}{install}}%
  \file{atbegshi.drv}{\from{atbegshi.dtx}{driver}}%
  \usedir{tex/generic/oberdiek}%
  \file{atbegshi.sty}{\from{atbegshi.dtx}{package}}%
  \usedir{doc/latex/oberdiek}%
  \file{atbegshi-example1.tex}{\from{atbegshi.dtx}{example1}}%
  \file{atbegshi-example2.tex}{\from{atbegshi.dtx}{example2}}%
  \usedir{doc/latex/oberdiek/test}%
  \file{atbegshi-test1.tex}{\from{atbegshi.dtx}{test1}}%
  \file{atbegshi-test2.tex}{\from{atbegshi.dtx}{test2}}%
  \file{atbegshi-test3.tex}{\from{atbegshi.dtx}{test3}}%
  \nopreamble
  \nopostamble
  \usedir{source/latex/oberdiek/catalogue}%
  \file{atbegshi.xml}{\from{atbegshi.dtx}{catalogue}}%
}

\catcode32=13\relax% active space
\let =\space%
\Msg{************************************************************************}
\Msg{*}
\Msg{* To finish the installation you have to move the following}
\Msg{* file into a directory searched by TeX:}
\Msg{*}
\Msg{*     atbegshi.sty}
\Msg{*}
\Msg{* To produce the documentation run the file `atbegshi.drv'}
\Msg{* through LaTeX.}
\Msg{*}
\Msg{* Happy TeXing!}
\Msg{*}
\Msg{************************************************************************}

\endbatchfile
%</install>
%<*ignore>
\fi
%</ignore>
%<*driver>
\NeedsTeXFormat{LaTeX2e}
\ProvidesFile{atbegshi.drv}%
  [2016/06/09 v1.18 At begin shipout hook (HO)]%
\documentclass{ltxdoc}
\usepackage{holtxdoc}[2011/11/22]
\begin{document}
  \DocInput{atbegshi.dtx}%
\end{document}
%</driver>
% \fi
%
%
% \CharacterTable
%  {Upper-case    \A\B\C\D\E\F\G\H\I\J\K\L\M\N\O\P\Q\R\S\T\U\V\W\X\Y\Z
%   Lower-case    \a\b\c\d\e\f\g\h\i\j\k\l\m\n\o\p\q\r\s\t\u\v\w\x\y\z
%   Digits        \0\1\2\3\4\5\6\7\8\9
%   Exclamation   \!     Double quote  \"     Hash (number) \#
%   Dollar        \$     Percent       \%     Ampersand     \&
%   Acute accent  \'     Left paren    \(     Right paren   \)
%   Asterisk      \*     Plus          \+     Comma         \,
%   Minus         \-     Point         \.     Solidus       \/
%   Colon         \:     Semicolon     \;     Less than     \<
%   Equals        \=     Greater than  \>     Question mark \?
%   Commercial at \@     Left bracket  \[     Backslash     \\
%   Right bracket \]     Circumflex    \^     Underscore    \_
%   Grave accent  \`     Left brace    \{     Vertical bar  \|
%   Right brace   \}     Tilde         \~}
%
% \GetFileInfo{atbegshi.drv}
%
% \title{The \xpackage{atbegshi} package}
% \date{2016/06/09 v1.18}
% \author{Heiko Oberdiek\thanks
% {Please report any issues at https://github.com/ho-tex/oberdiek/issues}\\
% \xemail{heiko.oberdiek at googlemail.com}}
%
% \maketitle
%
% \begin{abstract}
% This package is a modern reimplementation of package \xpackage{everyshi}
% without the burden of compatibility. It makes use of \eTeX's if available.
% Both \LaTeX\ and \plainTeX\ are supported.
% \end{abstract}
%
% \tableofcontents
%
% \section{Documentation}
%
% Package \xpackage{atbegshi} redefines \cs{shipout} to insert
% hooks for user code that is executed before the page is shipped out.
% The code may modify or even discard the output page.
% Three hooks are implemented:
% \begin{enumerate}
% \item A hook that is executed for every page, see\\
%       \cs{AtBeginShipout}
% \item A hook that is executed for the next page only, see\\
%       \cs{AtBeginShipoutNext}
% \item A hook that is only executed for the first page, see\\
%       \cs{AtBeginShipoutFirst}
% \end{enumerate}
% The hooks are executed in this order. The following three macros
% provide the user interface for adding code to these hooks:
%
% \begin{declcs}{AtBeginShipout} \M{code}\\
%   \cs{AtBeginShipoutBox}
% \end{declcs}
% Execute the \meta{code} for every page.
% The page contents is held in box register \cs{AtBeginShipoutBox}
% and may be modified. Use \cs{AtBeginShipoutDiscard} if you want to
% discard the page.
%
% \emph{Note:} Package \xpackage{everyshi} uses box register 255. With
% package \xpackage{atbegshi} you must use \cs{AtBeginShipoutBox} instead.
%
% If \LaTeX\ calls \cs{shipout} in \cs{@outputpage} (part of its output
% routine),
% the meaning of \cs{protect} is \cs{noexpand}. \LaTeX\ sets
% \cs{protect} to the appropriate \cs{@typeset@protect} in the box
% that is shipped out. This is too late for the hooks, they
% are called earlier in the redefined \cs{shipout}. Therefore
% package \xpackage{atbegshi} sets \cs{protect} to \cs{@typeset@protect}
% before it calls the hooks.
% (In \cs{EveryShipout} of package \xpackage{everyshi} the
% user is responsible for the correct setting of \cs{protect}.)
%
% \begin{declcs}{AtBeginShipoutNext} \M{code}
% \end{declcs}
% This reimplements package \xpackage{everyshi}'s \cs{AtNextShipout}.
% The \meta{code} is executed at shipout time of the next page only.
% It is just a convenience macro, it can be easily replaced by something
% like:
%\begin{quote}
%\begin{verbatim}
%\newcommand{\MyShipoutHook}{}%
%\AtBeginShipout{\MyShipoutHook}
%\gdef\MyShipoutHook{%
%  ... do something with next page ...
%  \gdef\MyShipoutHook{}%
% }
%\end{verbatim}
%\end{quote}
% (This can be necessary, if hook order does matter).
%
% \begin{declcs}{AtBeginShipoutFirst} \M{code}
% \end{declcs}
% This reimplements \LaTeX's \cs{AtBeginDvi}. This hook is
% usually used for \cs{special} commands that include PostScript
% header files. The \cs{code} is directly executed in a \cs{vbox}
% that is put at the beginning of the output page. Dealing with
% the output box \cs{AtBeginShipoutBox} is not necessary and
% not permitted here.
%
% \begin{declcs}{AtBeginShipoutDiscard}
% \end{declcs}
% This macro notifies package \xpackage{atbegshi} that
% the output page is discarded. The remaining hook code
% and the remaining hooks are not executed and the page is thrown away.
% Also \cs{deadcycles} is cleared to zero like an ordinary \cs{shipout}
% would do.
%
% \begin{declcs}{AtBeginShipoutInit}
% \end{declcs}
% Usually the redefinition of \cs{shipout} is delayed by
% \cs{AtBeginDocument} (if this macro exists). This can be
% too late, if other packages also redefines \cs{shipout}
% and the order does matter.
% \cs{AtBeginShipoutInit} forces the immediate redefinition
% of \cs{shipout}.
%
% \begin{declcs}{AtBeginShipoutAddToBox} \M{stuff}\\
%   \cs{AtBeginShipoutAddToBoxForeground} \M{stuff}
% \end{declcs}
% A quite common use case is the addition of \cs{special}
% or other whatsits to the page output box. Macro
% \cs{AtBeginShipoutAddToBox} puts \meta{stuff} in a
% box with zeroed dimensions. The box with the \meta{stuff}
% is put in the upper left corner of the shipout box
% \cs{AtBeginShipoutBox}. Macro \cs{AtBeginShipoutAddToBox}
% puts the \meta{stuff} in the background, the other macro
% \cs{AtBeginShipoutAddToBoxForeground} in the foreground
% after the original shipout box contents is set.
%
% A void shipout box (that means a discarded page)
% remains void that means \meta{stuff} is ignored in this case.
% The box type of \cs{AtBeginShipoutBox} is preserved.
% Also the box nesting level for the original contents
% of \cs{AtBeginShipoutBox} remains, for example, to avoid trouble
% with links across pages in case of \hologo{pdfTeX}.
%
% \begin{declcs}{AtBeginShipoutUpperLeft} \M{background material}
% \end{declcs}
% This is a macro that puts material in the background of
% box \cs{AtBeginShipoutBox}.
% The \meta{background material} is set in an \cs{hbox}, the
% reference point is the upper left corner of the output page.
% In case of \pdfTeX\ in PDF mode, the settings of
% \cs{pdfhorigin} and \cs{pdfvorigin} are respected.
%
% The macro \cs{AtBeginShipoutUpperLeft} is intended to be used
% in one of the hook setting macros, such as
% \cs{AtBeginShipout}, \cs{AtBeginShipoutFirst}, or
% \cs{AtBeginShipoutNext}.
%
% For \LaTeX\ users the \meta{background material} is set
% inside a \texttt{picture} environment:
% \begin{quote}
%   |\begin{picture}(0,0)|\\
%   \mbox{}\quad |\setlength{\unitlength}{1pt}%|\\
%   \mbox{}\quad \meta{background material}\\
%   |\end{picture}|
% \end{quote}
%
% \begin{declcs}{AtBeginShipoutUpperLeftForeground} \M{foreground material}
% \end{declcs}
% See \cs{AtBeginShipoutUpperLeft}. The difference is that the material
% is put in the foreground.
%
% \begin{declcs}{AtBeginShipoutOriginalShipout} \meta{box}
% \end{declcs}
% It stores the meaning of \cs{shipout} at the time this package
% is loaded.
%
% \begin{declcs}{AtBeginShipoutBoxWidth}\\
%   \cs{AtBeginShipoutBoxHeight}\\
%   \cs{AtBeginShipoutBoxDepth}\\
% \end{declcs}
% These macros store the dimensions of the output box \cs{AtBeginShipoutBox}
% before the original shipout is called. If \cs{shipout} is not
% redefined before the package loading or the box dimensions
% are not changed by the redefined \cs{shipout}, these macros
% contain the dimensions of the shipout box. These values can be
% remembered by \cs{label} and \cs{ref}. For example, this is done
% by the package module \xpackage{zref-pagelayout} of project \xpackage{zref}.
% The dimensions of the shipout page can be used in some \hologo{TeX}
% engines (\hologo{pdfTeX} in PDF mode, \hologo{XeTeX}) to calculate
% the media size of the shipout page if \cs{pdfpagewidth} and
% \cs{pdfpageheight} are not set.
%
% \subsection{Examples}
%
% \subsubsection{Example: circle in background}
%
% In this example we put a circle in the background in the middle
% of the paper.
%
%    \begin{macrocode}
%<*example1>
%    \end{macrocode}
%    \begin{macrocode}
\documentclass[a4paper]{article}
\usepackage{color}
\usepackage{atbegshi}
%    \end{macrocode}
%    Package \xpackage{picture} makes life a little easier, because
%    we can now also use length specifications in |picture|'s commands.
%    \begin{macrocode}
\usepackage{picture}
%    \end{macrocode}
%    Now we draw the circle in the middle of the paper.
%    \cs{put} moves downwards, because the origin is at the top
%    of the page, not at its bottom.
%    \begin{macrocode}
\AtBeginShipout{%
  \AtBeginShipoutUpperLeft{%
    \put(0.5\paperwidth,-0.5\paperheight){\circle{10}}%
  }%
}
\begin{document}
\section{Hello World}
\newpage
\AtBeginShipoutNext{%
  \AtBeginShipoutUpperLeft{%
    \color{red}%
    \put(0,-0.5\paperheight){\line(1,0){\paperwidth}}%
    \put(0.5\paperwidth, 0){\line(0,-1){\paperheight}}%
  }%
}
Only on this page we add a red cross.
\newpage
This page has the circle only.
\par
\vspace{\fill}
The next page will be discarded.
\newpage
\AtBeginShipoutNext{%
  \AtBeginShipoutDiscard
}
This page is discarded.
\newpage
The last page.
\end{document}
%    \end{macrocode}
%    \begin{macrocode}
%</example1>
%    \end{macrocode}
%
% \subsubsection{Example: adding TrimBox for dvipdfmx}
%
% Now an example from ``real life'' follows. Someone from the mailing  list
% for dvipdfmx wants to put a TrimBox on every page. If we use
% \cs{AtBeginShipout}, we have to put the \cs{special} inside the
% box \cs{AtBeginShipoutBox} that gets shipped out.
%    \begin{macrocode}
%<*example2>
\documentclass{minimal}
\usepackage{atbegshi}
\usepackage[
  dvipdfm,
  paperwidth=630bp,
  paperheight=810bp
]{geometry}
\AtBeginShipout{%
  \setbox\AtBeginShipoutBox=\hbox{%
    \special{pdf: put @thispage <</TrimBox[9 9 621 801]>>}%
    \box\AtBeginShipoutBox
  }%
}
\begin{document}
  First page
  \newpage
  Second page
\end{document}
%</example2>
%    \end{macrocode}
% Remember, in \cs{AtBeginShipoutBoxFirst} the \cs{setbox} wrapper code
% is implicitly given and the \cs{special} is used directly.
%
% \section{Method of \cs{shipout} overloading}
%
% \subsection{\cs{shipout}}
%
% The \TeX\ primitive command \cs{shipout} takes a box specification
% and puts the box as a new page in the output file.
% There are two kinds of box specifications:
% \begin{description}
% \item[Direct boxes:]
%   They are given by \cs{hbox}, \cs{vbox}, or \cs{vtop},\\
%   e.g. |\shipout\hbox{Hello World}|.
% \item[Indirect boxes:]
%   \cs{box} or \cs{copy} references a box register by number.
%   The box register contains the contents of the box.
%
%   \emph{Note:} \cs{box} also clears the box register globally.
% \end{description}
% Then we have to differentiate between void and empty boxes:
% \begin{description}
% \item[Void:]
%   Initially or after \cs{box} there is no box in the box register.
%   In this cases the box register is not empty, but \emph{void}.
% \item[Empty:]
%   A box with empty contents, such as
%   |\hbox{}| (= \cs{null}) or |\vbox{}| is an \emph{empty \cs{hbox}}
%   or \emph{empty \cs{vbox}}. If a box register holds such a
%   box, the box still exists, therefore the box register is
%   \emph{not void}.
% \end{description}
%
% \subsection{\cs{afterassignment}}
%
% We want to overload \cs{shipout} to do something with the box.
% It is quite impossible to do this reliable by catching the box
% using macro arguments. The
% variety of box specifications is too large, Examples:
% \begin{quote}
%   |\shipout\null|\\
%   |\shipout\vbox{...}|\\
%   |\shipout\vtop\bgroup ...\egroup|\\
%   |\shipout\box255|
% \end{quote}
% Even worse, the braces don't need to be balanced:
% \begin{quote}
%   |\shipout\hbox\bgroup}|\\
%   |\shipout\vbox{\egroup|
% \end{quote}
% Happily TeX provides a reliable way via \cs{afterassignment}.
% It takes a macro name and executes it just after the assignment.
%
% Now we can redefine \cs{shipout}. The box specification that
% follows \cs{shipout} is catched by \cs{setbox}. This is an
% assignment to a box register. \cs{afterassignment}
% notifies \TeX, that we want to call \cs{@test} right after
% the assignment:
%\begin{quote}
%\begin{verbatim}
%\shipout :=
%  \afterassignment\@test
%  \setbox\mybox=
%\end{verbatim}
%\end{quote}
% We have seen different box specifications. Indirect boxes are easy
% to understand:
% \begin{quote}
%   |\shipout\box0| $\Rightarrow$ |\setbox\mybox=\box0 |\fbox{\cs{@test}}
% \end{quote}
% However direct boxes can have arbitrary contents with lots of
% other assignments. It would be quite inpredictable if \TeX\ would
% put |\@test| after the first of such an assignment or after the
% box specification if the box lacks of assignments.
% Therefore \TeX\ puts |\@test| right at the beginning of the
% box specification, e.g:
% \begin{quote}
%   |\shipout\hbox{Hello World}|\\
%   $\Rightarrow$
%   |\setbox\mybox=\hbox{|\fbox{\cs{@test}}| Hello World}|
% \end{quote}
%
% \subsection{Test for direct or indirect boxes}
%
% Now we want to execute \cs{@test}, but where are we?
% We can be after the completed box assignment, if \cs{shipout}
% was called with an indirect box. Or we are right at the beginning
% of a direct box.
%
% \subsubsection{With \eTeX}
%
% With the \eTeX's extensions the answer is very easy:
% Being inside the direct box means that we are inside a new group.
% The new primitive command \cs{currentgrouplevel} tells how
% deeply the groups are currently nested. Macro \cs{@test} just
% compares the previously stored group level with the current one:
%\begin{quote}
%\begin{verbatim}
%\shipout :=
%  \edef\saved@grouplevel{\number\currentgrouplevel}
%  \afterassignment\@test
%  \setbox\mybox=
%
%\@test :=
%  \ifnum\saved@grouplevel=\currentgrouplevel
%    % case: indirect box, the assignment is completed
%    \@output
%  \else
%    % case: direct box, we are inside the box
%    \aftergroup\@outbox
%  \fi
%\end{verbatim}
%\end{quote}
%
% \subsubsection{Without \eTeX}
%
% Life becomes complicate without \eTeX. We cannot ask the
% group level. However, if we are inside a direct box,
% the box register \cs{mybox} is not yet changed by \cs{setbox}.
% Thus we need a special initial value and compare
% it in \cs{@test} with the current value of the box.
%
% What can be used as initial value? Arbitrary box contents cannot
% be compared. \TeX\ only tells us a few properties:
% \begin{itemize}
% \item Box type:\cs{ifhbox}, \cs{ifvbox}
% \item Dimensions: \cs{wd}, \cs{ht}, \cs{dp}
% \item Voidness: \cs{ifvoid}
% \end{itemize}
% Unhappily all these qualities even combined are not sufficient
% for constructing an initial box value, because \cs{shipout} can be
% called with a box that is accidently just the same as the choosen
% initial value.
%
% Nevertheless we have two alternatives for an initial value:
% \begin{itemize}
% \item A box of some type with some funny settings that are
% unlikely to occur in real life, e.g a height of |4911sp-\maxdimen|.
% \item A void box.
% \end{itemize}
% A collision between this initial value and an indirect \cs{shipout}
% box with just the same value is possible.
% Then \cs{@test} will make a wrong decision that it is executed
% inside a direct box and delays \cs{@output} by \cs{aftergroup}.
% Thus \cs{@output} is not called at the place we want.
% In contrary, the result is an uncertainty about the place:
% \begin{itemize}
% \item \cs{shipout} is used in a group that perhaps closes some
%   pages later. A bad place for \cs{@output}.
% \item Without a surrounding group \cs{aftergroup} effectively kills
%   its argument.
% \end{itemize}
% In the first case of a box with special dimensions we can even loose
% the page. However in the case of the void box, this effect is even
% desired, because the original \cs{shipout} does not output void boxes.
% All we have to do is to ensure that our box \cs{mybox} is always
% void except for the phase when the overloaded \cs{shipout} is executed.
% And secondly we must keep this semantics of \cs{shipout} for the void case
% in our macros, namely \cs{@output}.
%\begin{quote}
%\begin{verbatim}
%\shipout :=
%  % trick to get a void box \mybox
%  \begingroup
%    \setbox\mybox=\box\mybox
%  \endgroup
%  \afterassignment\@test
%  \setbox\mybox=
%
%\@test :=
%  \ifvoid\mybox
%    \aftergroup\@output
%  \else
%    \@output
%  \fi
%\end{verbatim}
%\end{quote}
% The nasty case is |\shipout\box\voidb@x| where the indirect box
% is void and that must not generate an output page.
% If a surrounding group is missing the output is ignored
% because of \cs{aftergroup}.
% Otherwise \@output is called some time
% later when the surrounding group closes.
% But \cs{mybox} is void outside the execution phase
% of the redefined \cs{shipout}. Also \cs{@output} checks for a
% void box and cancels the page output.
% The disadvantage remains that the hook in \cs{@output} is called
% for a page that will not be output.
%
% \subsubsection{\cs{lastkern} method}
%
% At the beginning of a new box, there is no \cs{kern}, the contents
% of the box is still empty and \cs{lastkern} returns 0\,pt.
% This can be used to distinguish between direct and indirect boxes:
% We execute \cs{setbox} in a box with a preceding non-zero kern.
% After an indirect box, \cs{lastkern} sees this kern, otherwise
% it returns 0\,pt.
%\begin{quote}
%\begin{verbatim}
%\shipout :=
%  \begingroup
%    \setbox\mybox=\hbox\bgroup
%      \kern1pt
%      \afterassignment\shipout@test
%      \global\setbox\mybox=
%\@test :=
%  \ifdim\lastkern=0pt
%    % direct box
%    \aftergroup\egroup
%    \aftergroup\endgroup
%    \aftergroup\@output
%  \else
%    \egroup
%    \endgroup
%    \@output
%  \fi
%\end{verbatim}
%\end{quote}
% We have two \cs{setbox} commands. The first creates a controlled
% context box where we can safely insert a \cs{kern}. We get rid of
% this temporarily used context box by putting the local \cs{setbox}
% in a group.
%
% After the group we want to have our shipout box in \cs{mybox}.
% Therefore we use a global assignment here.
%
% \subsection{Output}
%
% With or without \eTeX\ we ensure the original behaviour of
% \cs{shipout} that void boxes do not generate output pages.
%
% Now we can place the hook \cs{@hook} for the user code that
% wants to manipulate the output box.
%\begin{quote}
%\begin{verbatim}
%\@output :=
%  \ifvoid\mybox
%    % cancel output of void box
%  \else
%    \@hook
%    \ifvoid\mybox
%      % user code in \@hook could has voided the box
%    \else
%      \original@shipout\box\mybox
%    \fi
%  \fi
%\end{verbatim}
%\end{quote}
%
% \subsection{Separate box register}
%
% So far we have said nothing about the box number of \cs{mybox}.
% The following case that outputs the same page twice
% shows that we are not free in the use
% of the box register:
% \begin{quote}
%   |\shipout\copy<num> \shipout\box<num>|
% \end{quote}
% We manipulate the box by the hook and without \eTeX\ the box
% must even be voided. However, the use case above requires that
% the box contents does not change at all. Therefore we must
% reserve a separate box register to avoid collisions with
% user box registers.
%
% \emph{Note:} Box register number 255 is special for the output
% routine, because TeX complains if this box is not voided
% by the output routine.
% However, this requirement does not apply to \cs{shipout} at all.
% In fact \cs{shipout} does not change any box register. This is
% usually done by a call of \cs{box}, but the output routine
% can do it later \emph{after} invoking of \cs{shipout}.
%
% \subsection{Summary}
%
% \subsubsection{With \eTeX}
%
% Putting the pieces together we get for \eTeX:
%\begin{quote}
%\begin{verbatim}
%\newbox\mybox
%\let\original@shipout\shipout
%
%\shipout :=
%  \edef\saved@grouplevel{\number\currentgrouplevel}
%  \afterassignment\@test
%  \setbox\mybox=
%
%\@test :=
%  \ifnum\saved@grouplevel<\currentgrouplevel
%    \expandafter\aftergroup
%  \fi
%  \@output
%
%\@output :=
%  \ifvoid\mybox
%    % cancel output of void box
%  \else
%    \@hook
%    \ifvoid\mybox
%      % user code in \@hook could have voided the box
%    \else
%      \original@shipout\box\mybox
%    \fi
%  \fi
%\end{verbatim}
%\end{quote}
%
% \subsubsection{Without \eTeX, traditional way}
%
% And for \TeX\ without \eTeX:
%\begin{quote}
%\begin{verbatim}
%\newbox\mybox
%\begingroup
%   \setbox\mybox=\box\mybox % ensure \mybox is void
%\endgroup
%\let\original@shipout\shipout
%
%\shipout :=
%  % trick to get a void box \mybox
%  \begingroup
%    \setbox\mybox=\box\mybox
%  \endgroup
%  \afterassignment\@test
%  \setbox\mybox=
%
%\@test :=
%  \ifvoid\mybox
%    \expandafter\aftergroup
%  \fi
%  \@output
%
%\@output :=
%  \ifvoid\mybox
%    % cancel output of void box
%  \else
%    \@hook
%    \ifvoid\mybox
%      % user code in \@hook could have voided the box
%    \else
%      \original@shipout\box\mybox
%    \fi
%  \fi
%\end{verbatim}
%\end{quote}
%
% \subsubsection{\cs{lastkern} method}
%
% And for \TeX\ without \eTeX\ using the \cs{lastkern} method:
%\begin{quote}
%\begin{verbatim}
%\newbox\mybox
%\let\original@shipout\shipout
%
%\shipout :=
%  \begingroup
%  \setbox\mybox=\hbox\bgroup
%  \kern1pt
%  \afterassignment\@test
%  \setbox\mybox=
%
%\@test :=
%  \ifdim\lastkern=0pt
%    \expandafter\aftergroup
%  \fi
%  \@output
%
%\@output :=
%  \egroup
%  \endgroup
%  \ifvoid\mybox
%    % cancel output of void box
%  \else
%    \@hook
%    \ifvoid\mybox
%      % user code in \@hook could have voided the box
%    \else
%      \original@shipout\box\mybox
%    \fi
%  \fi
%\end{verbatim}
%\end{quote}
%
%
% \StopEventually{
% }
%
% \section{Implementation}
%
% Package \xpackage{atbegshi} uses \eTeX's \cs{currentgrouplevel},
% if it is available. Otherwise the \cs{lastkern} method is used.
%
%    \begin{macrocode}
%<*package>
%    \end{macrocode}
%
% \subsection{Reload check and package identification}
%    Reload check, especially if the package is not used with \LaTeX.
%    \begin{macrocode}
\begingroup\catcode61\catcode48\catcode32=10\relax%
  \catcode13=5 % ^^M
  \endlinechar=13 %
  \catcode35=6 % #
  \catcode39=12 % '
  \catcode44=12 % ,
  \catcode45=12 % -
  \catcode46=12 % .
  \catcode58=12 % :
  \catcode64=11 % @
  \catcode123=1 % {
  \catcode125=2 % }
  \expandafter\let\expandafter\x\csname ver@atbegshi.sty\endcsname
  \ifx\x\relax % plain-TeX, first loading
  \else
    \def\empty{}%
    \ifx\x\empty % LaTeX, first loading,
      % variable is initialized, but \ProvidesPackage not yet seen
    \else
      \expandafter\ifx\csname PackageInfo\endcsname\relax
        \def\x#1#2{%
          \immediate\write-1{Package #1 Info: #2.}%
        }%
      \else
        \def\x#1#2{\PackageInfo{#1}{#2, stopped}}%
      \fi
      \x{atbegshi}{The package is already loaded}%
      \aftergroup\endinput
    \fi
  \fi
\endgroup%
%    \end{macrocode}
%    Package identification:
%    \begin{macrocode}
\begingroup\catcode61\catcode48\catcode32=10\relax%
  \catcode13=5 % ^^M
  \endlinechar=13 %
  \catcode35=6 % #
  \catcode39=12 % '
  \catcode40=12 % (
  \catcode41=12 % )
  \catcode44=12 % ,
  \catcode45=12 % -
  \catcode46=12 % .
  \catcode47=12 % /
  \catcode58=12 % :
  \catcode64=11 % @
  \catcode91=12 % [
  \catcode93=12 % ]
  \catcode123=1 % {
  \catcode125=2 % }
  \expandafter\ifx\csname ProvidesPackage\endcsname\relax
    \def\x#1#2#3[#4]{\endgroup
      \immediate\write-1{Package: #3 #4}%
      \xdef#1{#4}%
    }%
  \else
    \def\x#1#2[#3]{\endgroup
      #2[{#3}]%
      \ifx#1\@undefined
        \xdef#1{#3}%
      \fi
      \ifx#1\relax
        \xdef#1{#3}%
      \fi
    }%
  \fi
\expandafter\x\csname ver@atbegshi.sty\endcsname
\ProvidesPackage{atbegshi}%
  [2016/06/09 v1.18 At begin shipout hook (HO)]%
%    \end{macrocode}
%
% \subsection{Catcodes}
%
%    \begin{macrocode}
\begingroup\catcode61\catcode48\catcode32=10\relax%
  \catcode13=5 % ^^M
  \endlinechar=13 %
  \catcode123=1 % {
  \catcode125=2 % }
  \catcode64=11 % @
  \def\x{\endgroup
    \expandafter\edef\csname AtBegShi@AtEnd\endcsname{%
      \endlinechar=\the\endlinechar\relax
      \catcode13=\the\catcode13\relax
      \catcode32=\the\catcode32\relax
      \catcode35=\the\catcode35\relax
      \catcode61=\the\catcode61\relax
      \catcode64=\the\catcode64\relax
      \catcode123=\the\catcode123\relax
      \catcode125=\the\catcode125\relax
    }%
  }%
\x\catcode61\catcode48\catcode32=10\relax%
\catcode13=5 % ^^M
\endlinechar=13 %
\catcode35=6 % #
\catcode64=11 % @
\catcode123=1 % {
\catcode125=2 % }
\def\TMP@EnsureCode#1#2{%
  \edef\AtBegShi@AtEnd{%
    \AtBegShi@AtEnd
    \catcode#1=\the\catcode#1\relax
  }%
  \catcode#1=#2\relax
}
\TMP@EnsureCode{40}{12}% (
\TMP@EnsureCode{41}{12}% )
\TMP@EnsureCode{44}{12}% ,
\TMP@EnsureCode{45}{12}% -
\TMP@EnsureCode{47}{12}% /
\TMP@EnsureCode{46}{12}% .
\TMP@EnsureCode{58}{12}% :
\TMP@EnsureCode{91}{12}% [
\TMP@EnsureCode{93}{12}% ]
\TMP@EnsureCode{94}{7}% ^(superscript)
\TMP@EnsureCode{96}{12}% `
\edef\AtBegShi@AtEnd{\AtBegShi@AtEnd\noexpand\endinput}
%    \end{macrocode}
%
% \subsection{Preparations}
%
%    \begin{macrocode}
\begingroup\expandafter\expandafter\expandafter\endgroup
\expandafter\ifx\csname RequirePackage\endcsname\relax
  \def\TMP@RequirePackage#1[#2]{%
    \begingroup\expandafter\expandafter\expandafter\endgroup
    \expandafter\ifx\csname ver@#1.sty\endcsname\relax
      \input #1.sty\relax
    \fi
  }%
  \TMP@RequirePackage{infwarerr}[2007/09/09]%
  \TMP@RequirePackage{ltxcmds}[2010/03/01]%
\else
  \RequirePackage{infwarerr}[2007/09/09]%
  \RequirePackage{ltxcmds}[2010/03/01]%
\fi
%    \end{macrocode}
%
%    \begin{macro}{\AtBegShi@CheckDefinable}
%    \begin{macrocode}
\begingroup\expandafter\expandafter\expandafter\endgroup
\expandafter\ifx\csname @ifdefinable\endcsname\relax
  \def\AtBegShi@CheckDefinable#1{%
    \ifcase\ifx#1\relax
             \ltx@one
           \else
             \ifx#1\@undefined
               \ltx@one
             \else
               \ltx@zero
             \fi
           \fi
      \@PackageError{atbegshi}{%
        \string#1\space is already defined%
      }\@ehd
    \fi
  }%
\else
  \def\AtBegShi@CheckDefinable#1{%
    \@ifdefinable{#1}{}%
  }%
\fi
%    \end{macrocode}
%    \end{macro}
%
%    \begin{macro}{\ifAtBegShi@Discarded}
%    \begin{macrocode}
\ltx@newif\ifAtBegShi@Discarded
%    \end{macrocode}
%    \end{macro}
%    \begin{macro}{\AtBeginShipoutDiscard}
%    \begin{macrocode}
\AtBegShi@CheckDefinable\AtBeginShipoutDiscard
\def\AtBeginShipoutDiscard{%
  \deadcycles=\ltx@zero
  \global\AtBegShi@Discardedtrue
}
%    \end{macrocode}
%    \end{macro}
%
%    \begin{macrocode}
\begingroup\expandafter\expandafter\expandafter\endgroup
\expandafter\ifx\csname currentgrouplevel\endcsname\relax
  \catcode`X=9  % ignore
  \catcode`E=14 % comment
\else
  \catcode`X=14 % comment
  \catcode`E=9  % ignore
\fi
%    \end{macrocode}
%
%    \begin{macro}{\AtBegShi@Shipout}
%    \begin{macrocode}
\def\AtBegShi@Shipout{%
X \begingroup
X \setbox\AtBeginShipoutBox=\hbox\bgroup
X \kern\p@
E \edef\AtBegShi@GroupLevel{\number\currentgrouplevel}%
  \afterassignment\AtBegShi@Test
X \global
  \setbox\AtBeginShipoutBox=%
}
%    \end{macrocode}
%    \end{macro}
%
%    \begin{macro}{\AtBegShi@Test}
%    \begin{macrocode}
\def\AtBegShi@Test{%
X \ifdim\lastkern=0pt %
E \ifnum\AtBegShi@GroupLevel<\currentgrouplevel
    \expandafter\aftergroup
  \fi
  \AtBegShi@Output
}
%    \end{macrocode}
%    \end{macro}
%
%    \begin{macro}{\AtBegShi@Output}
%    \begin{macrocode}
\def\AtBegShi@Output{%
X \egroup
X \endgroup
  \ifvoid\AtBeginShipoutBox
    \@PackageWarning{atbegshi}{Ignoring void shipout box}%
  \else
    \let\AtBegShi@OrgProtect\protect
    \csname set@typeset@protect\endcsname
    \global\AtBegShi@Discardedfalse
    \AtBegShi@Hook
    \expandafter\gdef\expandafter\AtBegShi@HookNext
    \expandafter{\expandafter}%
    \AtBegShi@HookNext
    \ifAtBegShi@Discarded
      \@PackageInfoNoLine{atbegshi}{Shipout page discarded}%
      \global\AtBegShi@Discardedfalse
      \begingroup
        \setbox\AtBeginShipoutBox\box\AtBeginShipoutBox
      \endgroup
      \let\protect\AtBegShi@OrgProtect
    \else
      \AtBegShi@First
      \let\protect\AtBegShi@OrgProtect
      \AtBegShi@GetBoxSize\AtBeginShipoutBox
      \ltx@ifundefined{AtNextShipout}{%
      }{%
        \AtNextShipout{\AtBegShi@GetBoxSize\@cclv}%
      }%
      \AtBeginShipoutOriginalShipout\box\AtBeginShipoutBox
    \fi
  \fi
}
%    \end{macrocode}
%    \end{macro}
%
%    \begin{macro}{\AtBegShi@GetBoxSize}
%    \begin{macrocode}
\def\AtBegShi@GetBoxSize#1{%
  \xdef\AtBeginShipoutBoxWidth{\the\wd#1}%
  \xdef\AtBeginShipoutBoxHeight{\the\ht#1}%
  \xdef\AtBeginShipoutBoxDepth{\the\dp#1}%
}
%    \end{macrocode}
%    \end{macro}
%    \begin{macro}{\AtBeginShipoutBoxWidth}
%    \begin{macrocode}
\def\AtBeginShipoutBoxWidth{0pt}
%    \end{macrocode}
%    \end{macro}
%    \begin{macro}{\AtBeginShipoutBoxHeight}
%    \begin{macrocode}
\def\AtBeginShipoutBoxHeight{0pt}
%    \end{macrocode}
%    \end{macro}
%    \begin{macro}{\AtBeginShipoutBoxDepth}
%    \begin{macrocode}
\def\AtBeginShipoutBoxDepth{0pt}
%    \end{macrocode}
%    \end{macro}
%
%    \begin{macrocode}
\catcode`\X=11 %
\catcode`\E=11 %
%    \end{macrocode}
%
%    \begin{macro}{\AtBegShi@First}
%    \begin{macrocode}
\def\AtBegShi@First{%
  \ifx\AtBegShi@HookFirst\ltx@empty
  \else
    \AtBeginShipoutAddToBox{\AtBegShi@HookFirst}%
  \fi
  \global\let\AtBegShi@First\ltx@empty
  \global\let\AtBeginShipoutFirst\AtBegShi@FirstDisabled
}
%    \end{macrocode}
%    \end{macro}
%
%    \begin{macro}{\AtBegShi@Hook}
%    \begin{macrocode}
\gdef\AtBegShi@Hook{}
%    \end{macrocode}
%    \end{macro}
%    \begin{macro}{\AtBegShi@HookNext}
%    \begin{macrocode}
\gdef\AtBegShi@HookNext{}
%    \end{macrocode}
%    \end{macro}
%    \begin{macro}{\AtBegShi@HookFirst}
%    \begin{macrocode}
\gdef\AtBegShi@HookFirst{}
%    \end{macrocode}
%    \end{macro}
%    \begin{macro}{\AtBeginShipout}
%    \begin{macrocode}
\AtBegShi@CheckDefinable\AtBeginShipout
\def\AtBeginShipout{%
  \AtBegShi@AddHook\AtBegShi@Hook
}
%    \end{macrocode}
%    \end{macro}
%    \begin{macro}{\AtBeginShipoutNext}
%    \begin{macrocode}
\AtBegShi@CheckDefinable\AtBeginShipoutNext
\def\AtBeginShipoutNext{%
  \AtBegShi@AddHook\AtBegShi@HookNext
}
%    \end{macrocode}
%    \end{macro}
%    \begin{macro}{\AtBeginShipoutFirst}
%    \begin{macrocode}
\AtBegShi@CheckDefinable\AtBeginShipoutFirst
\def\AtBeginShipoutFirst{%
  \AtBegShi@AddTo\AtBegShi@HookFirst
}
%    \end{macrocode}
%    \end{macro}
%    \begin{macro}{\AtBegShi@FirstDisabled}
%    \begin{macrocode}
\long\def\AtBegShi@FirstDisabled#1{%
  \@PackageWarning{atbegshi}{%
    First page is already shipped out, ignoring\MessageBreak
    \string\AtBeginShipoutFirst
  }%
}
%    \end{macrocode}
%    \end{macro}
%
%    \begin{macro}{\AtBegShi@AddTo}
%    \begin{macrocode}
\begingroup\expandafter\expandafter\expandafter\endgroup
\expandafter\ifx\csname g@addto@macro\endcsname\relax
  \long\def\AtBegShi@AddTo#1#2{%
    \begingroup
      \toks\ltx@zero\expandafter{#1#2}%
      \xdef#1{\the\toks\ltx@zero}%
    \endgroup
  }%
\else
  \let\AtBegShi@AddTo\g@addto@macro
\fi
%    \end{macrocode}
%    \end{macro}
%
%    \begin{macro}{\AtBegShi@AddHook}
%    \begin{macrocode}
\long\def\AtBegShi@AddHook#1#2{%
  \AtBegShi@AddTo#1{\AtBegShi@Item{#2}}%
}
%    \end{macrocode}
%    \end{macro}
%
%    \begin{macro}{\AtBegShi@Item}
%    \begin{macrocode}
\long\def\AtBegShi@Item#1{%
  \ifAtBegShi@Discarded
  \else
    #1%
    \ifAtBegShi@Discarded
    \else
      \ifvoid\AtBeginShipoutBox
        \@PackageWarning{atbegshi}{%
          Shipout box was voided by hook,\MessageBreak
          ignoring shipout box%
        }%
        \AtBeginShipoutDiscard
      \fi
    \fi
  \fi
}
%    \end{macrocode}
%    \end{macro}
%
%    \begin{macro}{\AtBeginShipoutInit}
%    \begin{macrocode}
\AtBegShi@CheckDefinable\AtBeginShipoutInit
\def\AtBeginShipoutInit{%
  \ltx@IfUndefined{newbox}{%
    \@PackageError{atbegshi}{%
      \string\AtBeginShipoutInit\space failed\MessageBreak
       because of missing \expandafter\string\csname newbox\endcsname
    }\@ehc
  }{%
    \csname newbox\endcsname\AtBeginShipoutBox
    \AtBegShi@CheckDefinable\AtBeginShipoutOriginalShipout
    \global\let\AtBeginShipoutOriginalShipout\shipout
    \global\let\shipout\AtBegShi@Shipout
  }%
  \gdef\AtBeginShipoutInit{}%
}
%    \end{macrocode}
%    \end{macro}
%    \begin{macrocode}
\begingroup\expandafter\expandafter\expandafter\endgroup
\expandafter\ifx\csname AtBeginDocument\endcsname\relax
  \AtBeginShipoutInit
\else
  \AtBeginDocument{\AtBeginShipoutInit}%
\fi
%    \end{macrocode}
%
% \subsection{Additions to the shipout box}
%
%    \begin{macro}{\AtBeginShipoutAddToBox}
%    \begin{macrocode}
\def\AtBeginShipoutAddToBox#1{%
  \ifhbox\AtBeginShipoutBox
    \edef\AtBegShi@restore{%
      \hfuzz=\the\hfuzz\relax
      \hbadness=\the\hbadness\relax
    }%
    \hfuzz=1073741823sp\relax
    \hbadness=2147483647\relax
    \setbox\AtBeginShipoutBox=\hbox to \wd\AtBeginShipoutBox{%
      \setbox\ltx@zero=\hbox{%
        \begingroup
          \AtBegShi@restore
          #1%
        \endgroup
      }%
      \wd\ltx@zero=0pt\relax
      \ht\ltx@zero=0pt\relax
      \dp\ltx@zero=0pt\relax
      \raise\ht\AtBeginShipoutBox\copy\ltx@zero
      \unhcopy\AtBeginShipoutBox
    }%
    \AtBegShi@restore
  \else
    \ifvbox\AtBeginShipoutBox
      \edef\AtBegShi@restore{%
        \vfuzz=\the\vfuzz\relax
        \vbadness=\the\vbadness\relax
        \dimen\ltx@zero=\the\dimen\ltx@zero\relax
      }%
      \edef\AtBegShi@restorebox{%
        \ht\AtBeginShipoutBox=\the\ht\AtBeginShipoutBox\relax
        \dp\AtBeginShipoutBox=\the\dp\AtBeginShipoutBox\relax
      }%
      \vfuzz=1073741823sp\relax
      \vbadness=2147483647\relax
      \dimen\ltx@zero=\ht\AtBeginShipoutBox
      \advance\dimen\ltx@zero by \dp\AtBeginShipoutBox
      \setbox\AtBeginShipoutBox=\vbox to \dimen\ltx@zero{%
        \setbox\ltx@zero=\hbox{%
          \begingroup
            \AtBegShi@restore
            #1%
          \endgroup
        }%
        \wd\ltx@zero=0pt\relax
        \ht\ltx@zero=0pt\relax
        \dp\ltx@zero=0pt\relax
        \baselineskip=0pt\relax
        \lineskip=0pt\relax
        \lineskiplimit=0pt\relax
        \copy\ltx@zero
        \unvbox\AtBeginShipoutBox
        \kern0pt%
      }%
      \AtBegShi@restore
      \AtBegShi@restorebox
    \fi
  \fi
}
%    \end{macrocode}
%    \end{macro}
%
%    \begin{macro}{\AtBeginShipoutAddToBoxForeground}
%    \begin{macrocode}
\def\AtBeginShipoutAddToBoxForeground#1{%
  \ifhbox\AtBeginShipoutBox
    \edef\AtBegShi@restore{%
      \hfuzz=\the\hfuzz\relax
      \hbadness=\the\hbadness\relax
    }%
    \hfuzz=1073741823sp\relax
    \hbadness=2147483647\relax
    \setbox\AtBeginShipoutBox=\hbox to \wd\AtBeginShipoutBox{%
      \unhcopy\AtBeginShipoutBox
      \kern-\wd\AtBeginShipoutBox
      \setbox\ltx@zero=\hbox{%
        \begingroup
          \AtBegShi@restore
          #1%
        \endgroup
      }%
      \wd\ltx@zero=0pt\relax
      \ht\ltx@zero=0pt\relax
      \dp\ltx@zero=0pt\relax
      \raise\ht\AtBeginShipoutBox\copy\ltx@zero
      \kern\wd\AtBeginShipoutBox
    }%
    \AtBegShi@restore
  \else
    \ifvbox\AtBeginShipoutBox
      \edef\AtBegShi@restore{%
        \vfuzz=\the\vfuzz\relax
        \vbadness=\the\vbadness\relax
        \dimen\ltx@zero=\the\dimen\ltx@zero\relax
      }%
      \edef\AtBegShi@restorebox{%
        \ht\AtBeginShipoutBox=\the\ht\AtBeginShipoutBox\relax
        \dp\AtBeginShipoutBox=\the\dp\AtBeginShipoutBox\relax
      }%
      \vfuzz=1073741823sp\relax
      \vbadness=2147483647\relax
      \dimen\ltx@zero=\ht\AtBeginShipoutBox
      \advance\dimen\ltx@zero by \dp\AtBeginShipoutBox
      \setbox\AtBeginShipoutBox=\vbox to \dimen\ltx@zero{%
        \setbox\ltx@zero=\hbox{%
          \begingroup
            \AtBegShi@restore
            #1%
          \endgroup
        }%
        \wd\ltx@zero=0pt\relax
        \ht\ltx@zero=0pt\relax
        \dp\ltx@zero=0pt\relax
        \baselineskip=0pt\relax
        \lineskip=0pt\relax
        \lineskiplimit=0pt\relax
        \unvbox\AtBeginShipoutBox
        \kern-\dimen\ltx@zero
        \copy\ltx@zero
        \kern\dimen\ltx@zero
      }%
      \AtBegShi@restore
      \AtBegShi@restorebox
    \fi
  \fi
}
%    \end{macrocode}
%    \end{macro}
%
% \subsection{Positioning}
%
%    \begin{macrocode}
\begingroup\expandafter\expandafter\expandafter\endgroup
\expandafter\ifx\csname RequirePackage\endcsname\relax
  \def\TMP@RequirePackage#1[#2]{%
    \begingroup\expandafter\expandafter\expandafter\endgroup
    \expandafter\ifx\csname ver@#1.sty\endcsname\relax
      \input #1.sty\relax
    \fi
  }%
  \TMP@RequirePackage{ifpdf}[2011/01/30]%
\else
  \RequirePackage{ifpdf}[2011/01/30]%
\fi
%    \end{macrocode}
%
%    \begin{macrocode}
\ifpdf
  \def\AtBegShi@horigin{%
    \ifx\pdfhorigin\@undefined\pdfvariable horigin\else\pdfhorigin\fi}%
  \def\AtBegShi@vorigin{%
    \ifx\pdfvorigin\@undefined\pdfvariable vorigin\else\pdfvorigin\fi}%
\else
  \def\AtBegShi@horigin{72.27pt}%
  \def\AtBegShi@vorigin{72.27pt}%
\fi
%    \end{macrocode}
%
%    \begin{macrocode}
\begingroup
\ifcase
    \expandafter\ifx\csname picture\endcsname\relax
      1%
    \else
      \expandafter\ifx\csname endpicture\endcsname\relax
        1%
      \else
        0%
      \fi
    \fi
  \endgroup
  \def\AtBegShi@BeginPicture{%
    \begingroup
    \picture(0,0)\relax
    \begingroup\expandafter\expandafter\expandafter\endgroup
    \expandafter\ifx\csname unitlength\endcsname\relax
    \else
      \unitlength=1pt\relax
    \fi
    \ignorespaces
  }%
  \def\AtBegShi@EndPicture{%
    \endpicture
    \endgroup
  }%
\else
  \endgroup
  \def\AtBegShi@BeginPicture{%
    \setbox\ltx@zero=\hbox\bgroup
    \begingroup
    \ignorespaces
  }%
  \def\AtBegShi@EndPicture{%
    \endgroup
    \egroup
    \ht\ltx@zero=0pt\relax
    \dp\ltx@zero=0pt\relax
    \copy\ltx@zero
  }%
\fi
%    \end{macrocode}
%
%    \begin{macro}{\AtBeginShipoutUpperLeft}
%    A surrounding \cs{rlap} is not necessary, because
%    the stuff is put in an \cs{hbox} with zero width.
%    \begin{macrocode}
\def\AtBeginShipoutUpperLeft#1{%
  \AtBeginShipoutAddToBox{%
    \kern-\AtBegShi@horigin\relax
    \vbox to 0pt{%
      \kern-\AtBegShi@vorigin\relax
      \AtBegShi@BeginPicture
      #1%
      \AtBegShi@EndPicture
      \vss
    }%
  }%
}
%    \end{macrocode}
%    \end{macro}
%
%    \begin{macro}{\AtBeginShipoutUpperLeftForeground}
%    \begin{macrocode}
\def\AtBeginShipoutUpperLeftForeground#1{%
  \AtBeginShipoutAddToBoxForeground{%
    \kern-\AtBegShi@horigin\relax
    \vbox to 0pt{%
      \kern-\AtBegShi@vorigin\relax
      \AtBegShi@BeginPicture
      #1%
      \AtBegShi@EndPicture
      \vss
    }%
  }%
}
%    \end{macrocode}
%    \end{macro}
%
% \subsection{Patches}
%
%    Patches for \LaTeX\ packages that redefine \cs{shipout}.
%    \LaTeX\ is now supposed to use \eTeX. Thus we do not patch,
%    without \LaTeX\ and \eTeX.
%    \begin{macrocode}
\def\AtBegShi@AbortIfUndefined#1{%
  \begingroup\expandafter\expandafter\expandafter\endgroup
  \expandafter\ifx\csname#1\endcsname\relax
    \expandafter\AtBegShi@AtEnd
  \fi
}
\AtBegShi@AbortIfUndefined{currentgrouplevel}%
\AtBegShi@AbortIfUndefined{AtBeginDocument}%
\AtBegShi@AbortIfUndefined{@ifpackageloaded}%
\AtBegShi@AbortIfUndefined{@ifclassloaded}%
%    \end{macrocode}
%
% \subsubsection{Package \xpackage{crop}}
%
% Fix of method and box.
%    \begin{macrocode}
\def\AtBegShi@PatchCrop{%
  \begingroup
    \def\AtBegShi@Crop@shipout{%
      \afterassignment\CROP@ship
      \setbox\@cclv=%
    }%
    \def\AtBegShi@Crop@ship{%
      \ifvoid\@cclv
        \expandafter\aftergroup
      \fi
      \CROP@@ship
    }%
    \def\AtBegShi@Crop@shiplist{%
      \lineskip\z@
      \lineskiplimit\z@
      \baselineskip\z@
      \CROP@kernel
      \box\@cclv
    }%
    \def\AtBegShi@Crop@@ship{%
      \CROP@shipout\vbox{%
        \CROP@shiplist
      }%
    }%
    \ifx\AtBegShi@Crop@ship\CROP@ship
      \ifx\AtBegShi@Crop@shiplist\CROP@shiplist
        \ifx\AtBegShi@Crop@@ship\CROP@@ship
          \let\AtBegShi@found\relax
          \ifx\shipout\AtBegShi@Crop@shipout
            \def\AtBegShi@found{\shipout}%
          \else\ifx\AtBeginShipoutOriginalShipout\AtBegShi@Crop@shipout
            \def\AtBegShi@found{\AtBeginShipoutOriginalShipout}%
          \else\ifx\@EveryShipout@Org@Shipout\AtBegShi@Crop@shipout
            \def\AtBegShi@found{\@EveryShipout@Org@Shipout}%
          \else\ifx\GPTorg@shipout\AtBegShi@Crop@shipout
            \def\AtBegShi@found{\GPTorg@shipout}%
          \else\ifx\THBorg@shipout\AtBegShi@Crop@shipout
            \def\AtBegShi@found{\THBorg@shipout}%
          \else\ifx\mem@oldshipout\AtBegShi@Crop@shipout
            \def\AtBegShi@found{\mem@oldshipout}%
          \fi\fi\fi\fi\fi\fi
          \ifx\AtBegShi@found\relax
          \else
            \expandafter\endgroup
            \expandafter\def\AtBegShi@found{%
              \edef\AtBegShi@GroupLevel{\number\currentgrouplevel}%
              \afterassignment\CROP@ship
              \setbox\AtBeginShipoutBox=%
            }%
            \def\CROP@ship{%
              \ifnum\AtBegShi@GroupLevel=\currentgrouplevel
              \else
                \expandafter\aftergroup
              \fi
              \CROP@@ship
            }%
            \def\CROP@shiplist{%
              \lineskip 0pt\relax
              \lineskiplimit 0pt\relax
              \baselineskip 0pt\relax
              \CROP@kernel
              \box\AtBeginShipoutBox
            }%
            \def\CROP@@ship{%
              \ifvoid\AtBeginShipoutBox
              \else
                \setbox\AtBeginShipoutBox=\vbox{%
                  \CROP@shiplist
                }%
                \AtBegShi@GetBoxSize\AtBeginShipoutBox
                \expandafter\CROP@shipout
                \expandafter\box
                \expandafter\AtBeginShipoutBox
              \fi
            }%
            \@PackageInfoNoLine{atbegshi}{Package `crop' patched}%
            \begingroup
          \fi
        \fi
      \fi
    \fi
  \endgroup
  \let\AtBegShi@PatchCrop\relax
}
\@ifpackageloaded{crop}{%
  \AtBegShi@PatchCrop
}{%
  \AtBeginDocument{\AtBegShi@PatchCrop}%
}
%    \end{macrocode}
%
% \subsubsection{Package \xpackage{everyshi}}
%
%     Fix of method. Use of box 255 is not changed.
%    \begin{macrocode}
\def\AtBegShi@PatchEveryshi{%
  \begingroup
    \long\def\AtBegShi@Everyshi@shipout{%
      \afterassignment\@EveryShipout@Test
      \global\setbox\@cclv= %
    }%
    \long\def\AtBegShi@Everyshi@Test{%
      \ifvoid\@cclv\relax
        \aftergroup\@EveryShipout@Output
      \else
        \@EveryShipout@Output
      \fi
    }%
    \ifx\AtBegShi@Everyshi@Test\@EveryShipout@Test
      \let\AtBegShi@found\relax
      \ifx\shipout\AtBegShi@Everyshi@shipout
        \def\AtBegShi@found{\shipout}%
      \else\ifx\AtBeginShipoutOriginalShipout\AtBegShi@Everyshi@shipout
        \def\AtBegShi@found{\AtBeginShipoutOriginalShipout}%
      \else\ifx\CROP@shipout\AtBegShi@Everyshi@shipout
        \def\AtBegShi@found{\CROP@shipout}%
      \else\ifx\GPTorg@shipout\AtBegShi@Everyshi@shipout
        \def\AtBegShi@found{\GPTorg@shipout}%
      \else\ifx\THBorg@shipout\AtBegShi@Everyshi@shipout
        \def\AtBegShi@found{\THBorg@shipout}%
      \else\ifx\mem@oldshipout\AtBegShi@Everyshi@shipout
        \def\AtBegShi@found{\mem@oldshipout}%
      \else
        \expandafter\ifx\csname @EveryShipout@Org@Shipout\endcsname
                        \relax
           \ifx\@EveryShipout@Shipout\AtBegShi@Everyshi@shipout
             \def\AtBegShi@found{\@EveryShipout@Shipout}%
           \fi
        \fi
      \fi\fi\fi\fi\fi\fi
      \ifx\AtBegShi@found\relax
      \else
        \expandafter\endgroup
        \expandafter\def\AtBegShi@found{%
          \edef\AtBegShi@GroupLevel{\number\currentgrouplevel}%
          \afterassignment\@EveryShipout@Test
          \setbox\AtBeginShipoutBox=%
        }%
        \def\@EveryShipout@Test{%
          \ifnum\AtBegShi@GroupLevel=\currentgrouplevel
          \else
            \expandafter\aftergroup
          \fi
          \AtBegShi@Everyshi@Output
        }%
        \def\AtBegShi@Everyshi@Output{%
          \ifvoid\AtBeginShipoutBox
          \else
            \global\setbox\ltx@cclv\box\AtBeginShipoutBox
            \expandafter\@EveryShipout@Output
          \fi
        }%
        \@PackageInfoNoLine{atbegshi}{Package `everyshi' patched}%
        \begingroup
      \fi
    \fi
  \endgroup
  \let\AtBegShi@PatchEveryshi\relax
}
\@ifpackageloaded{everyshi}{%
  \AtBegShi@PatchEveryshi
}{%
  \AtBeginDocument{\AtBegShi@PatchEveryshi}%
}
%    \end{macrocode}
%
% \subsubsection{Class \xpackage{memoir}}
%
%     Fix of method and box.
%    \begin{macrocode}
\def\AtBegShi@PatchMemoir{%
  \begingroup
    \def\AtBegShi@Memoir@shipout{%
      \afterassignment\mem@shipi
      \setbox\@cclv=%
    }%
    \def\AtBegShi@Memoir@shipi{%
      \ifvoid\@cclv
        \expandafter\aftergroup
      \fi
      \mem@shipii
    }%
    \def\AtBegShi@Memoir@shipiiA{%
      \mem@oldshipout\vbox{%
        \trimmarks
        \unvbox\@cclv
      }%
    }%
    \def\AtBegShi@Memoir@shipiiB{%
      \ifvoid\@cclv
        \mem@oldshipout\box\@cclv
      \else
        \mem@oldshipout\vbox{%
          \trimmarks
          \unvbox\@cclv
        }%
      \fi
    }%
    \def\AtBegShi@Memoir@PatchAB{%
      \ifvoid\AtBeginShipoutBox
      \else
        \setbox\AtBeginShipoutBox=\vbox{%
          \trimmarks
          \ifvbox\AtBeginShipoutBox
            \unvbox\AtBeginShipoutBox
          \else
            \box\AtBeginShipoutBox
          \fi
        }%
        \AtBegShi@GetBoxSize\AtBeginShipoutBox
        \expandafter\mem@oldshipout
        \expandafter\box
        \expandafter\AtBeginShipoutBox
      \fi
    }%
    \def\AtBegShi@Memoir@shipiiC{% 2008/08/07 v1.6180339a
      \ifvoid\@cclv
        \mem@oldshipout\box\@cclv
      \else
        \ifshowtrims
          \mem@oldshipout\vbox{\trimmarks\unvbox\@cclv}%
        \else
          \mem@oldshipout\box\@cclv
        \fi
      \fi
    }%
    \def\AtBegShi@Memoir@shipiiD{% 2011/03/06 v3.6j
      \ifvoid\@cclv
        \mem@oldshipout\box\@cclv
      \else
        \ifshowtrims
          \mem@oldshipout\vbox{%
            \trimmarks
            \nointerlineskip
            \box\@cclv
          }%
        \else
          \mem@oldshipout\box\@cclv
        \fi
      \fi
    }%
    \def\AtBegShi@Memoir@PatchCD{%
      \ifvoid\AtBeginShipoutBox
      \else
        \ifshowtrims
          \setbox\AtBeginShipoutBox=\vbox{%
            \trimmarks
            \nointerlineskip
            \box\AtBeginShipoutBox
          }%
        \fi
        \AtBegShi@GetBoxSize\AtBeginShipoutBox
        \expandafter\mem@oldshipout
        \expandafter\box
        \expandafter\AtBeginShipoutBox
      \fi
    }%
    \ifx\AtBegShi@Memoir@shipi\mem@shipi
      \let\AtBegShi@found\ltx@one
      \ifx\AtBegShi@Memoir@shipiiA\mem@shipii
        \let\AtBegShi@found\ltx@zero
        \global\let\AtBegShi@Memoir@PatchX\AtBegShi@Memoir@PatchAB
      \else\ifx\AtBegShi@Memoir@shipiiB\mem@shipii
        \let\AtBegShi@found\ltx@zero
        \global\let\AtBegShi@Memoir@PatchX\AtBegShi@Memoir@PatchAB
      \else\ifx\AtBegShi@Memoir@shipiiC\mem@shipii
        \let\AtBegShi@found\ltx@zero
        \global\let\AtBegShi@Memoir@PatchX\AtBegShi@Memoir@PatchCD
      \else\ifx\AtBegShi@Memoir@shipiiD\mem@shipii
        \let\AtBegShi@found\ltx@zero
        \global\let\AtBegShi@Memoir@PatchX\AtBegShi@Memoir@PatchCD
      \fi\fi\fi\fi
      \ifcase\AtBegShi@found
        \let\AtBegShi@found\relax
        \ifx\shipout\AtBegShi@Memoir@shipout
          \def\AtBegShi@found{\shipout}%
        \else\ifx\AtBeginShipoutOriginalShipout\AtBegShi@Memoir@shipout
          \def\AtBegShi@found{\AtBeginShipoutOriginalShipout}%
        \else\ifx\CROP@shipout\AtBegShi@Memoir@shipout
          \def\AtBegShi@found{\CROP@shipout}%
        \else\ifx\GPTorg@shipout\AtBegShi@Memoir@shipout
          \def\AtBegShi@found{\GPTorg@shipout}%
        \else\ifx\THBorg@shipout\AtBegShi@Memoir@shipout
          \def\AtBegShi@found{\THBorg@shipout}%
        \else\ifx\@EveryShipout@Org@Shipout\AtBegShi@Memoir@shipout
          \def\AtBegShi@found{\@EveryShipout@Org@Shipout}%
        \fi\fi\fi\fi\fi\fi
        \ifx\AtBegShi@found\relax
        \else
          \expandafter\endgroup
          \expandafter\def\AtBegShi@found{%
            \edef\AtBegShi@GroupLevel{\number\currentgrouplevel}%
            \afterassignment\mem@shipi
            \setbox\AtBeginShipoutBox=%
          }%
          \def\mem@shipi{%
            \ifnum\AtBegShi@GroupLevel=\currentgrouplevel
            \else
              \expandafter\aftergroup
            \fi
            \mem@shipii
          }%
          \let\mem@shipii\AtBegShi@Memoir@PatchX
          \@PackageInfoNoLine{atbegshi}{Class `memoir' patched}%
          \begingroup
        \fi
      \fi
    \fi
  \endgroup
  \let\AtBegShi@PatchMemoir\relax
}
\@ifclassloaded{memoir}{%
  \AtBegShi@PatchMemoir
}{%
  \AtBeginDocument{\AtBegShi@PatchMemoir}%
}
%    \end{macrocode}
%
%    \begin{macrocode}
\AtBegShi@AtEnd%
%</package>
%    \end{macrocode}
%
% \section{Test}
%
% \subsection{Catcode checks for loading}
%
%    \begin{macrocode}
%<*test1>
%    \end{macrocode}
%    \begin{macrocode}
\catcode`\{=1 %
\catcode`\}=2 %
\catcode`\#=6 %
\catcode`\@=11 %
\expandafter\ifx\csname count@\endcsname\relax
  \countdef\count@=255 %
\fi
\expandafter\ifx\csname @gobble\endcsname\relax
  \long\def\@gobble#1{}%
\fi
\expandafter\ifx\csname @firstofone\endcsname\relax
  \long\def\@firstofone#1{#1}%
\fi
\expandafter\ifx\csname loop\endcsname\relax
  \expandafter\@firstofone
\else
  \expandafter\@gobble
\fi
{%
  \def\loop#1\repeat{%
    \def\body{#1}%
    \iterate
  }%
  \def\iterate{%
    \body
      \let\next\iterate
    \else
      \let\next\relax
    \fi
    \next
  }%
  \let\repeat=\fi
}%
\def\RestoreCatcodes{}
\count@=0 %
\loop
  \edef\RestoreCatcodes{%
    \RestoreCatcodes
    \catcode\the\count@=\the\catcode\count@\relax
  }%
\ifnum\count@<255 %
  \advance\count@ 1 %
\repeat

\def\RangeCatcodeInvalid#1#2{%
  \count@=#1\relax
  \loop
    \catcode\count@=15 %
  \ifnum\count@<#2\relax
    \advance\count@ 1 %
  \repeat
}
\def\RangeCatcodeCheck#1#2#3{%
  \count@=#1\relax
  \loop
    \ifnum#3=\catcode\count@
    \else
      \errmessage{%
        Character \the\count@\space
        with wrong catcode \the\catcode\count@\space
        instead of \number#3%
      }%
    \fi
  \ifnum\count@<#2\relax
    \advance\count@ 1 %
  \repeat
}
\def\space{ }
\expandafter\ifx\csname LoadCommand\endcsname\relax
  \def\LoadCommand{\input atbegshi.sty\relax}%
\fi
\def\Test{%
  \RangeCatcodeInvalid{0}{47}%
  \RangeCatcodeInvalid{58}{64}%
  \RangeCatcodeInvalid{91}{96}%
  \RangeCatcodeInvalid{123}{255}%
  \catcode`\@=12 %
  \catcode`\\=0 %
  \catcode`\%=14 %
  \LoadCommand
  \RangeCatcodeCheck{0}{36}{15}%
  \RangeCatcodeCheck{37}{37}{14}%
  \RangeCatcodeCheck{38}{47}{15}%
  \RangeCatcodeCheck{48}{57}{12}%
  \RangeCatcodeCheck{58}{63}{15}%
  \RangeCatcodeCheck{64}{64}{12}%
  \RangeCatcodeCheck{65}{90}{11}%
  \RangeCatcodeCheck{91}{91}{15}%
  \RangeCatcodeCheck{92}{92}{0}%
  \RangeCatcodeCheck{93}{96}{15}%
  \RangeCatcodeCheck{97}{122}{11}%
  \RangeCatcodeCheck{123}{255}{15}%
  \RestoreCatcodes
}
\Test
\csname @@end\endcsname
\end
%    \end{macrocode}
%    \begin{macrocode}
%</test1>
%    \end{macrocode}
%
%    \begin{macrocode}
%<*test2>
\input atbegshi.sty\relax
\def\msg#{\immediate\write16}
\msg{File: atbegshi-test2.tex 2016/06/09 v1.18 Test file for plain-TeX}
\def\testmsg#1#2{%
  \msg{}%
  \msg{*** Test with box (#1), expected page output [#2]}% hash-ok
}

\newbox\voidbox
\def\void{\box\voidbox}
\begingroup
  \setbox\voidbox=\void
\endgroup

\count0=0\relax
\AtBeginShipout{%
  \global\advance\count0 by 1\relax
  \msg{* Inside \string\AtBeginShipout: [\the\count0]}%
}

\AtBeginShipoutFirst{%
  \msg{* Inside \string\AtBeginShipoutFirst}%
  Hello World%
}

\testmsg{\string\null}{1}
\shipout\null

\AtBeginShipoutFirst{%
  This is too late%
}

\testmsg{void}{}
\shipout\void

\testmsg{\string\copy255 (not void)}{2}
\setbox255\hbox{\vrule height 10bp width 10bp}
\shipout\copy255 %

\testmsg{\string\copy255 (again)}{3}
\shipout\copy255 %

\testmsg{\string\box255}{4}
\shipout\box255 %

\testmsg{\string\box255 (again)}{}
\shipout\box255 %

\testmsg{\string\hbox}{5}
\shipout\hbox{\vrule height 5bp width 20bp}

\testmsg{\string\vbox}{6}
\shipout\vbox{\hrule height 20bp width 5bp}

\testmsg{\string\null, voided by hook}{}
\def\VoidBox{%
  \begingroup
    \setbox\AtBeginShipoutBox=\box\AtBeginShipoutBox
  \endgroup
}
\AtBeginShipout{\VoidBox}
\shipout\null
\def\VoidBox{}

\msg{*** \string\begingroup}
\begingroup
  \testmsg{void}{}%
  \shipout\void
\msg{*** \string\endgroup}
\endgroup

\msg{*** \string\begingroup}
\begingroup
  \testmsg{void}{}%
  \shipout\void
  \testmsg{\string\null}{8}%
  \shipout\null
\msg{*** \string\endgroup}
\endgroup

\testmsg{output routine}{9}
Hello World
\vfill
\eject

\testmsg{\string\null\space(discarded)}{}
\AtBeginShipout{%
  \msg{* Inside \string\AtBeginShipout: DISCARD}%
  \AtBeginShipoutDiscard
}
\shipout\null

\end
%</test2>
%    \end{macrocode}
%
%    \begin{macrocode}
%<*test3>
\NeedsTeXFormat{LaTeX2e}
\ProvidesFile{atbegshi-test3.tex}[2016/06/09 v1.18 Test file for LaTeX]
\RequirePackage{color}
\pagecolor{yellow}
\documentclass[a5paper,showtrims]{memoir}
\usepackage{atbegshi}
\AtBeginShipout{%
  \setbox\AtBeginShipoutBox=\vbox{%
    \vbox to 0pt{%
      \kern-1.5in %
      \hbox to 0pt{%
        \kern-1.5in %
        \color{blue}%
        \rule{1in}{1in}%
        \hss
      }%
      \vss
    }%
    \hrule
    \hbox{\vrule\box\AtBeginShipoutBox\vrule}%
    \hrule
  }%
}
\usepackage{eso-pic}
\makeatletter
\@ifundefined{@EveryShipout@Init}{%
  \typeout{Test skipped}%
  \@@end
}{}
\@EveryShipout@Init
\let\@EveryShipout@Init\relax
\makeatother
\AddToShipoutPicture{%
  \hspace{.52\paperwidth}%
  \colorbox{cyan}{%
    \rule{0mm}{\paperheight}%
    \hspace{.48\paperwidth}%
  }%
}
%    \end{macrocode}
%    Newer versions of class \xclass{memoir} emulate package
%    \xpackage{crop} and prevents its loading. This is undone
%    in next line for this test file.
%    \begin{macrocode}
\expandafter\let\csname ver@crop.sty\endcsname\relax
\usepackage[color=red,cross,a4,center]{crop}
\begin{document}
\shipout\null
\shipout\box\csname voidb@x\endcsname
\section{Hello World}
\end{document}
%</test3>
%    \end{macrocode}
%
% \section{Installation}
%
% \subsection{Download}
%
% \paragraph{Package.} This package is available on
% CTAN\footnote{\url{http://ctan.org/pkg/atbegshi}}:
% \begin{description}
% \item[\CTAN{macros/latex/contrib/oberdiek/atbegshi.dtx}] The source file.
% \item[\CTAN{macros/latex/contrib/oberdiek/atbegshi.pdf}] Documentation.
% \end{description}
%
%
% \paragraph{Bundle.} All the packages of the bundle `oberdiek'
% are also available in a TDS compliant ZIP archive. There
% the packages are already unpacked and the documentation files
% are generated. The files and directories obey the TDS standard.
% \begin{description}
% \item[\CTAN{install/macros/latex/contrib/oberdiek.tds.zip}]
% \end{description}
% \emph{TDS} refers to the standard ``A Directory Structure
% for \TeX\ Files'' (\CTAN{tds/tds.pdf}). Directories
% with \xfile{texmf} in their name are usually organized this way.
%
% \subsection{Bundle installation}
%
% \paragraph{Unpacking.} Unpack the \xfile{oberdiek.tds.zip} in the
% TDS tree (also known as \xfile{texmf} tree) of your choice.
% Example (linux):
% \begin{quote}
%   |unzip oberdiek.tds.zip -d ~/texmf|
% \end{quote}
%
% \paragraph{Script installation.}
% Check the directory \xfile{TDS:scripts/oberdiek/} for
% scripts that need further installation steps.
% Package \xpackage{attachfile2} comes with the Perl script
% \xfile{pdfatfi.pl} that should be installed in such a way
% that it can be called as \texttt{pdfatfi}.
% Example (linux):
% \begin{quote}
%   |chmod +x scripts/oberdiek/pdfatfi.pl|\\
%   |cp scripts/oberdiek/pdfatfi.pl /usr/local/bin/|
% \end{quote}
%
% \subsection{Package installation}
%
% \paragraph{Unpacking.} The \xfile{.dtx} file is a self-extracting
% \docstrip\ archive. The files are extracted by running the
% \xfile{.dtx} through \plainTeX:
% \begin{quote}
%   \verb|tex atbegshi.dtx|
% \end{quote}
%
% \paragraph{TDS.} Now the different files must be moved into
% the different directories in your installation TDS tree
% (also known as \xfile{texmf} tree):
% \begin{quote}
% \def\t{^^A
% \begin{tabular}{@{}>{\ttfamily}l@{ $\rightarrow$ }>{\ttfamily}l@{}}
%   atbegshi.sty & tex/generic/oberdiek/atbegshi.sty\\
%   atbegshi.pdf & doc/latex/oberdiek/atbegshi.pdf\\
%   atbegshi-example1.tex & doc/latex/oberdiek/atbegshi-example1.tex\\
%   atbegshi-example2.tex & doc/latex/oberdiek/atbegshi-example2.tex\\
%   test/atbegshi-test1.tex & doc/latex/oberdiek/test/atbegshi-test1.tex\\
%   test/atbegshi-test2.tex & doc/latex/oberdiek/test/atbegshi-test2.tex\\
%   test/atbegshi-test3.tex & doc/latex/oberdiek/test/atbegshi-test3.tex\\
%   atbegshi.dtx & source/latex/oberdiek/atbegshi.dtx\\
% \end{tabular}^^A
% }^^A
% \sbox0{\t}^^A
% \ifdim\wd0>\linewidth
%   \begingroup
%     \advance\linewidth by\leftmargin
%     \advance\linewidth by\rightmargin
%   \edef\x{\endgroup
%     \def\noexpand\lw{\the\linewidth}^^A
%   }\x
%   \def\lwbox{^^A
%     \leavevmode
%     \hbox to \linewidth{^^A
%       \kern-\leftmargin\relax
%       \hss
%       \usebox0
%       \hss
%       \kern-\rightmargin\relax
%     }^^A
%   }^^A
%   \ifdim\wd0>\lw
%     \sbox0{\small\t}^^A
%     \ifdim\wd0>\linewidth
%       \ifdim\wd0>\lw
%         \sbox0{\footnotesize\t}^^A
%         \ifdim\wd0>\linewidth
%           \ifdim\wd0>\lw
%             \sbox0{\scriptsize\t}^^A
%             \ifdim\wd0>\linewidth
%               \ifdim\wd0>\lw
%                 \sbox0{\tiny\t}^^A
%                 \ifdim\wd0>\linewidth
%                   \lwbox
%                 \else
%                   \usebox0
%                 \fi
%               \else
%                 \lwbox
%               \fi
%             \else
%               \usebox0
%             \fi
%           \else
%             \lwbox
%           \fi
%         \else
%           \usebox0
%         \fi
%       \else
%         \lwbox
%       \fi
%     \else
%       \usebox0
%     \fi
%   \else
%     \lwbox
%   \fi
% \else
%   \usebox0
% \fi
% \end{quote}
% If you have a \xfile{docstrip.cfg} that configures and enables \docstrip's
% TDS installing feature, then some files can already be in the right
% place, see the documentation of \docstrip.
%
% \subsection{Refresh file name databases}
%
% If your \TeX~distribution
% (\teTeX, \mikTeX, \dots) relies on file name databases, you must refresh
% these. For example, \teTeX\ users run \verb|texhash| or
% \verb|mktexlsr|.
%
% \subsection{Some details for the interested}
%
% \paragraph{Attached source.}
%
% The PDF documentation on CTAN also includes the
% \xfile{.dtx} source file. It can be extracted by
% AcrobatReader 6 or higher. Another option is \textsf{pdftk},
% e.g. unpack the file into the current directory:
% \begin{quote}
%   \verb|pdftk atbegshi.pdf unpack_files output .|
% \end{quote}
%
% \paragraph{Unpacking with \LaTeX.}
% The \xfile{.dtx} chooses its action depending on the format:
% \begin{description}
% \item[\plainTeX:] Run \docstrip\ and extract the files.
% \item[\LaTeX:] Generate the documentation.
% \end{description}
% If you insist on using \LaTeX\ for \docstrip\ (really,
% \docstrip\ does not need \LaTeX), then inform the autodetect routine
% about your intention:
% \begin{quote}
%   \verb|latex \let\install=y\input{atbegshi.dtx}|
% \end{quote}
% Do not forget to quote the argument according to the demands
% of your shell.
%
% \paragraph{Generating the documentation.}
% You can use both the \xfile{.dtx} or the \xfile{.drv} to generate
% the documentation. The process can be configured by the
% configuration file \xfile{ltxdoc.cfg}. For instance, put this
% line into this file, if you want to have A4 as paper format:
% \begin{quote}
%   \verb|\PassOptionsToClass{a4paper}{article}|
% \end{quote}
% An example follows how to generate the
% documentation with pdf\LaTeX:
% \begin{quote}
%\begin{verbatim}
%pdflatex atbegshi.dtx
%makeindex -s gind.ist atbegshi.idx
%pdflatex atbegshi.dtx
%makeindex -s gind.ist atbegshi.idx
%pdflatex atbegshi.dtx
%\end{verbatim}
% \end{quote}
%
% \section{Catalogue}
%
% The following XML file can be used as source for the
% \href{http://mirror.ctan.org/help/Catalogue/catalogue.html}{\TeX\ Catalogue}.
% The elements \texttt{caption} and \texttt{description} are imported
% from the original XML file from the Catalogue.
% The name of the XML file in the Catalogue is \xfile{atbegshi.xml}.
%    \begin{macrocode}
%<*catalogue>
<?xml version='1.0' encoding='us-ascii'?>
<!DOCTYPE entry SYSTEM 'catalogue.dtd'>
<entry datestamp='$Date$' modifier='$Author$' id='atbegshi'>
  <name>atbegshi</name>
  <caption>Execute stuff at \shipout time.</caption>
  <authorref id='auth:oberdiek'/>
  <copyright owner='Heiko Oberdiek' year='2007-2011'/>
  <license type='lppl1.3'/>
  <version number='1.18'/>
  <description>
    This package is a modern reimplementation of package
    <xref refid='everyshi'>everyshi</xref>, providing various commands
    to be executed before a <tt>\shipout</tt> command.  It makes use of
    e-TeX&#x2019;s facilities if they are available.  The package may
    be used either with LaTeX or with plain TeX.
    <p/>
    The package is part of the <xref refid='oberdiek'>oberdiek</xref> bundle.
  </description>
  <documentation details='Package documentation'
      href='ctan:/macros/latex/contrib/oberdiek/atbegshi.pdf'/>
  <ctan file='true' path='/macros/latex/contrib/oberdiek/atbegshi.dtx'/>
  <miktex location='oberdiek'/>
  <texlive location='oberdiek'/>
  <install path='/macros/latex/contrib/oberdiek/oberdiek.tds.zip'/>
</entry>
%</catalogue>
%    \end{macrocode}
%
% \begin{History}
%   \begin{Version}{2007/04/17 v1.0}
%   \item
%     First version.
%   \end{Version}
%   \begin{Version}{2007/04/18 v1.1}
%   \item
%     New method based on \cs{lastkern} is used if \eTeX\ is missing.
%   \item
%     \cs{AtBeginShipoutDiscard} also resets \cs{deadcycles}.
%   \end{Version}
%   \begin{Version}{2007/04/19 v1.2}
%   \item
%     \cs{AtBeginShipoutEarly} removed for simplification reasons.
%   \item
%     Forgotten definition of \cs{AtBegShi@Info} added.
%   \item
%     Patches for packages \xpackage{crop} and \xpackage{everyshi}
%     and class \xclass{memoir} added.
%   \end{Version}
%   \begin{Version}{2007/04/26 v1.3}
%   \item
%     Use of package \xpackage{infwarerr}.
%   \item
%     Catcode section after generic header.
%   \end{Version}
%   \begin{Version}{2007/04/27 v1.4}
%   \item
%     Small optimizations.
%   \end{Version}
%   \begin{Version}{2007/06/06 v1.5}
%   \item
%     \cs{AtBeginShipoutUpperLeft} added.
%   \item
%     Example added.
%   \item
%     Fix in second test file for newer version of \xclass{memoir}.
%   \end{Version}
%   \begin{Version}{2007/09/09 v1.6}
%   \item
%     Catcode section rewritten.
%   \end{Version}
%   \begin{Version}{2008/07/18 v1.7}
%   \item
%     Documentation of \cs{AtBeginShipoutUpperLeft} fixed and extended.
%   \end{Version}
%   \begin{Version}{2008/07/19 v1.8}
%   \item
%     \cs{AtBeginShipoutUpperLeftForeground} added.
%   \end{Version}
%   \begin{Version}{2008/07/31 v1.9}
%   \item
%     Second example (TrimBox for dvipdfmx) added.
%   \item
%     No changes in package code.
%   \end{Version}
%   \begin{Version}{2009/12/02 v1.10}
%   \item
%     \cs{AtBeginShipoutOriginalShipout} added.
%   \item
%     Test file fixed.
%   \end{Version}
%   \begin{Version}{2010/03/01 v1.11}
%   \item
%     Compatibility with \iniTeX\ except for \cs{newbox}.
%   \end{Version}
%   \begin{Version}{2010/03/25 v1.12}
%   \item
%     \cs{AtBeginShipoutNext} can now be used inside
%     \cs{AtBeginShipoutNext}.
%   \end{Version}
%   \begin{Version}{2010/08/18 v1.13}
%   \item
%     Fixes for \cs{AtBegShi@CheckDefinable}.
%   \end{Version}
%   \begin{Version}{2010/12/02 v1.14}
%   \item
%     Remove the warning because of void box if the hook
%     calls \AtBeginShipoutDiscard.
%   \end{Version}
%   \begin{Version}{2011/01/30 v1.15}
%   \item
%     Already loaded package files are not input in \hologo{plainTeX}.
%   \end{Version}
%   \begin{Version}{2011/10/05 v1.16}
%   \item
%     \cs{AtBeginShipoutAddToBox}, \cs{AtBeginShipoutAddToBoxForeground} added.
%   \item
%     \cs{AtBeginShipoutBoxWidth}, \cs{AtBeginShipoutBoxHeight},
%     \cs{AtBeginShipoutBoxDepth} added.
%   \item
%     Updates for patches of class \xclass{memoir}.
%   \end{Version}
%   \begin{Version}{2016/05/16 v1.17}
%   \item
%     Documentation updates.
%   \end{Version}
%   \begin{Version}{2016/06/09 v1.18}
%   \item
%     Update for \cs{pdfhorign} in new LuaTeX.
%   \end{Version}
% \end{History}
%
% \PrintIndex
%
% \Finale
\endinput
|
% \end{quote}
% Do not forget to quote the argument according to the demands
% of your shell.
%
% \paragraph{Generating the documentation.}
% You can use both the \xfile{.dtx} or the \xfile{.drv} to generate
% the documentation. The process can be configured by the
% configuration file \xfile{ltxdoc.cfg}. For instance, put this
% line into this file, if you want to have A4 as paper format:
% \begin{quote}
%   \verb|\PassOptionsToClass{a4paper}{article}|
% \end{quote}
% An example follows how to generate the
% documentation with pdf\LaTeX:
% \begin{quote}
%\begin{verbatim}
%pdflatex atbegshi.dtx
%makeindex -s gind.ist atbegshi.idx
%pdflatex atbegshi.dtx
%makeindex -s gind.ist atbegshi.idx
%pdflatex atbegshi.dtx
%\end{verbatim}
% \end{quote}
%
% \section{Catalogue}
%
% The following XML file can be used as source for the
% \href{http://mirror.ctan.org/help/Catalogue/catalogue.html}{\TeX\ Catalogue}.
% The elements \texttt{caption} and \texttt{description} are imported
% from the original XML file from the Catalogue.
% The name of the XML file in the Catalogue is \xfile{atbegshi.xml}.
%    \begin{macrocode}
%<*catalogue>
<?xml version='1.0' encoding='us-ascii'?>
<!DOCTYPE entry SYSTEM 'catalogue.dtd'>
<entry datestamp='$Date$' modifier='$Author$' id='atbegshi'>
  <name>atbegshi</name>
  <caption>Execute stuff at \shipout time.</caption>
  <authorref id='auth:oberdiek'/>
  <copyright owner='Heiko Oberdiek' year='2007-2011'/>
  <license type='lppl1.3'/>
  <version number='1.18'/>
  <description>
    This package is a modern reimplementation of package
    <xref refid='everyshi'>everyshi</xref>, providing various commands
    to be executed before a <tt>\shipout</tt> command.  It makes use of
    e-TeX&#x2019;s facilities if they are available.  The package may
    be used either with LaTeX or with plain TeX.
    <p/>
    The package is part of the <xref refid='oberdiek'>oberdiek</xref> bundle.
  </description>
  <documentation details='Package documentation'
      href='ctan:/macros/latex/contrib/oberdiek/atbegshi.pdf'/>
  <ctan file='true' path='/macros/latex/contrib/oberdiek/atbegshi.dtx'/>
  <miktex location='oberdiek'/>
  <texlive location='oberdiek'/>
  <install path='/macros/latex/contrib/oberdiek/oberdiek.tds.zip'/>
</entry>
%</catalogue>
%    \end{macrocode}
%
% \begin{History}
%   \begin{Version}{2007/04/17 v1.0}
%   \item
%     First version.
%   \end{Version}
%   \begin{Version}{2007/04/18 v1.1}
%   \item
%     New method based on \cs{lastkern} is used if \eTeX\ is missing.
%   \item
%     \cs{AtBeginShipoutDiscard} also resets \cs{deadcycles}.
%   \end{Version}
%   \begin{Version}{2007/04/19 v1.2}
%   \item
%     \cs{AtBeginShipoutEarly} removed for simplification reasons.
%   \item
%     Forgotten definition of \cs{AtBegShi@Info} added.
%   \item
%     Patches for packages \xpackage{crop} and \xpackage{everyshi}
%     and class \xclass{memoir} added.
%   \end{Version}
%   \begin{Version}{2007/04/26 v1.3}
%   \item
%     Use of package \xpackage{infwarerr}.
%   \item
%     Catcode section after generic header.
%   \end{Version}
%   \begin{Version}{2007/04/27 v1.4}
%   \item
%     Small optimizations.
%   \end{Version}
%   \begin{Version}{2007/06/06 v1.5}
%   \item
%     \cs{AtBeginShipoutUpperLeft} added.
%   \item
%     Example added.
%   \item
%     Fix in second test file for newer version of \xclass{memoir}.
%   \end{Version}
%   \begin{Version}{2007/09/09 v1.6}
%   \item
%     Catcode section rewritten.
%   \end{Version}
%   \begin{Version}{2008/07/18 v1.7}
%   \item
%     Documentation of \cs{AtBeginShipoutUpperLeft} fixed and extended.
%   \end{Version}
%   \begin{Version}{2008/07/19 v1.8}
%   \item
%     \cs{AtBeginShipoutUpperLeftForeground} added.
%   \end{Version}
%   \begin{Version}{2008/07/31 v1.9}
%   \item
%     Second example (TrimBox for dvipdfmx) added.
%   \item
%     No changes in package code.
%   \end{Version}
%   \begin{Version}{2009/12/02 v1.10}
%   \item
%     \cs{AtBeginShipoutOriginalShipout} added.
%   \item
%     Test file fixed.
%   \end{Version}
%   \begin{Version}{2010/03/01 v1.11}
%   \item
%     Compatibility with \iniTeX\ except for \cs{newbox}.
%   \end{Version}
%   \begin{Version}{2010/03/25 v1.12}
%   \item
%     \cs{AtBeginShipoutNext} can now be used inside
%     \cs{AtBeginShipoutNext}.
%   \end{Version}
%   \begin{Version}{2010/08/18 v1.13}
%   \item
%     Fixes for \cs{AtBegShi@CheckDefinable}.
%   \end{Version}
%   \begin{Version}{2010/12/02 v1.14}
%   \item
%     Remove the warning because of void box if the hook
%     calls \AtBeginShipoutDiscard.
%   \end{Version}
%   \begin{Version}{2011/01/30 v1.15}
%   \item
%     Already loaded package files are not input in \hologo{plainTeX}.
%   \end{Version}
%   \begin{Version}{2011/10/05 v1.16}
%   \item
%     \cs{AtBeginShipoutAddToBox}, \cs{AtBeginShipoutAddToBoxForeground} added.
%   \item
%     \cs{AtBeginShipoutBoxWidth}, \cs{AtBeginShipoutBoxHeight},
%     \cs{AtBeginShipoutBoxDepth} added.
%   \item
%     Updates for patches of class \xclass{memoir}.
%   \end{Version}
%   \begin{Version}{2016/05/16 v1.17}
%   \item
%     Documentation updates.
%   \end{Version}
%   \begin{Version}{2016/06/09 v1.18}
%   \item
%     Update for \cs{pdfhorign} in new LuaTeX.
%   \end{Version}
% \end{History}
%
% \PrintIndex
%
% \Finale
\endinput

%        (quote the arguments according to the demands of your shell)
%
% Documentation:
%    (a) If atbegshi.drv is present:
%           latex atbegshi.drv
%    (b) Without atbegshi.drv:
%           latex atbegshi.dtx; ...
%    The class ltxdoc loads the configuration file ltxdoc.cfg
%    if available. Here you can specify further options, e.g.
%    use A4 as paper format:
%       \PassOptionsToClass{a4paper}{article}
%
%    Programm calls to get the documentation (example):
%       pdflatex atbegshi.dtx
%       makeindex -s gind.ist atbegshi.idx
%       pdflatex atbegshi.dtx
%       makeindex -s gind.ist atbegshi.idx
%       pdflatex atbegshi.dtx
%
% Installation:
%    TDS:tex/generic/oberdiek/atbegshi.sty
%    TDS:doc/latex/oberdiek/atbegshi.pdf
%    TDS:doc/latex/oberdiek/atbegshi-example1.tex
%    TDS:doc/latex/oberdiek/atbegshi-example2.tex
%    TDS:doc/latex/oberdiek/test/atbegshi-test1.tex
%    TDS:doc/latex/oberdiek/test/atbegshi-test2.tex
%    TDS:doc/latex/oberdiek/test/atbegshi-test3.tex
%    TDS:source/latex/oberdiek/atbegshi.dtx
%
%<*ignore>
\begingroup
  \catcode123=1 %
  \catcode125=2 %
  \def\x{LaTeX2e}%
\expandafter\endgroup
\ifcase 0\ifx\install y1\fi\expandafter
         \ifx\csname processbatchFile\endcsname\relax\else1\fi
         \ifx\fmtname\x\else 1\fi\relax
\else\csname fi\endcsname
%</ignore>
%<*install>
\input docstrip.tex
\Msg{************************************************************************}
\Msg{* Installation}
\Msg{* Package: atbegshi 2016/06/09 v1.18 At begin shipout hook (HO)}
\Msg{************************************************************************}

\keepsilent
\askforoverwritefalse

\let\MetaPrefix\relax
\preamble

This is a generated file.

Project: atbegshi
Version: 2016/06/09 v1.18

Copyright (C) 2007-2011 by
   Heiko Oberdiek <heiko.oberdiek at googlemail.com>

This work may be distributed and/or modified under the
conditions of the LaTeX Project Public License, either
version 1.3c of this license or (at your option) any later
version. This version of this license is in
   http://www.latex-project.org/lppl/lppl-1-3c.txt
and the latest version of this license is in
   http://www.latex-project.org/lppl.txt
and version 1.3 or later is part of all distributions of
LaTeX version 2005/12/01 or later.

This work has the LPPL maintenance status "maintained".

This Current Maintainer of this work is Heiko Oberdiek.

The Base Interpreter refers to any `TeX-Format',
because some files are installed in TDS:tex/generic//.

This work consists of the main source file atbegshi.dtx
and the derived files
   atbegshi.sty, atbegshi.pdf, atbegshi.ins, atbegshi.drv,
   atbegshi-example1.tex, atbegshi-example2.tex,
   atbegshi-test1.tex, atbegshi-test2.tex, atbegshi-test3.tex.

\endpreamble
\let\MetaPrefix\DoubleperCent

\generate{%
  \file{atbegshi.ins}{\from{atbegshi.dtx}{install}}%
  \file{atbegshi.drv}{\from{atbegshi.dtx}{driver}}%
  \usedir{tex/generic/oberdiek}%
  \file{atbegshi.sty}{\from{atbegshi.dtx}{package}}%
  \usedir{doc/latex/oberdiek}%
  \file{atbegshi-example1.tex}{\from{atbegshi.dtx}{example1}}%
  \file{atbegshi-example2.tex}{\from{atbegshi.dtx}{example2}}%
  \usedir{doc/latex/oberdiek/test}%
  \file{atbegshi-test1.tex}{\from{atbegshi.dtx}{test1}}%
  \file{atbegshi-test2.tex}{\from{atbegshi.dtx}{test2}}%
  \file{atbegshi-test3.tex}{\from{atbegshi.dtx}{test3}}%
  \nopreamble
  \nopostamble
  \usedir{source/latex/oberdiek/catalogue}%
  \file{atbegshi.xml}{\from{atbegshi.dtx}{catalogue}}%
}

\catcode32=13\relax% active space
\let =\space%
\Msg{************************************************************************}
\Msg{*}
\Msg{* To finish the installation you have to move the following}
\Msg{* file into a directory searched by TeX:}
\Msg{*}
\Msg{*     atbegshi.sty}
\Msg{*}
\Msg{* To produce the documentation run the file `atbegshi.drv'}
\Msg{* through LaTeX.}
\Msg{*}
\Msg{* Happy TeXing!}
\Msg{*}
\Msg{************************************************************************}

\endbatchfile
%</install>
%<*ignore>
\fi
%</ignore>
%<*driver>
\NeedsTeXFormat{LaTeX2e}
\ProvidesFile{atbegshi.drv}%
  [2016/06/09 v1.18 At begin shipout hook (HO)]%
\documentclass{ltxdoc}
\usepackage{holtxdoc}[2011/11/22]
\begin{document}
  \DocInput{atbegshi.dtx}%
\end{document}
%</driver>
% \fi
%
%
% \CharacterTable
%  {Upper-case    \A\B\C\D\E\F\G\H\I\J\K\L\M\N\O\P\Q\R\S\T\U\V\W\X\Y\Z
%   Lower-case    \a\b\c\d\e\f\g\h\i\j\k\l\m\n\o\p\q\r\s\t\u\v\w\x\y\z
%   Digits        \0\1\2\3\4\5\6\7\8\9
%   Exclamation   \!     Double quote  \"     Hash (number) \#
%   Dollar        \$     Percent       \%     Ampersand     \&
%   Acute accent  \'     Left paren    \(     Right paren   \)
%   Asterisk      \*     Plus          \+     Comma         \,
%   Minus         \-     Point         \.     Solidus       \/
%   Colon         \:     Semicolon     \;     Less than     \<
%   Equals        \=     Greater than  \>     Question mark \?
%   Commercial at \@     Left bracket  \[     Backslash     \\
%   Right bracket \]     Circumflex    \^     Underscore    \_
%   Grave accent  \`     Left brace    \{     Vertical bar  \|
%   Right brace   \}     Tilde         \~}
%
% \GetFileInfo{atbegshi.drv}
%
% \title{The \xpackage{atbegshi} package}
% \date{2016/06/09 v1.18}
% \author{Heiko Oberdiek\thanks
% {Please report any issues at https://github.com/ho-tex/oberdiek/issues}\\
% \xemail{heiko.oberdiek at googlemail.com}}
%
% \maketitle
%
% \begin{abstract}
% This package is a modern reimplementation of package \xpackage{everyshi}
% without the burden of compatibility. It makes use of \eTeX's if available.
% Both \LaTeX\ and \plainTeX\ are supported.
% \end{abstract}
%
% \tableofcontents
%
% \section{Documentation}
%
% Package \xpackage{atbegshi} redefines \cs{shipout} to insert
% hooks for user code that is executed before the page is shipped out.
% The code may modify or even discard the output page.
% Three hooks are implemented:
% \begin{enumerate}
% \item A hook that is executed for every page, see\\
%       \cs{AtBeginShipout}
% \item A hook that is executed for the next page only, see\\
%       \cs{AtBeginShipoutNext}
% \item A hook that is only executed for the first page, see\\
%       \cs{AtBeginShipoutFirst}
% \end{enumerate}
% The hooks are executed in this order. The following three macros
% provide the user interface for adding code to these hooks:
%
% \begin{declcs}{AtBeginShipout} \M{code}\\
%   \cs{AtBeginShipoutBox}
% \end{declcs}
% Execute the \meta{code} for every page.
% The page contents is held in box register \cs{AtBeginShipoutBox}
% and may be modified. Use \cs{AtBeginShipoutDiscard} if you want to
% discard the page.
%
% \emph{Note:} Package \xpackage{everyshi} uses box register 255. With
% package \xpackage{atbegshi} you must use \cs{AtBeginShipoutBox} instead.
%
% If \LaTeX\ calls \cs{shipout} in \cs{@outputpage} (part of its output
% routine),
% the meaning of \cs{protect} is \cs{noexpand}. \LaTeX\ sets
% \cs{protect} to the appropriate \cs{@typeset@protect} in the box
% that is shipped out. This is too late for the hooks, they
% are called earlier in the redefined \cs{shipout}. Therefore
% package \xpackage{atbegshi} sets \cs{protect} to \cs{@typeset@protect}
% before it calls the hooks.
% (In \cs{EveryShipout} of package \xpackage{everyshi} the
% user is responsible for the correct setting of \cs{protect}.)
%
% \begin{declcs}{AtBeginShipoutNext} \M{code}
% \end{declcs}
% This reimplements package \xpackage{everyshi}'s \cs{AtNextShipout}.
% The \meta{code} is executed at shipout time of the next page only.
% It is just a convenience macro, it can be easily replaced by something
% like:
%\begin{quote}
%\begin{verbatim}
%\newcommand{\MyShipoutHook}{}%
%\AtBeginShipout{\MyShipoutHook}
%\gdef\MyShipoutHook{%
%  ... do something with next page ...
%  \gdef\MyShipoutHook{}%
% }
%\end{verbatim}
%\end{quote}
% (This can be necessary, if hook order does matter).
%
% \begin{declcs}{AtBeginShipoutFirst} \M{code}
% \end{declcs}
% This reimplements \LaTeX's \cs{AtBeginDvi}. This hook is
% usually used for \cs{special} commands that include PostScript
% header files. The \cs{code} is directly executed in a \cs{vbox}
% that is put at the beginning of the output page. Dealing with
% the output box \cs{AtBeginShipoutBox} is not necessary and
% not permitted here.
%
% \begin{declcs}{AtBeginShipoutDiscard}
% \end{declcs}
% This macro notifies package \xpackage{atbegshi} that
% the output page is discarded. The remaining hook code
% and the remaining hooks are not executed and the page is thrown away.
% Also \cs{deadcycles} is cleared to zero like an ordinary \cs{shipout}
% would do.
%
% \begin{declcs}{AtBeginShipoutInit}
% \end{declcs}
% Usually the redefinition of \cs{shipout} is delayed by
% \cs{AtBeginDocument} (if this macro exists). This can be
% too late, if other packages also redefines \cs{shipout}
% and the order does matter.
% \cs{AtBeginShipoutInit} forces the immediate redefinition
% of \cs{shipout}.
%
% \begin{declcs}{AtBeginShipoutAddToBox} \M{stuff}\\
%   \cs{AtBeginShipoutAddToBoxForeground} \M{stuff}
% \end{declcs}
% A quite common use case is the addition of \cs{special}
% or other whatsits to the page output box. Macro
% \cs{AtBeginShipoutAddToBox} puts \meta{stuff} in a
% box with zeroed dimensions. The box with the \meta{stuff}
% is put in the upper left corner of the shipout box
% \cs{AtBeginShipoutBox}. Macro \cs{AtBeginShipoutAddToBox}
% puts the \meta{stuff} in the background, the other macro
% \cs{AtBeginShipoutAddToBoxForeground} in the foreground
% after the original shipout box contents is set.
%
% A void shipout box (that means a discarded page)
% remains void that means \meta{stuff} is ignored in this case.
% The box type of \cs{AtBeginShipoutBox} is preserved.
% Also the box nesting level for the original contents
% of \cs{AtBeginShipoutBox} remains, for example, to avoid trouble
% with links across pages in case of \hologo{pdfTeX}.
%
% \begin{declcs}{AtBeginShipoutUpperLeft} \M{background material}
% \end{declcs}
% This is a macro that puts material in the background of
% box \cs{AtBeginShipoutBox}.
% The \meta{background material} is set in an \cs{hbox}, the
% reference point is the upper left corner of the output page.
% In case of \pdfTeX\ in PDF mode, the settings of
% \cs{pdfhorigin} and \cs{pdfvorigin} are respected.
%
% The macro \cs{AtBeginShipoutUpperLeft} is intended to be used
% in one of the hook setting macros, such as
% \cs{AtBeginShipout}, \cs{AtBeginShipoutFirst}, or
% \cs{AtBeginShipoutNext}.
%
% For \LaTeX\ users the \meta{background material} is set
% inside a \texttt{picture} environment:
% \begin{quote}
%   |\begin{picture}(0,0)|\\
%   \mbox{}\quad |\setlength{\unitlength}{1pt}%|\\
%   \mbox{}\quad \meta{background material}\\
%   |\end{picture}|
% \end{quote}
%
% \begin{declcs}{AtBeginShipoutUpperLeftForeground} \M{foreground material}
% \end{declcs}
% See \cs{AtBeginShipoutUpperLeft}. The difference is that the material
% is put in the foreground.
%
% \begin{declcs}{AtBeginShipoutOriginalShipout} \meta{box}
% \end{declcs}
% It stores the meaning of \cs{shipout} at the time this package
% is loaded.
%
% \begin{declcs}{AtBeginShipoutBoxWidth}\\
%   \cs{AtBeginShipoutBoxHeight}\\
%   \cs{AtBeginShipoutBoxDepth}\\
% \end{declcs}
% These macros store the dimensions of the output box \cs{AtBeginShipoutBox}
% before the original shipout is called. If \cs{shipout} is not
% redefined before the package loading or the box dimensions
% are not changed by the redefined \cs{shipout}, these macros
% contain the dimensions of the shipout box. These values can be
% remembered by \cs{label} and \cs{ref}. For example, this is done
% by the package module \xpackage{zref-pagelayout} of project \xpackage{zref}.
% The dimensions of the shipout page can be used in some \hologo{TeX}
% engines (\hologo{pdfTeX} in PDF mode, \hologo{XeTeX}) to calculate
% the media size of the shipout page if \cs{pdfpagewidth} and
% \cs{pdfpageheight} are not set.
%
% \subsection{Examples}
%
% \subsubsection{Example: circle in background}
%
% In this example we put a circle in the background in the middle
% of the paper.
%
%    \begin{macrocode}
%<*example1>
%    \end{macrocode}
%    \begin{macrocode}
\documentclass[a4paper]{article}
\usepackage{color}
\usepackage{atbegshi}
%    \end{macrocode}
%    Package \xpackage{picture} makes life a little easier, because
%    we can now also use length specifications in |picture|'s commands.
%    \begin{macrocode}
\usepackage{picture}
%    \end{macrocode}
%    Now we draw the circle in the middle of the paper.
%    \cs{put} moves downwards, because the origin is at the top
%    of the page, not at its bottom.
%    \begin{macrocode}
\AtBeginShipout{%
  \AtBeginShipoutUpperLeft{%
    \put(0.5\paperwidth,-0.5\paperheight){\circle{10}}%
  }%
}
\begin{document}
\section{Hello World}
\newpage
\AtBeginShipoutNext{%
  \AtBeginShipoutUpperLeft{%
    \color{red}%
    \put(0,-0.5\paperheight){\line(1,0){\paperwidth}}%
    \put(0.5\paperwidth, 0){\line(0,-1){\paperheight}}%
  }%
}
Only on this page we add a red cross.
\newpage
This page has the circle only.
\par
\vspace{\fill}
The next page will be discarded.
\newpage
\AtBeginShipoutNext{%
  \AtBeginShipoutDiscard
}
This page is discarded.
\newpage
The last page.
\end{document}
%    \end{macrocode}
%    \begin{macrocode}
%</example1>
%    \end{macrocode}
%
% \subsubsection{Example: adding TrimBox for dvipdfmx}
%
% Now an example from ``real life'' follows. Someone from the mailing  list
% for dvipdfmx wants to put a TrimBox on every page. If we use
% \cs{AtBeginShipout}, we have to put the \cs{special} inside the
% box \cs{AtBeginShipoutBox} that gets shipped out.
%    \begin{macrocode}
%<*example2>
\documentclass{minimal}
\usepackage{atbegshi}
\usepackage[
  dvipdfm,
  paperwidth=630bp,
  paperheight=810bp
]{geometry}
\AtBeginShipout{%
  \setbox\AtBeginShipoutBox=\hbox{%
    \special{pdf: put @thispage <</TrimBox[9 9 621 801]>>}%
    \box\AtBeginShipoutBox
  }%
}
\begin{document}
  First page
  \newpage
  Second page
\end{document}
%</example2>
%    \end{macrocode}
% Remember, in \cs{AtBeginShipoutBoxFirst} the \cs{setbox} wrapper code
% is implicitly given and the \cs{special} is used directly.
%
% \section{Method of \cs{shipout} overloading}
%
% \subsection{\cs{shipout}}
%
% The \TeX\ primitive command \cs{shipout} takes a box specification
% and puts the box as a new page in the output file.
% There are two kinds of box specifications:
% \begin{description}
% \item[Direct boxes:]
%   They are given by \cs{hbox}, \cs{vbox}, or \cs{vtop},\\
%   e.g. |\shipout\hbox{Hello World}|.
% \item[Indirect boxes:]
%   \cs{box} or \cs{copy} references a box register by number.
%   The box register contains the contents of the box.
%
%   \emph{Note:} \cs{box} also clears the box register globally.
% \end{description}
% Then we have to differentiate between void and empty boxes:
% \begin{description}
% \item[Void:]
%   Initially or after \cs{box} there is no box in the box register.
%   In this cases the box register is not empty, but \emph{void}.
% \item[Empty:]
%   A box with empty contents, such as
%   |\hbox{}| (= \cs{null}) or |\vbox{}| is an \emph{empty \cs{hbox}}
%   or \emph{empty \cs{vbox}}. If a box register holds such a
%   box, the box still exists, therefore the box register is
%   \emph{not void}.
% \end{description}
%
% \subsection{\cs{afterassignment}}
%
% We want to overload \cs{shipout} to do something with the box.
% It is quite impossible to do this reliable by catching the box
% using macro arguments. The
% variety of box specifications is too large, Examples:
% \begin{quote}
%   |\shipout\null|\\
%   |\shipout\vbox{...}|\\
%   |\shipout\vtop\bgroup ...\egroup|\\
%   |\shipout\box255|
% \end{quote}
% Even worse, the braces don't need to be balanced:
% \begin{quote}
%   |\shipout\hbox\bgroup}|\\
%   |\shipout\vbox{\egroup|
% \end{quote}
% Happily TeX provides a reliable way via \cs{afterassignment}.
% It takes a macro name and executes it just after the assignment.
%
% Now we can redefine \cs{shipout}. The box specification that
% follows \cs{shipout} is catched by \cs{setbox}. This is an
% assignment to a box register. \cs{afterassignment}
% notifies \TeX, that we want to call \cs{@test} right after
% the assignment:
%\begin{quote}
%\begin{verbatim}
%\shipout :=
%  \afterassignment\@test
%  \setbox\mybox=
%\end{verbatim}
%\end{quote}
% We have seen different box specifications. Indirect boxes are easy
% to understand:
% \begin{quote}
%   |\shipout\box0| $\Rightarrow$ |\setbox\mybox=\box0 |\fbox{\cs{@test}}
% \end{quote}
% However direct boxes can have arbitrary contents with lots of
% other assignments. It would be quite inpredictable if \TeX\ would
% put |\@test| after the first of such an assignment or after the
% box specification if the box lacks of assignments.
% Therefore \TeX\ puts |\@test| right at the beginning of the
% box specification, e.g:
% \begin{quote}
%   |\shipout\hbox{Hello World}|\\
%   $\Rightarrow$
%   |\setbox\mybox=\hbox{|\fbox{\cs{@test}}| Hello World}|
% \end{quote}
%
% \subsection{Test for direct or indirect boxes}
%
% Now we want to execute \cs{@test}, but where are we?
% We can be after the completed box assignment, if \cs{shipout}
% was called with an indirect box. Or we are right at the beginning
% of a direct box.
%
% \subsubsection{With \eTeX}
%
% With the \eTeX's extensions the answer is very easy:
% Being inside the direct box means that we are inside a new group.
% The new primitive command \cs{currentgrouplevel} tells how
% deeply the groups are currently nested. Macro \cs{@test} just
% compares the previously stored group level with the current one:
%\begin{quote}
%\begin{verbatim}
%\shipout :=
%  \edef\saved@grouplevel{\number\currentgrouplevel}
%  \afterassignment\@test
%  \setbox\mybox=
%
%\@test :=
%  \ifnum\saved@grouplevel=\currentgrouplevel
%    % case: indirect box, the assignment is completed
%    \@output
%  \else
%    % case: direct box, we are inside the box
%    \aftergroup\@outbox
%  \fi
%\end{verbatim}
%\end{quote}
%
% \subsubsection{Without \eTeX}
%
% Life becomes complicate without \eTeX. We cannot ask the
% group level. However, if we are inside a direct box,
% the box register \cs{mybox} is not yet changed by \cs{setbox}.
% Thus we need a special initial value and compare
% it in \cs{@test} with the current value of the box.
%
% What can be used as initial value? Arbitrary box contents cannot
% be compared. \TeX\ only tells us a few properties:
% \begin{itemize}
% \item Box type:\cs{ifhbox}, \cs{ifvbox}
% \item Dimensions: \cs{wd}, \cs{ht}, \cs{dp}
% \item Voidness: \cs{ifvoid}
% \end{itemize}
% Unhappily all these qualities even combined are not sufficient
% for constructing an initial box value, because \cs{shipout} can be
% called with a box that is accidently just the same as the choosen
% initial value.
%
% Nevertheless we have two alternatives for an initial value:
% \begin{itemize}
% \item A box of some type with some funny settings that are
% unlikely to occur in real life, e.g a height of |4911sp-\maxdimen|.
% \item A void box.
% \end{itemize}
% A collision between this initial value and an indirect \cs{shipout}
% box with just the same value is possible.
% Then \cs{@test} will make a wrong decision that it is executed
% inside a direct box and delays \cs{@output} by \cs{aftergroup}.
% Thus \cs{@output} is not called at the place we want.
% In contrary, the result is an uncertainty about the place:
% \begin{itemize}
% \item \cs{shipout} is used in a group that perhaps closes some
%   pages later. A bad place for \cs{@output}.
% \item Without a surrounding group \cs{aftergroup} effectively kills
%   its argument.
% \end{itemize}
% In the first case of a box with special dimensions we can even loose
% the page. However in the case of the void box, this effect is even
% desired, because the original \cs{shipout} does not output void boxes.
% All we have to do is to ensure that our box \cs{mybox} is always
% void except for the phase when the overloaded \cs{shipout} is executed.
% And secondly we must keep this semantics of \cs{shipout} for the void case
% in our macros, namely \cs{@output}.
%\begin{quote}
%\begin{verbatim}
%\shipout :=
%  % trick to get a void box \mybox
%  \begingroup
%    \setbox\mybox=\box\mybox
%  \endgroup
%  \afterassignment\@test
%  \setbox\mybox=
%
%\@test :=
%  \ifvoid\mybox
%    \aftergroup\@output
%  \else
%    \@output
%  \fi
%\end{verbatim}
%\end{quote}
% The nasty case is |\shipout\box\voidb@x| where the indirect box
% is void and that must not generate an output page.
% If a surrounding group is missing the output is ignored
% because of \cs{aftergroup}.
% Otherwise \@output is called some time
% later when the surrounding group closes.
% But \cs{mybox} is void outside the execution phase
% of the redefined \cs{shipout}. Also \cs{@output} checks for a
% void box and cancels the page output.
% The disadvantage remains that the hook in \cs{@output} is called
% for a page that will not be output.
%
% \subsubsection{\cs{lastkern} method}
%
% At the beginning of a new box, there is no \cs{kern}, the contents
% of the box is still empty and \cs{lastkern} returns 0\,pt.
% This can be used to distinguish between direct and indirect boxes:
% We execute \cs{setbox} in a box with a preceding non-zero kern.
% After an indirect box, \cs{lastkern} sees this kern, otherwise
% it returns 0\,pt.
%\begin{quote}
%\begin{verbatim}
%\shipout :=
%  \begingroup
%    \setbox\mybox=\hbox\bgroup
%      \kern1pt
%      \afterassignment\shipout@test
%      \global\setbox\mybox=
%\@test :=
%  \ifdim\lastkern=0pt
%    % direct box
%    \aftergroup\egroup
%    \aftergroup\endgroup
%    \aftergroup\@output
%  \else
%    \egroup
%    \endgroup
%    \@output
%  \fi
%\end{verbatim}
%\end{quote}
% We have two \cs{setbox} commands. The first creates a controlled
% context box where we can safely insert a \cs{kern}. We get rid of
% this temporarily used context box by putting the local \cs{setbox}
% in a group.
%
% After the group we want to have our shipout box in \cs{mybox}.
% Therefore we use a global assignment here.
%
% \subsection{Output}
%
% With or without \eTeX\ we ensure the original behaviour of
% \cs{shipout} that void boxes do not generate output pages.
%
% Now we can place the hook \cs{@hook} for the user code that
% wants to manipulate the output box.
%\begin{quote}
%\begin{verbatim}
%\@output :=
%  \ifvoid\mybox
%    % cancel output of void box
%  \else
%    \@hook
%    \ifvoid\mybox
%      % user code in \@hook could has voided the box
%    \else
%      \original@shipout\box\mybox
%    \fi
%  \fi
%\end{verbatim}
%\end{quote}
%
% \subsection{Separate box register}
%
% So far we have said nothing about the box number of \cs{mybox}.
% The following case that outputs the same page twice
% shows that we are not free in the use
% of the box register:
% \begin{quote}
%   |\shipout\copy<num> \shipout\box<num>|
% \end{quote}
% We manipulate the box by the hook and without \eTeX\ the box
% must even be voided. However, the use case above requires that
% the box contents does not change at all. Therefore we must
% reserve a separate box register to avoid collisions with
% user box registers.
%
% \emph{Note:} Box register number 255 is special for the output
% routine, because TeX complains if this box is not voided
% by the output routine.
% However, this requirement does not apply to \cs{shipout} at all.
% In fact \cs{shipout} does not change any box register. This is
% usually done by a call of \cs{box}, but the output routine
% can do it later \emph{after} invoking of \cs{shipout}.
%
% \subsection{Summary}
%
% \subsubsection{With \eTeX}
%
% Putting the pieces together we get for \eTeX:
%\begin{quote}
%\begin{verbatim}
%\newbox\mybox
%\let\original@shipout\shipout
%
%\shipout :=
%  \edef\saved@grouplevel{\number\currentgrouplevel}
%  \afterassignment\@test
%  \setbox\mybox=
%
%\@test :=
%  \ifnum\saved@grouplevel<\currentgrouplevel
%    \expandafter\aftergroup
%  \fi
%  \@output
%
%\@output :=
%  \ifvoid\mybox
%    % cancel output of void box
%  \else
%    \@hook
%    \ifvoid\mybox
%      % user code in \@hook could have voided the box
%    \else
%      \original@shipout\box\mybox
%    \fi
%  \fi
%\end{verbatim}
%\end{quote}
%
% \subsubsection{Without \eTeX, traditional way}
%
% And for \TeX\ without \eTeX:
%\begin{quote}
%\begin{verbatim}
%\newbox\mybox
%\begingroup
%   \setbox\mybox=\box\mybox % ensure \mybox is void
%\endgroup
%\let\original@shipout\shipout
%
%\shipout :=
%  % trick to get a void box \mybox
%  \begingroup
%    \setbox\mybox=\box\mybox
%  \endgroup
%  \afterassignment\@test
%  \setbox\mybox=
%
%\@test :=
%  \ifvoid\mybox
%    \expandafter\aftergroup
%  \fi
%  \@output
%
%\@output :=
%  \ifvoid\mybox
%    % cancel output of void box
%  \else
%    \@hook
%    \ifvoid\mybox
%      % user code in \@hook could have voided the box
%    \else
%      \original@shipout\box\mybox
%    \fi
%  \fi
%\end{verbatim}
%\end{quote}
%
% \subsubsection{\cs{lastkern} method}
%
% And for \TeX\ without \eTeX\ using the \cs{lastkern} method:
%\begin{quote}
%\begin{verbatim}
%\newbox\mybox
%\let\original@shipout\shipout
%
%\shipout :=
%  \begingroup
%  \setbox\mybox=\hbox\bgroup
%  \kern1pt
%  \afterassignment\@test
%  \setbox\mybox=
%
%\@test :=
%  \ifdim\lastkern=0pt
%    \expandafter\aftergroup
%  \fi
%  \@output
%
%\@output :=
%  \egroup
%  \endgroup
%  \ifvoid\mybox
%    % cancel output of void box
%  \else
%    \@hook
%    \ifvoid\mybox
%      % user code in \@hook could have voided the box
%    \else
%      \original@shipout\box\mybox
%    \fi
%  \fi
%\end{verbatim}
%\end{quote}
%
%
% \StopEventually{
% }
%
% \section{Implementation}
%
% Package \xpackage{atbegshi} uses \eTeX's \cs{currentgrouplevel},
% if it is available. Otherwise the \cs{lastkern} method is used.
%
%    \begin{macrocode}
%<*package>
%    \end{macrocode}
%
% \subsection{Reload check and package identification}
%    Reload check, especially if the package is not used with \LaTeX.
%    \begin{macrocode}
\begingroup\catcode61\catcode48\catcode32=10\relax%
  \catcode13=5 % ^^M
  \endlinechar=13 %
  \catcode35=6 % #
  \catcode39=12 % '
  \catcode44=12 % ,
  \catcode45=12 % -
  \catcode46=12 % .
  \catcode58=12 % :
  \catcode64=11 % @
  \catcode123=1 % {
  \catcode125=2 % }
  \expandafter\let\expandafter\x\csname ver@atbegshi.sty\endcsname
  \ifx\x\relax % plain-TeX, first loading
  \else
    \def\empty{}%
    \ifx\x\empty % LaTeX, first loading,
      % variable is initialized, but \ProvidesPackage not yet seen
    \else
      \expandafter\ifx\csname PackageInfo\endcsname\relax
        \def\x#1#2{%
          \immediate\write-1{Package #1 Info: #2.}%
        }%
      \else
        \def\x#1#2{\PackageInfo{#1}{#2, stopped}}%
      \fi
      \x{atbegshi}{The package is already loaded}%
      \aftergroup\endinput
    \fi
  \fi
\endgroup%
%    \end{macrocode}
%    Package identification:
%    \begin{macrocode}
\begingroup\catcode61\catcode48\catcode32=10\relax%
  \catcode13=5 % ^^M
  \endlinechar=13 %
  \catcode35=6 % #
  \catcode39=12 % '
  \catcode40=12 % (
  \catcode41=12 % )
  \catcode44=12 % ,
  \catcode45=12 % -
  \catcode46=12 % .
  \catcode47=12 % /
  \catcode58=12 % :
  \catcode64=11 % @
  \catcode91=12 % [
  \catcode93=12 % ]
  \catcode123=1 % {
  \catcode125=2 % }
  \expandafter\ifx\csname ProvidesPackage\endcsname\relax
    \def\x#1#2#3[#4]{\endgroup
      \immediate\write-1{Package: #3 #4}%
      \xdef#1{#4}%
    }%
  \else
    \def\x#1#2[#3]{\endgroup
      #2[{#3}]%
      \ifx#1\@undefined
        \xdef#1{#3}%
      \fi
      \ifx#1\relax
        \xdef#1{#3}%
      \fi
    }%
  \fi
\expandafter\x\csname ver@atbegshi.sty\endcsname
\ProvidesPackage{atbegshi}%
  [2016/06/09 v1.18 At begin shipout hook (HO)]%
%    \end{macrocode}
%
% \subsection{Catcodes}
%
%    \begin{macrocode}
\begingroup\catcode61\catcode48\catcode32=10\relax%
  \catcode13=5 % ^^M
  \endlinechar=13 %
  \catcode123=1 % {
  \catcode125=2 % }
  \catcode64=11 % @
  \def\x{\endgroup
    \expandafter\edef\csname AtBegShi@AtEnd\endcsname{%
      \endlinechar=\the\endlinechar\relax
      \catcode13=\the\catcode13\relax
      \catcode32=\the\catcode32\relax
      \catcode35=\the\catcode35\relax
      \catcode61=\the\catcode61\relax
      \catcode64=\the\catcode64\relax
      \catcode123=\the\catcode123\relax
      \catcode125=\the\catcode125\relax
    }%
  }%
\x\catcode61\catcode48\catcode32=10\relax%
\catcode13=5 % ^^M
\endlinechar=13 %
\catcode35=6 % #
\catcode64=11 % @
\catcode123=1 % {
\catcode125=2 % }
\def\TMP@EnsureCode#1#2{%
  \edef\AtBegShi@AtEnd{%
    \AtBegShi@AtEnd
    \catcode#1=\the\catcode#1\relax
  }%
  \catcode#1=#2\relax
}
\TMP@EnsureCode{40}{12}% (
\TMP@EnsureCode{41}{12}% )
\TMP@EnsureCode{44}{12}% ,
\TMP@EnsureCode{45}{12}% -
\TMP@EnsureCode{47}{12}% /
\TMP@EnsureCode{46}{12}% .
\TMP@EnsureCode{58}{12}% :
\TMP@EnsureCode{91}{12}% [
\TMP@EnsureCode{93}{12}% ]
\TMP@EnsureCode{94}{7}% ^(superscript)
\TMP@EnsureCode{96}{12}% `
\edef\AtBegShi@AtEnd{\AtBegShi@AtEnd\noexpand\endinput}
%    \end{macrocode}
%
% \subsection{Preparations}
%
%    \begin{macrocode}
\begingroup\expandafter\expandafter\expandafter\endgroup
\expandafter\ifx\csname RequirePackage\endcsname\relax
  \def\TMP@RequirePackage#1[#2]{%
    \begingroup\expandafter\expandafter\expandafter\endgroup
    \expandafter\ifx\csname ver@#1.sty\endcsname\relax
      \input #1.sty\relax
    \fi
  }%
  \TMP@RequirePackage{infwarerr}[2007/09/09]%
  \TMP@RequirePackage{ltxcmds}[2010/03/01]%
\else
  \RequirePackage{infwarerr}[2007/09/09]%
  \RequirePackage{ltxcmds}[2010/03/01]%
\fi
%    \end{macrocode}
%
%    \begin{macro}{\AtBegShi@CheckDefinable}
%    \begin{macrocode}
\begingroup\expandafter\expandafter\expandafter\endgroup
\expandafter\ifx\csname @ifdefinable\endcsname\relax
  \def\AtBegShi@CheckDefinable#1{%
    \ifcase\ifx#1\relax
             \ltx@one
           \else
             \ifx#1\@undefined
               \ltx@one
             \else
               \ltx@zero
             \fi
           \fi
      \@PackageError{atbegshi}{%
        \string#1\space is already defined%
      }\@ehd
    \fi
  }%
\else
  \def\AtBegShi@CheckDefinable#1{%
    \@ifdefinable{#1}{}%
  }%
\fi
%    \end{macrocode}
%    \end{macro}
%
%    \begin{macro}{\ifAtBegShi@Discarded}
%    \begin{macrocode}
\ltx@newif\ifAtBegShi@Discarded
%    \end{macrocode}
%    \end{macro}
%    \begin{macro}{\AtBeginShipoutDiscard}
%    \begin{macrocode}
\AtBegShi@CheckDefinable\AtBeginShipoutDiscard
\def\AtBeginShipoutDiscard{%
  \deadcycles=\ltx@zero
  \global\AtBegShi@Discardedtrue
}
%    \end{macrocode}
%    \end{macro}
%
%    \begin{macrocode}
\begingroup\expandafter\expandafter\expandafter\endgroup
\expandafter\ifx\csname currentgrouplevel\endcsname\relax
  \catcode`X=9  % ignore
  \catcode`E=14 % comment
\else
  \catcode`X=14 % comment
  \catcode`E=9  % ignore
\fi
%    \end{macrocode}
%
%    \begin{macro}{\AtBegShi@Shipout}
%    \begin{macrocode}
\def\AtBegShi@Shipout{%
X \begingroup
X \setbox\AtBeginShipoutBox=\hbox\bgroup
X \kern\p@
E \edef\AtBegShi@GroupLevel{\number\currentgrouplevel}%
  \afterassignment\AtBegShi@Test
X \global
  \setbox\AtBeginShipoutBox=%
}
%    \end{macrocode}
%    \end{macro}
%
%    \begin{macro}{\AtBegShi@Test}
%    \begin{macrocode}
\def\AtBegShi@Test{%
X \ifdim\lastkern=0pt %
E \ifnum\AtBegShi@GroupLevel<\currentgrouplevel
    \expandafter\aftergroup
  \fi
  \AtBegShi@Output
}
%    \end{macrocode}
%    \end{macro}
%
%    \begin{macro}{\AtBegShi@Output}
%    \begin{macrocode}
\def\AtBegShi@Output{%
X \egroup
X \endgroup
  \ifvoid\AtBeginShipoutBox
    \@PackageWarning{atbegshi}{Ignoring void shipout box}%
  \else
    \let\AtBegShi@OrgProtect\protect
    \csname set@typeset@protect\endcsname
    \global\AtBegShi@Discardedfalse
    \AtBegShi@Hook
    \expandafter\gdef\expandafter\AtBegShi@HookNext
    \expandafter{\expandafter}%
    \AtBegShi@HookNext
    \ifAtBegShi@Discarded
      \@PackageInfoNoLine{atbegshi}{Shipout page discarded}%
      \global\AtBegShi@Discardedfalse
      \begingroup
        \setbox\AtBeginShipoutBox\box\AtBeginShipoutBox
      \endgroup
      \let\protect\AtBegShi@OrgProtect
    \else
      \AtBegShi@First
      \let\protect\AtBegShi@OrgProtect
      \AtBegShi@GetBoxSize\AtBeginShipoutBox
      \ltx@ifundefined{AtNextShipout}{%
      }{%
        \AtNextShipout{\AtBegShi@GetBoxSize\@cclv}%
      }%
      \AtBeginShipoutOriginalShipout\box\AtBeginShipoutBox
    \fi
  \fi
}
%    \end{macrocode}
%    \end{macro}
%
%    \begin{macro}{\AtBegShi@GetBoxSize}
%    \begin{macrocode}
\def\AtBegShi@GetBoxSize#1{%
  \xdef\AtBeginShipoutBoxWidth{\the\wd#1}%
  \xdef\AtBeginShipoutBoxHeight{\the\ht#1}%
  \xdef\AtBeginShipoutBoxDepth{\the\dp#1}%
}
%    \end{macrocode}
%    \end{macro}
%    \begin{macro}{\AtBeginShipoutBoxWidth}
%    \begin{macrocode}
\def\AtBeginShipoutBoxWidth{0pt}
%    \end{macrocode}
%    \end{macro}
%    \begin{macro}{\AtBeginShipoutBoxHeight}
%    \begin{macrocode}
\def\AtBeginShipoutBoxHeight{0pt}
%    \end{macrocode}
%    \end{macro}
%    \begin{macro}{\AtBeginShipoutBoxDepth}
%    \begin{macrocode}
\def\AtBeginShipoutBoxDepth{0pt}
%    \end{macrocode}
%    \end{macro}
%
%    \begin{macrocode}
\catcode`\X=11 %
\catcode`\E=11 %
%    \end{macrocode}
%
%    \begin{macro}{\AtBegShi@First}
%    \begin{macrocode}
\def\AtBegShi@First{%
  \ifx\AtBegShi@HookFirst\ltx@empty
  \else
    \AtBeginShipoutAddToBox{\AtBegShi@HookFirst}%
  \fi
  \global\let\AtBegShi@First\ltx@empty
  \global\let\AtBeginShipoutFirst\AtBegShi@FirstDisabled
}
%    \end{macrocode}
%    \end{macro}
%
%    \begin{macro}{\AtBegShi@Hook}
%    \begin{macrocode}
\gdef\AtBegShi@Hook{}
%    \end{macrocode}
%    \end{macro}
%    \begin{macro}{\AtBegShi@HookNext}
%    \begin{macrocode}
\gdef\AtBegShi@HookNext{}
%    \end{macrocode}
%    \end{macro}
%    \begin{macro}{\AtBegShi@HookFirst}
%    \begin{macrocode}
\gdef\AtBegShi@HookFirst{}
%    \end{macrocode}
%    \end{macro}
%    \begin{macro}{\AtBeginShipout}
%    \begin{macrocode}
\AtBegShi@CheckDefinable\AtBeginShipout
\def\AtBeginShipout{%
  \AtBegShi@AddHook\AtBegShi@Hook
}
%    \end{macrocode}
%    \end{macro}
%    \begin{macro}{\AtBeginShipoutNext}
%    \begin{macrocode}
\AtBegShi@CheckDefinable\AtBeginShipoutNext
\def\AtBeginShipoutNext{%
  \AtBegShi@AddHook\AtBegShi@HookNext
}
%    \end{macrocode}
%    \end{macro}
%    \begin{macro}{\AtBeginShipoutFirst}
%    \begin{macrocode}
\AtBegShi@CheckDefinable\AtBeginShipoutFirst
\def\AtBeginShipoutFirst{%
  \AtBegShi@AddTo\AtBegShi@HookFirst
}
%    \end{macrocode}
%    \end{macro}
%    \begin{macro}{\AtBegShi@FirstDisabled}
%    \begin{macrocode}
\long\def\AtBegShi@FirstDisabled#1{%
  \@PackageWarning{atbegshi}{%
    First page is already shipped out, ignoring\MessageBreak
    \string\AtBeginShipoutFirst
  }%
}
%    \end{macrocode}
%    \end{macro}
%
%    \begin{macro}{\AtBegShi@AddTo}
%    \begin{macrocode}
\begingroup\expandafter\expandafter\expandafter\endgroup
\expandafter\ifx\csname g@addto@macro\endcsname\relax
  \long\def\AtBegShi@AddTo#1#2{%
    \begingroup
      \toks\ltx@zero\expandafter{#1#2}%
      \xdef#1{\the\toks\ltx@zero}%
    \endgroup
  }%
\else
  \let\AtBegShi@AddTo\g@addto@macro
\fi
%    \end{macrocode}
%    \end{macro}
%
%    \begin{macro}{\AtBegShi@AddHook}
%    \begin{macrocode}
\long\def\AtBegShi@AddHook#1#2{%
  \AtBegShi@AddTo#1{\AtBegShi@Item{#2}}%
}
%    \end{macrocode}
%    \end{macro}
%
%    \begin{macro}{\AtBegShi@Item}
%    \begin{macrocode}
\long\def\AtBegShi@Item#1{%
  \ifAtBegShi@Discarded
  \else
    #1%
    \ifAtBegShi@Discarded
    \else
      \ifvoid\AtBeginShipoutBox
        \@PackageWarning{atbegshi}{%
          Shipout box was voided by hook,\MessageBreak
          ignoring shipout box%
        }%
        \AtBeginShipoutDiscard
      \fi
    \fi
  \fi
}
%    \end{macrocode}
%    \end{macro}
%
%    \begin{macro}{\AtBeginShipoutInit}
%    \begin{macrocode}
\AtBegShi@CheckDefinable\AtBeginShipoutInit
\def\AtBeginShipoutInit{%
  \ltx@IfUndefined{newbox}{%
    \@PackageError{atbegshi}{%
      \string\AtBeginShipoutInit\space failed\MessageBreak
       because of missing \expandafter\string\csname newbox\endcsname
    }\@ehc
  }{%
    \csname newbox\endcsname\AtBeginShipoutBox
    \AtBegShi@CheckDefinable\AtBeginShipoutOriginalShipout
    \global\let\AtBeginShipoutOriginalShipout\shipout
    \global\let\shipout\AtBegShi@Shipout
  }%
  \gdef\AtBeginShipoutInit{}%
}
%    \end{macrocode}
%    \end{macro}
%    \begin{macrocode}
\begingroup\expandafter\expandafter\expandafter\endgroup
\expandafter\ifx\csname AtBeginDocument\endcsname\relax
  \AtBeginShipoutInit
\else
  \AtBeginDocument{\AtBeginShipoutInit}%
\fi
%    \end{macrocode}
%
% \subsection{Additions to the shipout box}
%
%    \begin{macro}{\AtBeginShipoutAddToBox}
%    \begin{macrocode}
\def\AtBeginShipoutAddToBox#1{%
  \ifhbox\AtBeginShipoutBox
    \edef\AtBegShi@restore{%
      \hfuzz=\the\hfuzz\relax
      \hbadness=\the\hbadness\relax
    }%
    \hfuzz=1073741823sp\relax
    \hbadness=2147483647\relax
    \setbox\AtBeginShipoutBox=\hbox to \wd\AtBeginShipoutBox{%
      \setbox\ltx@zero=\hbox{%
        \begingroup
          \AtBegShi@restore
          #1%
        \endgroup
      }%
      \wd\ltx@zero=0pt\relax
      \ht\ltx@zero=0pt\relax
      \dp\ltx@zero=0pt\relax
      \raise\ht\AtBeginShipoutBox\copy\ltx@zero
      \unhcopy\AtBeginShipoutBox
    }%
    \AtBegShi@restore
  \else
    \ifvbox\AtBeginShipoutBox
      \edef\AtBegShi@restore{%
        \vfuzz=\the\vfuzz\relax
        \vbadness=\the\vbadness\relax
        \dimen\ltx@zero=\the\dimen\ltx@zero\relax
      }%
      \edef\AtBegShi@restorebox{%
        \ht\AtBeginShipoutBox=\the\ht\AtBeginShipoutBox\relax
        \dp\AtBeginShipoutBox=\the\dp\AtBeginShipoutBox\relax
      }%
      \vfuzz=1073741823sp\relax
      \vbadness=2147483647\relax
      \dimen\ltx@zero=\ht\AtBeginShipoutBox
      \advance\dimen\ltx@zero by \dp\AtBeginShipoutBox
      \setbox\AtBeginShipoutBox=\vbox to \dimen\ltx@zero{%
        \setbox\ltx@zero=\hbox{%
          \begingroup
            \AtBegShi@restore
            #1%
          \endgroup
        }%
        \wd\ltx@zero=0pt\relax
        \ht\ltx@zero=0pt\relax
        \dp\ltx@zero=0pt\relax
        \baselineskip=0pt\relax
        \lineskip=0pt\relax
        \lineskiplimit=0pt\relax
        \copy\ltx@zero
        \unvbox\AtBeginShipoutBox
        \kern0pt%
      }%
      \AtBegShi@restore
      \AtBegShi@restorebox
    \fi
  \fi
}
%    \end{macrocode}
%    \end{macro}
%
%    \begin{macro}{\AtBeginShipoutAddToBoxForeground}
%    \begin{macrocode}
\def\AtBeginShipoutAddToBoxForeground#1{%
  \ifhbox\AtBeginShipoutBox
    \edef\AtBegShi@restore{%
      \hfuzz=\the\hfuzz\relax
      \hbadness=\the\hbadness\relax
    }%
    \hfuzz=1073741823sp\relax
    \hbadness=2147483647\relax
    \setbox\AtBeginShipoutBox=\hbox to \wd\AtBeginShipoutBox{%
      \unhcopy\AtBeginShipoutBox
      \kern-\wd\AtBeginShipoutBox
      \setbox\ltx@zero=\hbox{%
        \begingroup
          \AtBegShi@restore
          #1%
        \endgroup
      }%
      \wd\ltx@zero=0pt\relax
      \ht\ltx@zero=0pt\relax
      \dp\ltx@zero=0pt\relax
      \raise\ht\AtBeginShipoutBox\copy\ltx@zero
      \kern\wd\AtBeginShipoutBox
    }%
    \AtBegShi@restore
  \else
    \ifvbox\AtBeginShipoutBox
      \edef\AtBegShi@restore{%
        \vfuzz=\the\vfuzz\relax
        \vbadness=\the\vbadness\relax
        \dimen\ltx@zero=\the\dimen\ltx@zero\relax
      }%
      \edef\AtBegShi@restorebox{%
        \ht\AtBeginShipoutBox=\the\ht\AtBeginShipoutBox\relax
        \dp\AtBeginShipoutBox=\the\dp\AtBeginShipoutBox\relax
      }%
      \vfuzz=1073741823sp\relax
      \vbadness=2147483647\relax
      \dimen\ltx@zero=\ht\AtBeginShipoutBox
      \advance\dimen\ltx@zero by \dp\AtBeginShipoutBox
      \setbox\AtBeginShipoutBox=\vbox to \dimen\ltx@zero{%
        \setbox\ltx@zero=\hbox{%
          \begingroup
            \AtBegShi@restore
            #1%
          \endgroup
        }%
        \wd\ltx@zero=0pt\relax
        \ht\ltx@zero=0pt\relax
        \dp\ltx@zero=0pt\relax
        \baselineskip=0pt\relax
        \lineskip=0pt\relax
        \lineskiplimit=0pt\relax
        \unvbox\AtBeginShipoutBox
        \kern-\dimen\ltx@zero
        \copy\ltx@zero
        \kern\dimen\ltx@zero
      }%
      \AtBegShi@restore
      \AtBegShi@restorebox
    \fi
  \fi
}
%    \end{macrocode}
%    \end{macro}
%
% \subsection{Positioning}
%
%    \begin{macrocode}
\begingroup\expandafter\expandafter\expandafter\endgroup
\expandafter\ifx\csname RequirePackage\endcsname\relax
  \def\TMP@RequirePackage#1[#2]{%
    \begingroup\expandafter\expandafter\expandafter\endgroup
    \expandafter\ifx\csname ver@#1.sty\endcsname\relax
      \input #1.sty\relax
    \fi
  }%
  \TMP@RequirePackage{ifpdf}[2011/01/30]%
\else
  \RequirePackage{ifpdf}[2011/01/30]%
\fi
%    \end{macrocode}
%
%    \begin{macrocode}
\ifpdf
  \def\AtBegShi@horigin{%
    \ifx\pdfhorigin\@undefined\pdfvariable horigin\else\pdfhorigin\fi}%
  \def\AtBegShi@vorigin{%
    \ifx\pdfvorigin\@undefined\pdfvariable vorigin\else\pdfvorigin\fi}%
\else
  \def\AtBegShi@horigin{72.27pt}%
  \def\AtBegShi@vorigin{72.27pt}%
\fi
%    \end{macrocode}
%
%    \begin{macrocode}
\begingroup
\ifcase
    \expandafter\ifx\csname picture\endcsname\relax
      1%
    \else
      \expandafter\ifx\csname endpicture\endcsname\relax
        1%
      \else
        0%
      \fi
    \fi
  \endgroup
  \def\AtBegShi@BeginPicture{%
    \begingroup
    \picture(0,0)\relax
    \begingroup\expandafter\expandafter\expandafter\endgroup
    \expandafter\ifx\csname unitlength\endcsname\relax
    \else
      \unitlength=1pt\relax
    \fi
    \ignorespaces
  }%
  \def\AtBegShi@EndPicture{%
    \endpicture
    \endgroup
  }%
\else
  \endgroup
  \def\AtBegShi@BeginPicture{%
    \setbox\ltx@zero=\hbox\bgroup
    \begingroup
    \ignorespaces
  }%
  \def\AtBegShi@EndPicture{%
    \endgroup
    \egroup
    \ht\ltx@zero=0pt\relax
    \dp\ltx@zero=0pt\relax
    \copy\ltx@zero
  }%
\fi
%    \end{macrocode}
%
%    \begin{macro}{\AtBeginShipoutUpperLeft}
%    A surrounding \cs{rlap} is not necessary, because
%    the stuff is put in an \cs{hbox} with zero width.
%    \begin{macrocode}
\def\AtBeginShipoutUpperLeft#1{%
  \AtBeginShipoutAddToBox{%
    \kern-\AtBegShi@horigin\relax
    \vbox to 0pt{%
      \kern-\AtBegShi@vorigin\relax
      \AtBegShi@BeginPicture
      #1%
      \AtBegShi@EndPicture
      \vss
    }%
  }%
}
%    \end{macrocode}
%    \end{macro}
%
%    \begin{macro}{\AtBeginShipoutUpperLeftForeground}
%    \begin{macrocode}
\def\AtBeginShipoutUpperLeftForeground#1{%
  \AtBeginShipoutAddToBoxForeground{%
    \kern-\AtBegShi@horigin\relax
    \vbox to 0pt{%
      \kern-\AtBegShi@vorigin\relax
      \AtBegShi@BeginPicture
      #1%
      \AtBegShi@EndPicture
      \vss
    }%
  }%
}
%    \end{macrocode}
%    \end{macro}
%
% \subsection{Patches}
%
%    Patches for \LaTeX\ packages that redefine \cs{shipout}.
%    \LaTeX\ is now supposed to use \eTeX. Thus we do not patch,
%    without \LaTeX\ and \eTeX.
%    \begin{macrocode}
\def\AtBegShi@AbortIfUndefined#1{%
  \begingroup\expandafter\expandafter\expandafter\endgroup
  \expandafter\ifx\csname#1\endcsname\relax
    \expandafter\AtBegShi@AtEnd
  \fi
}
\AtBegShi@AbortIfUndefined{currentgrouplevel}%
\AtBegShi@AbortIfUndefined{AtBeginDocument}%
\AtBegShi@AbortIfUndefined{@ifpackageloaded}%
\AtBegShi@AbortIfUndefined{@ifclassloaded}%
%    \end{macrocode}
%
% \subsubsection{Package \xpackage{crop}}
%
% Fix of method and box.
%    \begin{macrocode}
\def\AtBegShi@PatchCrop{%
  \begingroup
    \def\AtBegShi@Crop@shipout{%
      \afterassignment\CROP@ship
      \setbox\@cclv=%
    }%
    \def\AtBegShi@Crop@ship{%
      \ifvoid\@cclv
        \expandafter\aftergroup
      \fi
      \CROP@@ship
    }%
    \def\AtBegShi@Crop@shiplist{%
      \lineskip\z@
      \lineskiplimit\z@
      \baselineskip\z@
      \CROP@kernel
      \box\@cclv
    }%
    \def\AtBegShi@Crop@@ship{%
      \CROP@shipout\vbox{%
        \CROP@shiplist
      }%
    }%
    \ifx\AtBegShi@Crop@ship\CROP@ship
      \ifx\AtBegShi@Crop@shiplist\CROP@shiplist
        \ifx\AtBegShi@Crop@@ship\CROP@@ship
          \let\AtBegShi@found\relax
          \ifx\shipout\AtBegShi@Crop@shipout
            \def\AtBegShi@found{\shipout}%
          \else\ifx\AtBeginShipoutOriginalShipout\AtBegShi@Crop@shipout
            \def\AtBegShi@found{\AtBeginShipoutOriginalShipout}%
          \else\ifx\@EveryShipout@Org@Shipout\AtBegShi@Crop@shipout
            \def\AtBegShi@found{\@EveryShipout@Org@Shipout}%
          \else\ifx\GPTorg@shipout\AtBegShi@Crop@shipout
            \def\AtBegShi@found{\GPTorg@shipout}%
          \else\ifx\THBorg@shipout\AtBegShi@Crop@shipout
            \def\AtBegShi@found{\THBorg@shipout}%
          \else\ifx\mem@oldshipout\AtBegShi@Crop@shipout
            \def\AtBegShi@found{\mem@oldshipout}%
          \fi\fi\fi\fi\fi\fi
          \ifx\AtBegShi@found\relax
          \else
            \expandafter\endgroup
            \expandafter\def\AtBegShi@found{%
              \edef\AtBegShi@GroupLevel{\number\currentgrouplevel}%
              \afterassignment\CROP@ship
              \setbox\AtBeginShipoutBox=%
            }%
            \def\CROP@ship{%
              \ifnum\AtBegShi@GroupLevel=\currentgrouplevel
              \else
                \expandafter\aftergroup
              \fi
              \CROP@@ship
            }%
            \def\CROP@shiplist{%
              \lineskip 0pt\relax
              \lineskiplimit 0pt\relax
              \baselineskip 0pt\relax
              \CROP@kernel
              \box\AtBeginShipoutBox
            }%
            \def\CROP@@ship{%
              \ifvoid\AtBeginShipoutBox
              \else
                \setbox\AtBeginShipoutBox=\vbox{%
                  \CROP@shiplist
                }%
                \AtBegShi@GetBoxSize\AtBeginShipoutBox
                \expandafter\CROP@shipout
                \expandafter\box
                \expandafter\AtBeginShipoutBox
              \fi
            }%
            \@PackageInfoNoLine{atbegshi}{Package `crop' patched}%
            \begingroup
          \fi
        \fi
      \fi
    \fi
  \endgroup
  \let\AtBegShi@PatchCrop\relax
}
\@ifpackageloaded{crop}{%
  \AtBegShi@PatchCrop
}{%
  \AtBeginDocument{\AtBegShi@PatchCrop}%
}
%    \end{macrocode}
%
% \subsubsection{Package \xpackage{everyshi}}
%
%     Fix of method. Use of box 255 is not changed.
%    \begin{macrocode}
\def\AtBegShi@PatchEveryshi{%
  \begingroup
    \long\def\AtBegShi@Everyshi@shipout{%
      \afterassignment\@EveryShipout@Test
      \global\setbox\@cclv= %
    }%
    \long\def\AtBegShi@Everyshi@Test{%
      \ifvoid\@cclv\relax
        \aftergroup\@EveryShipout@Output
      \else
        \@EveryShipout@Output
      \fi
    }%
    \ifx\AtBegShi@Everyshi@Test\@EveryShipout@Test
      \let\AtBegShi@found\relax
      \ifx\shipout\AtBegShi@Everyshi@shipout
        \def\AtBegShi@found{\shipout}%
      \else\ifx\AtBeginShipoutOriginalShipout\AtBegShi@Everyshi@shipout
        \def\AtBegShi@found{\AtBeginShipoutOriginalShipout}%
      \else\ifx\CROP@shipout\AtBegShi@Everyshi@shipout
        \def\AtBegShi@found{\CROP@shipout}%
      \else\ifx\GPTorg@shipout\AtBegShi@Everyshi@shipout
        \def\AtBegShi@found{\GPTorg@shipout}%
      \else\ifx\THBorg@shipout\AtBegShi@Everyshi@shipout
        \def\AtBegShi@found{\THBorg@shipout}%
      \else\ifx\mem@oldshipout\AtBegShi@Everyshi@shipout
        \def\AtBegShi@found{\mem@oldshipout}%
      \else
        \expandafter\ifx\csname @EveryShipout@Org@Shipout\endcsname
                        \relax
           \ifx\@EveryShipout@Shipout\AtBegShi@Everyshi@shipout
             \def\AtBegShi@found{\@EveryShipout@Shipout}%
           \fi
        \fi
      \fi\fi\fi\fi\fi\fi
      \ifx\AtBegShi@found\relax
      \else
        \expandafter\endgroup
        \expandafter\def\AtBegShi@found{%
          \edef\AtBegShi@GroupLevel{\number\currentgrouplevel}%
          \afterassignment\@EveryShipout@Test
          \setbox\AtBeginShipoutBox=%
        }%
        \def\@EveryShipout@Test{%
          \ifnum\AtBegShi@GroupLevel=\currentgrouplevel
          \else
            \expandafter\aftergroup
          \fi
          \AtBegShi@Everyshi@Output
        }%
        \def\AtBegShi@Everyshi@Output{%
          \ifvoid\AtBeginShipoutBox
          \else
            \global\setbox\ltx@cclv\box\AtBeginShipoutBox
            \expandafter\@EveryShipout@Output
          \fi
        }%
        \@PackageInfoNoLine{atbegshi}{Package `everyshi' patched}%
        \begingroup
      \fi
    \fi
  \endgroup
  \let\AtBegShi@PatchEveryshi\relax
}
\@ifpackageloaded{everyshi}{%
  \AtBegShi@PatchEveryshi
}{%
  \AtBeginDocument{\AtBegShi@PatchEveryshi}%
}
%    \end{macrocode}
%
% \subsubsection{Class \xpackage{memoir}}
%
%     Fix of method and box.
%    \begin{macrocode}
\def\AtBegShi@PatchMemoir{%
  \begingroup
    \def\AtBegShi@Memoir@shipout{%
      \afterassignment\mem@shipi
      \setbox\@cclv=%
    }%
    \def\AtBegShi@Memoir@shipi{%
      \ifvoid\@cclv
        \expandafter\aftergroup
      \fi
      \mem@shipii
    }%
    \def\AtBegShi@Memoir@shipiiA{%
      \mem@oldshipout\vbox{%
        \trimmarks
        \unvbox\@cclv
      }%
    }%
    \def\AtBegShi@Memoir@shipiiB{%
      \ifvoid\@cclv
        \mem@oldshipout\box\@cclv
      \else
        \mem@oldshipout\vbox{%
          \trimmarks
          \unvbox\@cclv
        }%
      \fi
    }%
    \def\AtBegShi@Memoir@PatchAB{%
      \ifvoid\AtBeginShipoutBox
      \else
        \setbox\AtBeginShipoutBox=\vbox{%
          \trimmarks
          \ifvbox\AtBeginShipoutBox
            \unvbox\AtBeginShipoutBox
          \else
            \box\AtBeginShipoutBox
          \fi
        }%
        \AtBegShi@GetBoxSize\AtBeginShipoutBox
        \expandafter\mem@oldshipout
        \expandafter\box
        \expandafter\AtBeginShipoutBox
      \fi
    }%
    \def\AtBegShi@Memoir@shipiiC{% 2008/08/07 v1.6180339a
      \ifvoid\@cclv
        \mem@oldshipout\box\@cclv
      \else
        \ifshowtrims
          \mem@oldshipout\vbox{\trimmarks\unvbox\@cclv}%
        \else
          \mem@oldshipout\box\@cclv
        \fi
      \fi
    }%
    \def\AtBegShi@Memoir@shipiiD{% 2011/03/06 v3.6j
      \ifvoid\@cclv
        \mem@oldshipout\box\@cclv
      \else
        \ifshowtrims
          \mem@oldshipout\vbox{%
            \trimmarks
            \nointerlineskip
            \box\@cclv
          }%
        \else
          \mem@oldshipout\box\@cclv
        \fi
      \fi
    }%
    \def\AtBegShi@Memoir@PatchCD{%
      \ifvoid\AtBeginShipoutBox
      \else
        \ifshowtrims
          \setbox\AtBeginShipoutBox=\vbox{%
            \trimmarks
            \nointerlineskip
            \box\AtBeginShipoutBox
          }%
        \fi
        \AtBegShi@GetBoxSize\AtBeginShipoutBox
        \expandafter\mem@oldshipout
        \expandafter\box
        \expandafter\AtBeginShipoutBox
      \fi
    }%
    \ifx\AtBegShi@Memoir@shipi\mem@shipi
      \let\AtBegShi@found\ltx@one
      \ifx\AtBegShi@Memoir@shipiiA\mem@shipii
        \let\AtBegShi@found\ltx@zero
        \global\let\AtBegShi@Memoir@PatchX\AtBegShi@Memoir@PatchAB
      \else\ifx\AtBegShi@Memoir@shipiiB\mem@shipii
        \let\AtBegShi@found\ltx@zero
        \global\let\AtBegShi@Memoir@PatchX\AtBegShi@Memoir@PatchAB
      \else\ifx\AtBegShi@Memoir@shipiiC\mem@shipii
        \let\AtBegShi@found\ltx@zero
        \global\let\AtBegShi@Memoir@PatchX\AtBegShi@Memoir@PatchCD
      \else\ifx\AtBegShi@Memoir@shipiiD\mem@shipii
        \let\AtBegShi@found\ltx@zero
        \global\let\AtBegShi@Memoir@PatchX\AtBegShi@Memoir@PatchCD
      \fi\fi\fi\fi
      \ifcase\AtBegShi@found
        \let\AtBegShi@found\relax
        \ifx\shipout\AtBegShi@Memoir@shipout
          \def\AtBegShi@found{\shipout}%
        \else\ifx\AtBeginShipoutOriginalShipout\AtBegShi@Memoir@shipout
          \def\AtBegShi@found{\AtBeginShipoutOriginalShipout}%
        \else\ifx\CROP@shipout\AtBegShi@Memoir@shipout
          \def\AtBegShi@found{\CROP@shipout}%
        \else\ifx\GPTorg@shipout\AtBegShi@Memoir@shipout
          \def\AtBegShi@found{\GPTorg@shipout}%
        \else\ifx\THBorg@shipout\AtBegShi@Memoir@shipout
          \def\AtBegShi@found{\THBorg@shipout}%
        \else\ifx\@EveryShipout@Org@Shipout\AtBegShi@Memoir@shipout
          \def\AtBegShi@found{\@EveryShipout@Org@Shipout}%
        \fi\fi\fi\fi\fi\fi
        \ifx\AtBegShi@found\relax
        \else
          \expandafter\endgroup
          \expandafter\def\AtBegShi@found{%
            \edef\AtBegShi@GroupLevel{\number\currentgrouplevel}%
            \afterassignment\mem@shipi
            \setbox\AtBeginShipoutBox=%
          }%
          \def\mem@shipi{%
            \ifnum\AtBegShi@GroupLevel=\currentgrouplevel
            \else
              \expandafter\aftergroup
            \fi
            \mem@shipii
          }%
          \let\mem@shipii\AtBegShi@Memoir@PatchX
          \@PackageInfoNoLine{atbegshi}{Class `memoir' patched}%
          \begingroup
        \fi
      \fi
    \fi
  \endgroup
  \let\AtBegShi@PatchMemoir\relax
}
\@ifclassloaded{memoir}{%
  \AtBegShi@PatchMemoir
}{%
  \AtBeginDocument{\AtBegShi@PatchMemoir}%
}
%    \end{macrocode}
%
%    \begin{macrocode}
\AtBegShi@AtEnd%
%</package>
%    \end{macrocode}
%
% \section{Test}
%
% \subsection{Catcode checks for loading}
%
%    \begin{macrocode}
%<*test1>
%    \end{macrocode}
%    \begin{macrocode}
\catcode`\{=1 %
\catcode`\}=2 %
\catcode`\#=6 %
\catcode`\@=11 %
\expandafter\ifx\csname count@\endcsname\relax
  \countdef\count@=255 %
\fi
\expandafter\ifx\csname @gobble\endcsname\relax
  \long\def\@gobble#1{}%
\fi
\expandafter\ifx\csname @firstofone\endcsname\relax
  \long\def\@firstofone#1{#1}%
\fi
\expandafter\ifx\csname loop\endcsname\relax
  \expandafter\@firstofone
\else
  \expandafter\@gobble
\fi
{%
  \def\loop#1\repeat{%
    \def\body{#1}%
    \iterate
  }%
  \def\iterate{%
    \body
      \let\next\iterate
    \else
      \let\next\relax
    \fi
    \next
  }%
  \let\repeat=\fi
}%
\def\RestoreCatcodes{}
\count@=0 %
\loop
  \edef\RestoreCatcodes{%
    \RestoreCatcodes
    \catcode\the\count@=\the\catcode\count@\relax
  }%
\ifnum\count@<255 %
  \advance\count@ 1 %
\repeat

\def\RangeCatcodeInvalid#1#2{%
  \count@=#1\relax
  \loop
    \catcode\count@=15 %
  \ifnum\count@<#2\relax
    \advance\count@ 1 %
  \repeat
}
\def\RangeCatcodeCheck#1#2#3{%
  \count@=#1\relax
  \loop
    \ifnum#3=\catcode\count@
    \else
      \errmessage{%
        Character \the\count@\space
        with wrong catcode \the\catcode\count@\space
        instead of \number#3%
      }%
    \fi
  \ifnum\count@<#2\relax
    \advance\count@ 1 %
  \repeat
}
\def\space{ }
\expandafter\ifx\csname LoadCommand\endcsname\relax
  \def\LoadCommand{\input atbegshi.sty\relax}%
\fi
\def\Test{%
  \RangeCatcodeInvalid{0}{47}%
  \RangeCatcodeInvalid{58}{64}%
  \RangeCatcodeInvalid{91}{96}%
  \RangeCatcodeInvalid{123}{255}%
  \catcode`\@=12 %
  \catcode`\\=0 %
  \catcode`\%=14 %
  \LoadCommand
  \RangeCatcodeCheck{0}{36}{15}%
  \RangeCatcodeCheck{37}{37}{14}%
  \RangeCatcodeCheck{38}{47}{15}%
  \RangeCatcodeCheck{48}{57}{12}%
  \RangeCatcodeCheck{58}{63}{15}%
  \RangeCatcodeCheck{64}{64}{12}%
  \RangeCatcodeCheck{65}{90}{11}%
  \RangeCatcodeCheck{91}{91}{15}%
  \RangeCatcodeCheck{92}{92}{0}%
  \RangeCatcodeCheck{93}{96}{15}%
  \RangeCatcodeCheck{97}{122}{11}%
  \RangeCatcodeCheck{123}{255}{15}%
  \RestoreCatcodes
}
\Test
\csname @@end\endcsname
\end
%    \end{macrocode}
%    \begin{macrocode}
%</test1>
%    \end{macrocode}
%
%    \begin{macrocode}
%<*test2>
\input atbegshi.sty\relax
\def\msg#{\immediate\write16}
\msg{File: atbegshi-test2.tex 2016/06/09 v1.18 Test file for plain-TeX}
\def\testmsg#1#2{%
  \msg{}%
  \msg{*** Test with box (#1), expected page output [#2]}% hash-ok
}

\newbox\voidbox
\def\void{\box\voidbox}
\begingroup
  \setbox\voidbox=\void
\endgroup

\count0=0\relax
\AtBeginShipout{%
  \global\advance\count0 by 1\relax
  \msg{* Inside \string\AtBeginShipout: [\the\count0]}%
}

\AtBeginShipoutFirst{%
  \msg{* Inside \string\AtBeginShipoutFirst}%
  Hello World%
}

\testmsg{\string\null}{1}
\shipout\null

\AtBeginShipoutFirst{%
  This is too late%
}

\testmsg{void}{}
\shipout\void

\testmsg{\string\copy255 (not void)}{2}
\setbox255\hbox{\vrule height 10bp width 10bp}
\shipout\copy255 %

\testmsg{\string\copy255 (again)}{3}
\shipout\copy255 %

\testmsg{\string\box255}{4}
\shipout\box255 %

\testmsg{\string\box255 (again)}{}
\shipout\box255 %

\testmsg{\string\hbox}{5}
\shipout\hbox{\vrule height 5bp width 20bp}

\testmsg{\string\vbox}{6}
\shipout\vbox{\hrule height 20bp width 5bp}

\testmsg{\string\null, voided by hook}{}
\def\VoidBox{%
  \begingroup
    \setbox\AtBeginShipoutBox=\box\AtBeginShipoutBox
  \endgroup
}
\AtBeginShipout{\VoidBox}
\shipout\null
\def\VoidBox{}

\msg{*** \string\begingroup}
\begingroup
  \testmsg{void}{}%
  \shipout\void
\msg{*** \string\endgroup}
\endgroup

\msg{*** \string\begingroup}
\begingroup
  \testmsg{void}{}%
  \shipout\void
  \testmsg{\string\null}{8}%
  \shipout\null
\msg{*** \string\endgroup}
\endgroup

\testmsg{output routine}{9}
Hello World
\vfill
\eject

\testmsg{\string\null\space(discarded)}{}
\AtBeginShipout{%
  \msg{* Inside \string\AtBeginShipout: DISCARD}%
  \AtBeginShipoutDiscard
}
\shipout\null

\end
%</test2>
%    \end{macrocode}
%
%    \begin{macrocode}
%<*test3>
\NeedsTeXFormat{LaTeX2e}
\ProvidesFile{atbegshi-test3.tex}[2016/06/09 v1.18 Test file for LaTeX]
\RequirePackage{color}
\pagecolor{yellow}
\documentclass[a5paper,showtrims]{memoir}
\usepackage{atbegshi}
\AtBeginShipout{%
  \setbox\AtBeginShipoutBox=\vbox{%
    \vbox to 0pt{%
      \kern-1.5in %
      \hbox to 0pt{%
        \kern-1.5in %
        \color{blue}%
        \rule{1in}{1in}%
        \hss
      }%
      \vss
    }%
    \hrule
    \hbox{\vrule\box\AtBeginShipoutBox\vrule}%
    \hrule
  }%
}
\usepackage{eso-pic}
\makeatletter
\@ifundefined{@EveryShipout@Init}{%
  \typeout{Test skipped}%
  \@@end
}{}
\@EveryShipout@Init
\let\@EveryShipout@Init\relax
\makeatother
\AddToShipoutPicture{%
  \hspace{.52\paperwidth}%
  \colorbox{cyan}{%
    \rule{0mm}{\paperheight}%
    \hspace{.48\paperwidth}%
  }%
}
%    \end{macrocode}
%    Newer versions of class \xclass{memoir} emulate package
%    \xpackage{crop} and prevents its loading. This is undone
%    in next line for this test file.
%    \begin{macrocode}
\expandafter\let\csname ver@crop.sty\endcsname\relax
\usepackage[color=red,cross,a4,center]{crop}
\begin{document}
\shipout\null
\shipout\box\csname voidb@x\endcsname
\section{Hello World}
\end{document}
%</test3>
%    \end{macrocode}
%
% \section{Installation}
%
% \subsection{Download}
%
% \paragraph{Package.} This package is available on
% CTAN\footnote{\url{http://ctan.org/pkg/atbegshi}}:
% \begin{description}
% \item[\CTAN{macros/latex/contrib/oberdiek/atbegshi.dtx}] The source file.
% \item[\CTAN{macros/latex/contrib/oberdiek/atbegshi.pdf}] Documentation.
% \end{description}
%
%
% \paragraph{Bundle.} All the packages of the bundle `oberdiek'
% are also available in a TDS compliant ZIP archive. There
% the packages are already unpacked and the documentation files
% are generated. The files and directories obey the TDS standard.
% \begin{description}
% \item[\CTAN{install/macros/latex/contrib/oberdiek.tds.zip}]
% \end{description}
% \emph{TDS} refers to the standard ``A Directory Structure
% for \TeX\ Files'' (\CTAN{tds/tds.pdf}). Directories
% with \xfile{texmf} in their name are usually organized this way.
%
% \subsection{Bundle installation}
%
% \paragraph{Unpacking.} Unpack the \xfile{oberdiek.tds.zip} in the
% TDS tree (also known as \xfile{texmf} tree) of your choice.
% Example (linux):
% \begin{quote}
%   |unzip oberdiek.tds.zip -d ~/texmf|
% \end{quote}
%
% \paragraph{Script installation.}
% Check the directory \xfile{TDS:scripts/oberdiek/} for
% scripts that need further installation steps.
% Package \xpackage{attachfile2} comes with the Perl script
% \xfile{pdfatfi.pl} that should be installed in such a way
% that it can be called as \texttt{pdfatfi}.
% Example (linux):
% \begin{quote}
%   |chmod +x scripts/oberdiek/pdfatfi.pl|\\
%   |cp scripts/oberdiek/pdfatfi.pl /usr/local/bin/|
% \end{quote}
%
% \subsection{Package installation}
%
% \paragraph{Unpacking.} The \xfile{.dtx} file is a self-extracting
% \docstrip\ archive. The files are extracted by running the
% \xfile{.dtx} through \plainTeX:
% \begin{quote}
%   \verb|tex atbegshi.dtx|
% \end{quote}
%
% \paragraph{TDS.} Now the different files must be moved into
% the different directories in your installation TDS tree
% (also known as \xfile{texmf} tree):
% \begin{quote}
% \def\t{^^A
% \begin{tabular}{@{}>{\ttfamily}l@{ $\rightarrow$ }>{\ttfamily}l@{}}
%   atbegshi.sty & tex/generic/oberdiek/atbegshi.sty\\
%   atbegshi.pdf & doc/latex/oberdiek/atbegshi.pdf\\
%   atbegshi-example1.tex & doc/latex/oberdiek/atbegshi-example1.tex\\
%   atbegshi-example2.tex & doc/latex/oberdiek/atbegshi-example2.tex\\
%   test/atbegshi-test1.tex & doc/latex/oberdiek/test/atbegshi-test1.tex\\
%   test/atbegshi-test2.tex & doc/latex/oberdiek/test/atbegshi-test2.tex\\
%   test/atbegshi-test3.tex & doc/latex/oberdiek/test/atbegshi-test3.tex\\
%   atbegshi.dtx & source/latex/oberdiek/atbegshi.dtx\\
% \end{tabular}^^A
% }^^A
% \sbox0{\t}^^A
% \ifdim\wd0>\linewidth
%   \begingroup
%     \advance\linewidth by\leftmargin
%     \advance\linewidth by\rightmargin
%   \edef\x{\endgroup
%     \def\noexpand\lw{\the\linewidth}^^A
%   }\x
%   \def\lwbox{^^A
%     \leavevmode
%     \hbox to \linewidth{^^A
%       \kern-\leftmargin\relax
%       \hss
%       \usebox0
%       \hss
%       \kern-\rightmargin\relax
%     }^^A
%   }^^A
%   \ifdim\wd0>\lw
%     \sbox0{\small\t}^^A
%     \ifdim\wd0>\linewidth
%       \ifdim\wd0>\lw
%         \sbox0{\footnotesize\t}^^A
%         \ifdim\wd0>\linewidth
%           \ifdim\wd0>\lw
%             \sbox0{\scriptsize\t}^^A
%             \ifdim\wd0>\linewidth
%               \ifdim\wd0>\lw
%                 \sbox0{\tiny\t}^^A
%                 \ifdim\wd0>\linewidth
%                   \lwbox
%                 \else
%                   \usebox0
%                 \fi
%               \else
%                 \lwbox
%               \fi
%             \else
%               \usebox0
%             \fi
%           \else
%             \lwbox
%           \fi
%         \else
%           \usebox0
%         \fi
%       \else
%         \lwbox
%       \fi
%     \else
%       \usebox0
%     \fi
%   \else
%     \lwbox
%   \fi
% \else
%   \usebox0
% \fi
% \end{quote}
% If you have a \xfile{docstrip.cfg} that configures and enables \docstrip's
% TDS installing feature, then some files can already be in the right
% place, see the documentation of \docstrip.
%
% \subsection{Refresh file name databases}
%
% If your \TeX~distribution
% (\teTeX, \mikTeX, \dots) relies on file name databases, you must refresh
% these. For example, \teTeX\ users run \verb|texhash| or
% \verb|mktexlsr|.
%
% \subsection{Some details for the interested}
%
% \paragraph{Attached source.}
%
% The PDF documentation on CTAN also includes the
% \xfile{.dtx} source file. It can be extracted by
% AcrobatReader 6 or higher. Another option is \textsf{pdftk},
% e.g. unpack the file into the current directory:
% \begin{quote}
%   \verb|pdftk atbegshi.pdf unpack_files output .|
% \end{quote}
%
% \paragraph{Unpacking with \LaTeX.}
% The \xfile{.dtx} chooses its action depending on the format:
% \begin{description}
% \item[\plainTeX:] Run \docstrip\ and extract the files.
% \item[\LaTeX:] Generate the documentation.
% \end{description}
% If you insist on using \LaTeX\ for \docstrip\ (really,
% \docstrip\ does not need \LaTeX), then inform the autodetect routine
% about your intention:
% \begin{quote}
%   \verb|latex \let\install=y% \iffalse meta-comment
%
% File: atbegshi.dtx
% Version: 2016/06/09 v1.18
% Info: At begin shipout hook
%
% Copyright (C) 2007-2011 by
%    Heiko Oberdiek <heiko.oberdiek at googlemail.com>
%    2016
%    https://github.com/ho-tex/oberdiek/issues
%
% This work may be distributed and/or modified under the
% conditions of the LaTeX Project Public License, either
% version 1.3c of this license or (at your option) any later
% version. This version of this license is in
%    http://www.latex-project.org/lppl/lppl-1-3c.txt
% and the latest version of this license is in
%    http://www.latex-project.org/lppl.txt
% and version 1.3 or later is part of all distributions of
% LaTeX version 2005/12/01 or later.
%
% This work has the LPPL maintenance status "maintained".
%
% This Current Maintainer of this work is Heiko Oberdiek.
%
% The Base Interpreter refers to any `TeX-Format',
% because some files are installed in TDS:tex/generic//.
%
% This work consists of the main source file atbegshi.dtx
% and the derived files
%    atbegshi.sty, atbegshi.pdf, atbegshi.ins, atbegshi.drv,
%    atbegshi-example1.tex, atbegshi-example2.tex,
%    atbegshi-test1.tex, atbegshi-test2.tex, atbegshi-test3.tex.
%
% Distribution:
%    CTAN:macros/latex/contrib/oberdiek/atbegshi.dtx
%    CTAN:macros/latex/contrib/oberdiek/atbegshi.pdf
%
% Unpacking:
%    (a) If atbegshi.ins is present:
%           tex atbegshi.ins
%    (b) Without atbegshi.ins:
%           tex atbegshi.dtx
%    (c) If you insist on using LaTeX
%           latex \let\install=y% \iffalse meta-comment
%
% File: atbegshi.dtx
% Version: 2016/06/09 v1.18
% Info: At begin shipout hook
%
% Copyright (C) 2007-2011 by
%    Heiko Oberdiek <heiko.oberdiek at googlemail.com>
%    2016
%    https://github.com/ho-tex/oberdiek/issues
%
% This work may be distributed and/or modified under the
% conditions of the LaTeX Project Public License, either
% version 1.3c of this license or (at your option) any later
% version. This version of this license is in
%    http://www.latex-project.org/lppl/lppl-1-3c.txt
% and the latest version of this license is in
%    http://www.latex-project.org/lppl.txt
% and version 1.3 or later is part of all distributions of
% LaTeX version 2005/12/01 or later.
%
% This work has the LPPL maintenance status "maintained".
%
% This Current Maintainer of this work is Heiko Oberdiek.
%
% The Base Interpreter refers to any `TeX-Format',
% because some files are installed in TDS:tex/generic//.
%
% This work consists of the main source file atbegshi.dtx
% and the derived files
%    atbegshi.sty, atbegshi.pdf, atbegshi.ins, atbegshi.drv,
%    atbegshi-example1.tex, atbegshi-example2.tex,
%    atbegshi-test1.tex, atbegshi-test2.tex, atbegshi-test3.tex.
%
% Distribution:
%    CTAN:macros/latex/contrib/oberdiek/atbegshi.dtx
%    CTAN:macros/latex/contrib/oberdiek/atbegshi.pdf
%
% Unpacking:
%    (a) If atbegshi.ins is present:
%           tex atbegshi.ins
%    (b) Without atbegshi.ins:
%           tex atbegshi.dtx
%    (c) If you insist on using LaTeX
%           latex \let\install=y\input{atbegshi.dtx}
%        (quote the arguments according to the demands of your shell)
%
% Documentation:
%    (a) If atbegshi.drv is present:
%           latex atbegshi.drv
%    (b) Without atbegshi.drv:
%           latex atbegshi.dtx; ...
%    The class ltxdoc loads the configuration file ltxdoc.cfg
%    if available. Here you can specify further options, e.g.
%    use A4 as paper format:
%       \PassOptionsToClass{a4paper}{article}
%
%    Programm calls to get the documentation (example):
%       pdflatex atbegshi.dtx
%       makeindex -s gind.ist atbegshi.idx
%       pdflatex atbegshi.dtx
%       makeindex -s gind.ist atbegshi.idx
%       pdflatex atbegshi.dtx
%
% Installation:
%    TDS:tex/generic/oberdiek/atbegshi.sty
%    TDS:doc/latex/oberdiek/atbegshi.pdf
%    TDS:doc/latex/oberdiek/atbegshi-example1.tex
%    TDS:doc/latex/oberdiek/atbegshi-example2.tex
%    TDS:doc/latex/oberdiek/test/atbegshi-test1.tex
%    TDS:doc/latex/oberdiek/test/atbegshi-test2.tex
%    TDS:doc/latex/oberdiek/test/atbegshi-test3.tex
%    TDS:source/latex/oberdiek/atbegshi.dtx
%
%<*ignore>
\begingroup
  \catcode123=1 %
  \catcode125=2 %
  \def\x{LaTeX2e}%
\expandafter\endgroup
\ifcase 0\ifx\install y1\fi\expandafter
         \ifx\csname processbatchFile\endcsname\relax\else1\fi
         \ifx\fmtname\x\else 1\fi\relax
\else\csname fi\endcsname
%</ignore>
%<*install>
\input docstrip.tex
\Msg{************************************************************************}
\Msg{* Installation}
\Msg{* Package: atbegshi 2016/06/09 v1.18 At begin shipout hook (HO)}
\Msg{************************************************************************}

\keepsilent
\askforoverwritefalse

\let\MetaPrefix\relax
\preamble

This is a generated file.

Project: atbegshi
Version: 2016/06/09 v1.18

Copyright (C) 2007-2011 by
   Heiko Oberdiek <heiko.oberdiek at googlemail.com>

This work may be distributed and/or modified under the
conditions of the LaTeX Project Public License, either
version 1.3c of this license or (at your option) any later
version. This version of this license is in
   http://www.latex-project.org/lppl/lppl-1-3c.txt
and the latest version of this license is in
   http://www.latex-project.org/lppl.txt
and version 1.3 or later is part of all distributions of
LaTeX version 2005/12/01 or later.

This work has the LPPL maintenance status "maintained".

This Current Maintainer of this work is Heiko Oberdiek.

The Base Interpreter refers to any `TeX-Format',
because some files are installed in TDS:tex/generic//.

This work consists of the main source file atbegshi.dtx
and the derived files
   atbegshi.sty, atbegshi.pdf, atbegshi.ins, atbegshi.drv,
   atbegshi-example1.tex, atbegshi-example2.tex,
   atbegshi-test1.tex, atbegshi-test2.tex, atbegshi-test3.tex.

\endpreamble
\let\MetaPrefix\DoubleperCent

\generate{%
  \file{atbegshi.ins}{\from{atbegshi.dtx}{install}}%
  \file{atbegshi.drv}{\from{atbegshi.dtx}{driver}}%
  \usedir{tex/generic/oberdiek}%
  \file{atbegshi.sty}{\from{atbegshi.dtx}{package}}%
  \usedir{doc/latex/oberdiek}%
  \file{atbegshi-example1.tex}{\from{atbegshi.dtx}{example1}}%
  \file{atbegshi-example2.tex}{\from{atbegshi.dtx}{example2}}%
  \usedir{doc/latex/oberdiek/test}%
  \file{atbegshi-test1.tex}{\from{atbegshi.dtx}{test1}}%
  \file{atbegshi-test2.tex}{\from{atbegshi.dtx}{test2}}%
  \file{atbegshi-test3.tex}{\from{atbegshi.dtx}{test3}}%
  \nopreamble
  \nopostamble
  \usedir{source/latex/oberdiek/catalogue}%
  \file{atbegshi.xml}{\from{atbegshi.dtx}{catalogue}}%
}

\catcode32=13\relax% active space
\let =\space%
\Msg{************************************************************************}
\Msg{*}
\Msg{* To finish the installation you have to move the following}
\Msg{* file into a directory searched by TeX:}
\Msg{*}
\Msg{*     atbegshi.sty}
\Msg{*}
\Msg{* To produce the documentation run the file `atbegshi.drv'}
\Msg{* through LaTeX.}
\Msg{*}
\Msg{* Happy TeXing!}
\Msg{*}
\Msg{************************************************************************}

\endbatchfile
%</install>
%<*ignore>
\fi
%</ignore>
%<*driver>
\NeedsTeXFormat{LaTeX2e}
\ProvidesFile{atbegshi.drv}%
  [2016/06/09 v1.18 At begin shipout hook (HO)]%
\documentclass{ltxdoc}
\usepackage{holtxdoc}[2011/11/22]
\begin{document}
  \DocInput{atbegshi.dtx}%
\end{document}
%</driver>
% \fi
%
%
% \CharacterTable
%  {Upper-case    \A\B\C\D\E\F\G\H\I\J\K\L\M\N\O\P\Q\R\S\T\U\V\W\X\Y\Z
%   Lower-case    \a\b\c\d\e\f\g\h\i\j\k\l\m\n\o\p\q\r\s\t\u\v\w\x\y\z
%   Digits        \0\1\2\3\4\5\6\7\8\9
%   Exclamation   \!     Double quote  \"     Hash (number) \#
%   Dollar        \$     Percent       \%     Ampersand     \&
%   Acute accent  \'     Left paren    \(     Right paren   \)
%   Asterisk      \*     Plus          \+     Comma         \,
%   Minus         \-     Point         \.     Solidus       \/
%   Colon         \:     Semicolon     \;     Less than     \<
%   Equals        \=     Greater than  \>     Question mark \?
%   Commercial at \@     Left bracket  \[     Backslash     \\
%   Right bracket \]     Circumflex    \^     Underscore    \_
%   Grave accent  \`     Left brace    \{     Vertical bar  \|
%   Right brace   \}     Tilde         \~}
%
% \GetFileInfo{atbegshi.drv}
%
% \title{The \xpackage{atbegshi} package}
% \date{2016/06/09 v1.18}
% \author{Heiko Oberdiek\thanks
% {Please report any issues at https://github.com/ho-tex/oberdiek/issues}\\
% \xemail{heiko.oberdiek at googlemail.com}}
%
% \maketitle
%
% \begin{abstract}
% This package is a modern reimplementation of package \xpackage{everyshi}
% without the burden of compatibility. It makes use of \eTeX's if available.
% Both \LaTeX\ and \plainTeX\ are supported.
% \end{abstract}
%
% \tableofcontents
%
% \section{Documentation}
%
% Package \xpackage{atbegshi} redefines \cs{shipout} to insert
% hooks for user code that is executed before the page is shipped out.
% The code may modify or even discard the output page.
% Three hooks are implemented:
% \begin{enumerate}
% \item A hook that is executed for every page, see\\
%       \cs{AtBeginShipout}
% \item A hook that is executed for the next page only, see\\
%       \cs{AtBeginShipoutNext}
% \item A hook that is only executed for the first page, see\\
%       \cs{AtBeginShipoutFirst}
% \end{enumerate}
% The hooks are executed in this order. The following three macros
% provide the user interface for adding code to these hooks:
%
% \begin{declcs}{AtBeginShipout} \M{code}\\
%   \cs{AtBeginShipoutBox}
% \end{declcs}
% Execute the \meta{code} for every page.
% The page contents is held in box register \cs{AtBeginShipoutBox}
% and may be modified. Use \cs{AtBeginShipoutDiscard} if you want to
% discard the page.
%
% \emph{Note:} Package \xpackage{everyshi} uses box register 255. With
% package \xpackage{atbegshi} you must use \cs{AtBeginShipoutBox} instead.
%
% If \LaTeX\ calls \cs{shipout} in \cs{@outputpage} (part of its output
% routine),
% the meaning of \cs{protect} is \cs{noexpand}. \LaTeX\ sets
% \cs{protect} to the appropriate \cs{@typeset@protect} in the box
% that is shipped out. This is too late for the hooks, they
% are called earlier in the redefined \cs{shipout}. Therefore
% package \xpackage{atbegshi} sets \cs{protect} to \cs{@typeset@protect}
% before it calls the hooks.
% (In \cs{EveryShipout} of package \xpackage{everyshi} the
% user is responsible for the correct setting of \cs{protect}.)
%
% \begin{declcs}{AtBeginShipoutNext} \M{code}
% \end{declcs}
% This reimplements package \xpackage{everyshi}'s \cs{AtNextShipout}.
% The \meta{code} is executed at shipout time of the next page only.
% It is just a convenience macro, it can be easily replaced by something
% like:
%\begin{quote}
%\begin{verbatim}
%\newcommand{\MyShipoutHook}{}%
%\AtBeginShipout{\MyShipoutHook}
%\gdef\MyShipoutHook{%
%  ... do something with next page ...
%  \gdef\MyShipoutHook{}%
% }
%\end{verbatim}
%\end{quote}
% (This can be necessary, if hook order does matter).
%
% \begin{declcs}{AtBeginShipoutFirst} \M{code}
% \end{declcs}
% This reimplements \LaTeX's \cs{AtBeginDvi}. This hook is
% usually used for \cs{special} commands that include PostScript
% header files. The \cs{code} is directly executed in a \cs{vbox}
% that is put at the beginning of the output page. Dealing with
% the output box \cs{AtBeginShipoutBox} is not necessary and
% not permitted here.
%
% \begin{declcs}{AtBeginShipoutDiscard}
% \end{declcs}
% This macro notifies package \xpackage{atbegshi} that
% the output page is discarded. The remaining hook code
% and the remaining hooks are not executed and the page is thrown away.
% Also \cs{deadcycles} is cleared to zero like an ordinary \cs{shipout}
% would do.
%
% \begin{declcs}{AtBeginShipoutInit}
% \end{declcs}
% Usually the redefinition of \cs{shipout} is delayed by
% \cs{AtBeginDocument} (if this macro exists). This can be
% too late, if other packages also redefines \cs{shipout}
% and the order does matter.
% \cs{AtBeginShipoutInit} forces the immediate redefinition
% of \cs{shipout}.
%
% \begin{declcs}{AtBeginShipoutAddToBox} \M{stuff}\\
%   \cs{AtBeginShipoutAddToBoxForeground} \M{stuff}
% \end{declcs}
% A quite common use case is the addition of \cs{special}
% or other whatsits to the page output box. Macro
% \cs{AtBeginShipoutAddToBox} puts \meta{stuff} in a
% box with zeroed dimensions. The box with the \meta{stuff}
% is put in the upper left corner of the shipout box
% \cs{AtBeginShipoutBox}. Macro \cs{AtBeginShipoutAddToBox}
% puts the \meta{stuff} in the background, the other macro
% \cs{AtBeginShipoutAddToBoxForeground} in the foreground
% after the original shipout box contents is set.
%
% A void shipout box (that means a discarded page)
% remains void that means \meta{stuff} is ignored in this case.
% The box type of \cs{AtBeginShipoutBox} is preserved.
% Also the box nesting level for the original contents
% of \cs{AtBeginShipoutBox} remains, for example, to avoid trouble
% with links across pages in case of \hologo{pdfTeX}.
%
% \begin{declcs}{AtBeginShipoutUpperLeft} \M{background material}
% \end{declcs}
% This is a macro that puts material in the background of
% box \cs{AtBeginShipoutBox}.
% The \meta{background material} is set in an \cs{hbox}, the
% reference point is the upper left corner of the output page.
% In case of \pdfTeX\ in PDF mode, the settings of
% \cs{pdfhorigin} and \cs{pdfvorigin} are respected.
%
% The macro \cs{AtBeginShipoutUpperLeft} is intended to be used
% in one of the hook setting macros, such as
% \cs{AtBeginShipout}, \cs{AtBeginShipoutFirst}, or
% \cs{AtBeginShipoutNext}.
%
% For \LaTeX\ users the \meta{background material} is set
% inside a \texttt{picture} environment:
% \begin{quote}
%   |\begin{picture}(0,0)|\\
%   \mbox{}\quad |\setlength{\unitlength}{1pt}%|\\
%   \mbox{}\quad \meta{background material}\\
%   |\end{picture}|
% \end{quote}
%
% \begin{declcs}{AtBeginShipoutUpperLeftForeground} \M{foreground material}
% \end{declcs}
% See \cs{AtBeginShipoutUpperLeft}. The difference is that the material
% is put in the foreground.
%
% \begin{declcs}{AtBeginShipoutOriginalShipout} \meta{box}
% \end{declcs}
% It stores the meaning of \cs{shipout} at the time this package
% is loaded.
%
% \begin{declcs}{AtBeginShipoutBoxWidth}\\
%   \cs{AtBeginShipoutBoxHeight}\\
%   \cs{AtBeginShipoutBoxDepth}\\
% \end{declcs}
% These macros store the dimensions of the output box \cs{AtBeginShipoutBox}
% before the original shipout is called. If \cs{shipout} is not
% redefined before the package loading or the box dimensions
% are not changed by the redefined \cs{shipout}, these macros
% contain the dimensions of the shipout box. These values can be
% remembered by \cs{label} and \cs{ref}. For example, this is done
% by the package module \xpackage{zref-pagelayout} of project \xpackage{zref}.
% The dimensions of the shipout page can be used in some \hologo{TeX}
% engines (\hologo{pdfTeX} in PDF mode, \hologo{XeTeX}) to calculate
% the media size of the shipout page if \cs{pdfpagewidth} and
% \cs{pdfpageheight} are not set.
%
% \subsection{Examples}
%
% \subsubsection{Example: circle in background}
%
% In this example we put a circle in the background in the middle
% of the paper.
%
%    \begin{macrocode}
%<*example1>
%    \end{macrocode}
%    \begin{macrocode}
\documentclass[a4paper]{article}
\usepackage{color}
\usepackage{atbegshi}
%    \end{macrocode}
%    Package \xpackage{picture} makes life a little easier, because
%    we can now also use length specifications in |picture|'s commands.
%    \begin{macrocode}
\usepackage{picture}
%    \end{macrocode}
%    Now we draw the circle in the middle of the paper.
%    \cs{put} moves downwards, because the origin is at the top
%    of the page, not at its bottom.
%    \begin{macrocode}
\AtBeginShipout{%
  \AtBeginShipoutUpperLeft{%
    \put(0.5\paperwidth,-0.5\paperheight){\circle{10}}%
  }%
}
\begin{document}
\section{Hello World}
\newpage
\AtBeginShipoutNext{%
  \AtBeginShipoutUpperLeft{%
    \color{red}%
    \put(0,-0.5\paperheight){\line(1,0){\paperwidth}}%
    \put(0.5\paperwidth, 0){\line(0,-1){\paperheight}}%
  }%
}
Only on this page we add a red cross.
\newpage
This page has the circle only.
\par
\vspace{\fill}
The next page will be discarded.
\newpage
\AtBeginShipoutNext{%
  \AtBeginShipoutDiscard
}
This page is discarded.
\newpage
The last page.
\end{document}
%    \end{macrocode}
%    \begin{macrocode}
%</example1>
%    \end{macrocode}
%
% \subsubsection{Example: adding TrimBox for dvipdfmx}
%
% Now an example from ``real life'' follows. Someone from the mailing  list
% for dvipdfmx wants to put a TrimBox on every page. If we use
% \cs{AtBeginShipout}, we have to put the \cs{special} inside the
% box \cs{AtBeginShipoutBox} that gets shipped out.
%    \begin{macrocode}
%<*example2>
\documentclass{minimal}
\usepackage{atbegshi}
\usepackage[
  dvipdfm,
  paperwidth=630bp,
  paperheight=810bp
]{geometry}
\AtBeginShipout{%
  \setbox\AtBeginShipoutBox=\hbox{%
    \special{pdf: put @thispage <</TrimBox[9 9 621 801]>>}%
    \box\AtBeginShipoutBox
  }%
}
\begin{document}
  First page
  \newpage
  Second page
\end{document}
%</example2>
%    \end{macrocode}
% Remember, in \cs{AtBeginShipoutBoxFirst} the \cs{setbox} wrapper code
% is implicitly given and the \cs{special} is used directly.
%
% \section{Method of \cs{shipout} overloading}
%
% \subsection{\cs{shipout}}
%
% The \TeX\ primitive command \cs{shipout} takes a box specification
% and puts the box as a new page in the output file.
% There are two kinds of box specifications:
% \begin{description}
% \item[Direct boxes:]
%   They are given by \cs{hbox}, \cs{vbox}, or \cs{vtop},\\
%   e.g. |\shipout\hbox{Hello World}|.
% \item[Indirect boxes:]
%   \cs{box} or \cs{copy} references a box register by number.
%   The box register contains the contents of the box.
%
%   \emph{Note:} \cs{box} also clears the box register globally.
% \end{description}
% Then we have to differentiate between void and empty boxes:
% \begin{description}
% \item[Void:]
%   Initially or after \cs{box} there is no box in the box register.
%   In this cases the box register is not empty, but \emph{void}.
% \item[Empty:]
%   A box with empty contents, such as
%   |\hbox{}| (= \cs{null}) or |\vbox{}| is an \emph{empty \cs{hbox}}
%   or \emph{empty \cs{vbox}}. If a box register holds such a
%   box, the box still exists, therefore the box register is
%   \emph{not void}.
% \end{description}
%
% \subsection{\cs{afterassignment}}
%
% We want to overload \cs{shipout} to do something with the box.
% It is quite impossible to do this reliable by catching the box
% using macro arguments. The
% variety of box specifications is too large, Examples:
% \begin{quote}
%   |\shipout\null|\\
%   |\shipout\vbox{...}|\\
%   |\shipout\vtop\bgroup ...\egroup|\\
%   |\shipout\box255|
% \end{quote}
% Even worse, the braces don't need to be balanced:
% \begin{quote}
%   |\shipout\hbox\bgroup}|\\
%   |\shipout\vbox{\egroup|
% \end{quote}
% Happily TeX provides a reliable way via \cs{afterassignment}.
% It takes a macro name and executes it just after the assignment.
%
% Now we can redefine \cs{shipout}. The box specification that
% follows \cs{shipout} is catched by \cs{setbox}. This is an
% assignment to a box register. \cs{afterassignment}
% notifies \TeX, that we want to call \cs{@test} right after
% the assignment:
%\begin{quote}
%\begin{verbatim}
%\shipout :=
%  \afterassignment\@test
%  \setbox\mybox=
%\end{verbatim}
%\end{quote}
% We have seen different box specifications. Indirect boxes are easy
% to understand:
% \begin{quote}
%   |\shipout\box0| $\Rightarrow$ |\setbox\mybox=\box0 |\fbox{\cs{@test}}
% \end{quote}
% However direct boxes can have arbitrary contents with lots of
% other assignments. It would be quite inpredictable if \TeX\ would
% put |\@test| after the first of such an assignment or after the
% box specification if the box lacks of assignments.
% Therefore \TeX\ puts |\@test| right at the beginning of the
% box specification, e.g:
% \begin{quote}
%   |\shipout\hbox{Hello World}|\\
%   $\Rightarrow$
%   |\setbox\mybox=\hbox{|\fbox{\cs{@test}}| Hello World}|
% \end{quote}
%
% \subsection{Test for direct or indirect boxes}
%
% Now we want to execute \cs{@test}, but where are we?
% We can be after the completed box assignment, if \cs{shipout}
% was called with an indirect box. Or we are right at the beginning
% of a direct box.
%
% \subsubsection{With \eTeX}
%
% With the \eTeX's extensions the answer is very easy:
% Being inside the direct box means that we are inside a new group.
% The new primitive command \cs{currentgrouplevel} tells how
% deeply the groups are currently nested. Macro \cs{@test} just
% compares the previously stored group level with the current one:
%\begin{quote}
%\begin{verbatim}
%\shipout :=
%  \edef\saved@grouplevel{\number\currentgrouplevel}
%  \afterassignment\@test
%  \setbox\mybox=
%
%\@test :=
%  \ifnum\saved@grouplevel=\currentgrouplevel
%    % case: indirect box, the assignment is completed
%    \@output
%  \else
%    % case: direct box, we are inside the box
%    \aftergroup\@outbox
%  \fi
%\end{verbatim}
%\end{quote}
%
% \subsubsection{Without \eTeX}
%
% Life becomes complicate without \eTeX. We cannot ask the
% group level. However, if we are inside a direct box,
% the box register \cs{mybox} is not yet changed by \cs{setbox}.
% Thus we need a special initial value and compare
% it in \cs{@test} with the current value of the box.
%
% What can be used as initial value? Arbitrary box contents cannot
% be compared. \TeX\ only tells us a few properties:
% \begin{itemize}
% \item Box type:\cs{ifhbox}, \cs{ifvbox}
% \item Dimensions: \cs{wd}, \cs{ht}, \cs{dp}
% \item Voidness: \cs{ifvoid}
% \end{itemize}
% Unhappily all these qualities even combined are not sufficient
% for constructing an initial box value, because \cs{shipout} can be
% called with a box that is accidently just the same as the choosen
% initial value.
%
% Nevertheless we have two alternatives for an initial value:
% \begin{itemize}
% \item A box of some type with some funny settings that are
% unlikely to occur in real life, e.g a height of |4911sp-\maxdimen|.
% \item A void box.
% \end{itemize}
% A collision between this initial value and an indirect \cs{shipout}
% box with just the same value is possible.
% Then \cs{@test} will make a wrong decision that it is executed
% inside a direct box and delays \cs{@output} by \cs{aftergroup}.
% Thus \cs{@output} is not called at the place we want.
% In contrary, the result is an uncertainty about the place:
% \begin{itemize}
% \item \cs{shipout} is used in a group that perhaps closes some
%   pages later. A bad place for \cs{@output}.
% \item Without a surrounding group \cs{aftergroup} effectively kills
%   its argument.
% \end{itemize}
% In the first case of a box with special dimensions we can even loose
% the page. However in the case of the void box, this effect is even
% desired, because the original \cs{shipout} does not output void boxes.
% All we have to do is to ensure that our box \cs{mybox} is always
% void except for the phase when the overloaded \cs{shipout} is executed.
% And secondly we must keep this semantics of \cs{shipout} for the void case
% in our macros, namely \cs{@output}.
%\begin{quote}
%\begin{verbatim}
%\shipout :=
%  % trick to get a void box \mybox
%  \begingroup
%    \setbox\mybox=\box\mybox
%  \endgroup
%  \afterassignment\@test
%  \setbox\mybox=
%
%\@test :=
%  \ifvoid\mybox
%    \aftergroup\@output
%  \else
%    \@output
%  \fi
%\end{verbatim}
%\end{quote}
% The nasty case is |\shipout\box\voidb@x| where the indirect box
% is void and that must not generate an output page.
% If a surrounding group is missing the output is ignored
% because of \cs{aftergroup}.
% Otherwise \@output is called some time
% later when the surrounding group closes.
% But \cs{mybox} is void outside the execution phase
% of the redefined \cs{shipout}. Also \cs{@output} checks for a
% void box and cancels the page output.
% The disadvantage remains that the hook in \cs{@output} is called
% for a page that will not be output.
%
% \subsubsection{\cs{lastkern} method}
%
% At the beginning of a new box, there is no \cs{kern}, the contents
% of the box is still empty and \cs{lastkern} returns 0\,pt.
% This can be used to distinguish between direct and indirect boxes:
% We execute \cs{setbox} in a box with a preceding non-zero kern.
% After an indirect box, \cs{lastkern} sees this kern, otherwise
% it returns 0\,pt.
%\begin{quote}
%\begin{verbatim}
%\shipout :=
%  \begingroup
%    \setbox\mybox=\hbox\bgroup
%      \kern1pt
%      \afterassignment\shipout@test
%      \global\setbox\mybox=
%\@test :=
%  \ifdim\lastkern=0pt
%    % direct box
%    \aftergroup\egroup
%    \aftergroup\endgroup
%    \aftergroup\@output
%  \else
%    \egroup
%    \endgroup
%    \@output
%  \fi
%\end{verbatim}
%\end{quote}
% We have two \cs{setbox} commands. The first creates a controlled
% context box where we can safely insert a \cs{kern}. We get rid of
% this temporarily used context box by putting the local \cs{setbox}
% in a group.
%
% After the group we want to have our shipout box in \cs{mybox}.
% Therefore we use a global assignment here.
%
% \subsection{Output}
%
% With or without \eTeX\ we ensure the original behaviour of
% \cs{shipout} that void boxes do not generate output pages.
%
% Now we can place the hook \cs{@hook} for the user code that
% wants to manipulate the output box.
%\begin{quote}
%\begin{verbatim}
%\@output :=
%  \ifvoid\mybox
%    % cancel output of void box
%  \else
%    \@hook
%    \ifvoid\mybox
%      % user code in \@hook could has voided the box
%    \else
%      \original@shipout\box\mybox
%    \fi
%  \fi
%\end{verbatim}
%\end{quote}
%
% \subsection{Separate box register}
%
% So far we have said nothing about the box number of \cs{mybox}.
% The following case that outputs the same page twice
% shows that we are not free in the use
% of the box register:
% \begin{quote}
%   |\shipout\copy<num> \shipout\box<num>|
% \end{quote}
% We manipulate the box by the hook and without \eTeX\ the box
% must even be voided. However, the use case above requires that
% the box contents does not change at all. Therefore we must
% reserve a separate box register to avoid collisions with
% user box registers.
%
% \emph{Note:} Box register number 255 is special for the output
% routine, because TeX complains if this box is not voided
% by the output routine.
% However, this requirement does not apply to \cs{shipout} at all.
% In fact \cs{shipout} does not change any box register. This is
% usually done by a call of \cs{box}, but the output routine
% can do it later \emph{after} invoking of \cs{shipout}.
%
% \subsection{Summary}
%
% \subsubsection{With \eTeX}
%
% Putting the pieces together we get for \eTeX:
%\begin{quote}
%\begin{verbatim}
%\newbox\mybox
%\let\original@shipout\shipout
%
%\shipout :=
%  \edef\saved@grouplevel{\number\currentgrouplevel}
%  \afterassignment\@test
%  \setbox\mybox=
%
%\@test :=
%  \ifnum\saved@grouplevel<\currentgrouplevel
%    \expandafter\aftergroup
%  \fi
%  \@output
%
%\@output :=
%  \ifvoid\mybox
%    % cancel output of void box
%  \else
%    \@hook
%    \ifvoid\mybox
%      % user code in \@hook could have voided the box
%    \else
%      \original@shipout\box\mybox
%    \fi
%  \fi
%\end{verbatim}
%\end{quote}
%
% \subsubsection{Without \eTeX, traditional way}
%
% And for \TeX\ without \eTeX:
%\begin{quote}
%\begin{verbatim}
%\newbox\mybox
%\begingroup
%   \setbox\mybox=\box\mybox % ensure \mybox is void
%\endgroup
%\let\original@shipout\shipout
%
%\shipout :=
%  % trick to get a void box \mybox
%  \begingroup
%    \setbox\mybox=\box\mybox
%  \endgroup
%  \afterassignment\@test
%  \setbox\mybox=
%
%\@test :=
%  \ifvoid\mybox
%    \expandafter\aftergroup
%  \fi
%  \@output
%
%\@output :=
%  \ifvoid\mybox
%    % cancel output of void box
%  \else
%    \@hook
%    \ifvoid\mybox
%      % user code in \@hook could have voided the box
%    \else
%      \original@shipout\box\mybox
%    \fi
%  \fi
%\end{verbatim}
%\end{quote}
%
% \subsubsection{\cs{lastkern} method}
%
% And for \TeX\ without \eTeX\ using the \cs{lastkern} method:
%\begin{quote}
%\begin{verbatim}
%\newbox\mybox
%\let\original@shipout\shipout
%
%\shipout :=
%  \begingroup
%  \setbox\mybox=\hbox\bgroup
%  \kern1pt
%  \afterassignment\@test
%  \setbox\mybox=
%
%\@test :=
%  \ifdim\lastkern=0pt
%    \expandafter\aftergroup
%  \fi
%  \@output
%
%\@output :=
%  \egroup
%  \endgroup
%  \ifvoid\mybox
%    % cancel output of void box
%  \else
%    \@hook
%    \ifvoid\mybox
%      % user code in \@hook could have voided the box
%    \else
%      \original@shipout\box\mybox
%    \fi
%  \fi
%\end{verbatim}
%\end{quote}
%
%
% \StopEventually{
% }
%
% \section{Implementation}
%
% Package \xpackage{atbegshi} uses \eTeX's \cs{currentgrouplevel},
% if it is available. Otherwise the \cs{lastkern} method is used.
%
%    \begin{macrocode}
%<*package>
%    \end{macrocode}
%
% \subsection{Reload check and package identification}
%    Reload check, especially if the package is not used with \LaTeX.
%    \begin{macrocode}
\begingroup\catcode61\catcode48\catcode32=10\relax%
  \catcode13=5 % ^^M
  \endlinechar=13 %
  \catcode35=6 % #
  \catcode39=12 % '
  \catcode44=12 % ,
  \catcode45=12 % -
  \catcode46=12 % .
  \catcode58=12 % :
  \catcode64=11 % @
  \catcode123=1 % {
  \catcode125=2 % }
  \expandafter\let\expandafter\x\csname ver@atbegshi.sty\endcsname
  \ifx\x\relax % plain-TeX, first loading
  \else
    \def\empty{}%
    \ifx\x\empty % LaTeX, first loading,
      % variable is initialized, but \ProvidesPackage not yet seen
    \else
      \expandafter\ifx\csname PackageInfo\endcsname\relax
        \def\x#1#2{%
          \immediate\write-1{Package #1 Info: #2.}%
        }%
      \else
        \def\x#1#2{\PackageInfo{#1}{#2, stopped}}%
      \fi
      \x{atbegshi}{The package is already loaded}%
      \aftergroup\endinput
    \fi
  \fi
\endgroup%
%    \end{macrocode}
%    Package identification:
%    \begin{macrocode}
\begingroup\catcode61\catcode48\catcode32=10\relax%
  \catcode13=5 % ^^M
  \endlinechar=13 %
  \catcode35=6 % #
  \catcode39=12 % '
  \catcode40=12 % (
  \catcode41=12 % )
  \catcode44=12 % ,
  \catcode45=12 % -
  \catcode46=12 % .
  \catcode47=12 % /
  \catcode58=12 % :
  \catcode64=11 % @
  \catcode91=12 % [
  \catcode93=12 % ]
  \catcode123=1 % {
  \catcode125=2 % }
  \expandafter\ifx\csname ProvidesPackage\endcsname\relax
    \def\x#1#2#3[#4]{\endgroup
      \immediate\write-1{Package: #3 #4}%
      \xdef#1{#4}%
    }%
  \else
    \def\x#1#2[#3]{\endgroup
      #2[{#3}]%
      \ifx#1\@undefined
        \xdef#1{#3}%
      \fi
      \ifx#1\relax
        \xdef#1{#3}%
      \fi
    }%
  \fi
\expandafter\x\csname ver@atbegshi.sty\endcsname
\ProvidesPackage{atbegshi}%
  [2016/06/09 v1.18 At begin shipout hook (HO)]%
%    \end{macrocode}
%
% \subsection{Catcodes}
%
%    \begin{macrocode}
\begingroup\catcode61\catcode48\catcode32=10\relax%
  \catcode13=5 % ^^M
  \endlinechar=13 %
  \catcode123=1 % {
  \catcode125=2 % }
  \catcode64=11 % @
  \def\x{\endgroup
    \expandafter\edef\csname AtBegShi@AtEnd\endcsname{%
      \endlinechar=\the\endlinechar\relax
      \catcode13=\the\catcode13\relax
      \catcode32=\the\catcode32\relax
      \catcode35=\the\catcode35\relax
      \catcode61=\the\catcode61\relax
      \catcode64=\the\catcode64\relax
      \catcode123=\the\catcode123\relax
      \catcode125=\the\catcode125\relax
    }%
  }%
\x\catcode61\catcode48\catcode32=10\relax%
\catcode13=5 % ^^M
\endlinechar=13 %
\catcode35=6 % #
\catcode64=11 % @
\catcode123=1 % {
\catcode125=2 % }
\def\TMP@EnsureCode#1#2{%
  \edef\AtBegShi@AtEnd{%
    \AtBegShi@AtEnd
    \catcode#1=\the\catcode#1\relax
  }%
  \catcode#1=#2\relax
}
\TMP@EnsureCode{40}{12}% (
\TMP@EnsureCode{41}{12}% )
\TMP@EnsureCode{44}{12}% ,
\TMP@EnsureCode{45}{12}% -
\TMP@EnsureCode{47}{12}% /
\TMP@EnsureCode{46}{12}% .
\TMP@EnsureCode{58}{12}% :
\TMP@EnsureCode{91}{12}% [
\TMP@EnsureCode{93}{12}% ]
\TMP@EnsureCode{94}{7}% ^(superscript)
\TMP@EnsureCode{96}{12}% `
\edef\AtBegShi@AtEnd{\AtBegShi@AtEnd\noexpand\endinput}
%    \end{macrocode}
%
% \subsection{Preparations}
%
%    \begin{macrocode}
\begingroup\expandafter\expandafter\expandafter\endgroup
\expandafter\ifx\csname RequirePackage\endcsname\relax
  \def\TMP@RequirePackage#1[#2]{%
    \begingroup\expandafter\expandafter\expandafter\endgroup
    \expandafter\ifx\csname ver@#1.sty\endcsname\relax
      \input #1.sty\relax
    \fi
  }%
  \TMP@RequirePackage{infwarerr}[2007/09/09]%
  \TMP@RequirePackage{ltxcmds}[2010/03/01]%
\else
  \RequirePackage{infwarerr}[2007/09/09]%
  \RequirePackage{ltxcmds}[2010/03/01]%
\fi
%    \end{macrocode}
%
%    \begin{macro}{\AtBegShi@CheckDefinable}
%    \begin{macrocode}
\begingroup\expandafter\expandafter\expandafter\endgroup
\expandafter\ifx\csname @ifdefinable\endcsname\relax
  \def\AtBegShi@CheckDefinable#1{%
    \ifcase\ifx#1\relax
             \ltx@one
           \else
             \ifx#1\@undefined
               \ltx@one
             \else
               \ltx@zero
             \fi
           \fi
      \@PackageError{atbegshi}{%
        \string#1\space is already defined%
      }\@ehd
    \fi
  }%
\else
  \def\AtBegShi@CheckDefinable#1{%
    \@ifdefinable{#1}{}%
  }%
\fi
%    \end{macrocode}
%    \end{macro}
%
%    \begin{macro}{\ifAtBegShi@Discarded}
%    \begin{macrocode}
\ltx@newif\ifAtBegShi@Discarded
%    \end{macrocode}
%    \end{macro}
%    \begin{macro}{\AtBeginShipoutDiscard}
%    \begin{macrocode}
\AtBegShi@CheckDefinable\AtBeginShipoutDiscard
\def\AtBeginShipoutDiscard{%
  \deadcycles=\ltx@zero
  \global\AtBegShi@Discardedtrue
}
%    \end{macrocode}
%    \end{macro}
%
%    \begin{macrocode}
\begingroup\expandafter\expandafter\expandafter\endgroup
\expandafter\ifx\csname currentgrouplevel\endcsname\relax
  \catcode`X=9  % ignore
  \catcode`E=14 % comment
\else
  \catcode`X=14 % comment
  \catcode`E=9  % ignore
\fi
%    \end{macrocode}
%
%    \begin{macro}{\AtBegShi@Shipout}
%    \begin{macrocode}
\def\AtBegShi@Shipout{%
X \begingroup
X \setbox\AtBeginShipoutBox=\hbox\bgroup
X \kern\p@
E \edef\AtBegShi@GroupLevel{\number\currentgrouplevel}%
  \afterassignment\AtBegShi@Test
X \global
  \setbox\AtBeginShipoutBox=%
}
%    \end{macrocode}
%    \end{macro}
%
%    \begin{macro}{\AtBegShi@Test}
%    \begin{macrocode}
\def\AtBegShi@Test{%
X \ifdim\lastkern=0pt %
E \ifnum\AtBegShi@GroupLevel<\currentgrouplevel
    \expandafter\aftergroup
  \fi
  \AtBegShi@Output
}
%    \end{macrocode}
%    \end{macro}
%
%    \begin{macro}{\AtBegShi@Output}
%    \begin{macrocode}
\def\AtBegShi@Output{%
X \egroup
X \endgroup
  \ifvoid\AtBeginShipoutBox
    \@PackageWarning{atbegshi}{Ignoring void shipout box}%
  \else
    \let\AtBegShi@OrgProtect\protect
    \csname set@typeset@protect\endcsname
    \global\AtBegShi@Discardedfalse
    \AtBegShi@Hook
    \expandafter\gdef\expandafter\AtBegShi@HookNext
    \expandafter{\expandafter}%
    \AtBegShi@HookNext
    \ifAtBegShi@Discarded
      \@PackageInfoNoLine{atbegshi}{Shipout page discarded}%
      \global\AtBegShi@Discardedfalse
      \begingroup
        \setbox\AtBeginShipoutBox\box\AtBeginShipoutBox
      \endgroup
      \let\protect\AtBegShi@OrgProtect
    \else
      \AtBegShi@First
      \let\protect\AtBegShi@OrgProtect
      \AtBegShi@GetBoxSize\AtBeginShipoutBox
      \ltx@ifundefined{AtNextShipout}{%
      }{%
        \AtNextShipout{\AtBegShi@GetBoxSize\@cclv}%
      }%
      \AtBeginShipoutOriginalShipout\box\AtBeginShipoutBox
    \fi
  \fi
}
%    \end{macrocode}
%    \end{macro}
%
%    \begin{macro}{\AtBegShi@GetBoxSize}
%    \begin{macrocode}
\def\AtBegShi@GetBoxSize#1{%
  \xdef\AtBeginShipoutBoxWidth{\the\wd#1}%
  \xdef\AtBeginShipoutBoxHeight{\the\ht#1}%
  \xdef\AtBeginShipoutBoxDepth{\the\dp#1}%
}
%    \end{macrocode}
%    \end{macro}
%    \begin{macro}{\AtBeginShipoutBoxWidth}
%    \begin{macrocode}
\def\AtBeginShipoutBoxWidth{0pt}
%    \end{macrocode}
%    \end{macro}
%    \begin{macro}{\AtBeginShipoutBoxHeight}
%    \begin{macrocode}
\def\AtBeginShipoutBoxHeight{0pt}
%    \end{macrocode}
%    \end{macro}
%    \begin{macro}{\AtBeginShipoutBoxDepth}
%    \begin{macrocode}
\def\AtBeginShipoutBoxDepth{0pt}
%    \end{macrocode}
%    \end{macro}
%
%    \begin{macrocode}
\catcode`\X=11 %
\catcode`\E=11 %
%    \end{macrocode}
%
%    \begin{macro}{\AtBegShi@First}
%    \begin{macrocode}
\def\AtBegShi@First{%
  \ifx\AtBegShi@HookFirst\ltx@empty
  \else
    \AtBeginShipoutAddToBox{\AtBegShi@HookFirst}%
  \fi
  \global\let\AtBegShi@First\ltx@empty
  \global\let\AtBeginShipoutFirst\AtBegShi@FirstDisabled
}
%    \end{macrocode}
%    \end{macro}
%
%    \begin{macro}{\AtBegShi@Hook}
%    \begin{macrocode}
\gdef\AtBegShi@Hook{}
%    \end{macrocode}
%    \end{macro}
%    \begin{macro}{\AtBegShi@HookNext}
%    \begin{macrocode}
\gdef\AtBegShi@HookNext{}
%    \end{macrocode}
%    \end{macro}
%    \begin{macro}{\AtBegShi@HookFirst}
%    \begin{macrocode}
\gdef\AtBegShi@HookFirst{}
%    \end{macrocode}
%    \end{macro}
%    \begin{macro}{\AtBeginShipout}
%    \begin{macrocode}
\AtBegShi@CheckDefinable\AtBeginShipout
\def\AtBeginShipout{%
  \AtBegShi@AddHook\AtBegShi@Hook
}
%    \end{macrocode}
%    \end{macro}
%    \begin{macro}{\AtBeginShipoutNext}
%    \begin{macrocode}
\AtBegShi@CheckDefinable\AtBeginShipoutNext
\def\AtBeginShipoutNext{%
  \AtBegShi@AddHook\AtBegShi@HookNext
}
%    \end{macrocode}
%    \end{macro}
%    \begin{macro}{\AtBeginShipoutFirst}
%    \begin{macrocode}
\AtBegShi@CheckDefinable\AtBeginShipoutFirst
\def\AtBeginShipoutFirst{%
  \AtBegShi@AddTo\AtBegShi@HookFirst
}
%    \end{macrocode}
%    \end{macro}
%    \begin{macro}{\AtBegShi@FirstDisabled}
%    \begin{macrocode}
\long\def\AtBegShi@FirstDisabled#1{%
  \@PackageWarning{atbegshi}{%
    First page is already shipped out, ignoring\MessageBreak
    \string\AtBeginShipoutFirst
  }%
}
%    \end{macrocode}
%    \end{macro}
%
%    \begin{macro}{\AtBegShi@AddTo}
%    \begin{macrocode}
\begingroup\expandafter\expandafter\expandafter\endgroup
\expandafter\ifx\csname g@addto@macro\endcsname\relax
  \long\def\AtBegShi@AddTo#1#2{%
    \begingroup
      \toks\ltx@zero\expandafter{#1#2}%
      \xdef#1{\the\toks\ltx@zero}%
    \endgroup
  }%
\else
  \let\AtBegShi@AddTo\g@addto@macro
\fi
%    \end{macrocode}
%    \end{macro}
%
%    \begin{macro}{\AtBegShi@AddHook}
%    \begin{macrocode}
\long\def\AtBegShi@AddHook#1#2{%
  \AtBegShi@AddTo#1{\AtBegShi@Item{#2}}%
}
%    \end{macrocode}
%    \end{macro}
%
%    \begin{macro}{\AtBegShi@Item}
%    \begin{macrocode}
\long\def\AtBegShi@Item#1{%
  \ifAtBegShi@Discarded
  \else
    #1%
    \ifAtBegShi@Discarded
    \else
      \ifvoid\AtBeginShipoutBox
        \@PackageWarning{atbegshi}{%
          Shipout box was voided by hook,\MessageBreak
          ignoring shipout box%
        }%
        \AtBeginShipoutDiscard
      \fi
    \fi
  \fi
}
%    \end{macrocode}
%    \end{macro}
%
%    \begin{macro}{\AtBeginShipoutInit}
%    \begin{macrocode}
\AtBegShi@CheckDefinable\AtBeginShipoutInit
\def\AtBeginShipoutInit{%
  \ltx@IfUndefined{newbox}{%
    \@PackageError{atbegshi}{%
      \string\AtBeginShipoutInit\space failed\MessageBreak
       because of missing \expandafter\string\csname newbox\endcsname
    }\@ehc
  }{%
    \csname newbox\endcsname\AtBeginShipoutBox
    \AtBegShi@CheckDefinable\AtBeginShipoutOriginalShipout
    \global\let\AtBeginShipoutOriginalShipout\shipout
    \global\let\shipout\AtBegShi@Shipout
  }%
  \gdef\AtBeginShipoutInit{}%
}
%    \end{macrocode}
%    \end{macro}
%    \begin{macrocode}
\begingroup\expandafter\expandafter\expandafter\endgroup
\expandafter\ifx\csname AtBeginDocument\endcsname\relax
  \AtBeginShipoutInit
\else
  \AtBeginDocument{\AtBeginShipoutInit}%
\fi
%    \end{macrocode}
%
% \subsection{Additions to the shipout box}
%
%    \begin{macro}{\AtBeginShipoutAddToBox}
%    \begin{macrocode}
\def\AtBeginShipoutAddToBox#1{%
  \ifhbox\AtBeginShipoutBox
    \edef\AtBegShi@restore{%
      \hfuzz=\the\hfuzz\relax
      \hbadness=\the\hbadness\relax
    }%
    \hfuzz=1073741823sp\relax
    \hbadness=2147483647\relax
    \setbox\AtBeginShipoutBox=\hbox to \wd\AtBeginShipoutBox{%
      \setbox\ltx@zero=\hbox{%
        \begingroup
          \AtBegShi@restore
          #1%
        \endgroup
      }%
      \wd\ltx@zero=0pt\relax
      \ht\ltx@zero=0pt\relax
      \dp\ltx@zero=0pt\relax
      \raise\ht\AtBeginShipoutBox\copy\ltx@zero
      \unhcopy\AtBeginShipoutBox
    }%
    \AtBegShi@restore
  \else
    \ifvbox\AtBeginShipoutBox
      \edef\AtBegShi@restore{%
        \vfuzz=\the\vfuzz\relax
        \vbadness=\the\vbadness\relax
        \dimen\ltx@zero=\the\dimen\ltx@zero\relax
      }%
      \edef\AtBegShi@restorebox{%
        \ht\AtBeginShipoutBox=\the\ht\AtBeginShipoutBox\relax
        \dp\AtBeginShipoutBox=\the\dp\AtBeginShipoutBox\relax
      }%
      \vfuzz=1073741823sp\relax
      \vbadness=2147483647\relax
      \dimen\ltx@zero=\ht\AtBeginShipoutBox
      \advance\dimen\ltx@zero by \dp\AtBeginShipoutBox
      \setbox\AtBeginShipoutBox=\vbox to \dimen\ltx@zero{%
        \setbox\ltx@zero=\hbox{%
          \begingroup
            \AtBegShi@restore
            #1%
          \endgroup
        }%
        \wd\ltx@zero=0pt\relax
        \ht\ltx@zero=0pt\relax
        \dp\ltx@zero=0pt\relax
        \baselineskip=0pt\relax
        \lineskip=0pt\relax
        \lineskiplimit=0pt\relax
        \copy\ltx@zero
        \unvbox\AtBeginShipoutBox
        \kern0pt%
      }%
      \AtBegShi@restore
      \AtBegShi@restorebox
    \fi
  \fi
}
%    \end{macrocode}
%    \end{macro}
%
%    \begin{macro}{\AtBeginShipoutAddToBoxForeground}
%    \begin{macrocode}
\def\AtBeginShipoutAddToBoxForeground#1{%
  \ifhbox\AtBeginShipoutBox
    \edef\AtBegShi@restore{%
      \hfuzz=\the\hfuzz\relax
      \hbadness=\the\hbadness\relax
    }%
    \hfuzz=1073741823sp\relax
    \hbadness=2147483647\relax
    \setbox\AtBeginShipoutBox=\hbox to \wd\AtBeginShipoutBox{%
      \unhcopy\AtBeginShipoutBox
      \kern-\wd\AtBeginShipoutBox
      \setbox\ltx@zero=\hbox{%
        \begingroup
          \AtBegShi@restore
          #1%
        \endgroup
      }%
      \wd\ltx@zero=0pt\relax
      \ht\ltx@zero=0pt\relax
      \dp\ltx@zero=0pt\relax
      \raise\ht\AtBeginShipoutBox\copy\ltx@zero
      \kern\wd\AtBeginShipoutBox
    }%
    \AtBegShi@restore
  \else
    \ifvbox\AtBeginShipoutBox
      \edef\AtBegShi@restore{%
        \vfuzz=\the\vfuzz\relax
        \vbadness=\the\vbadness\relax
        \dimen\ltx@zero=\the\dimen\ltx@zero\relax
      }%
      \edef\AtBegShi@restorebox{%
        \ht\AtBeginShipoutBox=\the\ht\AtBeginShipoutBox\relax
        \dp\AtBeginShipoutBox=\the\dp\AtBeginShipoutBox\relax
      }%
      \vfuzz=1073741823sp\relax
      \vbadness=2147483647\relax
      \dimen\ltx@zero=\ht\AtBeginShipoutBox
      \advance\dimen\ltx@zero by \dp\AtBeginShipoutBox
      \setbox\AtBeginShipoutBox=\vbox to \dimen\ltx@zero{%
        \setbox\ltx@zero=\hbox{%
          \begingroup
            \AtBegShi@restore
            #1%
          \endgroup
        }%
        \wd\ltx@zero=0pt\relax
        \ht\ltx@zero=0pt\relax
        \dp\ltx@zero=0pt\relax
        \baselineskip=0pt\relax
        \lineskip=0pt\relax
        \lineskiplimit=0pt\relax
        \unvbox\AtBeginShipoutBox
        \kern-\dimen\ltx@zero
        \copy\ltx@zero
        \kern\dimen\ltx@zero
      }%
      \AtBegShi@restore
      \AtBegShi@restorebox
    \fi
  \fi
}
%    \end{macrocode}
%    \end{macro}
%
% \subsection{Positioning}
%
%    \begin{macrocode}
\begingroup\expandafter\expandafter\expandafter\endgroup
\expandafter\ifx\csname RequirePackage\endcsname\relax
  \def\TMP@RequirePackage#1[#2]{%
    \begingroup\expandafter\expandafter\expandafter\endgroup
    \expandafter\ifx\csname ver@#1.sty\endcsname\relax
      \input #1.sty\relax
    \fi
  }%
  \TMP@RequirePackage{ifpdf}[2011/01/30]%
\else
  \RequirePackage{ifpdf}[2011/01/30]%
\fi
%    \end{macrocode}
%
%    \begin{macrocode}
\ifpdf
  \def\AtBegShi@horigin{%
    \ifx\pdfhorigin\@undefined\pdfvariable horigin\else\pdfhorigin\fi}%
  \def\AtBegShi@vorigin{%
    \ifx\pdfvorigin\@undefined\pdfvariable vorigin\else\pdfvorigin\fi}%
\else
  \def\AtBegShi@horigin{72.27pt}%
  \def\AtBegShi@vorigin{72.27pt}%
\fi
%    \end{macrocode}
%
%    \begin{macrocode}
\begingroup
\ifcase
    \expandafter\ifx\csname picture\endcsname\relax
      1%
    \else
      \expandafter\ifx\csname endpicture\endcsname\relax
        1%
      \else
        0%
      \fi
    \fi
  \endgroup
  \def\AtBegShi@BeginPicture{%
    \begingroup
    \picture(0,0)\relax
    \begingroup\expandafter\expandafter\expandafter\endgroup
    \expandafter\ifx\csname unitlength\endcsname\relax
    \else
      \unitlength=1pt\relax
    \fi
    \ignorespaces
  }%
  \def\AtBegShi@EndPicture{%
    \endpicture
    \endgroup
  }%
\else
  \endgroup
  \def\AtBegShi@BeginPicture{%
    \setbox\ltx@zero=\hbox\bgroup
    \begingroup
    \ignorespaces
  }%
  \def\AtBegShi@EndPicture{%
    \endgroup
    \egroup
    \ht\ltx@zero=0pt\relax
    \dp\ltx@zero=0pt\relax
    \copy\ltx@zero
  }%
\fi
%    \end{macrocode}
%
%    \begin{macro}{\AtBeginShipoutUpperLeft}
%    A surrounding \cs{rlap} is not necessary, because
%    the stuff is put in an \cs{hbox} with zero width.
%    \begin{macrocode}
\def\AtBeginShipoutUpperLeft#1{%
  \AtBeginShipoutAddToBox{%
    \kern-\AtBegShi@horigin\relax
    \vbox to 0pt{%
      \kern-\AtBegShi@vorigin\relax
      \AtBegShi@BeginPicture
      #1%
      \AtBegShi@EndPicture
      \vss
    }%
  }%
}
%    \end{macrocode}
%    \end{macro}
%
%    \begin{macro}{\AtBeginShipoutUpperLeftForeground}
%    \begin{macrocode}
\def\AtBeginShipoutUpperLeftForeground#1{%
  \AtBeginShipoutAddToBoxForeground{%
    \kern-\AtBegShi@horigin\relax
    \vbox to 0pt{%
      \kern-\AtBegShi@vorigin\relax
      \AtBegShi@BeginPicture
      #1%
      \AtBegShi@EndPicture
      \vss
    }%
  }%
}
%    \end{macrocode}
%    \end{macro}
%
% \subsection{Patches}
%
%    Patches for \LaTeX\ packages that redefine \cs{shipout}.
%    \LaTeX\ is now supposed to use \eTeX. Thus we do not patch,
%    without \LaTeX\ and \eTeX.
%    \begin{macrocode}
\def\AtBegShi@AbortIfUndefined#1{%
  \begingroup\expandafter\expandafter\expandafter\endgroup
  \expandafter\ifx\csname#1\endcsname\relax
    \expandafter\AtBegShi@AtEnd
  \fi
}
\AtBegShi@AbortIfUndefined{currentgrouplevel}%
\AtBegShi@AbortIfUndefined{AtBeginDocument}%
\AtBegShi@AbortIfUndefined{@ifpackageloaded}%
\AtBegShi@AbortIfUndefined{@ifclassloaded}%
%    \end{macrocode}
%
% \subsubsection{Package \xpackage{crop}}
%
% Fix of method and box.
%    \begin{macrocode}
\def\AtBegShi@PatchCrop{%
  \begingroup
    \def\AtBegShi@Crop@shipout{%
      \afterassignment\CROP@ship
      \setbox\@cclv=%
    }%
    \def\AtBegShi@Crop@ship{%
      \ifvoid\@cclv
        \expandafter\aftergroup
      \fi
      \CROP@@ship
    }%
    \def\AtBegShi@Crop@shiplist{%
      \lineskip\z@
      \lineskiplimit\z@
      \baselineskip\z@
      \CROP@kernel
      \box\@cclv
    }%
    \def\AtBegShi@Crop@@ship{%
      \CROP@shipout\vbox{%
        \CROP@shiplist
      }%
    }%
    \ifx\AtBegShi@Crop@ship\CROP@ship
      \ifx\AtBegShi@Crop@shiplist\CROP@shiplist
        \ifx\AtBegShi@Crop@@ship\CROP@@ship
          \let\AtBegShi@found\relax
          \ifx\shipout\AtBegShi@Crop@shipout
            \def\AtBegShi@found{\shipout}%
          \else\ifx\AtBeginShipoutOriginalShipout\AtBegShi@Crop@shipout
            \def\AtBegShi@found{\AtBeginShipoutOriginalShipout}%
          \else\ifx\@EveryShipout@Org@Shipout\AtBegShi@Crop@shipout
            \def\AtBegShi@found{\@EveryShipout@Org@Shipout}%
          \else\ifx\GPTorg@shipout\AtBegShi@Crop@shipout
            \def\AtBegShi@found{\GPTorg@shipout}%
          \else\ifx\THBorg@shipout\AtBegShi@Crop@shipout
            \def\AtBegShi@found{\THBorg@shipout}%
          \else\ifx\mem@oldshipout\AtBegShi@Crop@shipout
            \def\AtBegShi@found{\mem@oldshipout}%
          \fi\fi\fi\fi\fi\fi
          \ifx\AtBegShi@found\relax
          \else
            \expandafter\endgroup
            \expandafter\def\AtBegShi@found{%
              \edef\AtBegShi@GroupLevel{\number\currentgrouplevel}%
              \afterassignment\CROP@ship
              \setbox\AtBeginShipoutBox=%
            }%
            \def\CROP@ship{%
              \ifnum\AtBegShi@GroupLevel=\currentgrouplevel
              \else
                \expandafter\aftergroup
              \fi
              \CROP@@ship
            }%
            \def\CROP@shiplist{%
              \lineskip 0pt\relax
              \lineskiplimit 0pt\relax
              \baselineskip 0pt\relax
              \CROP@kernel
              \box\AtBeginShipoutBox
            }%
            \def\CROP@@ship{%
              \ifvoid\AtBeginShipoutBox
              \else
                \setbox\AtBeginShipoutBox=\vbox{%
                  \CROP@shiplist
                }%
                \AtBegShi@GetBoxSize\AtBeginShipoutBox
                \expandafter\CROP@shipout
                \expandafter\box
                \expandafter\AtBeginShipoutBox
              \fi
            }%
            \@PackageInfoNoLine{atbegshi}{Package `crop' patched}%
            \begingroup
          \fi
        \fi
      \fi
    \fi
  \endgroup
  \let\AtBegShi@PatchCrop\relax
}
\@ifpackageloaded{crop}{%
  \AtBegShi@PatchCrop
}{%
  \AtBeginDocument{\AtBegShi@PatchCrop}%
}
%    \end{macrocode}
%
% \subsubsection{Package \xpackage{everyshi}}
%
%     Fix of method. Use of box 255 is not changed.
%    \begin{macrocode}
\def\AtBegShi@PatchEveryshi{%
  \begingroup
    \long\def\AtBegShi@Everyshi@shipout{%
      \afterassignment\@EveryShipout@Test
      \global\setbox\@cclv= %
    }%
    \long\def\AtBegShi@Everyshi@Test{%
      \ifvoid\@cclv\relax
        \aftergroup\@EveryShipout@Output
      \else
        \@EveryShipout@Output
      \fi
    }%
    \ifx\AtBegShi@Everyshi@Test\@EveryShipout@Test
      \let\AtBegShi@found\relax
      \ifx\shipout\AtBegShi@Everyshi@shipout
        \def\AtBegShi@found{\shipout}%
      \else\ifx\AtBeginShipoutOriginalShipout\AtBegShi@Everyshi@shipout
        \def\AtBegShi@found{\AtBeginShipoutOriginalShipout}%
      \else\ifx\CROP@shipout\AtBegShi@Everyshi@shipout
        \def\AtBegShi@found{\CROP@shipout}%
      \else\ifx\GPTorg@shipout\AtBegShi@Everyshi@shipout
        \def\AtBegShi@found{\GPTorg@shipout}%
      \else\ifx\THBorg@shipout\AtBegShi@Everyshi@shipout
        \def\AtBegShi@found{\THBorg@shipout}%
      \else\ifx\mem@oldshipout\AtBegShi@Everyshi@shipout
        \def\AtBegShi@found{\mem@oldshipout}%
      \else
        \expandafter\ifx\csname @EveryShipout@Org@Shipout\endcsname
                        \relax
           \ifx\@EveryShipout@Shipout\AtBegShi@Everyshi@shipout
             \def\AtBegShi@found{\@EveryShipout@Shipout}%
           \fi
        \fi
      \fi\fi\fi\fi\fi\fi
      \ifx\AtBegShi@found\relax
      \else
        \expandafter\endgroup
        \expandafter\def\AtBegShi@found{%
          \edef\AtBegShi@GroupLevel{\number\currentgrouplevel}%
          \afterassignment\@EveryShipout@Test
          \setbox\AtBeginShipoutBox=%
        }%
        \def\@EveryShipout@Test{%
          \ifnum\AtBegShi@GroupLevel=\currentgrouplevel
          \else
            \expandafter\aftergroup
          \fi
          \AtBegShi@Everyshi@Output
        }%
        \def\AtBegShi@Everyshi@Output{%
          \ifvoid\AtBeginShipoutBox
          \else
            \global\setbox\ltx@cclv\box\AtBeginShipoutBox
            \expandafter\@EveryShipout@Output
          \fi
        }%
        \@PackageInfoNoLine{atbegshi}{Package `everyshi' patched}%
        \begingroup
      \fi
    \fi
  \endgroup
  \let\AtBegShi@PatchEveryshi\relax
}
\@ifpackageloaded{everyshi}{%
  \AtBegShi@PatchEveryshi
}{%
  \AtBeginDocument{\AtBegShi@PatchEveryshi}%
}
%    \end{macrocode}
%
% \subsubsection{Class \xpackage{memoir}}
%
%     Fix of method and box.
%    \begin{macrocode}
\def\AtBegShi@PatchMemoir{%
  \begingroup
    \def\AtBegShi@Memoir@shipout{%
      \afterassignment\mem@shipi
      \setbox\@cclv=%
    }%
    \def\AtBegShi@Memoir@shipi{%
      \ifvoid\@cclv
        \expandafter\aftergroup
      \fi
      \mem@shipii
    }%
    \def\AtBegShi@Memoir@shipiiA{%
      \mem@oldshipout\vbox{%
        \trimmarks
        \unvbox\@cclv
      }%
    }%
    \def\AtBegShi@Memoir@shipiiB{%
      \ifvoid\@cclv
        \mem@oldshipout\box\@cclv
      \else
        \mem@oldshipout\vbox{%
          \trimmarks
          \unvbox\@cclv
        }%
      \fi
    }%
    \def\AtBegShi@Memoir@PatchAB{%
      \ifvoid\AtBeginShipoutBox
      \else
        \setbox\AtBeginShipoutBox=\vbox{%
          \trimmarks
          \ifvbox\AtBeginShipoutBox
            \unvbox\AtBeginShipoutBox
          \else
            \box\AtBeginShipoutBox
          \fi
        }%
        \AtBegShi@GetBoxSize\AtBeginShipoutBox
        \expandafter\mem@oldshipout
        \expandafter\box
        \expandafter\AtBeginShipoutBox
      \fi
    }%
    \def\AtBegShi@Memoir@shipiiC{% 2008/08/07 v1.6180339a
      \ifvoid\@cclv
        \mem@oldshipout\box\@cclv
      \else
        \ifshowtrims
          \mem@oldshipout\vbox{\trimmarks\unvbox\@cclv}%
        \else
          \mem@oldshipout\box\@cclv
        \fi
      \fi
    }%
    \def\AtBegShi@Memoir@shipiiD{% 2011/03/06 v3.6j
      \ifvoid\@cclv
        \mem@oldshipout\box\@cclv
      \else
        \ifshowtrims
          \mem@oldshipout\vbox{%
            \trimmarks
            \nointerlineskip
            \box\@cclv
          }%
        \else
          \mem@oldshipout\box\@cclv
        \fi
      \fi
    }%
    \def\AtBegShi@Memoir@PatchCD{%
      \ifvoid\AtBeginShipoutBox
      \else
        \ifshowtrims
          \setbox\AtBeginShipoutBox=\vbox{%
            \trimmarks
            \nointerlineskip
            \box\AtBeginShipoutBox
          }%
        \fi
        \AtBegShi@GetBoxSize\AtBeginShipoutBox
        \expandafter\mem@oldshipout
        \expandafter\box
        \expandafter\AtBeginShipoutBox
      \fi
    }%
    \ifx\AtBegShi@Memoir@shipi\mem@shipi
      \let\AtBegShi@found\ltx@one
      \ifx\AtBegShi@Memoir@shipiiA\mem@shipii
        \let\AtBegShi@found\ltx@zero
        \global\let\AtBegShi@Memoir@PatchX\AtBegShi@Memoir@PatchAB
      \else\ifx\AtBegShi@Memoir@shipiiB\mem@shipii
        \let\AtBegShi@found\ltx@zero
        \global\let\AtBegShi@Memoir@PatchX\AtBegShi@Memoir@PatchAB
      \else\ifx\AtBegShi@Memoir@shipiiC\mem@shipii
        \let\AtBegShi@found\ltx@zero
        \global\let\AtBegShi@Memoir@PatchX\AtBegShi@Memoir@PatchCD
      \else\ifx\AtBegShi@Memoir@shipiiD\mem@shipii
        \let\AtBegShi@found\ltx@zero
        \global\let\AtBegShi@Memoir@PatchX\AtBegShi@Memoir@PatchCD
      \fi\fi\fi\fi
      \ifcase\AtBegShi@found
        \let\AtBegShi@found\relax
        \ifx\shipout\AtBegShi@Memoir@shipout
          \def\AtBegShi@found{\shipout}%
        \else\ifx\AtBeginShipoutOriginalShipout\AtBegShi@Memoir@shipout
          \def\AtBegShi@found{\AtBeginShipoutOriginalShipout}%
        \else\ifx\CROP@shipout\AtBegShi@Memoir@shipout
          \def\AtBegShi@found{\CROP@shipout}%
        \else\ifx\GPTorg@shipout\AtBegShi@Memoir@shipout
          \def\AtBegShi@found{\GPTorg@shipout}%
        \else\ifx\THBorg@shipout\AtBegShi@Memoir@shipout
          \def\AtBegShi@found{\THBorg@shipout}%
        \else\ifx\@EveryShipout@Org@Shipout\AtBegShi@Memoir@shipout
          \def\AtBegShi@found{\@EveryShipout@Org@Shipout}%
        \fi\fi\fi\fi\fi\fi
        \ifx\AtBegShi@found\relax
        \else
          \expandafter\endgroup
          \expandafter\def\AtBegShi@found{%
            \edef\AtBegShi@GroupLevel{\number\currentgrouplevel}%
            \afterassignment\mem@shipi
            \setbox\AtBeginShipoutBox=%
          }%
          \def\mem@shipi{%
            \ifnum\AtBegShi@GroupLevel=\currentgrouplevel
            \else
              \expandafter\aftergroup
            \fi
            \mem@shipii
          }%
          \let\mem@shipii\AtBegShi@Memoir@PatchX
          \@PackageInfoNoLine{atbegshi}{Class `memoir' patched}%
          \begingroup
        \fi
      \fi
    \fi
  \endgroup
  \let\AtBegShi@PatchMemoir\relax
}
\@ifclassloaded{memoir}{%
  \AtBegShi@PatchMemoir
}{%
  \AtBeginDocument{\AtBegShi@PatchMemoir}%
}
%    \end{macrocode}
%
%    \begin{macrocode}
\AtBegShi@AtEnd%
%</package>
%    \end{macrocode}
%
% \section{Test}
%
% \subsection{Catcode checks for loading}
%
%    \begin{macrocode}
%<*test1>
%    \end{macrocode}
%    \begin{macrocode}
\catcode`\{=1 %
\catcode`\}=2 %
\catcode`\#=6 %
\catcode`\@=11 %
\expandafter\ifx\csname count@\endcsname\relax
  \countdef\count@=255 %
\fi
\expandafter\ifx\csname @gobble\endcsname\relax
  \long\def\@gobble#1{}%
\fi
\expandafter\ifx\csname @firstofone\endcsname\relax
  \long\def\@firstofone#1{#1}%
\fi
\expandafter\ifx\csname loop\endcsname\relax
  \expandafter\@firstofone
\else
  \expandafter\@gobble
\fi
{%
  \def\loop#1\repeat{%
    \def\body{#1}%
    \iterate
  }%
  \def\iterate{%
    \body
      \let\next\iterate
    \else
      \let\next\relax
    \fi
    \next
  }%
  \let\repeat=\fi
}%
\def\RestoreCatcodes{}
\count@=0 %
\loop
  \edef\RestoreCatcodes{%
    \RestoreCatcodes
    \catcode\the\count@=\the\catcode\count@\relax
  }%
\ifnum\count@<255 %
  \advance\count@ 1 %
\repeat

\def\RangeCatcodeInvalid#1#2{%
  \count@=#1\relax
  \loop
    \catcode\count@=15 %
  \ifnum\count@<#2\relax
    \advance\count@ 1 %
  \repeat
}
\def\RangeCatcodeCheck#1#2#3{%
  \count@=#1\relax
  \loop
    \ifnum#3=\catcode\count@
    \else
      \errmessage{%
        Character \the\count@\space
        with wrong catcode \the\catcode\count@\space
        instead of \number#3%
      }%
    \fi
  \ifnum\count@<#2\relax
    \advance\count@ 1 %
  \repeat
}
\def\space{ }
\expandafter\ifx\csname LoadCommand\endcsname\relax
  \def\LoadCommand{\input atbegshi.sty\relax}%
\fi
\def\Test{%
  \RangeCatcodeInvalid{0}{47}%
  \RangeCatcodeInvalid{58}{64}%
  \RangeCatcodeInvalid{91}{96}%
  \RangeCatcodeInvalid{123}{255}%
  \catcode`\@=12 %
  \catcode`\\=0 %
  \catcode`\%=14 %
  \LoadCommand
  \RangeCatcodeCheck{0}{36}{15}%
  \RangeCatcodeCheck{37}{37}{14}%
  \RangeCatcodeCheck{38}{47}{15}%
  \RangeCatcodeCheck{48}{57}{12}%
  \RangeCatcodeCheck{58}{63}{15}%
  \RangeCatcodeCheck{64}{64}{12}%
  \RangeCatcodeCheck{65}{90}{11}%
  \RangeCatcodeCheck{91}{91}{15}%
  \RangeCatcodeCheck{92}{92}{0}%
  \RangeCatcodeCheck{93}{96}{15}%
  \RangeCatcodeCheck{97}{122}{11}%
  \RangeCatcodeCheck{123}{255}{15}%
  \RestoreCatcodes
}
\Test
\csname @@end\endcsname
\end
%    \end{macrocode}
%    \begin{macrocode}
%</test1>
%    \end{macrocode}
%
%    \begin{macrocode}
%<*test2>
\input atbegshi.sty\relax
\def\msg#{\immediate\write16}
\msg{File: atbegshi-test2.tex 2016/06/09 v1.18 Test file for plain-TeX}
\def\testmsg#1#2{%
  \msg{}%
  \msg{*** Test with box (#1), expected page output [#2]}% hash-ok
}

\newbox\voidbox
\def\void{\box\voidbox}
\begingroup
  \setbox\voidbox=\void
\endgroup

\count0=0\relax
\AtBeginShipout{%
  \global\advance\count0 by 1\relax
  \msg{* Inside \string\AtBeginShipout: [\the\count0]}%
}

\AtBeginShipoutFirst{%
  \msg{* Inside \string\AtBeginShipoutFirst}%
  Hello World%
}

\testmsg{\string\null}{1}
\shipout\null

\AtBeginShipoutFirst{%
  This is too late%
}

\testmsg{void}{}
\shipout\void

\testmsg{\string\copy255 (not void)}{2}
\setbox255\hbox{\vrule height 10bp width 10bp}
\shipout\copy255 %

\testmsg{\string\copy255 (again)}{3}
\shipout\copy255 %

\testmsg{\string\box255}{4}
\shipout\box255 %

\testmsg{\string\box255 (again)}{}
\shipout\box255 %

\testmsg{\string\hbox}{5}
\shipout\hbox{\vrule height 5bp width 20bp}

\testmsg{\string\vbox}{6}
\shipout\vbox{\hrule height 20bp width 5bp}

\testmsg{\string\null, voided by hook}{}
\def\VoidBox{%
  \begingroup
    \setbox\AtBeginShipoutBox=\box\AtBeginShipoutBox
  \endgroup
}
\AtBeginShipout{\VoidBox}
\shipout\null
\def\VoidBox{}

\msg{*** \string\begingroup}
\begingroup
  \testmsg{void}{}%
  \shipout\void
\msg{*** \string\endgroup}
\endgroup

\msg{*** \string\begingroup}
\begingroup
  \testmsg{void}{}%
  \shipout\void
  \testmsg{\string\null}{8}%
  \shipout\null
\msg{*** \string\endgroup}
\endgroup

\testmsg{output routine}{9}
Hello World
\vfill
\eject

\testmsg{\string\null\space(discarded)}{}
\AtBeginShipout{%
  \msg{* Inside \string\AtBeginShipout: DISCARD}%
  \AtBeginShipoutDiscard
}
\shipout\null

\end
%</test2>
%    \end{macrocode}
%
%    \begin{macrocode}
%<*test3>
\NeedsTeXFormat{LaTeX2e}
\ProvidesFile{atbegshi-test3.tex}[2016/06/09 v1.18 Test file for LaTeX]
\RequirePackage{color}
\pagecolor{yellow}
\documentclass[a5paper,showtrims]{memoir}
\usepackage{atbegshi}
\AtBeginShipout{%
  \setbox\AtBeginShipoutBox=\vbox{%
    \vbox to 0pt{%
      \kern-1.5in %
      \hbox to 0pt{%
        \kern-1.5in %
        \color{blue}%
        \rule{1in}{1in}%
        \hss
      }%
      \vss
    }%
    \hrule
    \hbox{\vrule\box\AtBeginShipoutBox\vrule}%
    \hrule
  }%
}
\usepackage{eso-pic}
\makeatletter
\@ifundefined{@EveryShipout@Init}{%
  \typeout{Test skipped}%
  \@@end
}{}
\@EveryShipout@Init
\let\@EveryShipout@Init\relax
\makeatother
\AddToShipoutPicture{%
  \hspace{.52\paperwidth}%
  \colorbox{cyan}{%
    \rule{0mm}{\paperheight}%
    \hspace{.48\paperwidth}%
  }%
}
%    \end{macrocode}
%    Newer versions of class \xclass{memoir} emulate package
%    \xpackage{crop} and prevents its loading. This is undone
%    in next line for this test file.
%    \begin{macrocode}
\expandafter\let\csname ver@crop.sty\endcsname\relax
\usepackage[color=red,cross,a4,center]{crop}
\begin{document}
\shipout\null
\shipout\box\csname voidb@x\endcsname
\section{Hello World}
\end{document}
%</test3>
%    \end{macrocode}
%
% \section{Installation}
%
% \subsection{Download}
%
% \paragraph{Package.} This package is available on
% CTAN\footnote{\url{http://ctan.org/pkg/atbegshi}}:
% \begin{description}
% \item[\CTAN{macros/latex/contrib/oberdiek/atbegshi.dtx}] The source file.
% \item[\CTAN{macros/latex/contrib/oberdiek/atbegshi.pdf}] Documentation.
% \end{description}
%
%
% \paragraph{Bundle.} All the packages of the bundle `oberdiek'
% are also available in a TDS compliant ZIP archive. There
% the packages are already unpacked and the documentation files
% are generated. The files and directories obey the TDS standard.
% \begin{description}
% \item[\CTAN{install/macros/latex/contrib/oberdiek.tds.zip}]
% \end{description}
% \emph{TDS} refers to the standard ``A Directory Structure
% for \TeX\ Files'' (\CTAN{tds/tds.pdf}). Directories
% with \xfile{texmf} in their name are usually organized this way.
%
% \subsection{Bundle installation}
%
% \paragraph{Unpacking.} Unpack the \xfile{oberdiek.tds.zip} in the
% TDS tree (also known as \xfile{texmf} tree) of your choice.
% Example (linux):
% \begin{quote}
%   |unzip oberdiek.tds.zip -d ~/texmf|
% \end{quote}
%
% \paragraph{Script installation.}
% Check the directory \xfile{TDS:scripts/oberdiek/} for
% scripts that need further installation steps.
% Package \xpackage{attachfile2} comes with the Perl script
% \xfile{pdfatfi.pl} that should be installed in such a way
% that it can be called as \texttt{pdfatfi}.
% Example (linux):
% \begin{quote}
%   |chmod +x scripts/oberdiek/pdfatfi.pl|\\
%   |cp scripts/oberdiek/pdfatfi.pl /usr/local/bin/|
% \end{quote}
%
% \subsection{Package installation}
%
% \paragraph{Unpacking.} The \xfile{.dtx} file is a self-extracting
% \docstrip\ archive. The files are extracted by running the
% \xfile{.dtx} through \plainTeX:
% \begin{quote}
%   \verb|tex atbegshi.dtx|
% \end{quote}
%
% \paragraph{TDS.} Now the different files must be moved into
% the different directories in your installation TDS tree
% (also known as \xfile{texmf} tree):
% \begin{quote}
% \def\t{^^A
% \begin{tabular}{@{}>{\ttfamily}l@{ $\rightarrow$ }>{\ttfamily}l@{}}
%   atbegshi.sty & tex/generic/oberdiek/atbegshi.sty\\
%   atbegshi.pdf & doc/latex/oberdiek/atbegshi.pdf\\
%   atbegshi-example1.tex & doc/latex/oberdiek/atbegshi-example1.tex\\
%   atbegshi-example2.tex & doc/latex/oberdiek/atbegshi-example2.tex\\
%   test/atbegshi-test1.tex & doc/latex/oberdiek/test/atbegshi-test1.tex\\
%   test/atbegshi-test2.tex & doc/latex/oberdiek/test/atbegshi-test2.tex\\
%   test/atbegshi-test3.tex & doc/latex/oberdiek/test/atbegshi-test3.tex\\
%   atbegshi.dtx & source/latex/oberdiek/atbegshi.dtx\\
% \end{tabular}^^A
% }^^A
% \sbox0{\t}^^A
% \ifdim\wd0>\linewidth
%   \begingroup
%     \advance\linewidth by\leftmargin
%     \advance\linewidth by\rightmargin
%   \edef\x{\endgroup
%     \def\noexpand\lw{\the\linewidth}^^A
%   }\x
%   \def\lwbox{^^A
%     \leavevmode
%     \hbox to \linewidth{^^A
%       \kern-\leftmargin\relax
%       \hss
%       \usebox0
%       \hss
%       \kern-\rightmargin\relax
%     }^^A
%   }^^A
%   \ifdim\wd0>\lw
%     \sbox0{\small\t}^^A
%     \ifdim\wd0>\linewidth
%       \ifdim\wd0>\lw
%         \sbox0{\footnotesize\t}^^A
%         \ifdim\wd0>\linewidth
%           \ifdim\wd0>\lw
%             \sbox0{\scriptsize\t}^^A
%             \ifdim\wd0>\linewidth
%               \ifdim\wd0>\lw
%                 \sbox0{\tiny\t}^^A
%                 \ifdim\wd0>\linewidth
%                   \lwbox
%                 \else
%                   \usebox0
%                 \fi
%               \else
%                 \lwbox
%               \fi
%             \else
%               \usebox0
%             \fi
%           \else
%             \lwbox
%           \fi
%         \else
%           \usebox0
%         \fi
%       \else
%         \lwbox
%       \fi
%     \else
%       \usebox0
%     \fi
%   \else
%     \lwbox
%   \fi
% \else
%   \usebox0
% \fi
% \end{quote}
% If you have a \xfile{docstrip.cfg} that configures and enables \docstrip's
% TDS installing feature, then some files can already be in the right
% place, see the documentation of \docstrip.
%
% \subsection{Refresh file name databases}
%
% If your \TeX~distribution
% (\teTeX, \mikTeX, \dots) relies on file name databases, you must refresh
% these. For example, \teTeX\ users run \verb|texhash| or
% \verb|mktexlsr|.
%
% \subsection{Some details for the interested}
%
% \paragraph{Attached source.}
%
% The PDF documentation on CTAN also includes the
% \xfile{.dtx} source file. It can be extracted by
% AcrobatReader 6 or higher. Another option is \textsf{pdftk},
% e.g. unpack the file into the current directory:
% \begin{quote}
%   \verb|pdftk atbegshi.pdf unpack_files output .|
% \end{quote}
%
% \paragraph{Unpacking with \LaTeX.}
% The \xfile{.dtx} chooses its action depending on the format:
% \begin{description}
% \item[\plainTeX:] Run \docstrip\ and extract the files.
% \item[\LaTeX:] Generate the documentation.
% \end{description}
% If you insist on using \LaTeX\ for \docstrip\ (really,
% \docstrip\ does not need \LaTeX), then inform the autodetect routine
% about your intention:
% \begin{quote}
%   \verb|latex \let\install=y\input{atbegshi.dtx}|
% \end{quote}
% Do not forget to quote the argument according to the demands
% of your shell.
%
% \paragraph{Generating the documentation.}
% You can use both the \xfile{.dtx} or the \xfile{.drv} to generate
% the documentation. The process can be configured by the
% configuration file \xfile{ltxdoc.cfg}. For instance, put this
% line into this file, if you want to have A4 as paper format:
% \begin{quote}
%   \verb|\PassOptionsToClass{a4paper}{article}|
% \end{quote}
% An example follows how to generate the
% documentation with pdf\LaTeX:
% \begin{quote}
%\begin{verbatim}
%pdflatex atbegshi.dtx
%makeindex -s gind.ist atbegshi.idx
%pdflatex atbegshi.dtx
%makeindex -s gind.ist atbegshi.idx
%pdflatex atbegshi.dtx
%\end{verbatim}
% \end{quote}
%
% \section{Catalogue}
%
% The following XML file can be used as source for the
% \href{http://mirror.ctan.org/help/Catalogue/catalogue.html}{\TeX\ Catalogue}.
% The elements \texttt{caption} and \texttt{description} are imported
% from the original XML file from the Catalogue.
% The name of the XML file in the Catalogue is \xfile{atbegshi.xml}.
%    \begin{macrocode}
%<*catalogue>
<?xml version='1.0' encoding='us-ascii'?>
<!DOCTYPE entry SYSTEM 'catalogue.dtd'>
<entry datestamp='$Date$' modifier='$Author$' id='atbegshi'>
  <name>atbegshi</name>
  <caption>Execute stuff at \shipout time.</caption>
  <authorref id='auth:oberdiek'/>
  <copyright owner='Heiko Oberdiek' year='2007-2011'/>
  <license type='lppl1.3'/>
  <version number='1.18'/>
  <description>
    This package is a modern reimplementation of package
    <xref refid='everyshi'>everyshi</xref>, providing various commands
    to be executed before a <tt>\shipout</tt> command.  It makes use of
    e-TeX&#x2019;s facilities if they are available.  The package may
    be used either with LaTeX or with plain TeX.
    <p/>
    The package is part of the <xref refid='oberdiek'>oberdiek</xref> bundle.
  </description>
  <documentation details='Package documentation'
      href='ctan:/macros/latex/contrib/oberdiek/atbegshi.pdf'/>
  <ctan file='true' path='/macros/latex/contrib/oberdiek/atbegshi.dtx'/>
  <miktex location='oberdiek'/>
  <texlive location='oberdiek'/>
  <install path='/macros/latex/contrib/oberdiek/oberdiek.tds.zip'/>
</entry>
%</catalogue>
%    \end{macrocode}
%
% \begin{History}
%   \begin{Version}{2007/04/17 v1.0}
%   \item
%     First version.
%   \end{Version}
%   \begin{Version}{2007/04/18 v1.1}
%   \item
%     New method based on \cs{lastkern} is used if \eTeX\ is missing.
%   \item
%     \cs{AtBeginShipoutDiscard} also resets \cs{deadcycles}.
%   \end{Version}
%   \begin{Version}{2007/04/19 v1.2}
%   \item
%     \cs{AtBeginShipoutEarly} removed for simplification reasons.
%   \item
%     Forgotten definition of \cs{AtBegShi@Info} added.
%   \item
%     Patches for packages \xpackage{crop} and \xpackage{everyshi}
%     and class \xclass{memoir} added.
%   \end{Version}
%   \begin{Version}{2007/04/26 v1.3}
%   \item
%     Use of package \xpackage{infwarerr}.
%   \item
%     Catcode section after generic header.
%   \end{Version}
%   \begin{Version}{2007/04/27 v1.4}
%   \item
%     Small optimizations.
%   \end{Version}
%   \begin{Version}{2007/06/06 v1.5}
%   \item
%     \cs{AtBeginShipoutUpperLeft} added.
%   \item
%     Example added.
%   \item
%     Fix in second test file for newer version of \xclass{memoir}.
%   \end{Version}
%   \begin{Version}{2007/09/09 v1.6}
%   \item
%     Catcode section rewritten.
%   \end{Version}
%   \begin{Version}{2008/07/18 v1.7}
%   \item
%     Documentation of \cs{AtBeginShipoutUpperLeft} fixed and extended.
%   \end{Version}
%   \begin{Version}{2008/07/19 v1.8}
%   \item
%     \cs{AtBeginShipoutUpperLeftForeground} added.
%   \end{Version}
%   \begin{Version}{2008/07/31 v1.9}
%   \item
%     Second example (TrimBox for dvipdfmx) added.
%   \item
%     No changes in package code.
%   \end{Version}
%   \begin{Version}{2009/12/02 v1.10}
%   \item
%     \cs{AtBeginShipoutOriginalShipout} added.
%   \item
%     Test file fixed.
%   \end{Version}
%   \begin{Version}{2010/03/01 v1.11}
%   \item
%     Compatibility with \iniTeX\ except for \cs{newbox}.
%   \end{Version}
%   \begin{Version}{2010/03/25 v1.12}
%   \item
%     \cs{AtBeginShipoutNext} can now be used inside
%     \cs{AtBeginShipoutNext}.
%   \end{Version}
%   \begin{Version}{2010/08/18 v1.13}
%   \item
%     Fixes for \cs{AtBegShi@CheckDefinable}.
%   \end{Version}
%   \begin{Version}{2010/12/02 v1.14}
%   \item
%     Remove the warning because of void box if the hook
%     calls \AtBeginShipoutDiscard.
%   \end{Version}
%   \begin{Version}{2011/01/30 v1.15}
%   \item
%     Already loaded package files are not input in \hologo{plainTeX}.
%   \end{Version}
%   \begin{Version}{2011/10/05 v1.16}
%   \item
%     \cs{AtBeginShipoutAddToBox}, \cs{AtBeginShipoutAddToBoxForeground} added.
%   \item
%     \cs{AtBeginShipoutBoxWidth}, \cs{AtBeginShipoutBoxHeight},
%     \cs{AtBeginShipoutBoxDepth} added.
%   \item
%     Updates for patches of class \xclass{memoir}.
%   \end{Version}
%   \begin{Version}{2016/05/16 v1.17}
%   \item
%     Documentation updates.
%   \end{Version}
%   \begin{Version}{2016/06/09 v1.18}
%   \item
%     Update for \cs{pdfhorign} in new LuaTeX.
%   \end{Version}
% \end{History}
%
% \PrintIndex
%
% \Finale
\endinput

%        (quote the arguments according to the demands of your shell)
%
% Documentation:
%    (a) If atbegshi.drv is present:
%           latex atbegshi.drv
%    (b) Without atbegshi.drv:
%           latex atbegshi.dtx; ...
%    The class ltxdoc loads the configuration file ltxdoc.cfg
%    if available. Here you can specify further options, e.g.
%    use A4 as paper format:
%       \PassOptionsToClass{a4paper}{article}
%
%    Programm calls to get the documentation (example):
%       pdflatex atbegshi.dtx
%       makeindex -s gind.ist atbegshi.idx
%       pdflatex atbegshi.dtx
%       makeindex -s gind.ist atbegshi.idx
%       pdflatex atbegshi.dtx
%
% Installation:
%    TDS:tex/generic/oberdiek/atbegshi.sty
%    TDS:doc/latex/oberdiek/atbegshi.pdf
%    TDS:doc/latex/oberdiek/atbegshi-example1.tex
%    TDS:doc/latex/oberdiek/atbegshi-example2.tex
%    TDS:doc/latex/oberdiek/test/atbegshi-test1.tex
%    TDS:doc/latex/oberdiek/test/atbegshi-test2.tex
%    TDS:doc/latex/oberdiek/test/atbegshi-test3.tex
%    TDS:source/latex/oberdiek/atbegshi.dtx
%
%<*ignore>
\begingroup
  \catcode123=1 %
  \catcode125=2 %
  \def\x{LaTeX2e}%
\expandafter\endgroup
\ifcase 0\ifx\install y1\fi\expandafter
         \ifx\csname processbatchFile\endcsname\relax\else1\fi
         \ifx\fmtname\x\else 1\fi\relax
\else\csname fi\endcsname
%</ignore>
%<*install>
\input docstrip.tex
\Msg{************************************************************************}
\Msg{* Installation}
\Msg{* Package: atbegshi 2016/06/09 v1.18 At begin shipout hook (HO)}
\Msg{************************************************************************}

\keepsilent
\askforoverwritefalse

\let\MetaPrefix\relax
\preamble

This is a generated file.

Project: atbegshi
Version: 2016/06/09 v1.18

Copyright (C) 2007-2011 by
   Heiko Oberdiek <heiko.oberdiek at googlemail.com>

This work may be distributed and/or modified under the
conditions of the LaTeX Project Public License, either
version 1.3c of this license or (at your option) any later
version. This version of this license is in
   http://www.latex-project.org/lppl/lppl-1-3c.txt
and the latest version of this license is in
   http://www.latex-project.org/lppl.txt
and version 1.3 or later is part of all distributions of
LaTeX version 2005/12/01 or later.

This work has the LPPL maintenance status "maintained".

This Current Maintainer of this work is Heiko Oberdiek.

The Base Interpreter refers to any `TeX-Format',
because some files are installed in TDS:tex/generic//.

This work consists of the main source file atbegshi.dtx
and the derived files
   atbegshi.sty, atbegshi.pdf, atbegshi.ins, atbegshi.drv,
   atbegshi-example1.tex, atbegshi-example2.tex,
   atbegshi-test1.tex, atbegshi-test2.tex, atbegshi-test3.tex.

\endpreamble
\let\MetaPrefix\DoubleperCent

\generate{%
  \file{atbegshi.ins}{\from{atbegshi.dtx}{install}}%
  \file{atbegshi.drv}{\from{atbegshi.dtx}{driver}}%
  \usedir{tex/generic/oberdiek}%
  \file{atbegshi.sty}{\from{atbegshi.dtx}{package}}%
  \usedir{doc/latex/oberdiek}%
  \file{atbegshi-example1.tex}{\from{atbegshi.dtx}{example1}}%
  \file{atbegshi-example2.tex}{\from{atbegshi.dtx}{example2}}%
  \usedir{doc/latex/oberdiek/test}%
  \file{atbegshi-test1.tex}{\from{atbegshi.dtx}{test1}}%
  \file{atbegshi-test2.tex}{\from{atbegshi.dtx}{test2}}%
  \file{atbegshi-test3.tex}{\from{atbegshi.dtx}{test3}}%
  \nopreamble
  \nopostamble
  \usedir{source/latex/oberdiek/catalogue}%
  \file{atbegshi.xml}{\from{atbegshi.dtx}{catalogue}}%
}

\catcode32=13\relax% active space
\let =\space%
\Msg{************************************************************************}
\Msg{*}
\Msg{* To finish the installation you have to move the following}
\Msg{* file into a directory searched by TeX:}
\Msg{*}
\Msg{*     atbegshi.sty}
\Msg{*}
\Msg{* To produce the documentation run the file `atbegshi.drv'}
\Msg{* through LaTeX.}
\Msg{*}
\Msg{* Happy TeXing!}
\Msg{*}
\Msg{************************************************************************}

\endbatchfile
%</install>
%<*ignore>
\fi
%</ignore>
%<*driver>
\NeedsTeXFormat{LaTeX2e}
\ProvidesFile{atbegshi.drv}%
  [2016/06/09 v1.18 At begin shipout hook (HO)]%
\documentclass{ltxdoc}
\usepackage{holtxdoc}[2011/11/22]
\begin{document}
  \DocInput{atbegshi.dtx}%
\end{document}
%</driver>
% \fi
%
%
% \CharacterTable
%  {Upper-case    \A\B\C\D\E\F\G\H\I\J\K\L\M\N\O\P\Q\R\S\T\U\V\W\X\Y\Z
%   Lower-case    \a\b\c\d\e\f\g\h\i\j\k\l\m\n\o\p\q\r\s\t\u\v\w\x\y\z
%   Digits        \0\1\2\3\4\5\6\7\8\9
%   Exclamation   \!     Double quote  \"     Hash (number) \#
%   Dollar        \$     Percent       \%     Ampersand     \&
%   Acute accent  \'     Left paren    \(     Right paren   \)
%   Asterisk      \*     Plus          \+     Comma         \,
%   Minus         \-     Point         \.     Solidus       \/
%   Colon         \:     Semicolon     \;     Less than     \<
%   Equals        \=     Greater than  \>     Question mark \?
%   Commercial at \@     Left bracket  \[     Backslash     \\
%   Right bracket \]     Circumflex    \^     Underscore    \_
%   Grave accent  \`     Left brace    \{     Vertical bar  \|
%   Right brace   \}     Tilde         \~}
%
% \GetFileInfo{atbegshi.drv}
%
% \title{The \xpackage{atbegshi} package}
% \date{2016/06/09 v1.18}
% \author{Heiko Oberdiek\thanks
% {Please report any issues at https://github.com/ho-tex/oberdiek/issues}\\
% \xemail{heiko.oberdiek at googlemail.com}}
%
% \maketitle
%
% \begin{abstract}
% This package is a modern reimplementation of package \xpackage{everyshi}
% without the burden of compatibility. It makes use of \eTeX's if available.
% Both \LaTeX\ and \plainTeX\ are supported.
% \end{abstract}
%
% \tableofcontents
%
% \section{Documentation}
%
% Package \xpackage{atbegshi} redefines \cs{shipout} to insert
% hooks for user code that is executed before the page is shipped out.
% The code may modify or even discard the output page.
% Three hooks are implemented:
% \begin{enumerate}
% \item A hook that is executed for every page, see\\
%       \cs{AtBeginShipout}
% \item A hook that is executed for the next page only, see\\
%       \cs{AtBeginShipoutNext}
% \item A hook that is only executed for the first page, see\\
%       \cs{AtBeginShipoutFirst}
% \end{enumerate}
% The hooks are executed in this order. The following three macros
% provide the user interface for adding code to these hooks:
%
% \begin{declcs}{AtBeginShipout} \M{code}\\
%   \cs{AtBeginShipoutBox}
% \end{declcs}
% Execute the \meta{code} for every page.
% The page contents is held in box register \cs{AtBeginShipoutBox}
% and may be modified. Use \cs{AtBeginShipoutDiscard} if you want to
% discard the page.
%
% \emph{Note:} Package \xpackage{everyshi} uses box register 255. With
% package \xpackage{atbegshi} you must use \cs{AtBeginShipoutBox} instead.
%
% If \LaTeX\ calls \cs{shipout} in \cs{@outputpage} (part of its output
% routine),
% the meaning of \cs{protect} is \cs{noexpand}. \LaTeX\ sets
% \cs{protect} to the appropriate \cs{@typeset@protect} in the box
% that is shipped out. This is too late for the hooks, they
% are called earlier in the redefined \cs{shipout}. Therefore
% package \xpackage{atbegshi} sets \cs{protect} to \cs{@typeset@protect}
% before it calls the hooks.
% (In \cs{EveryShipout} of package \xpackage{everyshi} the
% user is responsible for the correct setting of \cs{protect}.)
%
% \begin{declcs}{AtBeginShipoutNext} \M{code}
% \end{declcs}
% This reimplements package \xpackage{everyshi}'s \cs{AtNextShipout}.
% The \meta{code} is executed at shipout time of the next page only.
% It is just a convenience macro, it can be easily replaced by something
% like:
%\begin{quote}
%\begin{verbatim}
%\newcommand{\MyShipoutHook}{}%
%\AtBeginShipout{\MyShipoutHook}
%\gdef\MyShipoutHook{%
%  ... do something with next page ...
%  \gdef\MyShipoutHook{}%
% }
%\end{verbatim}
%\end{quote}
% (This can be necessary, if hook order does matter).
%
% \begin{declcs}{AtBeginShipoutFirst} \M{code}
% \end{declcs}
% This reimplements \LaTeX's \cs{AtBeginDvi}. This hook is
% usually used for \cs{special} commands that include PostScript
% header files. The \cs{code} is directly executed in a \cs{vbox}
% that is put at the beginning of the output page. Dealing with
% the output box \cs{AtBeginShipoutBox} is not necessary and
% not permitted here.
%
% \begin{declcs}{AtBeginShipoutDiscard}
% \end{declcs}
% This macro notifies package \xpackage{atbegshi} that
% the output page is discarded. The remaining hook code
% and the remaining hooks are not executed and the page is thrown away.
% Also \cs{deadcycles} is cleared to zero like an ordinary \cs{shipout}
% would do.
%
% \begin{declcs}{AtBeginShipoutInit}
% \end{declcs}
% Usually the redefinition of \cs{shipout} is delayed by
% \cs{AtBeginDocument} (if this macro exists). This can be
% too late, if other packages also redefines \cs{shipout}
% and the order does matter.
% \cs{AtBeginShipoutInit} forces the immediate redefinition
% of \cs{shipout}.
%
% \begin{declcs}{AtBeginShipoutAddToBox} \M{stuff}\\
%   \cs{AtBeginShipoutAddToBoxForeground} \M{stuff}
% \end{declcs}
% A quite common use case is the addition of \cs{special}
% or other whatsits to the page output box. Macro
% \cs{AtBeginShipoutAddToBox} puts \meta{stuff} in a
% box with zeroed dimensions. The box with the \meta{stuff}
% is put in the upper left corner of the shipout box
% \cs{AtBeginShipoutBox}. Macro \cs{AtBeginShipoutAddToBox}
% puts the \meta{stuff} in the background, the other macro
% \cs{AtBeginShipoutAddToBoxForeground} in the foreground
% after the original shipout box contents is set.
%
% A void shipout box (that means a discarded page)
% remains void that means \meta{stuff} is ignored in this case.
% The box type of \cs{AtBeginShipoutBox} is preserved.
% Also the box nesting level for the original contents
% of \cs{AtBeginShipoutBox} remains, for example, to avoid trouble
% with links across pages in case of \hologo{pdfTeX}.
%
% \begin{declcs}{AtBeginShipoutUpperLeft} \M{background material}
% \end{declcs}
% This is a macro that puts material in the background of
% box \cs{AtBeginShipoutBox}.
% The \meta{background material} is set in an \cs{hbox}, the
% reference point is the upper left corner of the output page.
% In case of \pdfTeX\ in PDF mode, the settings of
% \cs{pdfhorigin} and \cs{pdfvorigin} are respected.
%
% The macro \cs{AtBeginShipoutUpperLeft} is intended to be used
% in one of the hook setting macros, such as
% \cs{AtBeginShipout}, \cs{AtBeginShipoutFirst}, or
% \cs{AtBeginShipoutNext}.
%
% For \LaTeX\ users the \meta{background material} is set
% inside a \texttt{picture} environment:
% \begin{quote}
%   |\begin{picture}(0,0)|\\
%   \mbox{}\quad |\setlength{\unitlength}{1pt}%|\\
%   \mbox{}\quad \meta{background material}\\
%   |\end{picture}|
% \end{quote}
%
% \begin{declcs}{AtBeginShipoutUpperLeftForeground} \M{foreground material}
% \end{declcs}
% See \cs{AtBeginShipoutUpperLeft}. The difference is that the material
% is put in the foreground.
%
% \begin{declcs}{AtBeginShipoutOriginalShipout} \meta{box}
% \end{declcs}
% It stores the meaning of \cs{shipout} at the time this package
% is loaded.
%
% \begin{declcs}{AtBeginShipoutBoxWidth}\\
%   \cs{AtBeginShipoutBoxHeight}\\
%   \cs{AtBeginShipoutBoxDepth}\\
% \end{declcs}
% These macros store the dimensions of the output box \cs{AtBeginShipoutBox}
% before the original shipout is called. If \cs{shipout} is not
% redefined before the package loading or the box dimensions
% are not changed by the redefined \cs{shipout}, these macros
% contain the dimensions of the shipout box. These values can be
% remembered by \cs{label} and \cs{ref}. For example, this is done
% by the package module \xpackage{zref-pagelayout} of project \xpackage{zref}.
% The dimensions of the shipout page can be used in some \hologo{TeX}
% engines (\hologo{pdfTeX} in PDF mode, \hologo{XeTeX}) to calculate
% the media size of the shipout page if \cs{pdfpagewidth} and
% \cs{pdfpageheight} are not set.
%
% \subsection{Examples}
%
% \subsubsection{Example: circle in background}
%
% In this example we put a circle in the background in the middle
% of the paper.
%
%    \begin{macrocode}
%<*example1>
%    \end{macrocode}
%    \begin{macrocode}
\documentclass[a4paper]{article}
\usepackage{color}
\usepackage{atbegshi}
%    \end{macrocode}
%    Package \xpackage{picture} makes life a little easier, because
%    we can now also use length specifications in |picture|'s commands.
%    \begin{macrocode}
\usepackage{picture}
%    \end{macrocode}
%    Now we draw the circle in the middle of the paper.
%    \cs{put} moves downwards, because the origin is at the top
%    of the page, not at its bottom.
%    \begin{macrocode}
\AtBeginShipout{%
  \AtBeginShipoutUpperLeft{%
    \put(0.5\paperwidth,-0.5\paperheight){\circle{10}}%
  }%
}
\begin{document}
\section{Hello World}
\newpage
\AtBeginShipoutNext{%
  \AtBeginShipoutUpperLeft{%
    \color{red}%
    \put(0,-0.5\paperheight){\line(1,0){\paperwidth}}%
    \put(0.5\paperwidth, 0){\line(0,-1){\paperheight}}%
  }%
}
Only on this page we add a red cross.
\newpage
This page has the circle only.
\par
\vspace{\fill}
The next page will be discarded.
\newpage
\AtBeginShipoutNext{%
  \AtBeginShipoutDiscard
}
This page is discarded.
\newpage
The last page.
\end{document}
%    \end{macrocode}
%    \begin{macrocode}
%</example1>
%    \end{macrocode}
%
% \subsubsection{Example: adding TrimBox for dvipdfmx}
%
% Now an example from ``real life'' follows. Someone from the mailing  list
% for dvipdfmx wants to put a TrimBox on every page. If we use
% \cs{AtBeginShipout}, we have to put the \cs{special} inside the
% box \cs{AtBeginShipoutBox} that gets shipped out.
%    \begin{macrocode}
%<*example2>
\documentclass{minimal}
\usepackage{atbegshi}
\usepackage[
  dvipdfm,
  paperwidth=630bp,
  paperheight=810bp
]{geometry}
\AtBeginShipout{%
  \setbox\AtBeginShipoutBox=\hbox{%
    \special{pdf: put @thispage <</TrimBox[9 9 621 801]>>}%
    \box\AtBeginShipoutBox
  }%
}
\begin{document}
  First page
  \newpage
  Second page
\end{document}
%</example2>
%    \end{macrocode}
% Remember, in \cs{AtBeginShipoutBoxFirst} the \cs{setbox} wrapper code
% is implicitly given and the \cs{special} is used directly.
%
% \section{Method of \cs{shipout} overloading}
%
% \subsection{\cs{shipout}}
%
% The \TeX\ primitive command \cs{shipout} takes a box specification
% and puts the box as a new page in the output file.
% There are two kinds of box specifications:
% \begin{description}
% \item[Direct boxes:]
%   They are given by \cs{hbox}, \cs{vbox}, or \cs{vtop},\\
%   e.g. |\shipout\hbox{Hello World}|.
% \item[Indirect boxes:]
%   \cs{box} or \cs{copy} references a box register by number.
%   The box register contains the contents of the box.
%
%   \emph{Note:} \cs{box} also clears the box register globally.
% \end{description}
% Then we have to differentiate between void and empty boxes:
% \begin{description}
% \item[Void:]
%   Initially or after \cs{box} there is no box in the box register.
%   In this cases the box register is not empty, but \emph{void}.
% \item[Empty:]
%   A box with empty contents, such as
%   |\hbox{}| (= \cs{null}) or |\vbox{}| is an \emph{empty \cs{hbox}}
%   or \emph{empty \cs{vbox}}. If a box register holds such a
%   box, the box still exists, therefore the box register is
%   \emph{not void}.
% \end{description}
%
% \subsection{\cs{afterassignment}}
%
% We want to overload \cs{shipout} to do something with the box.
% It is quite impossible to do this reliable by catching the box
% using macro arguments. The
% variety of box specifications is too large, Examples:
% \begin{quote}
%   |\shipout\null|\\
%   |\shipout\vbox{...}|\\
%   |\shipout\vtop\bgroup ...\egroup|\\
%   |\shipout\box255|
% \end{quote}
% Even worse, the braces don't need to be balanced:
% \begin{quote}
%   |\shipout\hbox\bgroup}|\\
%   |\shipout\vbox{\egroup|
% \end{quote}
% Happily TeX provides a reliable way via \cs{afterassignment}.
% It takes a macro name and executes it just after the assignment.
%
% Now we can redefine \cs{shipout}. The box specification that
% follows \cs{shipout} is catched by \cs{setbox}. This is an
% assignment to a box register. \cs{afterassignment}
% notifies \TeX, that we want to call \cs{@test} right after
% the assignment:
%\begin{quote}
%\begin{verbatim}
%\shipout :=
%  \afterassignment\@test
%  \setbox\mybox=
%\end{verbatim}
%\end{quote}
% We have seen different box specifications. Indirect boxes are easy
% to understand:
% \begin{quote}
%   |\shipout\box0| $\Rightarrow$ |\setbox\mybox=\box0 |\fbox{\cs{@test}}
% \end{quote}
% However direct boxes can have arbitrary contents with lots of
% other assignments. It would be quite inpredictable if \TeX\ would
% put |\@test| after the first of such an assignment or after the
% box specification if the box lacks of assignments.
% Therefore \TeX\ puts |\@test| right at the beginning of the
% box specification, e.g:
% \begin{quote}
%   |\shipout\hbox{Hello World}|\\
%   $\Rightarrow$
%   |\setbox\mybox=\hbox{|\fbox{\cs{@test}}| Hello World}|
% \end{quote}
%
% \subsection{Test for direct or indirect boxes}
%
% Now we want to execute \cs{@test}, but where are we?
% We can be after the completed box assignment, if \cs{shipout}
% was called with an indirect box. Or we are right at the beginning
% of a direct box.
%
% \subsubsection{With \eTeX}
%
% With the \eTeX's extensions the answer is very easy:
% Being inside the direct box means that we are inside a new group.
% The new primitive command \cs{currentgrouplevel} tells how
% deeply the groups are currently nested. Macro \cs{@test} just
% compares the previously stored group level with the current one:
%\begin{quote}
%\begin{verbatim}
%\shipout :=
%  \edef\saved@grouplevel{\number\currentgrouplevel}
%  \afterassignment\@test
%  \setbox\mybox=
%
%\@test :=
%  \ifnum\saved@grouplevel=\currentgrouplevel
%    % case: indirect box, the assignment is completed
%    \@output
%  \else
%    % case: direct box, we are inside the box
%    \aftergroup\@outbox
%  \fi
%\end{verbatim}
%\end{quote}
%
% \subsubsection{Without \eTeX}
%
% Life becomes complicate without \eTeX. We cannot ask the
% group level. However, if we are inside a direct box,
% the box register \cs{mybox} is not yet changed by \cs{setbox}.
% Thus we need a special initial value and compare
% it in \cs{@test} with the current value of the box.
%
% What can be used as initial value? Arbitrary box contents cannot
% be compared. \TeX\ only tells us a few properties:
% \begin{itemize}
% \item Box type:\cs{ifhbox}, \cs{ifvbox}
% \item Dimensions: \cs{wd}, \cs{ht}, \cs{dp}
% \item Voidness: \cs{ifvoid}
% \end{itemize}
% Unhappily all these qualities even combined are not sufficient
% for constructing an initial box value, because \cs{shipout} can be
% called with a box that is accidently just the same as the choosen
% initial value.
%
% Nevertheless we have two alternatives for an initial value:
% \begin{itemize}
% \item A box of some type with some funny settings that are
% unlikely to occur in real life, e.g a height of |4911sp-\maxdimen|.
% \item A void box.
% \end{itemize}
% A collision between this initial value and an indirect \cs{shipout}
% box with just the same value is possible.
% Then \cs{@test} will make a wrong decision that it is executed
% inside a direct box and delays \cs{@output} by \cs{aftergroup}.
% Thus \cs{@output} is not called at the place we want.
% In contrary, the result is an uncertainty about the place:
% \begin{itemize}
% \item \cs{shipout} is used in a group that perhaps closes some
%   pages later. A bad place for \cs{@output}.
% \item Without a surrounding group \cs{aftergroup} effectively kills
%   its argument.
% \end{itemize}
% In the first case of a box with special dimensions we can even loose
% the page. However in the case of the void box, this effect is even
% desired, because the original \cs{shipout} does not output void boxes.
% All we have to do is to ensure that our box \cs{mybox} is always
% void except for the phase when the overloaded \cs{shipout} is executed.
% And secondly we must keep this semantics of \cs{shipout} for the void case
% in our macros, namely \cs{@output}.
%\begin{quote}
%\begin{verbatim}
%\shipout :=
%  % trick to get a void box \mybox
%  \begingroup
%    \setbox\mybox=\box\mybox
%  \endgroup
%  \afterassignment\@test
%  \setbox\mybox=
%
%\@test :=
%  \ifvoid\mybox
%    \aftergroup\@output
%  \else
%    \@output
%  \fi
%\end{verbatim}
%\end{quote}
% The nasty case is |\shipout\box\voidb@x| where the indirect box
% is void and that must not generate an output page.
% If a surrounding group is missing the output is ignored
% because of \cs{aftergroup}.
% Otherwise \@output is called some time
% later when the surrounding group closes.
% But \cs{mybox} is void outside the execution phase
% of the redefined \cs{shipout}. Also \cs{@output} checks for a
% void box and cancels the page output.
% The disadvantage remains that the hook in \cs{@output} is called
% for a page that will not be output.
%
% \subsubsection{\cs{lastkern} method}
%
% At the beginning of a new box, there is no \cs{kern}, the contents
% of the box is still empty and \cs{lastkern} returns 0\,pt.
% This can be used to distinguish between direct and indirect boxes:
% We execute \cs{setbox} in a box with a preceding non-zero kern.
% After an indirect box, \cs{lastkern} sees this kern, otherwise
% it returns 0\,pt.
%\begin{quote}
%\begin{verbatim}
%\shipout :=
%  \begingroup
%    \setbox\mybox=\hbox\bgroup
%      \kern1pt
%      \afterassignment\shipout@test
%      \global\setbox\mybox=
%\@test :=
%  \ifdim\lastkern=0pt
%    % direct box
%    \aftergroup\egroup
%    \aftergroup\endgroup
%    \aftergroup\@output
%  \else
%    \egroup
%    \endgroup
%    \@output
%  \fi
%\end{verbatim}
%\end{quote}
% We have two \cs{setbox} commands. The first creates a controlled
% context box where we can safely insert a \cs{kern}. We get rid of
% this temporarily used context box by putting the local \cs{setbox}
% in a group.
%
% After the group we want to have our shipout box in \cs{mybox}.
% Therefore we use a global assignment here.
%
% \subsection{Output}
%
% With or without \eTeX\ we ensure the original behaviour of
% \cs{shipout} that void boxes do not generate output pages.
%
% Now we can place the hook \cs{@hook} for the user code that
% wants to manipulate the output box.
%\begin{quote}
%\begin{verbatim}
%\@output :=
%  \ifvoid\mybox
%    % cancel output of void box
%  \else
%    \@hook
%    \ifvoid\mybox
%      % user code in \@hook could has voided the box
%    \else
%      \original@shipout\box\mybox
%    \fi
%  \fi
%\end{verbatim}
%\end{quote}
%
% \subsection{Separate box register}
%
% So far we have said nothing about the box number of \cs{mybox}.
% The following case that outputs the same page twice
% shows that we are not free in the use
% of the box register:
% \begin{quote}
%   |\shipout\copy<num> \shipout\box<num>|
% \end{quote}
% We manipulate the box by the hook and without \eTeX\ the box
% must even be voided. However, the use case above requires that
% the box contents does not change at all. Therefore we must
% reserve a separate box register to avoid collisions with
% user box registers.
%
% \emph{Note:} Box register number 255 is special for the output
% routine, because TeX complains if this box is not voided
% by the output routine.
% However, this requirement does not apply to \cs{shipout} at all.
% In fact \cs{shipout} does not change any box register. This is
% usually done by a call of \cs{box}, but the output routine
% can do it later \emph{after} invoking of \cs{shipout}.
%
% \subsection{Summary}
%
% \subsubsection{With \eTeX}
%
% Putting the pieces together we get for \eTeX:
%\begin{quote}
%\begin{verbatim}
%\newbox\mybox
%\let\original@shipout\shipout
%
%\shipout :=
%  \edef\saved@grouplevel{\number\currentgrouplevel}
%  \afterassignment\@test
%  \setbox\mybox=
%
%\@test :=
%  \ifnum\saved@grouplevel<\currentgrouplevel
%    \expandafter\aftergroup
%  \fi
%  \@output
%
%\@output :=
%  \ifvoid\mybox
%    % cancel output of void box
%  \else
%    \@hook
%    \ifvoid\mybox
%      % user code in \@hook could have voided the box
%    \else
%      \original@shipout\box\mybox
%    \fi
%  \fi
%\end{verbatim}
%\end{quote}
%
% \subsubsection{Without \eTeX, traditional way}
%
% And for \TeX\ without \eTeX:
%\begin{quote}
%\begin{verbatim}
%\newbox\mybox
%\begingroup
%   \setbox\mybox=\box\mybox % ensure \mybox is void
%\endgroup
%\let\original@shipout\shipout
%
%\shipout :=
%  % trick to get a void box \mybox
%  \begingroup
%    \setbox\mybox=\box\mybox
%  \endgroup
%  \afterassignment\@test
%  \setbox\mybox=
%
%\@test :=
%  \ifvoid\mybox
%    \expandafter\aftergroup
%  \fi
%  \@output
%
%\@output :=
%  \ifvoid\mybox
%    % cancel output of void box
%  \else
%    \@hook
%    \ifvoid\mybox
%      % user code in \@hook could have voided the box
%    \else
%      \original@shipout\box\mybox
%    \fi
%  \fi
%\end{verbatim}
%\end{quote}
%
% \subsubsection{\cs{lastkern} method}
%
% And for \TeX\ without \eTeX\ using the \cs{lastkern} method:
%\begin{quote}
%\begin{verbatim}
%\newbox\mybox
%\let\original@shipout\shipout
%
%\shipout :=
%  \begingroup
%  \setbox\mybox=\hbox\bgroup
%  \kern1pt
%  \afterassignment\@test
%  \setbox\mybox=
%
%\@test :=
%  \ifdim\lastkern=0pt
%    \expandafter\aftergroup
%  \fi
%  \@output
%
%\@output :=
%  \egroup
%  \endgroup
%  \ifvoid\mybox
%    % cancel output of void box
%  \else
%    \@hook
%    \ifvoid\mybox
%      % user code in \@hook could have voided the box
%    \else
%      \original@shipout\box\mybox
%    \fi
%  \fi
%\end{verbatim}
%\end{quote}
%
%
% \StopEventually{
% }
%
% \section{Implementation}
%
% Package \xpackage{atbegshi} uses \eTeX's \cs{currentgrouplevel},
% if it is available. Otherwise the \cs{lastkern} method is used.
%
%    \begin{macrocode}
%<*package>
%    \end{macrocode}
%
% \subsection{Reload check and package identification}
%    Reload check, especially if the package is not used with \LaTeX.
%    \begin{macrocode}
\begingroup\catcode61\catcode48\catcode32=10\relax%
  \catcode13=5 % ^^M
  \endlinechar=13 %
  \catcode35=6 % #
  \catcode39=12 % '
  \catcode44=12 % ,
  \catcode45=12 % -
  \catcode46=12 % .
  \catcode58=12 % :
  \catcode64=11 % @
  \catcode123=1 % {
  \catcode125=2 % }
  \expandafter\let\expandafter\x\csname ver@atbegshi.sty\endcsname
  \ifx\x\relax % plain-TeX, first loading
  \else
    \def\empty{}%
    \ifx\x\empty % LaTeX, first loading,
      % variable is initialized, but \ProvidesPackage not yet seen
    \else
      \expandafter\ifx\csname PackageInfo\endcsname\relax
        \def\x#1#2{%
          \immediate\write-1{Package #1 Info: #2.}%
        }%
      \else
        \def\x#1#2{\PackageInfo{#1}{#2, stopped}}%
      \fi
      \x{atbegshi}{The package is already loaded}%
      \aftergroup\endinput
    \fi
  \fi
\endgroup%
%    \end{macrocode}
%    Package identification:
%    \begin{macrocode}
\begingroup\catcode61\catcode48\catcode32=10\relax%
  \catcode13=5 % ^^M
  \endlinechar=13 %
  \catcode35=6 % #
  \catcode39=12 % '
  \catcode40=12 % (
  \catcode41=12 % )
  \catcode44=12 % ,
  \catcode45=12 % -
  \catcode46=12 % .
  \catcode47=12 % /
  \catcode58=12 % :
  \catcode64=11 % @
  \catcode91=12 % [
  \catcode93=12 % ]
  \catcode123=1 % {
  \catcode125=2 % }
  \expandafter\ifx\csname ProvidesPackage\endcsname\relax
    \def\x#1#2#3[#4]{\endgroup
      \immediate\write-1{Package: #3 #4}%
      \xdef#1{#4}%
    }%
  \else
    \def\x#1#2[#3]{\endgroup
      #2[{#3}]%
      \ifx#1\@undefined
        \xdef#1{#3}%
      \fi
      \ifx#1\relax
        \xdef#1{#3}%
      \fi
    }%
  \fi
\expandafter\x\csname ver@atbegshi.sty\endcsname
\ProvidesPackage{atbegshi}%
  [2016/06/09 v1.18 At begin shipout hook (HO)]%
%    \end{macrocode}
%
% \subsection{Catcodes}
%
%    \begin{macrocode}
\begingroup\catcode61\catcode48\catcode32=10\relax%
  \catcode13=5 % ^^M
  \endlinechar=13 %
  \catcode123=1 % {
  \catcode125=2 % }
  \catcode64=11 % @
  \def\x{\endgroup
    \expandafter\edef\csname AtBegShi@AtEnd\endcsname{%
      \endlinechar=\the\endlinechar\relax
      \catcode13=\the\catcode13\relax
      \catcode32=\the\catcode32\relax
      \catcode35=\the\catcode35\relax
      \catcode61=\the\catcode61\relax
      \catcode64=\the\catcode64\relax
      \catcode123=\the\catcode123\relax
      \catcode125=\the\catcode125\relax
    }%
  }%
\x\catcode61\catcode48\catcode32=10\relax%
\catcode13=5 % ^^M
\endlinechar=13 %
\catcode35=6 % #
\catcode64=11 % @
\catcode123=1 % {
\catcode125=2 % }
\def\TMP@EnsureCode#1#2{%
  \edef\AtBegShi@AtEnd{%
    \AtBegShi@AtEnd
    \catcode#1=\the\catcode#1\relax
  }%
  \catcode#1=#2\relax
}
\TMP@EnsureCode{40}{12}% (
\TMP@EnsureCode{41}{12}% )
\TMP@EnsureCode{44}{12}% ,
\TMP@EnsureCode{45}{12}% -
\TMP@EnsureCode{47}{12}% /
\TMP@EnsureCode{46}{12}% .
\TMP@EnsureCode{58}{12}% :
\TMP@EnsureCode{91}{12}% [
\TMP@EnsureCode{93}{12}% ]
\TMP@EnsureCode{94}{7}% ^(superscript)
\TMP@EnsureCode{96}{12}% `
\edef\AtBegShi@AtEnd{\AtBegShi@AtEnd\noexpand\endinput}
%    \end{macrocode}
%
% \subsection{Preparations}
%
%    \begin{macrocode}
\begingroup\expandafter\expandafter\expandafter\endgroup
\expandafter\ifx\csname RequirePackage\endcsname\relax
  \def\TMP@RequirePackage#1[#2]{%
    \begingroup\expandafter\expandafter\expandafter\endgroup
    \expandafter\ifx\csname ver@#1.sty\endcsname\relax
      \input #1.sty\relax
    \fi
  }%
  \TMP@RequirePackage{infwarerr}[2007/09/09]%
  \TMP@RequirePackage{ltxcmds}[2010/03/01]%
\else
  \RequirePackage{infwarerr}[2007/09/09]%
  \RequirePackage{ltxcmds}[2010/03/01]%
\fi
%    \end{macrocode}
%
%    \begin{macro}{\AtBegShi@CheckDefinable}
%    \begin{macrocode}
\begingroup\expandafter\expandafter\expandafter\endgroup
\expandafter\ifx\csname @ifdefinable\endcsname\relax
  \def\AtBegShi@CheckDefinable#1{%
    \ifcase\ifx#1\relax
             \ltx@one
           \else
             \ifx#1\@undefined
               \ltx@one
             \else
               \ltx@zero
             \fi
           \fi
      \@PackageError{atbegshi}{%
        \string#1\space is already defined%
      }\@ehd
    \fi
  }%
\else
  \def\AtBegShi@CheckDefinable#1{%
    \@ifdefinable{#1}{}%
  }%
\fi
%    \end{macrocode}
%    \end{macro}
%
%    \begin{macro}{\ifAtBegShi@Discarded}
%    \begin{macrocode}
\ltx@newif\ifAtBegShi@Discarded
%    \end{macrocode}
%    \end{macro}
%    \begin{macro}{\AtBeginShipoutDiscard}
%    \begin{macrocode}
\AtBegShi@CheckDefinable\AtBeginShipoutDiscard
\def\AtBeginShipoutDiscard{%
  \deadcycles=\ltx@zero
  \global\AtBegShi@Discardedtrue
}
%    \end{macrocode}
%    \end{macro}
%
%    \begin{macrocode}
\begingroup\expandafter\expandafter\expandafter\endgroup
\expandafter\ifx\csname currentgrouplevel\endcsname\relax
  \catcode`X=9  % ignore
  \catcode`E=14 % comment
\else
  \catcode`X=14 % comment
  \catcode`E=9  % ignore
\fi
%    \end{macrocode}
%
%    \begin{macro}{\AtBegShi@Shipout}
%    \begin{macrocode}
\def\AtBegShi@Shipout{%
X \begingroup
X \setbox\AtBeginShipoutBox=\hbox\bgroup
X \kern\p@
E \edef\AtBegShi@GroupLevel{\number\currentgrouplevel}%
  \afterassignment\AtBegShi@Test
X \global
  \setbox\AtBeginShipoutBox=%
}
%    \end{macrocode}
%    \end{macro}
%
%    \begin{macro}{\AtBegShi@Test}
%    \begin{macrocode}
\def\AtBegShi@Test{%
X \ifdim\lastkern=0pt %
E \ifnum\AtBegShi@GroupLevel<\currentgrouplevel
    \expandafter\aftergroup
  \fi
  \AtBegShi@Output
}
%    \end{macrocode}
%    \end{macro}
%
%    \begin{macro}{\AtBegShi@Output}
%    \begin{macrocode}
\def\AtBegShi@Output{%
X \egroup
X \endgroup
  \ifvoid\AtBeginShipoutBox
    \@PackageWarning{atbegshi}{Ignoring void shipout box}%
  \else
    \let\AtBegShi@OrgProtect\protect
    \csname set@typeset@protect\endcsname
    \global\AtBegShi@Discardedfalse
    \AtBegShi@Hook
    \expandafter\gdef\expandafter\AtBegShi@HookNext
    \expandafter{\expandafter}%
    \AtBegShi@HookNext
    \ifAtBegShi@Discarded
      \@PackageInfoNoLine{atbegshi}{Shipout page discarded}%
      \global\AtBegShi@Discardedfalse
      \begingroup
        \setbox\AtBeginShipoutBox\box\AtBeginShipoutBox
      \endgroup
      \let\protect\AtBegShi@OrgProtect
    \else
      \AtBegShi@First
      \let\protect\AtBegShi@OrgProtect
      \AtBegShi@GetBoxSize\AtBeginShipoutBox
      \ltx@ifundefined{AtNextShipout}{%
      }{%
        \AtNextShipout{\AtBegShi@GetBoxSize\@cclv}%
      }%
      \AtBeginShipoutOriginalShipout\box\AtBeginShipoutBox
    \fi
  \fi
}
%    \end{macrocode}
%    \end{macro}
%
%    \begin{macro}{\AtBegShi@GetBoxSize}
%    \begin{macrocode}
\def\AtBegShi@GetBoxSize#1{%
  \xdef\AtBeginShipoutBoxWidth{\the\wd#1}%
  \xdef\AtBeginShipoutBoxHeight{\the\ht#1}%
  \xdef\AtBeginShipoutBoxDepth{\the\dp#1}%
}
%    \end{macrocode}
%    \end{macro}
%    \begin{macro}{\AtBeginShipoutBoxWidth}
%    \begin{macrocode}
\def\AtBeginShipoutBoxWidth{0pt}
%    \end{macrocode}
%    \end{macro}
%    \begin{macro}{\AtBeginShipoutBoxHeight}
%    \begin{macrocode}
\def\AtBeginShipoutBoxHeight{0pt}
%    \end{macrocode}
%    \end{macro}
%    \begin{macro}{\AtBeginShipoutBoxDepth}
%    \begin{macrocode}
\def\AtBeginShipoutBoxDepth{0pt}
%    \end{macrocode}
%    \end{macro}
%
%    \begin{macrocode}
\catcode`\X=11 %
\catcode`\E=11 %
%    \end{macrocode}
%
%    \begin{macro}{\AtBegShi@First}
%    \begin{macrocode}
\def\AtBegShi@First{%
  \ifx\AtBegShi@HookFirst\ltx@empty
  \else
    \AtBeginShipoutAddToBox{\AtBegShi@HookFirst}%
  \fi
  \global\let\AtBegShi@First\ltx@empty
  \global\let\AtBeginShipoutFirst\AtBegShi@FirstDisabled
}
%    \end{macrocode}
%    \end{macro}
%
%    \begin{macro}{\AtBegShi@Hook}
%    \begin{macrocode}
\gdef\AtBegShi@Hook{}
%    \end{macrocode}
%    \end{macro}
%    \begin{macro}{\AtBegShi@HookNext}
%    \begin{macrocode}
\gdef\AtBegShi@HookNext{}
%    \end{macrocode}
%    \end{macro}
%    \begin{macro}{\AtBegShi@HookFirst}
%    \begin{macrocode}
\gdef\AtBegShi@HookFirst{}
%    \end{macrocode}
%    \end{macro}
%    \begin{macro}{\AtBeginShipout}
%    \begin{macrocode}
\AtBegShi@CheckDefinable\AtBeginShipout
\def\AtBeginShipout{%
  \AtBegShi@AddHook\AtBegShi@Hook
}
%    \end{macrocode}
%    \end{macro}
%    \begin{macro}{\AtBeginShipoutNext}
%    \begin{macrocode}
\AtBegShi@CheckDefinable\AtBeginShipoutNext
\def\AtBeginShipoutNext{%
  \AtBegShi@AddHook\AtBegShi@HookNext
}
%    \end{macrocode}
%    \end{macro}
%    \begin{macro}{\AtBeginShipoutFirst}
%    \begin{macrocode}
\AtBegShi@CheckDefinable\AtBeginShipoutFirst
\def\AtBeginShipoutFirst{%
  \AtBegShi@AddTo\AtBegShi@HookFirst
}
%    \end{macrocode}
%    \end{macro}
%    \begin{macro}{\AtBegShi@FirstDisabled}
%    \begin{macrocode}
\long\def\AtBegShi@FirstDisabled#1{%
  \@PackageWarning{atbegshi}{%
    First page is already shipped out, ignoring\MessageBreak
    \string\AtBeginShipoutFirst
  }%
}
%    \end{macrocode}
%    \end{macro}
%
%    \begin{macro}{\AtBegShi@AddTo}
%    \begin{macrocode}
\begingroup\expandafter\expandafter\expandafter\endgroup
\expandafter\ifx\csname g@addto@macro\endcsname\relax
  \long\def\AtBegShi@AddTo#1#2{%
    \begingroup
      \toks\ltx@zero\expandafter{#1#2}%
      \xdef#1{\the\toks\ltx@zero}%
    \endgroup
  }%
\else
  \let\AtBegShi@AddTo\g@addto@macro
\fi
%    \end{macrocode}
%    \end{macro}
%
%    \begin{macro}{\AtBegShi@AddHook}
%    \begin{macrocode}
\long\def\AtBegShi@AddHook#1#2{%
  \AtBegShi@AddTo#1{\AtBegShi@Item{#2}}%
}
%    \end{macrocode}
%    \end{macro}
%
%    \begin{macro}{\AtBegShi@Item}
%    \begin{macrocode}
\long\def\AtBegShi@Item#1{%
  \ifAtBegShi@Discarded
  \else
    #1%
    \ifAtBegShi@Discarded
    \else
      \ifvoid\AtBeginShipoutBox
        \@PackageWarning{atbegshi}{%
          Shipout box was voided by hook,\MessageBreak
          ignoring shipout box%
        }%
        \AtBeginShipoutDiscard
      \fi
    \fi
  \fi
}
%    \end{macrocode}
%    \end{macro}
%
%    \begin{macro}{\AtBeginShipoutInit}
%    \begin{macrocode}
\AtBegShi@CheckDefinable\AtBeginShipoutInit
\def\AtBeginShipoutInit{%
  \ltx@IfUndefined{newbox}{%
    \@PackageError{atbegshi}{%
      \string\AtBeginShipoutInit\space failed\MessageBreak
       because of missing \expandafter\string\csname newbox\endcsname
    }\@ehc
  }{%
    \csname newbox\endcsname\AtBeginShipoutBox
    \AtBegShi@CheckDefinable\AtBeginShipoutOriginalShipout
    \global\let\AtBeginShipoutOriginalShipout\shipout
    \global\let\shipout\AtBegShi@Shipout
  }%
  \gdef\AtBeginShipoutInit{}%
}
%    \end{macrocode}
%    \end{macro}
%    \begin{macrocode}
\begingroup\expandafter\expandafter\expandafter\endgroup
\expandafter\ifx\csname AtBeginDocument\endcsname\relax
  \AtBeginShipoutInit
\else
  \AtBeginDocument{\AtBeginShipoutInit}%
\fi
%    \end{macrocode}
%
% \subsection{Additions to the shipout box}
%
%    \begin{macro}{\AtBeginShipoutAddToBox}
%    \begin{macrocode}
\def\AtBeginShipoutAddToBox#1{%
  \ifhbox\AtBeginShipoutBox
    \edef\AtBegShi@restore{%
      \hfuzz=\the\hfuzz\relax
      \hbadness=\the\hbadness\relax
    }%
    \hfuzz=1073741823sp\relax
    \hbadness=2147483647\relax
    \setbox\AtBeginShipoutBox=\hbox to \wd\AtBeginShipoutBox{%
      \setbox\ltx@zero=\hbox{%
        \begingroup
          \AtBegShi@restore
          #1%
        \endgroup
      }%
      \wd\ltx@zero=0pt\relax
      \ht\ltx@zero=0pt\relax
      \dp\ltx@zero=0pt\relax
      \raise\ht\AtBeginShipoutBox\copy\ltx@zero
      \unhcopy\AtBeginShipoutBox
    }%
    \AtBegShi@restore
  \else
    \ifvbox\AtBeginShipoutBox
      \edef\AtBegShi@restore{%
        \vfuzz=\the\vfuzz\relax
        \vbadness=\the\vbadness\relax
        \dimen\ltx@zero=\the\dimen\ltx@zero\relax
      }%
      \edef\AtBegShi@restorebox{%
        \ht\AtBeginShipoutBox=\the\ht\AtBeginShipoutBox\relax
        \dp\AtBeginShipoutBox=\the\dp\AtBeginShipoutBox\relax
      }%
      \vfuzz=1073741823sp\relax
      \vbadness=2147483647\relax
      \dimen\ltx@zero=\ht\AtBeginShipoutBox
      \advance\dimen\ltx@zero by \dp\AtBeginShipoutBox
      \setbox\AtBeginShipoutBox=\vbox to \dimen\ltx@zero{%
        \setbox\ltx@zero=\hbox{%
          \begingroup
            \AtBegShi@restore
            #1%
          \endgroup
        }%
        \wd\ltx@zero=0pt\relax
        \ht\ltx@zero=0pt\relax
        \dp\ltx@zero=0pt\relax
        \baselineskip=0pt\relax
        \lineskip=0pt\relax
        \lineskiplimit=0pt\relax
        \copy\ltx@zero
        \unvbox\AtBeginShipoutBox
        \kern0pt%
      }%
      \AtBegShi@restore
      \AtBegShi@restorebox
    \fi
  \fi
}
%    \end{macrocode}
%    \end{macro}
%
%    \begin{macro}{\AtBeginShipoutAddToBoxForeground}
%    \begin{macrocode}
\def\AtBeginShipoutAddToBoxForeground#1{%
  \ifhbox\AtBeginShipoutBox
    \edef\AtBegShi@restore{%
      \hfuzz=\the\hfuzz\relax
      \hbadness=\the\hbadness\relax
    }%
    \hfuzz=1073741823sp\relax
    \hbadness=2147483647\relax
    \setbox\AtBeginShipoutBox=\hbox to \wd\AtBeginShipoutBox{%
      \unhcopy\AtBeginShipoutBox
      \kern-\wd\AtBeginShipoutBox
      \setbox\ltx@zero=\hbox{%
        \begingroup
          \AtBegShi@restore
          #1%
        \endgroup
      }%
      \wd\ltx@zero=0pt\relax
      \ht\ltx@zero=0pt\relax
      \dp\ltx@zero=0pt\relax
      \raise\ht\AtBeginShipoutBox\copy\ltx@zero
      \kern\wd\AtBeginShipoutBox
    }%
    \AtBegShi@restore
  \else
    \ifvbox\AtBeginShipoutBox
      \edef\AtBegShi@restore{%
        \vfuzz=\the\vfuzz\relax
        \vbadness=\the\vbadness\relax
        \dimen\ltx@zero=\the\dimen\ltx@zero\relax
      }%
      \edef\AtBegShi@restorebox{%
        \ht\AtBeginShipoutBox=\the\ht\AtBeginShipoutBox\relax
        \dp\AtBeginShipoutBox=\the\dp\AtBeginShipoutBox\relax
      }%
      \vfuzz=1073741823sp\relax
      \vbadness=2147483647\relax
      \dimen\ltx@zero=\ht\AtBeginShipoutBox
      \advance\dimen\ltx@zero by \dp\AtBeginShipoutBox
      \setbox\AtBeginShipoutBox=\vbox to \dimen\ltx@zero{%
        \setbox\ltx@zero=\hbox{%
          \begingroup
            \AtBegShi@restore
            #1%
          \endgroup
        }%
        \wd\ltx@zero=0pt\relax
        \ht\ltx@zero=0pt\relax
        \dp\ltx@zero=0pt\relax
        \baselineskip=0pt\relax
        \lineskip=0pt\relax
        \lineskiplimit=0pt\relax
        \unvbox\AtBeginShipoutBox
        \kern-\dimen\ltx@zero
        \copy\ltx@zero
        \kern\dimen\ltx@zero
      }%
      \AtBegShi@restore
      \AtBegShi@restorebox
    \fi
  \fi
}
%    \end{macrocode}
%    \end{macro}
%
% \subsection{Positioning}
%
%    \begin{macrocode}
\begingroup\expandafter\expandafter\expandafter\endgroup
\expandafter\ifx\csname RequirePackage\endcsname\relax
  \def\TMP@RequirePackage#1[#2]{%
    \begingroup\expandafter\expandafter\expandafter\endgroup
    \expandafter\ifx\csname ver@#1.sty\endcsname\relax
      \input #1.sty\relax
    \fi
  }%
  \TMP@RequirePackage{ifpdf}[2011/01/30]%
\else
  \RequirePackage{ifpdf}[2011/01/30]%
\fi
%    \end{macrocode}
%
%    \begin{macrocode}
\ifpdf
  \def\AtBegShi@horigin{%
    \ifx\pdfhorigin\@undefined\pdfvariable horigin\else\pdfhorigin\fi}%
  \def\AtBegShi@vorigin{%
    \ifx\pdfvorigin\@undefined\pdfvariable vorigin\else\pdfvorigin\fi}%
\else
  \def\AtBegShi@horigin{72.27pt}%
  \def\AtBegShi@vorigin{72.27pt}%
\fi
%    \end{macrocode}
%
%    \begin{macrocode}
\begingroup
\ifcase
    \expandafter\ifx\csname picture\endcsname\relax
      1%
    \else
      \expandafter\ifx\csname endpicture\endcsname\relax
        1%
      \else
        0%
      \fi
    \fi
  \endgroup
  \def\AtBegShi@BeginPicture{%
    \begingroup
    \picture(0,0)\relax
    \begingroup\expandafter\expandafter\expandafter\endgroup
    \expandafter\ifx\csname unitlength\endcsname\relax
    \else
      \unitlength=1pt\relax
    \fi
    \ignorespaces
  }%
  \def\AtBegShi@EndPicture{%
    \endpicture
    \endgroup
  }%
\else
  \endgroup
  \def\AtBegShi@BeginPicture{%
    \setbox\ltx@zero=\hbox\bgroup
    \begingroup
    \ignorespaces
  }%
  \def\AtBegShi@EndPicture{%
    \endgroup
    \egroup
    \ht\ltx@zero=0pt\relax
    \dp\ltx@zero=0pt\relax
    \copy\ltx@zero
  }%
\fi
%    \end{macrocode}
%
%    \begin{macro}{\AtBeginShipoutUpperLeft}
%    A surrounding \cs{rlap} is not necessary, because
%    the stuff is put in an \cs{hbox} with zero width.
%    \begin{macrocode}
\def\AtBeginShipoutUpperLeft#1{%
  \AtBeginShipoutAddToBox{%
    \kern-\AtBegShi@horigin\relax
    \vbox to 0pt{%
      \kern-\AtBegShi@vorigin\relax
      \AtBegShi@BeginPicture
      #1%
      \AtBegShi@EndPicture
      \vss
    }%
  }%
}
%    \end{macrocode}
%    \end{macro}
%
%    \begin{macro}{\AtBeginShipoutUpperLeftForeground}
%    \begin{macrocode}
\def\AtBeginShipoutUpperLeftForeground#1{%
  \AtBeginShipoutAddToBoxForeground{%
    \kern-\AtBegShi@horigin\relax
    \vbox to 0pt{%
      \kern-\AtBegShi@vorigin\relax
      \AtBegShi@BeginPicture
      #1%
      \AtBegShi@EndPicture
      \vss
    }%
  }%
}
%    \end{macrocode}
%    \end{macro}
%
% \subsection{Patches}
%
%    Patches for \LaTeX\ packages that redefine \cs{shipout}.
%    \LaTeX\ is now supposed to use \eTeX. Thus we do not patch,
%    without \LaTeX\ and \eTeX.
%    \begin{macrocode}
\def\AtBegShi@AbortIfUndefined#1{%
  \begingroup\expandafter\expandafter\expandafter\endgroup
  \expandafter\ifx\csname#1\endcsname\relax
    \expandafter\AtBegShi@AtEnd
  \fi
}
\AtBegShi@AbortIfUndefined{currentgrouplevel}%
\AtBegShi@AbortIfUndefined{AtBeginDocument}%
\AtBegShi@AbortIfUndefined{@ifpackageloaded}%
\AtBegShi@AbortIfUndefined{@ifclassloaded}%
%    \end{macrocode}
%
% \subsubsection{Package \xpackage{crop}}
%
% Fix of method and box.
%    \begin{macrocode}
\def\AtBegShi@PatchCrop{%
  \begingroup
    \def\AtBegShi@Crop@shipout{%
      \afterassignment\CROP@ship
      \setbox\@cclv=%
    }%
    \def\AtBegShi@Crop@ship{%
      \ifvoid\@cclv
        \expandafter\aftergroup
      \fi
      \CROP@@ship
    }%
    \def\AtBegShi@Crop@shiplist{%
      \lineskip\z@
      \lineskiplimit\z@
      \baselineskip\z@
      \CROP@kernel
      \box\@cclv
    }%
    \def\AtBegShi@Crop@@ship{%
      \CROP@shipout\vbox{%
        \CROP@shiplist
      }%
    }%
    \ifx\AtBegShi@Crop@ship\CROP@ship
      \ifx\AtBegShi@Crop@shiplist\CROP@shiplist
        \ifx\AtBegShi@Crop@@ship\CROP@@ship
          \let\AtBegShi@found\relax
          \ifx\shipout\AtBegShi@Crop@shipout
            \def\AtBegShi@found{\shipout}%
          \else\ifx\AtBeginShipoutOriginalShipout\AtBegShi@Crop@shipout
            \def\AtBegShi@found{\AtBeginShipoutOriginalShipout}%
          \else\ifx\@EveryShipout@Org@Shipout\AtBegShi@Crop@shipout
            \def\AtBegShi@found{\@EveryShipout@Org@Shipout}%
          \else\ifx\GPTorg@shipout\AtBegShi@Crop@shipout
            \def\AtBegShi@found{\GPTorg@shipout}%
          \else\ifx\THBorg@shipout\AtBegShi@Crop@shipout
            \def\AtBegShi@found{\THBorg@shipout}%
          \else\ifx\mem@oldshipout\AtBegShi@Crop@shipout
            \def\AtBegShi@found{\mem@oldshipout}%
          \fi\fi\fi\fi\fi\fi
          \ifx\AtBegShi@found\relax
          \else
            \expandafter\endgroup
            \expandafter\def\AtBegShi@found{%
              \edef\AtBegShi@GroupLevel{\number\currentgrouplevel}%
              \afterassignment\CROP@ship
              \setbox\AtBeginShipoutBox=%
            }%
            \def\CROP@ship{%
              \ifnum\AtBegShi@GroupLevel=\currentgrouplevel
              \else
                \expandafter\aftergroup
              \fi
              \CROP@@ship
            }%
            \def\CROP@shiplist{%
              \lineskip 0pt\relax
              \lineskiplimit 0pt\relax
              \baselineskip 0pt\relax
              \CROP@kernel
              \box\AtBeginShipoutBox
            }%
            \def\CROP@@ship{%
              \ifvoid\AtBeginShipoutBox
              \else
                \setbox\AtBeginShipoutBox=\vbox{%
                  \CROP@shiplist
                }%
                \AtBegShi@GetBoxSize\AtBeginShipoutBox
                \expandafter\CROP@shipout
                \expandafter\box
                \expandafter\AtBeginShipoutBox
              \fi
            }%
            \@PackageInfoNoLine{atbegshi}{Package `crop' patched}%
            \begingroup
          \fi
        \fi
      \fi
    \fi
  \endgroup
  \let\AtBegShi@PatchCrop\relax
}
\@ifpackageloaded{crop}{%
  \AtBegShi@PatchCrop
}{%
  \AtBeginDocument{\AtBegShi@PatchCrop}%
}
%    \end{macrocode}
%
% \subsubsection{Package \xpackage{everyshi}}
%
%     Fix of method. Use of box 255 is not changed.
%    \begin{macrocode}
\def\AtBegShi@PatchEveryshi{%
  \begingroup
    \long\def\AtBegShi@Everyshi@shipout{%
      \afterassignment\@EveryShipout@Test
      \global\setbox\@cclv= %
    }%
    \long\def\AtBegShi@Everyshi@Test{%
      \ifvoid\@cclv\relax
        \aftergroup\@EveryShipout@Output
      \else
        \@EveryShipout@Output
      \fi
    }%
    \ifx\AtBegShi@Everyshi@Test\@EveryShipout@Test
      \let\AtBegShi@found\relax
      \ifx\shipout\AtBegShi@Everyshi@shipout
        \def\AtBegShi@found{\shipout}%
      \else\ifx\AtBeginShipoutOriginalShipout\AtBegShi@Everyshi@shipout
        \def\AtBegShi@found{\AtBeginShipoutOriginalShipout}%
      \else\ifx\CROP@shipout\AtBegShi@Everyshi@shipout
        \def\AtBegShi@found{\CROP@shipout}%
      \else\ifx\GPTorg@shipout\AtBegShi@Everyshi@shipout
        \def\AtBegShi@found{\GPTorg@shipout}%
      \else\ifx\THBorg@shipout\AtBegShi@Everyshi@shipout
        \def\AtBegShi@found{\THBorg@shipout}%
      \else\ifx\mem@oldshipout\AtBegShi@Everyshi@shipout
        \def\AtBegShi@found{\mem@oldshipout}%
      \else
        \expandafter\ifx\csname @EveryShipout@Org@Shipout\endcsname
                        \relax
           \ifx\@EveryShipout@Shipout\AtBegShi@Everyshi@shipout
             \def\AtBegShi@found{\@EveryShipout@Shipout}%
           \fi
        \fi
      \fi\fi\fi\fi\fi\fi
      \ifx\AtBegShi@found\relax
      \else
        \expandafter\endgroup
        \expandafter\def\AtBegShi@found{%
          \edef\AtBegShi@GroupLevel{\number\currentgrouplevel}%
          \afterassignment\@EveryShipout@Test
          \setbox\AtBeginShipoutBox=%
        }%
        \def\@EveryShipout@Test{%
          \ifnum\AtBegShi@GroupLevel=\currentgrouplevel
          \else
            \expandafter\aftergroup
          \fi
          \AtBegShi@Everyshi@Output
        }%
        \def\AtBegShi@Everyshi@Output{%
          \ifvoid\AtBeginShipoutBox
          \else
            \global\setbox\ltx@cclv\box\AtBeginShipoutBox
            \expandafter\@EveryShipout@Output
          \fi
        }%
        \@PackageInfoNoLine{atbegshi}{Package `everyshi' patched}%
        \begingroup
      \fi
    \fi
  \endgroup
  \let\AtBegShi@PatchEveryshi\relax
}
\@ifpackageloaded{everyshi}{%
  \AtBegShi@PatchEveryshi
}{%
  \AtBeginDocument{\AtBegShi@PatchEveryshi}%
}
%    \end{macrocode}
%
% \subsubsection{Class \xpackage{memoir}}
%
%     Fix of method and box.
%    \begin{macrocode}
\def\AtBegShi@PatchMemoir{%
  \begingroup
    \def\AtBegShi@Memoir@shipout{%
      \afterassignment\mem@shipi
      \setbox\@cclv=%
    }%
    \def\AtBegShi@Memoir@shipi{%
      \ifvoid\@cclv
        \expandafter\aftergroup
      \fi
      \mem@shipii
    }%
    \def\AtBegShi@Memoir@shipiiA{%
      \mem@oldshipout\vbox{%
        \trimmarks
        \unvbox\@cclv
      }%
    }%
    \def\AtBegShi@Memoir@shipiiB{%
      \ifvoid\@cclv
        \mem@oldshipout\box\@cclv
      \else
        \mem@oldshipout\vbox{%
          \trimmarks
          \unvbox\@cclv
        }%
      \fi
    }%
    \def\AtBegShi@Memoir@PatchAB{%
      \ifvoid\AtBeginShipoutBox
      \else
        \setbox\AtBeginShipoutBox=\vbox{%
          \trimmarks
          \ifvbox\AtBeginShipoutBox
            \unvbox\AtBeginShipoutBox
          \else
            \box\AtBeginShipoutBox
          \fi
        }%
        \AtBegShi@GetBoxSize\AtBeginShipoutBox
        \expandafter\mem@oldshipout
        \expandafter\box
        \expandafter\AtBeginShipoutBox
      \fi
    }%
    \def\AtBegShi@Memoir@shipiiC{% 2008/08/07 v1.6180339a
      \ifvoid\@cclv
        \mem@oldshipout\box\@cclv
      \else
        \ifshowtrims
          \mem@oldshipout\vbox{\trimmarks\unvbox\@cclv}%
        \else
          \mem@oldshipout\box\@cclv
        \fi
      \fi
    }%
    \def\AtBegShi@Memoir@shipiiD{% 2011/03/06 v3.6j
      \ifvoid\@cclv
        \mem@oldshipout\box\@cclv
      \else
        \ifshowtrims
          \mem@oldshipout\vbox{%
            \trimmarks
            \nointerlineskip
            \box\@cclv
          }%
        \else
          \mem@oldshipout\box\@cclv
        \fi
      \fi
    }%
    \def\AtBegShi@Memoir@PatchCD{%
      \ifvoid\AtBeginShipoutBox
      \else
        \ifshowtrims
          \setbox\AtBeginShipoutBox=\vbox{%
            \trimmarks
            \nointerlineskip
            \box\AtBeginShipoutBox
          }%
        \fi
        \AtBegShi@GetBoxSize\AtBeginShipoutBox
        \expandafter\mem@oldshipout
        \expandafter\box
        \expandafter\AtBeginShipoutBox
      \fi
    }%
    \ifx\AtBegShi@Memoir@shipi\mem@shipi
      \let\AtBegShi@found\ltx@one
      \ifx\AtBegShi@Memoir@shipiiA\mem@shipii
        \let\AtBegShi@found\ltx@zero
        \global\let\AtBegShi@Memoir@PatchX\AtBegShi@Memoir@PatchAB
      \else\ifx\AtBegShi@Memoir@shipiiB\mem@shipii
        \let\AtBegShi@found\ltx@zero
        \global\let\AtBegShi@Memoir@PatchX\AtBegShi@Memoir@PatchAB
      \else\ifx\AtBegShi@Memoir@shipiiC\mem@shipii
        \let\AtBegShi@found\ltx@zero
        \global\let\AtBegShi@Memoir@PatchX\AtBegShi@Memoir@PatchCD
      \else\ifx\AtBegShi@Memoir@shipiiD\mem@shipii
        \let\AtBegShi@found\ltx@zero
        \global\let\AtBegShi@Memoir@PatchX\AtBegShi@Memoir@PatchCD
      \fi\fi\fi\fi
      \ifcase\AtBegShi@found
        \let\AtBegShi@found\relax
        \ifx\shipout\AtBegShi@Memoir@shipout
          \def\AtBegShi@found{\shipout}%
        \else\ifx\AtBeginShipoutOriginalShipout\AtBegShi@Memoir@shipout
          \def\AtBegShi@found{\AtBeginShipoutOriginalShipout}%
        \else\ifx\CROP@shipout\AtBegShi@Memoir@shipout
          \def\AtBegShi@found{\CROP@shipout}%
        \else\ifx\GPTorg@shipout\AtBegShi@Memoir@shipout
          \def\AtBegShi@found{\GPTorg@shipout}%
        \else\ifx\THBorg@shipout\AtBegShi@Memoir@shipout
          \def\AtBegShi@found{\THBorg@shipout}%
        \else\ifx\@EveryShipout@Org@Shipout\AtBegShi@Memoir@shipout
          \def\AtBegShi@found{\@EveryShipout@Org@Shipout}%
        \fi\fi\fi\fi\fi\fi
        \ifx\AtBegShi@found\relax
        \else
          \expandafter\endgroup
          \expandafter\def\AtBegShi@found{%
            \edef\AtBegShi@GroupLevel{\number\currentgrouplevel}%
            \afterassignment\mem@shipi
            \setbox\AtBeginShipoutBox=%
          }%
          \def\mem@shipi{%
            \ifnum\AtBegShi@GroupLevel=\currentgrouplevel
            \else
              \expandafter\aftergroup
            \fi
            \mem@shipii
          }%
          \let\mem@shipii\AtBegShi@Memoir@PatchX
          \@PackageInfoNoLine{atbegshi}{Class `memoir' patched}%
          \begingroup
        \fi
      \fi
    \fi
  \endgroup
  \let\AtBegShi@PatchMemoir\relax
}
\@ifclassloaded{memoir}{%
  \AtBegShi@PatchMemoir
}{%
  \AtBeginDocument{\AtBegShi@PatchMemoir}%
}
%    \end{macrocode}
%
%    \begin{macrocode}
\AtBegShi@AtEnd%
%</package>
%    \end{macrocode}
%
% \section{Test}
%
% \subsection{Catcode checks for loading}
%
%    \begin{macrocode}
%<*test1>
%    \end{macrocode}
%    \begin{macrocode}
\catcode`\{=1 %
\catcode`\}=2 %
\catcode`\#=6 %
\catcode`\@=11 %
\expandafter\ifx\csname count@\endcsname\relax
  \countdef\count@=255 %
\fi
\expandafter\ifx\csname @gobble\endcsname\relax
  \long\def\@gobble#1{}%
\fi
\expandafter\ifx\csname @firstofone\endcsname\relax
  \long\def\@firstofone#1{#1}%
\fi
\expandafter\ifx\csname loop\endcsname\relax
  \expandafter\@firstofone
\else
  \expandafter\@gobble
\fi
{%
  \def\loop#1\repeat{%
    \def\body{#1}%
    \iterate
  }%
  \def\iterate{%
    \body
      \let\next\iterate
    \else
      \let\next\relax
    \fi
    \next
  }%
  \let\repeat=\fi
}%
\def\RestoreCatcodes{}
\count@=0 %
\loop
  \edef\RestoreCatcodes{%
    \RestoreCatcodes
    \catcode\the\count@=\the\catcode\count@\relax
  }%
\ifnum\count@<255 %
  \advance\count@ 1 %
\repeat

\def\RangeCatcodeInvalid#1#2{%
  \count@=#1\relax
  \loop
    \catcode\count@=15 %
  \ifnum\count@<#2\relax
    \advance\count@ 1 %
  \repeat
}
\def\RangeCatcodeCheck#1#2#3{%
  \count@=#1\relax
  \loop
    \ifnum#3=\catcode\count@
    \else
      \errmessage{%
        Character \the\count@\space
        with wrong catcode \the\catcode\count@\space
        instead of \number#3%
      }%
    \fi
  \ifnum\count@<#2\relax
    \advance\count@ 1 %
  \repeat
}
\def\space{ }
\expandafter\ifx\csname LoadCommand\endcsname\relax
  \def\LoadCommand{\input atbegshi.sty\relax}%
\fi
\def\Test{%
  \RangeCatcodeInvalid{0}{47}%
  \RangeCatcodeInvalid{58}{64}%
  \RangeCatcodeInvalid{91}{96}%
  \RangeCatcodeInvalid{123}{255}%
  \catcode`\@=12 %
  \catcode`\\=0 %
  \catcode`\%=14 %
  \LoadCommand
  \RangeCatcodeCheck{0}{36}{15}%
  \RangeCatcodeCheck{37}{37}{14}%
  \RangeCatcodeCheck{38}{47}{15}%
  \RangeCatcodeCheck{48}{57}{12}%
  \RangeCatcodeCheck{58}{63}{15}%
  \RangeCatcodeCheck{64}{64}{12}%
  \RangeCatcodeCheck{65}{90}{11}%
  \RangeCatcodeCheck{91}{91}{15}%
  \RangeCatcodeCheck{92}{92}{0}%
  \RangeCatcodeCheck{93}{96}{15}%
  \RangeCatcodeCheck{97}{122}{11}%
  \RangeCatcodeCheck{123}{255}{15}%
  \RestoreCatcodes
}
\Test
\csname @@end\endcsname
\end
%    \end{macrocode}
%    \begin{macrocode}
%</test1>
%    \end{macrocode}
%
%    \begin{macrocode}
%<*test2>
\input atbegshi.sty\relax
\def\msg#{\immediate\write16}
\msg{File: atbegshi-test2.tex 2016/06/09 v1.18 Test file for plain-TeX}
\def\testmsg#1#2{%
  \msg{}%
  \msg{*** Test with box (#1), expected page output [#2]}% hash-ok
}

\newbox\voidbox
\def\void{\box\voidbox}
\begingroup
  \setbox\voidbox=\void
\endgroup

\count0=0\relax
\AtBeginShipout{%
  \global\advance\count0 by 1\relax
  \msg{* Inside \string\AtBeginShipout: [\the\count0]}%
}

\AtBeginShipoutFirst{%
  \msg{* Inside \string\AtBeginShipoutFirst}%
  Hello World%
}

\testmsg{\string\null}{1}
\shipout\null

\AtBeginShipoutFirst{%
  This is too late%
}

\testmsg{void}{}
\shipout\void

\testmsg{\string\copy255 (not void)}{2}
\setbox255\hbox{\vrule height 10bp width 10bp}
\shipout\copy255 %

\testmsg{\string\copy255 (again)}{3}
\shipout\copy255 %

\testmsg{\string\box255}{4}
\shipout\box255 %

\testmsg{\string\box255 (again)}{}
\shipout\box255 %

\testmsg{\string\hbox}{5}
\shipout\hbox{\vrule height 5bp width 20bp}

\testmsg{\string\vbox}{6}
\shipout\vbox{\hrule height 20bp width 5bp}

\testmsg{\string\null, voided by hook}{}
\def\VoidBox{%
  \begingroup
    \setbox\AtBeginShipoutBox=\box\AtBeginShipoutBox
  \endgroup
}
\AtBeginShipout{\VoidBox}
\shipout\null
\def\VoidBox{}

\msg{*** \string\begingroup}
\begingroup
  \testmsg{void}{}%
  \shipout\void
\msg{*** \string\endgroup}
\endgroup

\msg{*** \string\begingroup}
\begingroup
  \testmsg{void}{}%
  \shipout\void
  \testmsg{\string\null}{8}%
  \shipout\null
\msg{*** \string\endgroup}
\endgroup

\testmsg{output routine}{9}
Hello World
\vfill
\eject

\testmsg{\string\null\space(discarded)}{}
\AtBeginShipout{%
  \msg{* Inside \string\AtBeginShipout: DISCARD}%
  \AtBeginShipoutDiscard
}
\shipout\null

\end
%</test2>
%    \end{macrocode}
%
%    \begin{macrocode}
%<*test3>
\NeedsTeXFormat{LaTeX2e}
\ProvidesFile{atbegshi-test3.tex}[2016/06/09 v1.18 Test file for LaTeX]
\RequirePackage{color}
\pagecolor{yellow}
\documentclass[a5paper,showtrims]{memoir}
\usepackage{atbegshi}
\AtBeginShipout{%
  \setbox\AtBeginShipoutBox=\vbox{%
    \vbox to 0pt{%
      \kern-1.5in %
      \hbox to 0pt{%
        \kern-1.5in %
        \color{blue}%
        \rule{1in}{1in}%
        \hss
      }%
      \vss
    }%
    \hrule
    \hbox{\vrule\box\AtBeginShipoutBox\vrule}%
    \hrule
  }%
}
\usepackage{eso-pic}
\makeatletter
\@ifundefined{@EveryShipout@Init}{%
  \typeout{Test skipped}%
  \@@end
}{}
\@EveryShipout@Init
\let\@EveryShipout@Init\relax
\makeatother
\AddToShipoutPicture{%
  \hspace{.52\paperwidth}%
  \colorbox{cyan}{%
    \rule{0mm}{\paperheight}%
    \hspace{.48\paperwidth}%
  }%
}
%    \end{macrocode}
%    Newer versions of class \xclass{memoir} emulate package
%    \xpackage{crop} and prevents its loading. This is undone
%    in next line for this test file.
%    \begin{macrocode}
\expandafter\let\csname ver@crop.sty\endcsname\relax
\usepackage[color=red,cross,a4,center]{crop}
\begin{document}
\shipout\null
\shipout\box\csname voidb@x\endcsname
\section{Hello World}
\end{document}
%</test3>
%    \end{macrocode}
%
% \section{Installation}
%
% \subsection{Download}
%
% \paragraph{Package.} This package is available on
% CTAN\footnote{\url{http://ctan.org/pkg/atbegshi}}:
% \begin{description}
% \item[\CTAN{macros/latex/contrib/oberdiek/atbegshi.dtx}] The source file.
% \item[\CTAN{macros/latex/contrib/oberdiek/atbegshi.pdf}] Documentation.
% \end{description}
%
%
% \paragraph{Bundle.} All the packages of the bundle `oberdiek'
% are also available in a TDS compliant ZIP archive. There
% the packages are already unpacked and the documentation files
% are generated. The files and directories obey the TDS standard.
% \begin{description}
% \item[\CTAN{install/macros/latex/contrib/oberdiek.tds.zip}]
% \end{description}
% \emph{TDS} refers to the standard ``A Directory Structure
% for \TeX\ Files'' (\CTAN{tds/tds.pdf}). Directories
% with \xfile{texmf} in their name are usually organized this way.
%
% \subsection{Bundle installation}
%
% \paragraph{Unpacking.} Unpack the \xfile{oberdiek.tds.zip} in the
% TDS tree (also known as \xfile{texmf} tree) of your choice.
% Example (linux):
% \begin{quote}
%   |unzip oberdiek.tds.zip -d ~/texmf|
% \end{quote}
%
% \paragraph{Script installation.}
% Check the directory \xfile{TDS:scripts/oberdiek/} for
% scripts that need further installation steps.
% Package \xpackage{attachfile2} comes with the Perl script
% \xfile{pdfatfi.pl} that should be installed in such a way
% that it can be called as \texttt{pdfatfi}.
% Example (linux):
% \begin{quote}
%   |chmod +x scripts/oberdiek/pdfatfi.pl|\\
%   |cp scripts/oberdiek/pdfatfi.pl /usr/local/bin/|
% \end{quote}
%
% \subsection{Package installation}
%
% \paragraph{Unpacking.} The \xfile{.dtx} file is a self-extracting
% \docstrip\ archive. The files are extracted by running the
% \xfile{.dtx} through \plainTeX:
% \begin{quote}
%   \verb|tex atbegshi.dtx|
% \end{quote}
%
% \paragraph{TDS.} Now the different files must be moved into
% the different directories in your installation TDS tree
% (also known as \xfile{texmf} tree):
% \begin{quote}
% \def\t{^^A
% \begin{tabular}{@{}>{\ttfamily}l@{ $\rightarrow$ }>{\ttfamily}l@{}}
%   atbegshi.sty & tex/generic/oberdiek/atbegshi.sty\\
%   atbegshi.pdf & doc/latex/oberdiek/atbegshi.pdf\\
%   atbegshi-example1.tex & doc/latex/oberdiek/atbegshi-example1.tex\\
%   atbegshi-example2.tex & doc/latex/oberdiek/atbegshi-example2.tex\\
%   test/atbegshi-test1.tex & doc/latex/oberdiek/test/atbegshi-test1.tex\\
%   test/atbegshi-test2.tex & doc/latex/oberdiek/test/atbegshi-test2.tex\\
%   test/atbegshi-test3.tex & doc/latex/oberdiek/test/atbegshi-test3.tex\\
%   atbegshi.dtx & source/latex/oberdiek/atbegshi.dtx\\
% \end{tabular}^^A
% }^^A
% \sbox0{\t}^^A
% \ifdim\wd0>\linewidth
%   \begingroup
%     \advance\linewidth by\leftmargin
%     \advance\linewidth by\rightmargin
%   \edef\x{\endgroup
%     \def\noexpand\lw{\the\linewidth}^^A
%   }\x
%   \def\lwbox{^^A
%     \leavevmode
%     \hbox to \linewidth{^^A
%       \kern-\leftmargin\relax
%       \hss
%       \usebox0
%       \hss
%       \kern-\rightmargin\relax
%     }^^A
%   }^^A
%   \ifdim\wd0>\lw
%     \sbox0{\small\t}^^A
%     \ifdim\wd0>\linewidth
%       \ifdim\wd0>\lw
%         \sbox0{\footnotesize\t}^^A
%         \ifdim\wd0>\linewidth
%           \ifdim\wd0>\lw
%             \sbox0{\scriptsize\t}^^A
%             \ifdim\wd0>\linewidth
%               \ifdim\wd0>\lw
%                 \sbox0{\tiny\t}^^A
%                 \ifdim\wd0>\linewidth
%                   \lwbox
%                 \else
%                   \usebox0
%                 \fi
%               \else
%                 \lwbox
%               \fi
%             \else
%               \usebox0
%             \fi
%           \else
%             \lwbox
%           \fi
%         \else
%           \usebox0
%         \fi
%       \else
%         \lwbox
%       \fi
%     \else
%       \usebox0
%     \fi
%   \else
%     \lwbox
%   \fi
% \else
%   \usebox0
% \fi
% \end{quote}
% If you have a \xfile{docstrip.cfg} that configures and enables \docstrip's
% TDS installing feature, then some files can already be in the right
% place, see the documentation of \docstrip.
%
% \subsection{Refresh file name databases}
%
% If your \TeX~distribution
% (\teTeX, \mikTeX, \dots) relies on file name databases, you must refresh
% these. For example, \teTeX\ users run \verb|texhash| or
% \verb|mktexlsr|.
%
% \subsection{Some details for the interested}
%
% \paragraph{Attached source.}
%
% The PDF documentation on CTAN also includes the
% \xfile{.dtx} source file. It can be extracted by
% AcrobatReader 6 or higher. Another option is \textsf{pdftk},
% e.g. unpack the file into the current directory:
% \begin{quote}
%   \verb|pdftk atbegshi.pdf unpack_files output .|
% \end{quote}
%
% \paragraph{Unpacking with \LaTeX.}
% The \xfile{.dtx} chooses its action depending on the format:
% \begin{description}
% \item[\plainTeX:] Run \docstrip\ and extract the files.
% \item[\LaTeX:] Generate the documentation.
% \end{description}
% If you insist on using \LaTeX\ for \docstrip\ (really,
% \docstrip\ does not need \LaTeX), then inform the autodetect routine
% about your intention:
% \begin{quote}
%   \verb|latex \let\install=y% \iffalse meta-comment
%
% File: atbegshi.dtx
% Version: 2016/06/09 v1.18
% Info: At begin shipout hook
%
% Copyright (C) 2007-2011 by
%    Heiko Oberdiek <heiko.oberdiek at googlemail.com>
%    2016
%    https://github.com/ho-tex/oberdiek/issues
%
% This work may be distributed and/or modified under the
% conditions of the LaTeX Project Public License, either
% version 1.3c of this license or (at your option) any later
% version. This version of this license is in
%    http://www.latex-project.org/lppl/lppl-1-3c.txt
% and the latest version of this license is in
%    http://www.latex-project.org/lppl.txt
% and version 1.3 or later is part of all distributions of
% LaTeX version 2005/12/01 or later.
%
% This work has the LPPL maintenance status "maintained".
%
% This Current Maintainer of this work is Heiko Oberdiek.
%
% The Base Interpreter refers to any `TeX-Format',
% because some files are installed in TDS:tex/generic//.
%
% This work consists of the main source file atbegshi.dtx
% and the derived files
%    atbegshi.sty, atbegshi.pdf, atbegshi.ins, atbegshi.drv,
%    atbegshi-example1.tex, atbegshi-example2.tex,
%    atbegshi-test1.tex, atbegshi-test2.tex, atbegshi-test3.tex.
%
% Distribution:
%    CTAN:macros/latex/contrib/oberdiek/atbegshi.dtx
%    CTAN:macros/latex/contrib/oberdiek/atbegshi.pdf
%
% Unpacking:
%    (a) If atbegshi.ins is present:
%           tex atbegshi.ins
%    (b) Without atbegshi.ins:
%           tex atbegshi.dtx
%    (c) If you insist on using LaTeX
%           latex \let\install=y\input{atbegshi.dtx}
%        (quote the arguments according to the demands of your shell)
%
% Documentation:
%    (a) If atbegshi.drv is present:
%           latex atbegshi.drv
%    (b) Without atbegshi.drv:
%           latex atbegshi.dtx; ...
%    The class ltxdoc loads the configuration file ltxdoc.cfg
%    if available. Here you can specify further options, e.g.
%    use A4 as paper format:
%       \PassOptionsToClass{a4paper}{article}
%
%    Programm calls to get the documentation (example):
%       pdflatex atbegshi.dtx
%       makeindex -s gind.ist atbegshi.idx
%       pdflatex atbegshi.dtx
%       makeindex -s gind.ist atbegshi.idx
%       pdflatex atbegshi.dtx
%
% Installation:
%    TDS:tex/generic/oberdiek/atbegshi.sty
%    TDS:doc/latex/oberdiek/atbegshi.pdf
%    TDS:doc/latex/oberdiek/atbegshi-example1.tex
%    TDS:doc/latex/oberdiek/atbegshi-example2.tex
%    TDS:doc/latex/oberdiek/test/atbegshi-test1.tex
%    TDS:doc/latex/oberdiek/test/atbegshi-test2.tex
%    TDS:doc/latex/oberdiek/test/atbegshi-test3.tex
%    TDS:source/latex/oberdiek/atbegshi.dtx
%
%<*ignore>
\begingroup
  \catcode123=1 %
  \catcode125=2 %
  \def\x{LaTeX2e}%
\expandafter\endgroup
\ifcase 0\ifx\install y1\fi\expandafter
         \ifx\csname processbatchFile\endcsname\relax\else1\fi
         \ifx\fmtname\x\else 1\fi\relax
\else\csname fi\endcsname
%</ignore>
%<*install>
\input docstrip.tex
\Msg{************************************************************************}
\Msg{* Installation}
\Msg{* Package: atbegshi 2016/06/09 v1.18 At begin shipout hook (HO)}
\Msg{************************************************************************}

\keepsilent
\askforoverwritefalse

\let\MetaPrefix\relax
\preamble

This is a generated file.

Project: atbegshi
Version: 2016/06/09 v1.18

Copyright (C) 2007-2011 by
   Heiko Oberdiek <heiko.oberdiek at googlemail.com>

This work may be distributed and/or modified under the
conditions of the LaTeX Project Public License, either
version 1.3c of this license or (at your option) any later
version. This version of this license is in
   http://www.latex-project.org/lppl/lppl-1-3c.txt
and the latest version of this license is in
   http://www.latex-project.org/lppl.txt
and version 1.3 or later is part of all distributions of
LaTeX version 2005/12/01 or later.

This work has the LPPL maintenance status "maintained".

This Current Maintainer of this work is Heiko Oberdiek.

The Base Interpreter refers to any `TeX-Format',
because some files are installed in TDS:tex/generic//.

This work consists of the main source file atbegshi.dtx
and the derived files
   atbegshi.sty, atbegshi.pdf, atbegshi.ins, atbegshi.drv,
   atbegshi-example1.tex, atbegshi-example2.tex,
   atbegshi-test1.tex, atbegshi-test2.tex, atbegshi-test3.tex.

\endpreamble
\let\MetaPrefix\DoubleperCent

\generate{%
  \file{atbegshi.ins}{\from{atbegshi.dtx}{install}}%
  \file{atbegshi.drv}{\from{atbegshi.dtx}{driver}}%
  \usedir{tex/generic/oberdiek}%
  \file{atbegshi.sty}{\from{atbegshi.dtx}{package}}%
  \usedir{doc/latex/oberdiek}%
  \file{atbegshi-example1.tex}{\from{atbegshi.dtx}{example1}}%
  \file{atbegshi-example2.tex}{\from{atbegshi.dtx}{example2}}%
  \usedir{doc/latex/oberdiek/test}%
  \file{atbegshi-test1.tex}{\from{atbegshi.dtx}{test1}}%
  \file{atbegshi-test2.tex}{\from{atbegshi.dtx}{test2}}%
  \file{atbegshi-test3.tex}{\from{atbegshi.dtx}{test3}}%
  \nopreamble
  \nopostamble
  \usedir{source/latex/oberdiek/catalogue}%
  \file{atbegshi.xml}{\from{atbegshi.dtx}{catalogue}}%
}

\catcode32=13\relax% active space
\let =\space%
\Msg{************************************************************************}
\Msg{*}
\Msg{* To finish the installation you have to move the following}
\Msg{* file into a directory searched by TeX:}
\Msg{*}
\Msg{*     atbegshi.sty}
\Msg{*}
\Msg{* To produce the documentation run the file `atbegshi.drv'}
\Msg{* through LaTeX.}
\Msg{*}
\Msg{* Happy TeXing!}
\Msg{*}
\Msg{************************************************************************}

\endbatchfile
%</install>
%<*ignore>
\fi
%</ignore>
%<*driver>
\NeedsTeXFormat{LaTeX2e}
\ProvidesFile{atbegshi.drv}%
  [2016/06/09 v1.18 At begin shipout hook (HO)]%
\documentclass{ltxdoc}
\usepackage{holtxdoc}[2011/11/22]
\begin{document}
  \DocInput{atbegshi.dtx}%
\end{document}
%</driver>
% \fi
%
%
% \CharacterTable
%  {Upper-case    \A\B\C\D\E\F\G\H\I\J\K\L\M\N\O\P\Q\R\S\T\U\V\W\X\Y\Z
%   Lower-case    \a\b\c\d\e\f\g\h\i\j\k\l\m\n\o\p\q\r\s\t\u\v\w\x\y\z
%   Digits        \0\1\2\3\4\5\6\7\8\9
%   Exclamation   \!     Double quote  \"     Hash (number) \#
%   Dollar        \$     Percent       \%     Ampersand     \&
%   Acute accent  \'     Left paren    \(     Right paren   \)
%   Asterisk      \*     Plus          \+     Comma         \,
%   Minus         \-     Point         \.     Solidus       \/
%   Colon         \:     Semicolon     \;     Less than     \<
%   Equals        \=     Greater than  \>     Question mark \?
%   Commercial at \@     Left bracket  \[     Backslash     \\
%   Right bracket \]     Circumflex    \^     Underscore    \_
%   Grave accent  \`     Left brace    \{     Vertical bar  \|
%   Right brace   \}     Tilde         \~}
%
% \GetFileInfo{atbegshi.drv}
%
% \title{The \xpackage{atbegshi} package}
% \date{2016/06/09 v1.18}
% \author{Heiko Oberdiek\thanks
% {Please report any issues at https://github.com/ho-tex/oberdiek/issues}\\
% \xemail{heiko.oberdiek at googlemail.com}}
%
% \maketitle
%
% \begin{abstract}
% This package is a modern reimplementation of package \xpackage{everyshi}
% without the burden of compatibility. It makes use of \eTeX's if available.
% Both \LaTeX\ and \plainTeX\ are supported.
% \end{abstract}
%
% \tableofcontents
%
% \section{Documentation}
%
% Package \xpackage{atbegshi} redefines \cs{shipout} to insert
% hooks for user code that is executed before the page is shipped out.
% The code may modify or even discard the output page.
% Three hooks are implemented:
% \begin{enumerate}
% \item A hook that is executed for every page, see\\
%       \cs{AtBeginShipout}
% \item A hook that is executed for the next page only, see\\
%       \cs{AtBeginShipoutNext}
% \item A hook that is only executed for the first page, see\\
%       \cs{AtBeginShipoutFirst}
% \end{enumerate}
% The hooks are executed in this order. The following three macros
% provide the user interface for adding code to these hooks:
%
% \begin{declcs}{AtBeginShipout} \M{code}\\
%   \cs{AtBeginShipoutBox}
% \end{declcs}
% Execute the \meta{code} for every page.
% The page contents is held in box register \cs{AtBeginShipoutBox}
% and may be modified. Use \cs{AtBeginShipoutDiscard} if you want to
% discard the page.
%
% \emph{Note:} Package \xpackage{everyshi} uses box register 255. With
% package \xpackage{atbegshi} you must use \cs{AtBeginShipoutBox} instead.
%
% If \LaTeX\ calls \cs{shipout} in \cs{@outputpage} (part of its output
% routine),
% the meaning of \cs{protect} is \cs{noexpand}. \LaTeX\ sets
% \cs{protect} to the appropriate \cs{@typeset@protect} in the box
% that is shipped out. This is too late for the hooks, they
% are called earlier in the redefined \cs{shipout}. Therefore
% package \xpackage{atbegshi} sets \cs{protect} to \cs{@typeset@protect}
% before it calls the hooks.
% (In \cs{EveryShipout} of package \xpackage{everyshi} the
% user is responsible for the correct setting of \cs{protect}.)
%
% \begin{declcs}{AtBeginShipoutNext} \M{code}
% \end{declcs}
% This reimplements package \xpackage{everyshi}'s \cs{AtNextShipout}.
% The \meta{code} is executed at shipout time of the next page only.
% It is just a convenience macro, it can be easily replaced by something
% like:
%\begin{quote}
%\begin{verbatim}
%\newcommand{\MyShipoutHook}{}%
%\AtBeginShipout{\MyShipoutHook}
%\gdef\MyShipoutHook{%
%  ... do something with next page ...
%  \gdef\MyShipoutHook{}%
% }
%\end{verbatim}
%\end{quote}
% (This can be necessary, if hook order does matter).
%
% \begin{declcs}{AtBeginShipoutFirst} \M{code}
% \end{declcs}
% This reimplements \LaTeX's \cs{AtBeginDvi}. This hook is
% usually used for \cs{special} commands that include PostScript
% header files. The \cs{code} is directly executed in a \cs{vbox}
% that is put at the beginning of the output page. Dealing with
% the output box \cs{AtBeginShipoutBox} is not necessary and
% not permitted here.
%
% \begin{declcs}{AtBeginShipoutDiscard}
% \end{declcs}
% This macro notifies package \xpackage{atbegshi} that
% the output page is discarded. The remaining hook code
% and the remaining hooks are not executed and the page is thrown away.
% Also \cs{deadcycles} is cleared to zero like an ordinary \cs{shipout}
% would do.
%
% \begin{declcs}{AtBeginShipoutInit}
% \end{declcs}
% Usually the redefinition of \cs{shipout} is delayed by
% \cs{AtBeginDocument} (if this macro exists). This can be
% too late, if other packages also redefines \cs{shipout}
% and the order does matter.
% \cs{AtBeginShipoutInit} forces the immediate redefinition
% of \cs{shipout}.
%
% \begin{declcs}{AtBeginShipoutAddToBox} \M{stuff}\\
%   \cs{AtBeginShipoutAddToBoxForeground} \M{stuff}
% \end{declcs}
% A quite common use case is the addition of \cs{special}
% or other whatsits to the page output box. Macro
% \cs{AtBeginShipoutAddToBox} puts \meta{stuff} in a
% box with zeroed dimensions. The box with the \meta{stuff}
% is put in the upper left corner of the shipout box
% \cs{AtBeginShipoutBox}. Macro \cs{AtBeginShipoutAddToBox}
% puts the \meta{stuff} in the background, the other macro
% \cs{AtBeginShipoutAddToBoxForeground} in the foreground
% after the original shipout box contents is set.
%
% A void shipout box (that means a discarded page)
% remains void that means \meta{stuff} is ignored in this case.
% The box type of \cs{AtBeginShipoutBox} is preserved.
% Also the box nesting level for the original contents
% of \cs{AtBeginShipoutBox} remains, for example, to avoid trouble
% with links across pages in case of \hologo{pdfTeX}.
%
% \begin{declcs}{AtBeginShipoutUpperLeft} \M{background material}
% \end{declcs}
% This is a macro that puts material in the background of
% box \cs{AtBeginShipoutBox}.
% The \meta{background material} is set in an \cs{hbox}, the
% reference point is the upper left corner of the output page.
% In case of \pdfTeX\ in PDF mode, the settings of
% \cs{pdfhorigin} and \cs{pdfvorigin} are respected.
%
% The macro \cs{AtBeginShipoutUpperLeft} is intended to be used
% in one of the hook setting macros, such as
% \cs{AtBeginShipout}, \cs{AtBeginShipoutFirst}, or
% \cs{AtBeginShipoutNext}.
%
% For \LaTeX\ users the \meta{background material} is set
% inside a \texttt{picture} environment:
% \begin{quote}
%   |\begin{picture}(0,0)|\\
%   \mbox{}\quad |\setlength{\unitlength}{1pt}%|\\
%   \mbox{}\quad \meta{background material}\\
%   |\end{picture}|
% \end{quote}
%
% \begin{declcs}{AtBeginShipoutUpperLeftForeground} \M{foreground material}
% \end{declcs}
% See \cs{AtBeginShipoutUpperLeft}. The difference is that the material
% is put in the foreground.
%
% \begin{declcs}{AtBeginShipoutOriginalShipout} \meta{box}
% \end{declcs}
% It stores the meaning of \cs{shipout} at the time this package
% is loaded.
%
% \begin{declcs}{AtBeginShipoutBoxWidth}\\
%   \cs{AtBeginShipoutBoxHeight}\\
%   \cs{AtBeginShipoutBoxDepth}\\
% \end{declcs}
% These macros store the dimensions of the output box \cs{AtBeginShipoutBox}
% before the original shipout is called. If \cs{shipout} is not
% redefined before the package loading or the box dimensions
% are not changed by the redefined \cs{shipout}, these macros
% contain the dimensions of the shipout box. These values can be
% remembered by \cs{label} and \cs{ref}. For example, this is done
% by the package module \xpackage{zref-pagelayout} of project \xpackage{zref}.
% The dimensions of the shipout page can be used in some \hologo{TeX}
% engines (\hologo{pdfTeX} in PDF mode, \hologo{XeTeX}) to calculate
% the media size of the shipout page if \cs{pdfpagewidth} and
% \cs{pdfpageheight} are not set.
%
% \subsection{Examples}
%
% \subsubsection{Example: circle in background}
%
% In this example we put a circle in the background in the middle
% of the paper.
%
%    \begin{macrocode}
%<*example1>
%    \end{macrocode}
%    \begin{macrocode}
\documentclass[a4paper]{article}
\usepackage{color}
\usepackage{atbegshi}
%    \end{macrocode}
%    Package \xpackage{picture} makes life a little easier, because
%    we can now also use length specifications in |picture|'s commands.
%    \begin{macrocode}
\usepackage{picture}
%    \end{macrocode}
%    Now we draw the circle in the middle of the paper.
%    \cs{put} moves downwards, because the origin is at the top
%    of the page, not at its bottom.
%    \begin{macrocode}
\AtBeginShipout{%
  \AtBeginShipoutUpperLeft{%
    \put(0.5\paperwidth,-0.5\paperheight){\circle{10}}%
  }%
}
\begin{document}
\section{Hello World}
\newpage
\AtBeginShipoutNext{%
  \AtBeginShipoutUpperLeft{%
    \color{red}%
    \put(0,-0.5\paperheight){\line(1,0){\paperwidth}}%
    \put(0.5\paperwidth, 0){\line(0,-1){\paperheight}}%
  }%
}
Only on this page we add a red cross.
\newpage
This page has the circle only.
\par
\vspace{\fill}
The next page will be discarded.
\newpage
\AtBeginShipoutNext{%
  \AtBeginShipoutDiscard
}
This page is discarded.
\newpage
The last page.
\end{document}
%    \end{macrocode}
%    \begin{macrocode}
%</example1>
%    \end{macrocode}
%
% \subsubsection{Example: adding TrimBox for dvipdfmx}
%
% Now an example from ``real life'' follows. Someone from the mailing  list
% for dvipdfmx wants to put a TrimBox on every page. If we use
% \cs{AtBeginShipout}, we have to put the \cs{special} inside the
% box \cs{AtBeginShipoutBox} that gets shipped out.
%    \begin{macrocode}
%<*example2>
\documentclass{minimal}
\usepackage{atbegshi}
\usepackage[
  dvipdfm,
  paperwidth=630bp,
  paperheight=810bp
]{geometry}
\AtBeginShipout{%
  \setbox\AtBeginShipoutBox=\hbox{%
    \special{pdf: put @thispage <</TrimBox[9 9 621 801]>>}%
    \box\AtBeginShipoutBox
  }%
}
\begin{document}
  First page
  \newpage
  Second page
\end{document}
%</example2>
%    \end{macrocode}
% Remember, in \cs{AtBeginShipoutBoxFirst} the \cs{setbox} wrapper code
% is implicitly given and the \cs{special} is used directly.
%
% \section{Method of \cs{shipout} overloading}
%
% \subsection{\cs{shipout}}
%
% The \TeX\ primitive command \cs{shipout} takes a box specification
% and puts the box as a new page in the output file.
% There are two kinds of box specifications:
% \begin{description}
% \item[Direct boxes:]
%   They are given by \cs{hbox}, \cs{vbox}, or \cs{vtop},\\
%   e.g. |\shipout\hbox{Hello World}|.
% \item[Indirect boxes:]
%   \cs{box} or \cs{copy} references a box register by number.
%   The box register contains the contents of the box.
%
%   \emph{Note:} \cs{box} also clears the box register globally.
% \end{description}
% Then we have to differentiate between void and empty boxes:
% \begin{description}
% \item[Void:]
%   Initially or after \cs{box} there is no box in the box register.
%   In this cases the box register is not empty, but \emph{void}.
% \item[Empty:]
%   A box with empty contents, such as
%   |\hbox{}| (= \cs{null}) or |\vbox{}| is an \emph{empty \cs{hbox}}
%   or \emph{empty \cs{vbox}}. If a box register holds such a
%   box, the box still exists, therefore the box register is
%   \emph{not void}.
% \end{description}
%
% \subsection{\cs{afterassignment}}
%
% We want to overload \cs{shipout} to do something with the box.
% It is quite impossible to do this reliable by catching the box
% using macro arguments. The
% variety of box specifications is too large, Examples:
% \begin{quote}
%   |\shipout\null|\\
%   |\shipout\vbox{...}|\\
%   |\shipout\vtop\bgroup ...\egroup|\\
%   |\shipout\box255|
% \end{quote}
% Even worse, the braces don't need to be balanced:
% \begin{quote}
%   |\shipout\hbox\bgroup}|\\
%   |\shipout\vbox{\egroup|
% \end{quote}
% Happily TeX provides a reliable way via \cs{afterassignment}.
% It takes a macro name and executes it just after the assignment.
%
% Now we can redefine \cs{shipout}. The box specification that
% follows \cs{shipout} is catched by \cs{setbox}. This is an
% assignment to a box register. \cs{afterassignment}
% notifies \TeX, that we want to call \cs{@test} right after
% the assignment:
%\begin{quote}
%\begin{verbatim}
%\shipout :=
%  \afterassignment\@test
%  \setbox\mybox=
%\end{verbatim}
%\end{quote}
% We have seen different box specifications. Indirect boxes are easy
% to understand:
% \begin{quote}
%   |\shipout\box0| $\Rightarrow$ |\setbox\mybox=\box0 |\fbox{\cs{@test}}
% \end{quote}
% However direct boxes can have arbitrary contents with lots of
% other assignments. It would be quite inpredictable if \TeX\ would
% put |\@test| after the first of such an assignment or after the
% box specification if the box lacks of assignments.
% Therefore \TeX\ puts |\@test| right at the beginning of the
% box specification, e.g:
% \begin{quote}
%   |\shipout\hbox{Hello World}|\\
%   $\Rightarrow$
%   |\setbox\mybox=\hbox{|\fbox{\cs{@test}}| Hello World}|
% \end{quote}
%
% \subsection{Test for direct or indirect boxes}
%
% Now we want to execute \cs{@test}, but where are we?
% We can be after the completed box assignment, if \cs{shipout}
% was called with an indirect box. Or we are right at the beginning
% of a direct box.
%
% \subsubsection{With \eTeX}
%
% With the \eTeX's extensions the answer is very easy:
% Being inside the direct box means that we are inside a new group.
% The new primitive command \cs{currentgrouplevel} tells how
% deeply the groups are currently nested. Macro \cs{@test} just
% compares the previously stored group level with the current one:
%\begin{quote}
%\begin{verbatim}
%\shipout :=
%  \edef\saved@grouplevel{\number\currentgrouplevel}
%  \afterassignment\@test
%  \setbox\mybox=
%
%\@test :=
%  \ifnum\saved@grouplevel=\currentgrouplevel
%    % case: indirect box, the assignment is completed
%    \@output
%  \else
%    % case: direct box, we are inside the box
%    \aftergroup\@outbox
%  \fi
%\end{verbatim}
%\end{quote}
%
% \subsubsection{Without \eTeX}
%
% Life becomes complicate without \eTeX. We cannot ask the
% group level. However, if we are inside a direct box,
% the box register \cs{mybox} is not yet changed by \cs{setbox}.
% Thus we need a special initial value and compare
% it in \cs{@test} with the current value of the box.
%
% What can be used as initial value? Arbitrary box contents cannot
% be compared. \TeX\ only tells us a few properties:
% \begin{itemize}
% \item Box type:\cs{ifhbox}, \cs{ifvbox}
% \item Dimensions: \cs{wd}, \cs{ht}, \cs{dp}
% \item Voidness: \cs{ifvoid}
% \end{itemize}
% Unhappily all these qualities even combined are not sufficient
% for constructing an initial box value, because \cs{shipout} can be
% called with a box that is accidently just the same as the choosen
% initial value.
%
% Nevertheless we have two alternatives for an initial value:
% \begin{itemize}
% \item A box of some type with some funny settings that are
% unlikely to occur in real life, e.g a height of |4911sp-\maxdimen|.
% \item A void box.
% \end{itemize}
% A collision between this initial value and an indirect \cs{shipout}
% box with just the same value is possible.
% Then \cs{@test} will make a wrong decision that it is executed
% inside a direct box and delays \cs{@output} by \cs{aftergroup}.
% Thus \cs{@output} is not called at the place we want.
% In contrary, the result is an uncertainty about the place:
% \begin{itemize}
% \item \cs{shipout} is used in a group that perhaps closes some
%   pages later. A bad place for \cs{@output}.
% \item Without a surrounding group \cs{aftergroup} effectively kills
%   its argument.
% \end{itemize}
% In the first case of a box with special dimensions we can even loose
% the page. However in the case of the void box, this effect is even
% desired, because the original \cs{shipout} does not output void boxes.
% All we have to do is to ensure that our box \cs{mybox} is always
% void except for the phase when the overloaded \cs{shipout} is executed.
% And secondly we must keep this semantics of \cs{shipout} for the void case
% in our macros, namely \cs{@output}.
%\begin{quote}
%\begin{verbatim}
%\shipout :=
%  % trick to get a void box \mybox
%  \begingroup
%    \setbox\mybox=\box\mybox
%  \endgroup
%  \afterassignment\@test
%  \setbox\mybox=
%
%\@test :=
%  \ifvoid\mybox
%    \aftergroup\@output
%  \else
%    \@output
%  \fi
%\end{verbatim}
%\end{quote}
% The nasty case is |\shipout\box\voidb@x| where the indirect box
% is void and that must not generate an output page.
% If a surrounding group is missing the output is ignored
% because of \cs{aftergroup}.
% Otherwise \@output is called some time
% later when the surrounding group closes.
% But \cs{mybox} is void outside the execution phase
% of the redefined \cs{shipout}. Also \cs{@output} checks for a
% void box and cancels the page output.
% The disadvantage remains that the hook in \cs{@output} is called
% for a page that will not be output.
%
% \subsubsection{\cs{lastkern} method}
%
% At the beginning of a new box, there is no \cs{kern}, the contents
% of the box is still empty and \cs{lastkern} returns 0\,pt.
% This can be used to distinguish between direct and indirect boxes:
% We execute \cs{setbox} in a box with a preceding non-zero kern.
% After an indirect box, \cs{lastkern} sees this kern, otherwise
% it returns 0\,pt.
%\begin{quote}
%\begin{verbatim}
%\shipout :=
%  \begingroup
%    \setbox\mybox=\hbox\bgroup
%      \kern1pt
%      \afterassignment\shipout@test
%      \global\setbox\mybox=
%\@test :=
%  \ifdim\lastkern=0pt
%    % direct box
%    \aftergroup\egroup
%    \aftergroup\endgroup
%    \aftergroup\@output
%  \else
%    \egroup
%    \endgroup
%    \@output
%  \fi
%\end{verbatim}
%\end{quote}
% We have two \cs{setbox} commands. The first creates a controlled
% context box where we can safely insert a \cs{kern}. We get rid of
% this temporarily used context box by putting the local \cs{setbox}
% in a group.
%
% After the group we want to have our shipout box in \cs{mybox}.
% Therefore we use a global assignment here.
%
% \subsection{Output}
%
% With or without \eTeX\ we ensure the original behaviour of
% \cs{shipout} that void boxes do not generate output pages.
%
% Now we can place the hook \cs{@hook} for the user code that
% wants to manipulate the output box.
%\begin{quote}
%\begin{verbatim}
%\@output :=
%  \ifvoid\mybox
%    % cancel output of void box
%  \else
%    \@hook
%    \ifvoid\mybox
%      % user code in \@hook could has voided the box
%    \else
%      \original@shipout\box\mybox
%    \fi
%  \fi
%\end{verbatim}
%\end{quote}
%
% \subsection{Separate box register}
%
% So far we have said nothing about the box number of \cs{mybox}.
% The following case that outputs the same page twice
% shows that we are not free in the use
% of the box register:
% \begin{quote}
%   |\shipout\copy<num> \shipout\box<num>|
% \end{quote}
% We manipulate the box by the hook and without \eTeX\ the box
% must even be voided. However, the use case above requires that
% the box contents does not change at all. Therefore we must
% reserve a separate box register to avoid collisions with
% user box registers.
%
% \emph{Note:} Box register number 255 is special for the output
% routine, because TeX complains if this box is not voided
% by the output routine.
% However, this requirement does not apply to \cs{shipout} at all.
% In fact \cs{shipout} does not change any box register. This is
% usually done by a call of \cs{box}, but the output routine
% can do it later \emph{after} invoking of \cs{shipout}.
%
% \subsection{Summary}
%
% \subsubsection{With \eTeX}
%
% Putting the pieces together we get for \eTeX:
%\begin{quote}
%\begin{verbatim}
%\newbox\mybox
%\let\original@shipout\shipout
%
%\shipout :=
%  \edef\saved@grouplevel{\number\currentgrouplevel}
%  \afterassignment\@test
%  \setbox\mybox=
%
%\@test :=
%  \ifnum\saved@grouplevel<\currentgrouplevel
%    \expandafter\aftergroup
%  \fi
%  \@output
%
%\@output :=
%  \ifvoid\mybox
%    % cancel output of void box
%  \else
%    \@hook
%    \ifvoid\mybox
%      % user code in \@hook could have voided the box
%    \else
%      \original@shipout\box\mybox
%    \fi
%  \fi
%\end{verbatim}
%\end{quote}
%
% \subsubsection{Without \eTeX, traditional way}
%
% And for \TeX\ without \eTeX:
%\begin{quote}
%\begin{verbatim}
%\newbox\mybox
%\begingroup
%   \setbox\mybox=\box\mybox % ensure \mybox is void
%\endgroup
%\let\original@shipout\shipout
%
%\shipout :=
%  % trick to get a void box \mybox
%  \begingroup
%    \setbox\mybox=\box\mybox
%  \endgroup
%  \afterassignment\@test
%  \setbox\mybox=
%
%\@test :=
%  \ifvoid\mybox
%    \expandafter\aftergroup
%  \fi
%  \@output
%
%\@output :=
%  \ifvoid\mybox
%    % cancel output of void box
%  \else
%    \@hook
%    \ifvoid\mybox
%      % user code in \@hook could have voided the box
%    \else
%      \original@shipout\box\mybox
%    \fi
%  \fi
%\end{verbatim}
%\end{quote}
%
% \subsubsection{\cs{lastkern} method}
%
% And for \TeX\ without \eTeX\ using the \cs{lastkern} method:
%\begin{quote}
%\begin{verbatim}
%\newbox\mybox
%\let\original@shipout\shipout
%
%\shipout :=
%  \begingroup
%  \setbox\mybox=\hbox\bgroup
%  \kern1pt
%  \afterassignment\@test
%  \setbox\mybox=
%
%\@test :=
%  \ifdim\lastkern=0pt
%    \expandafter\aftergroup
%  \fi
%  \@output
%
%\@output :=
%  \egroup
%  \endgroup
%  \ifvoid\mybox
%    % cancel output of void box
%  \else
%    \@hook
%    \ifvoid\mybox
%      % user code in \@hook could have voided the box
%    \else
%      \original@shipout\box\mybox
%    \fi
%  \fi
%\end{verbatim}
%\end{quote}
%
%
% \StopEventually{
% }
%
% \section{Implementation}
%
% Package \xpackage{atbegshi} uses \eTeX's \cs{currentgrouplevel},
% if it is available. Otherwise the \cs{lastkern} method is used.
%
%    \begin{macrocode}
%<*package>
%    \end{macrocode}
%
% \subsection{Reload check and package identification}
%    Reload check, especially if the package is not used with \LaTeX.
%    \begin{macrocode}
\begingroup\catcode61\catcode48\catcode32=10\relax%
  \catcode13=5 % ^^M
  \endlinechar=13 %
  \catcode35=6 % #
  \catcode39=12 % '
  \catcode44=12 % ,
  \catcode45=12 % -
  \catcode46=12 % .
  \catcode58=12 % :
  \catcode64=11 % @
  \catcode123=1 % {
  \catcode125=2 % }
  \expandafter\let\expandafter\x\csname ver@atbegshi.sty\endcsname
  \ifx\x\relax % plain-TeX, first loading
  \else
    \def\empty{}%
    \ifx\x\empty % LaTeX, first loading,
      % variable is initialized, but \ProvidesPackage not yet seen
    \else
      \expandafter\ifx\csname PackageInfo\endcsname\relax
        \def\x#1#2{%
          \immediate\write-1{Package #1 Info: #2.}%
        }%
      \else
        \def\x#1#2{\PackageInfo{#1}{#2, stopped}}%
      \fi
      \x{atbegshi}{The package is already loaded}%
      \aftergroup\endinput
    \fi
  \fi
\endgroup%
%    \end{macrocode}
%    Package identification:
%    \begin{macrocode}
\begingroup\catcode61\catcode48\catcode32=10\relax%
  \catcode13=5 % ^^M
  \endlinechar=13 %
  \catcode35=6 % #
  \catcode39=12 % '
  \catcode40=12 % (
  \catcode41=12 % )
  \catcode44=12 % ,
  \catcode45=12 % -
  \catcode46=12 % .
  \catcode47=12 % /
  \catcode58=12 % :
  \catcode64=11 % @
  \catcode91=12 % [
  \catcode93=12 % ]
  \catcode123=1 % {
  \catcode125=2 % }
  \expandafter\ifx\csname ProvidesPackage\endcsname\relax
    \def\x#1#2#3[#4]{\endgroup
      \immediate\write-1{Package: #3 #4}%
      \xdef#1{#4}%
    }%
  \else
    \def\x#1#2[#3]{\endgroup
      #2[{#3}]%
      \ifx#1\@undefined
        \xdef#1{#3}%
      \fi
      \ifx#1\relax
        \xdef#1{#3}%
      \fi
    }%
  \fi
\expandafter\x\csname ver@atbegshi.sty\endcsname
\ProvidesPackage{atbegshi}%
  [2016/06/09 v1.18 At begin shipout hook (HO)]%
%    \end{macrocode}
%
% \subsection{Catcodes}
%
%    \begin{macrocode}
\begingroup\catcode61\catcode48\catcode32=10\relax%
  \catcode13=5 % ^^M
  \endlinechar=13 %
  \catcode123=1 % {
  \catcode125=2 % }
  \catcode64=11 % @
  \def\x{\endgroup
    \expandafter\edef\csname AtBegShi@AtEnd\endcsname{%
      \endlinechar=\the\endlinechar\relax
      \catcode13=\the\catcode13\relax
      \catcode32=\the\catcode32\relax
      \catcode35=\the\catcode35\relax
      \catcode61=\the\catcode61\relax
      \catcode64=\the\catcode64\relax
      \catcode123=\the\catcode123\relax
      \catcode125=\the\catcode125\relax
    }%
  }%
\x\catcode61\catcode48\catcode32=10\relax%
\catcode13=5 % ^^M
\endlinechar=13 %
\catcode35=6 % #
\catcode64=11 % @
\catcode123=1 % {
\catcode125=2 % }
\def\TMP@EnsureCode#1#2{%
  \edef\AtBegShi@AtEnd{%
    \AtBegShi@AtEnd
    \catcode#1=\the\catcode#1\relax
  }%
  \catcode#1=#2\relax
}
\TMP@EnsureCode{40}{12}% (
\TMP@EnsureCode{41}{12}% )
\TMP@EnsureCode{44}{12}% ,
\TMP@EnsureCode{45}{12}% -
\TMP@EnsureCode{47}{12}% /
\TMP@EnsureCode{46}{12}% .
\TMP@EnsureCode{58}{12}% :
\TMP@EnsureCode{91}{12}% [
\TMP@EnsureCode{93}{12}% ]
\TMP@EnsureCode{94}{7}% ^(superscript)
\TMP@EnsureCode{96}{12}% `
\edef\AtBegShi@AtEnd{\AtBegShi@AtEnd\noexpand\endinput}
%    \end{macrocode}
%
% \subsection{Preparations}
%
%    \begin{macrocode}
\begingroup\expandafter\expandafter\expandafter\endgroup
\expandafter\ifx\csname RequirePackage\endcsname\relax
  \def\TMP@RequirePackage#1[#2]{%
    \begingroup\expandafter\expandafter\expandafter\endgroup
    \expandafter\ifx\csname ver@#1.sty\endcsname\relax
      \input #1.sty\relax
    \fi
  }%
  \TMP@RequirePackage{infwarerr}[2007/09/09]%
  \TMP@RequirePackage{ltxcmds}[2010/03/01]%
\else
  \RequirePackage{infwarerr}[2007/09/09]%
  \RequirePackage{ltxcmds}[2010/03/01]%
\fi
%    \end{macrocode}
%
%    \begin{macro}{\AtBegShi@CheckDefinable}
%    \begin{macrocode}
\begingroup\expandafter\expandafter\expandafter\endgroup
\expandafter\ifx\csname @ifdefinable\endcsname\relax
  \def\AtBegShi@CheckDefinable#1{%
    \ifcase\ifx#1\relax
             \ltx@one
           \else
             \ifx#1\@undefined
               \ltx@one
             \else
               \ltx@zero
             \fi
           \fi
      \@PackageError{atbegshi}{%
        \string#1\space is already defined%
      }\@ehd
    \fi
  }%
\else
  \def\AtBegShi@CheckDefinable#1{%
    \@ifdefinable{#1}{}%
  }%
\fi
%    \end{macrocode}
%    \end{macro}
%
%    \begin{macro}{\ifAtBegShi@Discarded}
%    \begin{macrocode}
\ltx@newif\ifAtBegShi@Discarded
%    \end{macrocode}
%    \end{macro}
%    \begin{macro}{\AtBeginShipoutDiscard}
%    \begin{macrocode}
\AtBegShi@CheckDefinable\AtBeginShipoutDiscard
\def\AtBeginShipoutDiscard{%
  \deadcycles=\ltx@zero
  \global\AtBegShi@Discardedtrue
}
%    \end{macrocode}
%    \end{macro}
%
%    \begin{macrocode}
\begingroup\expandafter\expandafter\expandafter\endgroup
\expandafter\ifx\csname currentgrouplevel\endcsname\relax
  \catcode`X=9  % ignore
  \catcode`E=14 % comment
\else
  \catcode`X=14 % comment
  \catcode`E=9  % ignore
\fi
%    \end{macrocode}
%
%    \begin{macro}{\AtBegShi@Shipout}
%    \begin{macrocode}
\def\AtBegShi@Shipout{%
X \begingroup
X \setbox\AtBeginShipoutBox=\hbox\bgroup
X \kern\p@
E \edef\AtBegShi@GroupLevel{\number\currentgrouplevel}%
  \afterassignment\AtBegShi@Test
X \global
  \setbox\AtBeginShipoutBox=%
}
%    \end{macrocode}
%    \end{macro}
%
%    \begin{macro}{\AtBegShi@Test}
%    \begin{macrocode}
\def\AtBegShi@Test{%
X \ifdim\lastkern=0pt %
E \ifnum\AtBegShi@GroupLevel<\currentgrouplevel
    \expandafter\aftergroup
  \fi
  \AtBegShi@Output
}
%    \end{macrocode}
%    \end{macro}
%
%    \begin{macro}{\AtBegShi@Output}
%    \begin{macrocode}
\def\AtBegShi@Output{%
X \egroup
X \endgroup
  \ifvoid\AtBeginShipoutBox
    \@PackageWarning{atbegshi}{Ignoring void shipout box}%
  \else
    \let\AtBegShi@OrgProtect\protect
    \csname set@typeset@protect\endcsname
    \global\AtBegShi@Discardedfalse
    \AtBegShi@Hook
    \expandafter\gdef\expandafter\AtBegShi@HookNext
    \expandafter{\expandafter}%
    \AtBegShi@HookNext
    \ifAtBegShi@Discarded
      \@PackageInfoNoLine{atbegshi}{Shipout page discarded}%
      \global\AtBegShi@Discardedfalse
      \begingroup
        \setbox\AtBeginShipoutBox\box\AtBeginShipoutBox
      \endgroup
      \let\protect\AtBegShi@OrgProtect
    \else
      \AtBegShi@First
      \let\protect\AtBegShi@OrgProtect
      \AtBegShi@GetBoxSize\AtBeginShipoutBox
      \ltx@ifundefined{AtNextShipout}{%
      }{%
        \AtNextShipout{\AtBegShi@GetBoxSize\@cclv}%
      }%
      \AtBeginShipoutOriginalShipout\box\AtBeginShipoutBox
    \fi
  \fi
}
%    \end{macrocode}
%    \end{macro}
%
%    \begin{macro}{\AtBegShi@GetBoxSize}
%    \begin{macrocode}
\def\AtBegShi@GetBoxSize#1{%
  \xdef\AtBeginShipoutBoxWidth{\the\wd#1}%
  \xdef\AtBeginShipoutBoxHeight{\the\ht#1}%
  \xdef\AtBeginShipoutBoxDepth{\the\dp#1}%
}
%    \end{macrocode}
%    \end{macro}
%    \begin{macro}{\AtBeginShipoutBoxWidth}
%    \begin{macrocode}
\def\AtBeginShipoutBoxWidth{0pt}
%    \end{macrocode}
%    \end{macro}
%    \begin{macro}{\AtBeginShipoutBoxHeight}
%    \begin{macrocode}
\def\AtBeginShipoutBoxHeight{0pt}
%    \end{macrocode}
%    \end{macro}
%    \begin{macro}{\AtBeginShipoutBoxDepth}
%    \begin{macrocode}
\def\AtBeginShipoutBoxDepth{0pt}
%    \end{macrocode}
%    \end{macro}
%
%    \begin{macrocode}
\catcode`\X=11 %
\catcode`\E=11 %
%    \end{macrocode}
%
%    \begin{macro}{\AtBegShi@First}
%    \begin{macrocode}
\def\AtBegShi@First{%
  \ifx\AtBegShi@HookFirst\ltx@empty
  \else
    \AtBeginShipoutAddToBox{\AtBegShi@HookFirst}%
  \fi
  \global\let\AtBegShi@First\ltx@empty
  \global\let\AtBeginShipoutFirst\AtBegShi@FirstDisabled
}
%    \end{macrocode}
%    \end{macro}
%
%    \begin{macro}{\AtBegShi@Hook}
%    \begin{macrocode}
\gdef\AtBegShi@Hook{}
%    \end{macrocode}
%    \end{macro}
%    \begin{macro}{\AtBegShi@HookNext}
%    \begin{macrocode}
\gdef\AtBegShi@HookNext{}
%    \end{macrocode}
%    \end{macro}
%    \begin{macro}{\AtBegShi@HookFirst}
%    \begin{macrocode}
\gdef\AtBegShi@HookFirst{}
%    \end{macrocode}
%    \end{macro}
%    \begin{macro}{\AtBeginShipout}
%    \begin{macrocode}
\AtBegShi@CheckDefinable\AtBeginShipout
\def\AtBeginShipout{%
  \AtBegShi@AddHook\AtBegShi@Hook
}
%    \end{macrocode}
%    \end{macro}
%    \begin{macro}{\AtBeginShipoutNext}
%    \begin{macrocode}
\AtBegShi@CheckDefinable\AtBeginShipoutNext
\def\AtBeginShipoutNext{%
  \AtBegShi@AddHook\AtBegShi@HookNext
}
%    \end{macrocode}
%    \end{macro}
%    \begin{macro}{\AtBeginShipoutFirst}
%    \begin{macrocode}
\AtBegShi@CheckDefinable\AtBeginShipoutFirst
\def\AtBeginShipoutFirst{%
  \AtBegShi@AddTo\AtBegShi@HookFirst
}
%    \end{macrocode}
%    \end{macro}
%    \begin{macro}{\AtBegShi@FirstDisabled}
%    \begin{macrocode}
\long\def\AtBegShi@FirstDisabled#1{%
  \@PackageWarning{atbegshi}{%
    First page is already shipped out, ignoring\MessageBreak
    \string\AtBeginShipoutFirst
  }%
}
%    \end{macrocode}
%    \end{macro}
%
%    \begin{macro}{\AtBegShi@AddTo}
%    \begin{macrocode}
\begingroup\expandafter\expandafter\expandafter\endgroup
\expandafter\ifx\csname g@addto@macro\endcsname\relax
  \long\def\AtBegShi@AddTo#1#2{%
    \begingroup
      \toks\ltx@zero\expandafter{#1#2}%
      \xdef#1{\the\toks\ltx@zero}%
    \endgroup
  }%
\else
  \let\AtBegShi@AddTo\g@addto@macro
\fi
%    \end{macrocode}
%    \end{macro}
%
%    \begin{macro}{\AtBegShi@AddHook}
%    \begin{macrocode}
\long\def\AtBegShi@AddHook#1#2{%
  \AtBegShi@AddTo#1{\AtBegShi@Item{#2}}%
}
%    \end{macrocode}
%    \end{macro}
%
%    \begin{macro}{\AtBegShi@Item}
%    \begin{macrocode}
\long\def\AtBegShi@Item#1{%
  \ifAtBegShi@Discarded
  \else
    #1%
    \ifAtBegShi@Discarded
    \else
      \ifvoid\AtBeginShipoutBox
        \@PackageWarning{atbegshi}{%
          Shipout box was voided by hook,\MessageBreak
          ignoring shipout box%
        }%
        \AtBeginShipoutDiscard
      \fi
    \fi
  \fi
}
%    \end{macrocode}
%    \end{macro}
%
%    \begin{macro}{\AtBeginShipoutInit}
%    \begin{macrocode}
\AtBegShi@CheckDefinable\AtBeginShipoutInit
\def\AtBeginShipoutInit{%
  \ltx@IfUndefined{newbox}{%
    \@PackageError{atbegshi}{%
      \string\AtBeginShipoutInit\space failed\MessageBreak
       because of missing \expandafter\string\csname newbox\endcsname
    }\@ehc
  }{%
    \csname newbox\endcsname\AtBeginShipoutBox
    \AtBegShi@CheckDefinable\AtBeginShipoutOriginalShipout
    \global\let\AtBeginShipoutOriginalShipout\shipout
    \global\let\shipout\AtBegShi@Shipout
  }%
  \gdef\AtBeginShipoutInit{}%
}
%    \end{macrocode}
%    \end{macro}
%    \begin{macrocode}
\begingroup\expandafter\expandafter\expandafter\endgroup
\expandafter\ifx\csname AtBeginDocument\endcsname\relax
  \AtBeginShipoutInit
\else
  \AtBeginDocument{\AtBeginShipoutInit}%
\fi
%    \end{macrocode}
%
% \subsection{Additions to the shipout box}
%
%    \begin{macro}{\AtBeginShipoutAddToBox}
%    \begin{macrocode}
\def\AtBeginShipoutAddToBox#1{%
  \ifhbox\AtBeginShipoutBox
    \edef\AtBegShi@restore{%
      \hfuzz=\the\hfuzz\relax
      \hbadness=\the\hbadness\relax
    }%
    \hfuzz=1073741823sp\relax
    \hbadness=2147483647\relax
    \setbox\AtBeginShipoutBox=\hbox to \wd\AtBeginShipoutBox{%
      \setbox\ltx@zero=\hbox{%
        \begingroup
          \AtBegShi@restore
          #1%
        \endgroup
      }%
      \wd\ltx@zero=0pt\relax
      \ht\ltx@zero=0pt\relax
      \dp\ltx@zero=0pt\relax
      \raise\ht\AtBeginShipoutBox\copy\ltx@zero
      \unhcopy\AtBeginShipoutBox
    }%
    \AtBegShi@restore
  \else
    \ifvbox\AtBeginShipoutBox
      \edef\AtBegShi@restore{%
        \vfuzz=\the\vfuzz\relax
        \vbadness=\the\vbadness\relax
        \dimen\ltx@zero=\the\dimen\ltx@zero\relax
      }%
      \edef\AtBegShi@restorebox{%
        \ht\AtBeginShipoutBox=\the\ht\AtBeginShipoutBox\relax
        \dp\AtBeginShipoutBox=\the\dp\AtBeginShipoutBox\relax
      }%
      \vfuzz=1073741823sp\relax
      \vbadness=2147483647\relax
      \dimen\ltx@zero=\ht\AtBeginShipoutBox
      \advance\dimen\ltx@zero by \dp\AtBeginShipoutBox
      \setbox\AtBeginShipoutBox=\vbox to \dimen\ltx@zero{%
        \setbox\ltx@zero=\hbox{%
          \begingroup
            \AtBegShi@restore
            #1%
          \endgroup
        }%
        \wd\ltx@zero=0pt\relax
        \ht\ltx@zero=0pt\relax
        \dp\ltx@zero=0pt\relax
        \baselineskip=0pt\relax
        \lineskip=0pt\relax
        \lineskiplimit=0pt\relax
        \copy\ltx@zero
        \unvbox\AtBeginShipoutBox
        \kern0pt%
      }%
      \AtBegShi@restore
      \AtBegShi@restorebox
    \fi
  \fi
}
%    \end{macrocode}
%    \end{macro}
%
%    \begin{macro}{\AtBeginShipoutAddToBoxForeground}
%    \begin{macrocode}
\def\AtBeginShipoutAddToBoxForeground#1{%
  \ifhbox\AtBeginShipoutBox
    \edef\AtBegShi@restore{%
      \hfuzz=\the\hfuzz\relax
      \hbadness=\the\hbadness\relax
    }%
    \hfuzz=1073741823sp\relax
    \hbadness=2147483647\relax
    \setbox\AtBeginShipoutBox=\hbox to \wd\AtBeginShipoutBox{%
      \unhcopy\AtBeginShipoutBox
      \kern-\wd\AtBeginShipoutBox
      \setbox\ltx@zero=\hbox{%
        \begingroup
          \AtBegShi@restore
          #1%
        \endgroup
      }%
      \wd\ltx@zero=0pt\relax
      \ht\ltx@zero=0pt\relax
      \dp\ltx@zero=0pt\relax
      \raise\ht\AtBeginShipoutBox\copy\ltx@zero
      \kern\wd\AtBeginShipoutBox
    }%
    \AtBegShi@restore
  \else
    \ifvbox\AtBeginShipoutBox
      \edef\AtBegShi@restore{%
        \vfuzz=\the\vfuzz\relax
        \vbadness=\the\vbadness\relax
        \dimen\ltx@zero=\the\dimen\ltx@zero\relax
      }%
      \edef\AtBegShi@restorebox{%
        \ht\AtBeginShipoutBox=\the\ht\AtBeginShipoutBox\relax
        \dp\AtBeginShipoutBox=\the\dp\AtBeginShipoutBox\relax
      }%
      \vfuzz=1073741823sp\relax
      \vbadness=2147483647\relax
      \dimen\ltx@zero=\ht\AtBeginShipoutBox
      \advance\dimen\ltx@zero by \dp\AtBeginShipoutBox
      \setbox\AtBeginShipoutBox=\vbox to \dimen\ltx@zero{%
        \setbox\ltx@zero=\hbox{%
          \begingroup
            \AtBegShi@restore
            #1%
          \endgroup
        }%
        \wd\ltx@zero=0pt\relax
        \ht\ltx@zero=0pt\relax
        \dp\ltx@zero=0pt\relax
        \baselineskip=0pt\relax
        \lineskip=0pt\relax
        \lineskiplimit=0pt\relax
        \unvbox\AtBeginShipoutBox
        \kern-\dimen\ltx@zero
        \copy\ltx@zero
        \kern\dimen\ltx@zero
      }%
      \AtBegShi@restore
      \AtBegShi@restorebox
    \fi
  \fi
}
%    \end{macrocode}
%    \end{macro}
%
% \subsection{Positioning}
%
%    \begin{macrocode}
\begingroup\expandafter\expandafter\expandafter\endgroup
\expandafter\ifx\csname RequirePackage\endcsname\relax
  \def\TMP@RequirePackage#1[#2]{%
    \begingroup\expandafter\expandafter\expandafter\endgroup
    \expandafter\ifx\csname ver@#1.sty\endcsname\relax
      \input #1.sty\relax
    \fi
  }%
  \TMP@RequirePackage{ifpdf}[2011/01/30]%
\else
  \RequirePackage{ifpdf}[2011/01/30]%
\fi
%    \end{macrocode}
%
%    \begin{macrocode}
\ifpdf
  \def\AtBegShi@horigin{%
    \ifx\pdfhorigin\@undefined\pdfvariable horigin\else\pdfhorigin\fi}%
  \def\AtBegShi@vorigin{%
    \ifx\pdfvorigin\@undefined\pdfvariable vorigin\else\pdfvorigin\fi}%
\else
  \def\AtBegShi@horigin{72.27pt}%
  \def\AtBegShi@vorigin{72.27pt}%
\fi
%    \end{macrocode}
%
%    \begin{macrocode}
\begingroup
\ifcase
    \expandafter\ifx\csname picture\endcsname\relax
      1%
    \else
      \expandafter\ifx\csname endpicture\endcsname\relax
        1%
      \else
        0%
      \fi
    \fi
  \endgroup
  \def\AtBegShi@BeginPicture{%
    \begingroup
    \picture(0,0)\relax
    \begingroup\expandafter\expandafter\expandafter\endgroup
    \expandafter\ifx\csname unitlength\endcsname\relax
    \else
      \unitlength=1pt\relax
    \fi
    \ignorespaces
  }%
  \def\AtBegShi@EndPicture{%
    \endpicture
    \endgroup
  }%
\else
  \endgroup
  \def\AtBegShi@BeginPicture{%
    \setbox\ltx@zero=\hbox\bgroup
    \begingroup
    \ignorespaces
  }%
  \def\AtBegShi@EndPicture{%
    \endgroup
    \egroup
    \ht\ltx@zero=0pt\relax
    \dp\ltx@zero=0pt\relax
    \copy\ltx@zero
  }%
\fi
%    \end{macrocode}
%
%    \begin{macro}{\AtBeginShipoutUpperLeft}
%    A surrounding \cs{rlap} is not necessary, because
%    the stuff is put in an \cs{hbox} with zero width.
%    \begin{macrocode}
\def\AtBeginShipoutUpperLeft#1{%
  \AtBeginShipoutAddToBox{%
    \kern-\AtBegShi@horigin\relax
    \vbox to 0pt{%
      \kern-\AtBegShi@vorigin\relax
      \AtBegShi@BeginPicture
      #1%
      \AtBegShi@EndPicture
      \vss
    }%
  }%
}
%    \end{macrocode}
%    \end{macro}
%
%    \begin{macro}{\AtBeginShipoutUpperLeftForeground}
%    \begin{macrocode}
\def\AtBeginShipoutUpperLeftForeground#1{%
  \AtBeginShipoutAddToBoxForeground{%
    \kern-\AtBegShi@horigin\relax
    \vbox to 0pt{%
      \kern-\AtBegShi@vorigin\relax
      \AtBegShi@BeginPicture
      #1%
      \AtBegShi@EndPicture
      \vss
    }%
  }%
}
%    \end{macrocode}
%    \end{macro}
%
% \subsection{Patches}
%
%    Patches for \LaTeX\ packages that redefine \cs{shipout}.
%    \LaTeX\ is now supposed to use \eTeX. Thus we do not patch,
%    without \LaTeX\ and \eTeX.
%    \begin{macrocode}
\def\AtBegShi@AbortIfUndefined#1{%
  \begingroup\expandafter\expandafter\expandafter\endgroup
  \expandafter\ifx\csname#1\endcsname\relax
    \expandafter\AtBegShi@AtEnd
  \fi
}
\AtBegShi@AbortIfUndefined{currentgrouplevel}%
\AtBegShi@AbortIfUndefined{AtBeginDocument}%
\AtBegShi@AbortIfUndefined{@ifpackageloaded}%
\AtBegShi@AbortIfUndefined{@ifclassloaded}%
%    \end{macrocode}
%
% \subsubsection{Package \xpackage{crop}}
%
% Fix of method and box.
%    \begin{macrocode}
\def\AtBegShi@PatchCrop{%
  \begingroup
    \def\AtBegShi@Crop@shipout{%
      \afterassignment\CROP@ship
      \setbox\@cclv=%
    }%
    \def\AtBegShi@Crop@ship{%
      \ifvoid\@cclv
        \expandafter\aftergroup
      \fi
      \CROP@@ship
    }%
    \def\AtBegShi@Crop@shiplist{%
      \lineskip\z@
      \lineskiplimit\z@
      \baselineskip\z@
      \CROP@kernel
      \box\@cclv
    }%
    \def\AtBegShi@Crop@@ship{%
      \CROP@shipout\vbox{%
        \CROP@shiplist
      }%
    }%
    \ifx\AtBegShi@Crop@ship\CROP@ship
      \ifx\AtBegShi@Crop@shiplist\CROP@shiplist
        \ifx\AtBegShi@Crop@@ship\CROP@@ship
          \let\AtBegShi@found\relax
          \ifx\shipout\AtBegShi@Crop@shipout
            \def\AtBegShi@found{\shipout}%
          \else\ifx\AtBeginShipoutOriginalShipout\AtBegShi@Crop@shipout
            \def\AtBegShi@found{\AtBeginShipoutOriginalShipout}%
          \else\ifx\@EveryShipout@Org@Shipout\AtBegShi@Crop@shipout
            \def\AtBegShi@found{\@EveryShipout@Org@Shipout}%
          \else\ifx\GPTorg@shipout\AtBegShi@Crop@shipout
            \def\AtBegShi@found{\GPTorg@shipout}%
          \else\ifx\THBorg@shipout\AtBegShi@Crop@shipout
            \def\AtBegShi@found{\THBorg@shipout}%
          \else\ifx\mem@oldshipout\AtBegShi@Crop@shipout
            \def\AtBegShi@found{\mem@oldshipout}%
          \fi\fi\fi\fi\fi\fi
          \ifx\AtBegShi@found\relax
          \else
            \expandafter\endgroup
            \expandafter\def\AtBegShi@found{%
              \edef\AtBegShi@GroupLevel{\number\currentgrouplevel}%
              \afterassignment\CROP@ship
              \setbox\AtBeginShipoutBox=%
            }%
            \def\CROP@ship{%
              \ifnum\AtBegShi@GroupLevel=\currentgrouplevel
              \else
                \expandafter\aftergroup
              \fi
              \CROP@@ship
            }%
            \def\CROP@shiplist{%
              \lineskip 0pt\relax
              \lineskiplimit 0pt\relax
              \baselineskip 0pt\relax
              \CROP@kernel
              \box\AtBeginShipoutBox
            }%
            \def\CROP@@ship{%
              \ifvoid\AtBeginShipoutBox
              \else
                \setbox\AtBeginShipoutBox=\vbox{%
                  \CROP@shiplist
                }%
                \AtBegShi@GetBoxSize\AtBeginShipoutBox
                \expandafter\CROP@shipout
                \expandafter\box
                \expandafter\AtBeginShipoutBox
              \fi
            }%
            \@PackageInfoNoLine{atbegshi}{Package `crop' patched}%
            \begingroup
          \fi
        \fi
      \fi
    \fi
  \endgroup
  \let\AtBegShi@PatchCrop\relax
}
\@ifpackageloaded{crop}{%
  \AtBegShi@PatchCrop
}{%
  \AtBeginDocument{\AtBegShi@PatchCrop}%
}
%    \end{macrocode}
%
% \subsubsection{Package \xpackage{everyshi}}
%
%     Fix of method. Use of box 255 is not changed.
%    \begin{macrocode}
\def\AtBegShi@PatchEveryshi{%
  \begingroup
    \long\def\AtBegShi@Everyshi@shipout{%
      \afterassignment\@EveryShipout@Test
      \global\setbox\@cclv= %
    }%
    \long\def\AtBegShi@Everyshi@Test{%
      \ifvoid\@cclv\relax
        \aftergroup\@EveryShipout@Output
      \else
        \@EveryShipout@Output
      \fi
    }%
    \ifx\AtBegShi@Everyshi@Test\@EveryShipout@Test
      \let\AtBegShi@found\relax
      \ifx\shipout\AtBegShi@Everyshi@shipout
        \def\AtBegShi@found{\shipout}%
      \else\ifx\AtBeginShipoutOriginalShipout\AtBegShi@Everyshi@shipout
        \def\AtBegShi@found{\AtBeginShipoutOriginalShipout}%
      \else\ifx\CROP@shipout\AtBegShi@Everyshi@shipout
        \def\AtBegShi@found{\CROP@shipout}%
      \else\ifx\GPTorg@shipout\AtBegShi@Everyshi@shipout
        \def\AtBegShi@found{\GPTorg@shipout}%
      \else\ifx\THBorg@shipout\AtBegShi@Everyshi@shipout
        \def\AtBegShi@found{\THBorg@shipout}%
      \else\ifx\mem@oldshipout\AtBegShi@Everyshi@shipout
        \def\AtBegShi@found{\mem@oldshipout}%
      \else
        \expandafter\ifx\csname @EveryShipout@Org@Shipout\endcsname
                        \relax
           \ifx\@EveryShipout@Shipout\AtBegShi@Everyshi@shipout
             \def\AtBegShi@found{\@EveryShipout@Shipout}%
           \fi
        \fi
      \fi\fi\fi\fi\fi\fi
      \ifx\AtBegShi@found\relax
      \else
        \expandafter\endgroup
        \expandafter\def\AtBegShi@found{%
          \edef\AtBegShi@GroupLevel{\number\currentgrouplevel}%
          \afterassignment\@EveryShipout@Test
          \setbox\AtBeginShipoutBox=%
        }%
        \def\@EveryShipout@Test{%
          \ifnum\AtBegShi@GroupLevel=\currentgrouplevel
          \else
            \expandafter\aftergroup
          \fi
          \AtBegShi@Everyshi@Output
        }%
        \def\AtBegShi@Everyshi@Output{%
          \ifvoid\AtBeginShipoutBox
          \else
            \global\setbox\ltx@cclv\box\AtBeginShipoutBox
            \expandafter\@EveryShipout@Output
          \fi
        }%
        \@PackageInfoNoLine{atbegshi}{Package `everyshi' patched}%
        \begingroup
      \fi
    \fi
  \endgroup
  \let\AtBegShi@PatchEveryshi\relax
}
\@ifpackageloaded{everyshi}{%
  \AtBegShi@PatchEveryshi
}{%
  \AtBeginDocument{\AtBegShi@PatchEveryshi}%
}
%    \end{macrocode}
%
% \subsubsection{Class \xpackage{memoir}}
%
%     Fix of method and box.
%    \begin{macrocode}
\def\AtBegShi@PatchMemoir{%
  \begingroup
    \def\AtBegShi@Memoir@shipout{%
      \afterassignment\mem@shipi
      \setbox\@cclv=%
    }%
    \def\AtBegShi@Memoir@shipi{%
      \ifvoid\@cclv
        \expandafter\aftergroup
      \fi
      \mem@shipii
    }%
    \def\AtBegShi@Memoir@shipiiA{%
      \mem@oldshipout\vbox{%
        \trimmarks
        \unvbox\@cclv
      }%
    }%
    \def\AtBegShi@Memoir@shipiiB{%
      \ifvoid\@cclv
        \mem@oldshipout\box\@cclv
      \else
        \mem@oldshipout\vbox{%
          \trimmarks
          \unvbox\@cclv
        }%
      \fi
    }%
    \def\AtBegShi@Memoir@PatchAB{%
      \ifvoid\AtBeginShipoutBox
      \else
        \setbox\AtBeginShipoutBox=\vbox{%
          \trimmarks
          \ifvbox\AtBeginShipoutBox
            \unvbox\AtBeginShipoutBox
          \else
            \box\AtBeginShipoutBox
          \fi
        }%
        \AtBegShi@GetBoxSize\AtBeginShipoutBox
        \expandafter\mem@oldshipout
        \expandafter\box
        \expandafter\AtBeginShipoutBox
      \fi
    }%
    \def\AtBegShi@Memoir@shipiiC{% 2008/08/07 v1.6180339a
      \ifvoid\@cclv
        \mem@oldshipout\box\@cclv
      \else
        \ifshowtrims
          \mem@oldshipout\vbox{\trimmarks\unvbox\@cclv}%
        \else
          \mem@oldshipout\box\@cclv
        \fi
      \fi
    }%
    \def\AtBegShi@Memoir@shipiiD{% 2011/03/06 v3.6j
      \ifvoid\@cclv
        \mem@oldshipout\box\@cclv
      \else
        \ifshowtrims
          \mem@oldshipout\vbox{%
            \trimmarks
            \nointerlineskip
            \box\@cclv
          }%
        \else
          \mem@oldshipout\box\@cclv
        \fi
      \fi
    }%
    \def\AtBegShi@Memoir@PatchCD{%
      \ifvoid\AtBeginShipoutBox
      \else
        \ifshowtrims
          \setbox\AtBeginShipoutBox=\vbox{%
            \trimmarks
            \nointerlineskip
            \box\AtBeginShipoutBox
          }%
        \fi
        \AtBegShi@GetBoxSize\AtBeginShipoutBox
        \expandafter\mem@oldshipout
        \expandafter\box
        \expandafter\AtBeginShipoutBox
      \fi
    }%
    \ifx\AtBegShi@Memoir@shipi\mem@shipi
      \let\AtBegShi@found\ltx@one
      \ifx\AtBegShi@Memoir@shipiiA\mem@shipii
        \let\AtBegShi@found\ltx@zero
        \global\let\AtBegShi@Memoir@PatchX\AtBegShi@Memoir@PatchAB
      \else\ifx\AtBegShi@Memoir@shipiiB\mem@shipii
        \let\AtBegShi@found\ltx@zero
        \global\let\AtBegShi@Memoir@PatchX\AtBegShi@Memoir@PatchAB
      \else\ifx\AtBegShi@Memoir@shipiiC\mem@shipii
        \let\AtBegShi@found\ltx@zero
        \global\let\AtBegShi@Memoir@PatchX\AtBegShi@Memoir@PatchCD
      \else\ifx\AtBegShi@Memoir@shipiiD\mem@shipii
        \let\AtBegShi@found\ltx@zero
        \global\let\AtBegShi@Memoir@PatchX\AtBegShi@Memoir@PatchCD
      \fi\fi\fi\fi
      \ifcase\AtBegShi@found
        \let\AtBegShi@found\relax
        \ifx\shipout\AtBegShi@Memoir@shipout
          \def\AtBegShi@found{\shipout}%
        \else\ifx\AtBeginShipoutOriginalShipout\AtBegShi@Memoir@shipout
          \def\AtBegShi@found{\AtBeginShipoutOriginalShipout}%
        \else\ifx\CROP@shipout\AtBegShi@Memoir@shipout
          \def\AtBegShi@found{\CROP@shipout}%
        \else\ifx\GPTorg@shipout\AtBegShi@Memoir@shipout
          \def\AtBegShi@found{\GPTorg@shipout}%
        \else\ifx\THBorg@shipout\AtBegShi@Memoir@shipout
          \def\AtBegShi@found{\THBorg@shipout}%
        \else\ifx\@EveryShipout@Org@Shipout\AtBegShi@Memoir@shipout
          \def\AtBegShi@found{\@EveryShipout@Org@Shipout}%
        \fi\fi\fi\fi\fi\fi
        \ifx\AtBegShi@found\relax
        \else
          \expandafter\endgroup
          \expandafter\def\AtBegShi@found{%
            \edef\AtBegShi@GroupLevel{\number\currentgrouplevel}%
            \afterassignment\mem@shipi
            \setbox\AtBeginShipoutBox=%
          }%
          \def\mem@shipi{%
            \ifnum\AtBegShi@GroupLevel=\currentgrouplevel
            \else
              \expandafter\aftergroup
            \fi
            \mem@shipii
          }%
          \let\mem@shipii\AtBegShi@Memoir@PatchX
          \@PackageInfoNoLine{atbegshi}{Class `memoir' patched}%
          \begingroup
        \fi
      \fi
    \fi
  \endgroup
  \let\AtBegShi@PatchMemoir\relax
}
\@ifclassloaded{memoir}{%
  \AtBegShi@PatchMemoir
}{%
  \AtBeginDocument{\AtBegShi@PatchMemoir}%
}
%    \end{macrocode}
%
%    \begin{macrocode}
\AtBegShi@AtEnd%
%</package>
%    \end{macrocode}
%
% \section{Test}
%
% \subsection{Catcode checks for loading}
%
%    \begin{macrocode}
%<*test1>
%    \end{macrocode}
%    \begin{macrocode}
\catcode`\{=1 %
\catcode`\}=2 %
\catcode`\#=6 %
\catcode`\@=11 %
\expandafter\ifx\csname count@\endcsname\relax
  \countdef\count@=255 %
\fi
\expandafter\ifx\csname @gobble\endcsname\relax
  \long\def\@gobble#1{}%
\fi
\expandafter\ifx\csname @firstofone\endcsname\relax
  \long\def\@firstofone#1{#1}%
\fi
\expandafter\ifx\csname loop\endcsname\relax
  \expandafter\@firstofone
\else
  \expandafter\@gobble
\fi
{%
  \def\loop#1\repeat{%
    \def\body{#1}%
    \iterate
  }%
  \def\iterate{%
    \body
      \let\next\iterate
    \else
      \let\next\relax
    \fi
    \next
  }%
  \let\repeat=\fi
}%
\def\RestoreCatcodes{}
\count@=0 %
\loop
  \edef\RestoreCatcodes{%
    \RestoreCatcodes
    \catcode\the\count@=\the\catcode\count@\relax
  }%
\ifnum\count@<255 %
  \advance\count@ 1 %
\repeat

\def\RangeCatcodeInvalid#1#2{%
  \count@=#1\relax
  \loop
    \catcode\count@=15 %
  \ifnum\count@<#2\relax
    \advance\count@ 1 %
  \repeat
}
\def\RangeCatcodeCheck#1#2#3{%
  \count@=#1\relax
  \loop
    \ifnum#3=\catcode\count@
    \else
      \errmessage{%
        Character \the\count@\space
        with wrong catcode \the\catcode\count@\space
        instead of \number#3%
      }%
    \fi
  \ifnum\count@<#2\relax
    \advance\count@ 1 %
  \repeat
}
\def\space{ }
\expandafter\ifx\csname LoadCommand\endcsname\relax
  \def\LoadCommand{\input atbegshi.sty\relax}%
\fi
\def\Test{%
  \RangeCatcodeInvalid{0}{47}%
  \RangeCatcodeInvalid{58}{64}%
  \RangeCatcodeInvalid{91}{96}%
  \RangeCatcodeInvalid{123}{255}%
  \catcode`\@=12 %
  \catcode`\\=0 %
  \catcode`\%=14 %
  \LoadCommand
  \RangeCatcodeCheck{0}{36}{15}%
  \RangeCatcodeCheck{37}{37}{14}%
  \RangeCatcodeCheck{38}{47}{15}%
  \RangeCatcodeCheck{48}{57}{12}%
  \RangeCatcodeCheck{58}{63}{15}%
  \RangeCatcodeCheck{64}{64}{12}%
  \RangeCatcodeCheck{65}{90}{11}%
  \RangeCatcodeCheck{91}{91}{15}%
  \RangeCatcodeCheck{92}{92}{0}%
  \RangeCatcodeCheck{93}{96}{15}%
  \RangeCatcodeCheck{97}{122}{11}%
  \RangeCatcodeCheck{123}{255}{15}%
  \RestoreCatcodes
}
\Test
\csname @@end\endcsname
\end
%    \end{macrocode}
%    \begin{macrocode}
%</test1>
%    \end{macrocode}
%
%    \begin{macrocode}
%<*test2>
\input atbegshi.sty\relax
\def\msg#{\immediate\write16}
\msg{File: atbegshi-test2.tex 2016/06/09 v1.18 Test file for plain-TeX}
\def\testmsg#1#2{%
  \msg{}%
  \msg{*** Test with box (#1), expected page output [#2]}% hash-ok
}

\newbox\voidbox
\def\void{\box\voidbox}
\begingroup
  \setbox\voidbox=\void
\endgroup

\count0=0\relax
\AtBeginShipout{%
  \global\advance\count0 by 1\relax
  \msg{* Inside \string\AtBeginShipout: [\the\count0]}%
}

\AtBeginShipoutFirst{%
  \msg{* Inside \string\AtBeginShipoutFirst}%
  Hello World%
}

\testmsg{\string\null}{1}
\shipout\null

\AtBeginShipoutFirst{%
  This is too late%
}

\testmsg{void}{}
\shipout\void

\testmsg{\string\copy255 (not void)}{2}
\setbox255\hbox{\vrule height 10bp width 10bp}
\shipout\copy255 %

\testmsg{\string\copy255 (again)}{3}
\shipout\copy255 %

\testmsg{\string\box255}{4}
\shipout\box255 %

\testmsg{\string\box255 (again)}{}
\shipout\box255 %

\testmsg{\string\hbox}{5}
\shipout\hbox{\vrule height 5bp width 20bp}

\testmsg{\string\vbox}{6}
\shipout\vbox{\hrule height 20bp width 5bp}

\testmsg{\string\null, voided by hook}{}
\def\VoidBox{%
  \begingroup
    \setbox\AtBeginShipoutBox=\box\AtBeginShipoutBox
  \endgroup
}
\AtBeginShipout{\VoidBox}
\shipout\null
\def\VoidBox{}

\msg{*** \string\begingroup}
\begingroup
  \testmsg{void}{}%
  \shipout\void
\msg{*** \string\endgroup}
\endgroup

\msg{*** \string\begingroup}
\begingroup
  \testmsg{void}{}%
  \shipout\void
  \testmsg{\string\null}{8}%
  \shipout\null
\msg{*** \string\endgroup}
\endgroup

\testmsg{output routine}{9}
Hello World
\vfill
\eject

\testmsg{\string\null\space(discarded)}{}
\AtBeginShipout{%
  \msg{* Inside \string\AtBeginShipout: DISCARD}%
  \AtBeginShipoutDiscard
}
\shipout\null

\end
%</test2>
%    \end{macrocode}
%
%    \begin{macrocode}
%<*test3>
\NeedsTeXFormat{LaTeX2e}
\ProvidesFile{atbegshi-test3.tex}[2016/06/09 v1.18 Test file for LaTeX]
\RequirePackage{color}
\pagecolor{yellow}
\documentclass[a5paper,showtrims]{memoir}
\usepackage{atbegshi}
\AtBeginShipout{%
  \setbox\AtBeginShipoutBox=\vbox{%
    \vbox to 0pt{%
      \kern-1.5in %
      \hbox to 0pt{%
        \kern-1.5in %
        \color{blue}%
        \rule{1in}{1in}%
        \hss
      }%
      \vss
    }%
    \hrule
    \hbox{\vrule\box\AtBeginShipoutBox\vrule}%
    \hrule
  }%
}
\usepackage{eso-pic}
\makeatletter
\@ifundefined{@EveryShipout@Init}{%
  \typeout{Test skipped}%
  \@@end
}{}
\@EveryShipout@Init
\let\@EveryShipout@Init\relax
\makeatother
\AddToShipoutPicture{%
  \hspace{.52\paperwidth}%
  \colorbox{cyan}{%
    \rule{0mm}{\paperheight}%
    \hspace{.48\paperwidth}%
  }%
}
%    \end{macrocode}
%    Newer versions of class \xclass{memoir} emulate package
%    \xpackage{crop} and prevents its loading. This is undone
%    in next line for this test file.
%    \begin{macrocode}
\expandafter\let\csname ver@crop.sty\endcsname\relax
\usepackage[color=red,cross,a4,center]{crop}
\begin{document}
\shipout\null
\shipout\box\csname voidb@x\endcsname
\section{Hello World}
\end{document}
%</test3>
%    \end{macrocode}
%
% \section{Installation}
%
% \subsection{Download}
%
% \paragraph{Package.} This package is available on
% CTAN\footnote{\url{http://ctan.org/pkg/atbegshi}}:
% \begin{description}
% \item[\CTAN{macros/latex/contrib/oberdiek/atbegshi.dtx}] The source file.
% \item[\CTAN{macros/latex/contrib/oberdiek/atbegshi.pdf}] Documentation.
% \end{description}
%
%
% \paragraph{Bundle.} All the packages of the bundle `oberdiek'
% are also available in a TDS compliant ZIP archive. There
% the packages are already unpacked and the documentation files
% are generated. The files and directories obey the TDS standard.
% \begin{description}
% \item[\CTAN{install/macros/latex/contrib/oberdiek.tds.zip}]
% \end{description}
% \emph{TDS} refers to the standard ``A Directory Structure
% for \TeX\ Files'' (\CTAN{tds/tds.pdf}). Directories
% with \xfile{texmf} in their name are usually organized this way.
%
% \subsection{Bundle installation}
%
% \paragraph{Unpacking.} Unpack the \xfile{oberdiek.tds.zip} in the
% TDS tree (also known as \xfile{texmf} tree) of your choice.
% Example (linux):
% \begin{quote}
%   |unzip oberdiek.tds.zip -d ~/texmf|
% \end{quote}
%
% \paragraph{Script installation.}
% Check the directory \xfile{TDS:scripts/oberdiek/} for
% scripts that need further installation steps.
% Package \xpackage{attachfile2} comes with the Perl script
% \xfile{pdfatfi.pl} that should be installed in such a way
% that it can be called as \texttt{pdfatfi}.
% Example (linux):
% \begin{quote}
%   |chmod +x scripts/oberdiek/pdfatfi.pl|\\
%   |cp scripts/oberdiek/pdfatfi.pl /usr/local/bin/|
% \end{quote}
%
% \subsection{Package installation}
%
% \paragraph{Unpacking.} The \xfile{.dtx} file is a self-extracting
% \docstrip\ archive. The files are extracted by running the
% \xfile{.dtx} through \plainTeX:
% \begin{quote}
%   \verb|tex atbegshi.dtx|
% \end{quote}
%
% \paragraph{TDS.} Now the different files must be moved into
% the different directories in your installation TDS tree
% (also known as \xfile{texmf} tree):
% \begin{quote}
% \def\t{^^A
% \begin{tabular}{@{}>{\ttfamily}l@{ $\rightarrow$ }>{\ttfamily}l@{}}
%   atbegshi.sty & tex/generic/oberdiek/atbegshi.sty\\
%   atbegshi.pdf & doc/latex/oberdiek/atbegshi.pdf\\
%   atbegshi-example1.tex & doc/latex/oberdiek/atbegshi-example1.tex\\
%   atbegshi-example2.tex & doc/latex/oberdiek/atbegshi-example2.tex\\
%   test/atbegshi-test1.tex & doc/latex/oberdiek/test/atbegshi-test1.tex\\
%   test/atbegshi-test2.tex & doc/latex/oberdiek/test/atbegshi-test2.tex\\
%   test/atbegshi-test3.tex & doc/latex/oberdiek/test/atbegshi-test3.tex\\
%   atbegshi.dtx & source/latex/oberdiek/atbegshi.dtx\\
% \end{tabular}^^A
% }^^A
% \sbox0{\t}^^A
% \ifdim\wd0>\linewidth
%   \begingroup
%     \advance\linewidth by\leftmargin
%     \advance\linewidth by\rightmargin
%   \edef\x{\endgroup
%     \def\noexpand\lw{\the\linewidth}^^A
%   }\x
%   \def\lwbox{^^A
%     \leavevmode
%     \hbox to \linewidth{^^A
%       \kern-\leftmargin\relax
%       \hss
%       \usebox0
%       \hss
%       \kern-\rightmargin\relax
%     }^^A
%   }^^A
%   \ifdim\wd0>\lw
%     \sbox0{\small\t}^^A
%     \ifdim\wd0>\linewidth
%       \ifdim\wd0>\lw
%         \sbox0{\footnotesize\t}^^A
%         \ifdim\wd0>\linewidth
%           \ifdim\wd0>\lw
%             \sbox0{\scriptsize\t}^^A
%             \ifdim\wd0>\linewidth
%               \ifdim\wd0>\lw
%                 \sbox0{\tiny\t}^^A
%                 \ifdim\wd0>\linewidth
%                   \lwbox
%                 \else
%                   \usebox0
%                 \fi
%               \else
%                 \lwbox
%               \fi
%             \else
%               \usebox0
%             \fi
%           \else
%             \lwbox
%           \fi
%         \else
%           \usebox0
%         \fi
%       \else
%         \lwbox
%       \fi
%     \else
%       \usebox0
%     \fi
%   \else
%     \lwbox
%   \fi
% \else
%   \usebox0
% \fi
% \end{quote}
% If you have a \xfile{docstrip.cfg} that configures and enables \docstrip's
% TDS installing feature, then some files can already be in the right
% place, see the documentation of \docstrip.
%
% \subsection{Refresh file name databases}
%
% If your \TeX~distribution
% (\teTeX, \mikTeX, \dots) relies on file name databases, you must refresh
% these. For example, \teTeX\ users run \verb|texhash| or
% \verb|mktexlsr|.
%
% \subsection{Some details for the interested}
%
% \paragraph{Attached source.}
%
% The PDF documentation on CTAN also includes the
% \xfile{.dtx} source file. It can be extracted by
% AcrobatReader 6 or higher. Another option is \textsf{pdftk},
% e.g. unpack the file into the current directory:
% \begin{quote}
%   \verb|pdftk atbegshi.pdf unpack_files output .|
% \end{quote}
%
% \paragraph{Unpacking with \LaTeX.}
% The \xfile{.dtx} chooses its action depending on the format:
% \begin{description}
% \item[\plainTeX:] Run \docstrip\ and extract the files.
% \item[\LaTeX:] Generate the documentation.
% \end{description}
% If you insist on using \LaTeX\ for \docstrip\ (really,
% \docstrip\ does not need \LaTeX), then inform the autodetect routine
% about your intention:
% \begin{quote}
%   \verb|latex \let\install=y\input{atbegshi.dtx}|
% \end{quote}
% Do not forget to quote the argument according to the demands
% of your shell.
%
% \paragraph{Generating the documentation.}
% You can use both the \xfile{.dtx} or the \xfile{.drv} to generate
% the documentation. The process can be configured by the
% configuration file \xfile{ltxdoc.cfg}. For instance, put this
% line into this file, if you want to have A4 as paper format:
% \begin{quote}
%   \verb|\PassOptionsToClass{a4paper}{article}|
% \end{quote}
% An example follows how to generate the
% documentation with pdf\LaTeX:
% \begin{quote}
%\begin{verbatim}
%pdflatex atbegshi.dtx
%makeindex -s gind.ist atbegshi.idx
%pdflatex atbegshi.dtx
%makeindex -s gind.ist atbegshi.idx
%pdflatex atbegshi.dtx
%\end{verbatim}
% \end{quote}
%
% \section{Catalogue}
%
% The following XML file can be used as source for the
% \href{http://mirror.ctan.org/help/Catalogue/catalogue.html}{\TeX\ Catalogue}.
% The elements \texttt{caption} and \texttt{description} are imported
% from the original XML file from the Catalogue.
% The name of the XML file in the Catalogue is \xfile{atbegshi.xml}.
%    \begin{macrocode}
%<*catalogue>
<?xml version='1.0' encoding='us-ascii'?>
<!DOCTYPE entry SYSTEM 'catalogue.dtd'>
<entry datestamp='$Date$' modifier='$Author$' id='atbegshi'>
  <name>atbegshi</name>
  <caption>Execute stuff at \shipout time.</caption>
  <authorref id='auth:oberdiek'/>
  <copyright owner='Heiko Oberdiek' year='2007-2011'/>
  <license type='lppl1.3'/>
  <version number='1.18'/>
  <description>
    This package is a modern reimplementation of package
    <xref refid='everyshi'>everyshi</xref>, providing various commands
    to be executed before a <tt>\shipout</tt> command.  It makes use of
    e-TeX&#x2019;s facilities if they are available.  The package may
    be used either with LaTeX or with plain TeX.
    <p/>
    The package is part of the <xref refid='oberdiek'>oberdiek</xref> bundle.
  </description>
  <documentation details='Package documentation'
      href='ctan:/macros/latex/contrib/oberdiek/atbegshi.pdf'/>
  <ctan file='true' path='/macros/latex/contrib/oberdiek/atbegshi.dtx'/>
  <miktex location='oberdiek'/>
  <texlive location='oberdiek'/>
  <install path='/macros/latex/contrib/oberdiek/oberdiek.tds.zip'/>
</entry>
%</catalogue>
%    \end{macrocode}
%
% \begin{History}
%   \begin{Version}{2007/04/17 v1.0}
%   \item
%     First version.
%   \end{Version}
%   \begin{Version}{2007/04/18 v1.1}
%   \item
%     New method based on \cs{lastkern} is used if \eTeX\ is missing.
%   \item
%     \cs{AtBeginShipoutDiscard} also resets \cs{deadcycles}.
%   \end{Version}
%   \begin{Version}{2007/04/19 v1.2}
%   \item
%     \cs{AtBeginShipoutEarly} removed for simplification reasons.
%   \item
%     Forgotten definition of \cs{AtBegShi@Info} added.
%   \item
%     Patches for packages \xpackage{crop} and \xpackage{everyshi}
%     and class \xclass{memoir} added.
%   \end{Version}
%   \begin{Version}{2007/04/26 v1.3}
%   \item
%     Use of package \xpackage{infwarerr}.
%   \item
%     Catcode section after generic header.
%   \end{Version}
%   \begin{Version}{2007/04/27 v1.4}
%   \item
%     Small optimizations.
%   \end{Version}
%   \begin{Version}{2007/06/06 v1.5}
%   \item
%     \cs{AtBeginShipoutUpperLeft} added.
%   \item
%     Example added.
%   \item
%     Fix in second test file for newer version of \xclass{memoir}.
%   \end{Version}
%   \begin{Version}{2007/09/09 v1.6}
%   \item
%     Catcode section rewritten.
%   \end{Version}
%   \begin{Version}{2008/07/18 v1.7}
%   \item
%     Documentation of \cs{AtBeginShipoutUpperLeft} fixed and extended.
%   \end{Version}
%   \begin{Version}{2008/07/19 v1.8}
%   \item
%     \cs{AtBeginShipoutUpperLeftForeground} added.
%   \end{Version}
%   \begin{Version}{2008/07/31 v1.9}
%   \item
%     Second example (TrimBox for dvipdfmx) added.
%   \item
%     No changes in package code.
%   \end{Version}
%   \begin{Version}{2009/12/02 v1.10}
%   \item
%     \cs{AtBeginShipoutOriginalShipout} added.
%   \item
%     Test file fixed.
%   \end{Version}
%   \begin{Version}{2010/03/01 v1.11}
%   \item
%     Compatibility with \iniTeX\ except for \cs{newbox}.
%   \end{Version}
%   \begin{Version}{2010/03/25 v1.12}
%   \item
%     \cs{AtBeginShipoutNext} can now be used inside
%     \cs{AtBeginShipoutNext}.
%   \end{Version}
%   \begin{Version}{2010/08/18 v1.13}
%   \item
%     Fixes for \cs{AtBegShi@CheckDefinable}.
%   \end{Version}
%   \begin{Version}{2010/12/02 v1.14}
%   \item
%     Remove the warning because of void box if the hook
%     calls \AtBeginShipoutDiscard.
%   \end{Version}
%   \begin{Version}{2011/01/30 v1.15}
%   \item
%     Already loaded package files are not input in \hologo{plainTeX}.
%   \end{Version}
%   \begin{Version}{2011/10/05 v1.16}
%   \item
%     \cs{AtBeginShipoutAddToBox}, \cs{AtBeginShipoutAddToBoxForeground} added.
%   \item
%     \cs{AtBeginShipoutBoxWidth}, \cs{AtBeginShipoutBoxHeight},
%     \cs{AtBeginShipoutBoxDepth} added.
%   \item
%     Updates for patches of class \xclass{memoir}.
%   \end{Version}
%   \begin{Version}{2016/05/16 v1.17}
%   \item
%     Documentation updates.
%   \end{Version}
%   \begin{Version}{2016/06/09 v1.18}
%   \item
%     Update for \cs{pdfhorign} in new LuaTeX.
%   \end{Version}
% \end{History}
%
% \PrintIndex
%
% \Finale
\endinput
|
% \end{quote}
% Do not forget to quote the argument according to the demands
% of your shell.
%
% \paragraph{Generating the documentation.}
% You can use both the \xfile{.dtx} or the \xfile{.drv} to generate
% the documentation. The process can be configured by the
% configuration file \xfile{ltxdoc.cfg}. For instance, put this
% line into this file, if you want to have A4 as paper format:
% \begin{quote}
%   \verb|\PassOptionsToClass{a4paper}{article}|
% \end{quote}
% An example follows how to generate the
% documentation with pdf\LaTeX:
% \begin{quote}
%\begin{verbatim}
%pdflatex atbegshi.dtx
%makeindex -s gind.ist atbegshi.idx
%pdflatex atbegshi.dtx
%makeindex -s gind.ist atbegshi.idx
%pdflatex atbegshi.dtx
%\end{verbatim}
% \end{quote}
%
% \section{Catalogue}
%
% The following XML file can be used as source for the
% \href{http://mirror.ctan.org/help/Catalogue/catalogue.html}{\TeX\ Catalogue}.
% The elements \texttt{caption} and \texttt{description} are imported
% from the original XML file from the Catalogue.
% The name of the XML file in the Catalogue is \xfile{atbegshi.xml}.
%    \begin{macrocode}
%<*catalogue>
<?xml version='1.0' encoding='us-ascii'?>
<!DOCTYPE entry SYSTEM 'catalogue.dtd'>
<entry datestamp='$Date$' modifier='$Author$' id='atbegshi'>
  <name>atbegshi</name>
  <caption>Execute stuff at \shipout time.</caption>
  <authorref id='auth:oberdiek'/>
  <copyright owner='Heiko Oberdiek' year='2007-2011'/>
  <license type='lppl1.3'/>
  <version number='1.18'/>
  <description>
    This package is a modern reimplementation of package
    <xref refid='everyshi'>everyshi</xref>, providing various commands
    to be executed before a <tt>\shipout</tt> command.  It makes use of
    e-TeX&#x2019;s facilities if they are available.  The package may
    be used either with LaTeX or with plain TeX.
    <p/>
    The package is part of the <xref refid='oberdiek'>oberdiek</xref> bundle.
  </description>
  <documentation details='Package documentation'
      href='ctan:/macros/latex/contrib/oberdiek/atbegshi.pdf'/>
  <ctan file='true' path='/macros/latex/contrib/oberdiek/atbegshi.dtx'/>
  <miktex location='oberdiek'/>
  <texlive location='oberdiek'/>
  <install path='/macros/latex/contrib/oberdiek/oberdiek.tds.zip'/>
</entry>
%</catalogue>
%    \end{macrocode}
%
% \begin{History}
%   \begin{Version}{2007/04/17 v1.0}
%   \item
%     First version.
%   \end{Version}
%   \begin{Version}{2007/04/18 v1.1}
%   \item
%     New method based on \cs{lastkern} is used if \eTeX\ is missing.
%   \item
%     \cs{AtBeginShipoutDiscard} also resets \cs{deadcycles}.
%   \end{Version}
%   \begin{Version}{2007/04/19 v1.2}
%   \item
%     \cs{AtBeginShipoutEarly} removed for simplification reasons.
%   \item
%     Forgotten definition of \cs{AtBegShi@Info} added.
%   \item
%     Patches for packages \xpackage{crop} and \xpackage{everyshi}
%     and class \xclass{memoir} added.
%   \end{Version}
%   \begin{Version}{2007/04/26 v1.3}
%   \item
%     Use of package \xpackage{infwarerr}.
%   \item
%     Catcode section after generic header.
%   \end{Version}
%   \begin{Version}{2007/04/27 v1.4}
%   \item
%     Small optimizations.
%   \end{Version}
%   \begin{Version}{2007/06/06 v1.5}
%   \item
%     \cs{AtBeginShipoutUpperLeft} added.
%   \item
%     Example added.
%   \item
%     Fix in second test file for newer version of \xclass{memoir}.
%   \end{Version}
%   \begin{Version}{2007/09/09 v1.6}
%   \item
%     Catcode section rewritten.
%   \end{Version}
%   \begin{Version}{2008/07/18 v1.7}
%   \item
%     Documentation of \cs{AtBeginShipoutUpperLeft} fixed and extended.
%   \end{Version}
%   \begin{Version}{2008/07/19 v1.8}
%   \item
%     \cs{AtBeginShipoutUpperLeftForeground} added.
%   \end{Version}
%   \begin{Version}{2008/07/31 v1.9}
%   \item
%     Second example (TrimBox for dvipdfmx) added.
%   \item
%     No changes in package code.
%   \end{Version}
%   \begin{Version}{2009/12/02 v1.10}
%   \item
%     \cs{AtBeginShipoutOriginalShipout} added.
%   \item
%     Test file fixed.
%   \end{Version}
%   \begin{Version}{2010/03/01 v1.11}
%   \item
%     Compatibility with \iniTeX\ except for \cs{newbox}.
%   \end{Version}
%   \begin{Version}{2010/03/25 v1.12}
%   \item
%     \cs{AtBeginShipoutNext} can now be used inside
%     \cs{AtBeginShipoutNext}.
%   \end{Version}
%   \begin{Version}{2010/08/18 v1.13}
%   \item
%     Fixes for \cs{AtBegShi@CheckDefinable}.
%   \end{Version}
%   \begin{Version}{2010/12/02 v1.14}
%   \item
%     Remove the warning because of void box if the hook
%     calls \AtBeginShipoutDiscard.
%   \end{Version}
%   \begin{Version}{2011/01/30 v1.15}
%   \item
%     Already loaded package files are not input in \hologo{plainTeX}.
%   \end{Version}
%   \begin{Version}{2011/10/05 v1.16}
%   \item
%     \cs{AtBeginShipoutAddToBox}, \cs{AtBeginShipoutAddToBoxForeground} added.
%   \item
%     \cs{AtBeginShipoutBoxWidth}, \cs{AtBeginShipoutBoxHeight},
%     \cs{AtBeginShipoutBoxDepth} added.
%   \item
%     Updates for patches of class \xclass{memoir}.
%   \end{Version}
%   \begin{Version}{2016/05/16 v1.17}
%   \item
%     Documentation updates.
%   \end{Version}
%   \begin{Version}{2016/06/09 v1.18}
%   \item
%     Update for \cs{pdfhorign} in new LuaTeX.
%   \end{Version}
% \end{History}
%
% \PrintIndex
%
% \Finale
\endinput
|
% \end{quote}
% Do not forget to quote the argument according to the demands
% of your shell.
%
% \paragraph{Generating the documentation.}
% You can use both the \xfile{.dtx} or the \xfile{.drv} to generate
% the documentation. The process can be configured by the
% configuration file \xfile{ltxdoc.cfg}. For instance, put this
% line into this file, if you want to have A4 as paper format:
% \begin{quote}
%   \verb|\PassOptionsToClass{a4paper}{article}|
% \end{quote}
% An example follows how to generate the
% documentation with pdf\LaTeX:
% \begin{quote}
%\begin{verbatim}
%pdflatex atbegshi.dtx
%makeindex -s gind.ist atbegshi.idx
%pdflatex atbegshi.dtx
%makeindex -s gind.ist atbegshi.idx
%pdflatex atbegshi.dtx
%\end{verbatim}
% \end{quote}
%
% \section{Catalogue}
%
% The following XML file can be used as source for the
% \href{http://mirror.ctan.org/help/Catalogue/catalogue.html}{\TeX\ Catalogue}.
% The elements \texttt{caption} and \texttt{description} are imported
% from the original XML file from the Catalogue.
% The name of the XML file in the Catalogue is \xfile{atbegshi.xml}.
%    \begin{macrocode}
%<*catalogue>
<?xml version='1.0' encoding='us-ascii'?>
<!DOCTYPE entry SYSTEM 'catalogue.dtd'>
<entry datestamp='$Date$' modifier='$Author$' id='atbegshi'>
  <name>atbegshi</name>
  <caption>Execute stuff at \shipout time.</caption>
  <authorref id='auth:oberdiek'/>
  <copyright owner='Heiko Oberdiek' year='2007-2011'/>
  <license type='lppl1.3'/>
  <version number='1.18'/>
  <description>
    This package is a modern reimplementation of package
    <xref refid='everyshi'>everyshi</xref>, providing various commands
    to be executed before a <tt>\shipout</tt> command.  It makes use of
    e-TeX&#x2019;s facilities if they are available.  The package may
    be used either with LaTeX or with plain TeX.
    <p/>
    The package is part of the <xref refid='oberdiek'>oberdiek</xref> bundle.
  </description>
  <documentation details='Package documentation'
      href='ctan:/macros/latex/contrib/oberdiek/atbegshi.pdf'/>
  <ctan file='true' path='/macros/latex/contrib/oberdiek/atbegshi.dtx'/>
  <miktex location='oberdiek'/>
  <texlive location='oberdiek'/>
  <install path='/macros/latex/contrib/oberdiek/oberdiek.tds.zip'/>
</entry>
%</catalogue>
%    \end{macrocode}
%
% \begin{History}
%   \begin{Version}{2007/04/17 v1.0}
%   \item
%     First version.
%   \end{Version}
%   \begin{Version}{2007/04/18 v1.1}
%   \item
%     New method based on \cs{lastkern} is used if \eTeX\ is missing.
%   \item
%     \cs{AtBeginShipoutDiscard} also resets \cs{deadcycles}.
%   \end{Version}
%   \begin{Version}{2007/04/19 v1.2}
%   \item
%     \cs{AtBeginShipoutEarly} removed for simplification reasons.
%   \item
%     Forgotten definition of \cs{AtBegShi@Info} added.
%   \item
%     Patches for packages \xpackage{crop} and \xpackage{everyshi}
%     and class \xclass{memoir} added.
%   \end{Version}
%   \begin{Version}{2007/04/26 v1.3}
%   \item
%     Use of package \xpackage{infwarerr}.
%   \item
%     Catcode section after generic header.
%   \end{Version}
%   \begin{Version}{2007/04/27 v1.4}
%   \item
%     Small optimizations.
%   \end{Version}
%   \begin{Version}{2007/06/06 v1.5}
%   \item
%     \cs{AtBeginShipoutUpperLeft} added.
%   \item
%     Example added.
%   \item
%     Fix in second test file for newer version of \xclass{memoir}.
%   \end{Version}
%   \begin{Version}{2007/09/09 v1.6}
%   \item
%     Catcode section rewritten.
%   \end{Version}
%   \begin{Version}{2008/07/18 v1.7}
%   \item
%     Documentation of \cs{AtBeginShipoutUpperLeft} fixed and extended.
%   \end{Version}
%   \begin{Version}{2008/07/19 v1.8}
%   \item
%     \cs{AtBeginShipoutUpperLeftForeground} added.
%   \end{Version}
%   \begin{Version}{2008/07/31 v1.9}
%   \item
%     Second example (TrimBox for dvipdfmx) added.
%   \item
%     No changes in package code.
%   \end{Version}
%   \begin{Version}{2009/12/02 v1.10}
%   \item
%     \cs{AtBeginShipoutOriginalShipout} added.
%   \item
%     Test file fixed.
%   \end{Version}
%   \begin{Version}{2010/03/01 v1.11}
%   \item
%     Compatibility with \iniTeX\ except for \cs{newbox}.
%   \end{Version}
%   \begin{Version}{2010/03/25 v1.12}
%   \item
%     \cs{AtBeginShipoutNext} can now be used inside
%     \cs{AtBeginShipoutNext}.
%   \end{Version}
%   \begin{Version}{2010/08/18 v1.13}
%   \item
%     Fixes for \cs{AtBegShi@CheckDefinable}.
%   \end{Version}
%   \begin{Version}{2010/12/02 v1.14}
%   \item
%     Remove the warning because of void box if the hook
%     calls \AtBeginShipoutDiscard.
%   \end{Version}
%   \begin{Version}{2011/01/30 v1.15}
%   \item
%     Already loaded package files are not input in \hologo{plainTeX}.
%   \end{Version}
%   \begin{Version}{2011/10/05 v1.16}
%   \item
%     \cs{AtBeginShipoutAddToBox}, \cs{AtBeginShipoutAddToBoxForeground} added.
%   \item
%     \cs{AtBeginShipoutBoxWidth}, \cs{AtBeginShipoutBoxHeight},
%     \cs{AtBeginShipoutBoxDepth} added.
%   \item
%     Updates for patches of class \xclass{memoir}.
%   \end{Version}
%   \begin{Version}{2016/05/16 v1.17}
%   \item
%     Documentation updates.
%   \end{Version}
%   \begin{Version}{2016/06/09 v1.18}
%   \item
%     Update for \cs{pdfhorign} in new LuaTeX.
%   \end{Version}
% \end{History}
%
% \PrintIndex
%
% \Finale
\endinput
|
% \end{quote}
% Do not forget to quote the argument according to the demands
% of your shell.
%
% \paragraph{Generating the documentation.}
% You can use both the \xfile{.dtx} or the \xfile{.drv} to generate
% the documentation. The process can be configured by the
% configuration file \xfile{ltxdoc.cfg}. For instance, put this
% line into this file, if you want to have A4 as paper format:
% \begin{quote}
%   \verb|\PassOptionsToClass{a4paper}{article}|
% \end{quote}
% An example follows how to generate the
% documentation with pdf\LaTeX:
% \begin{quote}
%\begin{verbatim}
%pdflatex atbegshi.dtx
%makeindex -s gind.ist atbegshi.idx
%pdflatex atbegshi.dtx
%makeindex -s gind.ist atbegshi.idx
%pdflatex atbegshi.dtx
%\end{verbatim}
% \end{quote}
%
% \section{Catalogue}
%
% The following XML file can be used as source for the
% \href{http://mirror.ctan.org/help/Catalogue/catalogue.html}{\TeX\ Catalogue}.
% The elements \texttt{caption} and \texttt{description} are imported
% from the original XML file from the Catalogue.
% The name of the XML file in the Catalogue is \xfile{atbegshi.xml}.
%    \begin{macrocode}
%<*catalogue>
<?xml version='1.0' encoding='us-ascii'?>
<!DOCTYPE entry SYSTEM 'catalogue.dtd'>
<entry datestamp='$Date$' modifier='$Author$' id='atbegshi'>
  <name>atbegshi</name>
  <caption>Execute stuff at \shipout time.</caption>
  <authorref id='auth:oberdiek'/>
  <copyright owner='Heiko Oberdiek' year='2007-2011'/>
  <license type='lppl1.3'/>
  <version number='1.18'/>
  <description>
    This package is a modern reimplementation of package
    <xref refid='everyshi'>everyshi</xref>, providing various commands
    to be executed before a <tt>\shipout</tt> command.  It makes use of
    e-TeX&#x2019;s facilities if they are available.  The package may
    be used either with LaTeX or with plain TeX.
    <p/>
    The package is part of the <xref refid='oberdiek'>oberdiek</xref> bundle.
  </description>
  <documentation details='Package documentation'
      href='ctan:/macros/latex/contrib/oberdiek/atbegshi.pdf'/>
  <ctan file='true' path='/macros/latex/contrib/oberdiek/atbegshi.dtx'/>
  <miktex location='oberdiek'/>
  <texlive location='oberdiek'/>
  <install path='/macros/latex/contrib/oberdiek/oberdiek.tds.zip'/>
</entry>
%</catalogue>
%    \end{macrocode}
%
% \begin{History}
%   \begin{Version}{2007/04/17 v1.0}
%   \item
%     First version.
%   \end{Version}
%   \begin{Version}{2007/04/18 v1.1}
%   \item
%     New method based on \cs{lastkern} is used if \eTeX\ is missing.
%   \item
%     \cs{AtBeginShipoutDiscard} also resets \cs{deadcycles}.
%   \end{Version}
%   \begin{Version}{2007/04/19 v1.2}
%   \item
%     \cs{AtBeginShipoutEarly} removed for simplification reasons.
%   \item
%     Forgotten definition of \cs{AtBegShi@Info} added.
%   \item
%     Patches for packages \xpackage{crop} and \xpackage{everyshi}
%     and class \xclass{memoir} added.
%   \end{Version}
%   \begin{Version}{2007/04/26 v1.3}
%   \item
%     Use of package \xpackage{infwarerr}.
%   \item
%     Catcode section after generic header.
%   \end{Version}
%   \begin{Version}{2007/04/27 v1.4}
%   \item
%     Small optimizations.
%   \end{Version}
%   \begin{Version}{2007/06/06 v1.5}
%   \item
%     \cs{AtBeginShipoutUpperLeft} added.
%   \item
%     Example added.
%   \item
%     Fix in second test file for newer version of \xclass{memoir}.
%   \end{Version}
%   \begin{Version}{2007/09/09 v1.6}
%   \item
%     Catcode section rewritten.
%   \end{Version}
%   \begin{Version}{2008/07/18 v1.7}
%   \item
%     Documentation of \cs{AtBeginShipoutUpperLeft} fixed and extended.
%   \end{Version}
%   \begin{Version}{2008/07/19 v1.8}
%   \item
%     \cs{AtBeginShipoutUpperLeftForeground} added.
%   \end{Version}
%   \begin{Version}{2008/07/31 v1.9}
%   \item
%     Second example (TrimBox for dvipdfmx) added.
%   \item
%     No changes in package code.
%   \end{Version}
%   \begin{Version}{2009/12/02 v1.10}
%   \item
%     \cs{AtBeginShipoutOriginalShipout} added.
%   \item
%     Test file fixed.
%   \end{Version}
%   \begin{Version}{2010/03/01 v1.11}
%   \item
%     Compatibility with \iniTeX\ except for \cs{newbox}.
%   \end{Version}
%   \begin{Version}{2010/03/25 v1.12}
%   \item
%     \cs{AtBeginShipoutNext} can now be used inside
%     \cs{AtBeginShipoutNext}.
%   \end{Version}
%   \begin{Version}{2010/08/18 v1.13}
%   \item
%     Fixes for \cs{AtBegShi@CheckDefinable}.
%   \end{Version}
%   \begin{Version}{2010/12/02 v1.14}
%   \item
%     Remove the warning because of void box if the hook
%     calls \AtBeginShipoutDiscard.
%   \end{Version}
%   \begin{Version}{2011/01/30 v1.15}
%   \item
%     Already loaded package files are not input in \hologo{plainTeX}.
%   \end{Version}
%   \begin{Version}{2011/10/05 v1.16}
%   \item
%     \cs{AtBeginShipoutAddToBox}, \cs{AtBeginShipoutAddToBoxForeground} added.
%   \item
%     \cs{AtBeginShipoutBoxWidth}, \cs{AtBeginShipoutBoxHeight},
%     \cs{AtBeginShipoutBoxDepth} added.
%   \item
%     Updates for patches of class \xclass{memoir}.
%   \end{Version}
%   \begin{Version}{2016/05/16 v1.17}
%   \item
%     Documentation updates.
%   \end{Version}
%   \begin{Version}{2016/06/09 v1.18}
%   \item
%     Update for \cs{pdfhorign} in new LuaTeX.
%   \end{Version}
% \end{History}
%
% \PrintIndex
%
% \Finale
\endinput
