% \iffalse meta-comment
%
% Copyright 1993 1994 1995 1996 1997 1998 1999 2000 2001 2002 2003 2004 2005 2006 2007 2008 2009
% The LaTeX3 Project and any individual authors listed elsewhere
% in this file. 
% 
% This file is part of the LaTeX base system.
% -------------------------------------------
% 
% It may be distributed and/or modified under the
% conditions of the LaTeX Project Public License, either version 1.3c
% of this license or (at your option) any later version.
% The latest version of this license is in
%    http://www.latex-project.org/lppl.txt
% and version 1.3c or later is part of all distributions of LaTeX 
% version 2005/12/01 or later.
% 
% This file has the LPPL maintenance status "maintained".
% 
% The list of all files belonging to the LaTeX base distribution is
% given in the file `manifest.txt'. See also `legal.txt' for additional
% information.
% 
% The list of derived (unpacked) files belonging to the distribution 
% and covered by LPPL is defined by the unpacking scripts (with 
% extension .ins) which are part of the distribution.
% 
% \fi
% Filename: ltnews05.tex

% This is issue 5 of LaTeX News.

\documentclass
%    [lw35fonts]
   {ltnews}

% \usepackage[T1]{fontenc}

\publicationmonth{June}
\publicationyear{1996}
\publicationissue{5}

\begin{document}

\maketitle

\section{Welcome to \LaTeXNews~5}
This issue of \emph{\LaTeXNews} accompanies the fifth release of the
new standard \LaTeX{}, \LaTeXe.

\section{Extra possibilities for section headings}
Most \LaTeX\ sectioning commands are defined using
\verb|\@startsection|.
For example, the \textsf{article} class defines:
\begin{small}
\begin{verbatim}
\newcommand\section{\@startsection 
 {section}{1}{0pt}{-3.5ex plus-1ex minus-.2ex}%
 {2.3ex plus.2ex}{\normalfont\Large\bfseries}}
\end{verbatim}
\end{small}
The last argument specifies the style in which the section heading is
to be typeset.

The new feature added at this release is that at the \emph{end} of
this argument you may specify a command that \emph{takes an argument}.
This command will be applied to the section number and heading.
For example, one could use the \verb|\MakeUppercase| command to
produce uppercase headings. A package or class file could contain:
\begin{small}
\begin{verbatim}
\renewcommand\section{\@startsection 
 {section}{1}{0pt}{-3.5ex plus-1ex minus-.2ex}%
 {2.3ex plus.2ex}{\normalfont\Large\MakeUppercase}}
\end{verbatim}
\end{small}
to produce section headings using uppercase medium weight text, rather
than the bold text used by \textsf{article}. Note that, like the font
choice, the uppercasing applies only to the actual heading (including
any automatically generated section number), not to the text as it may
appear in the running head or table of contents.

\section{The `openany' option in the `book' class}
The \textsf{openany} option allows chapter and similar openings to
occur on left hand pages. Previously this option only affected
\verb|\chapter| and \verb|\backmatter|. It now also affects
\verb|\part|, \verb|\frontmatter| and \verb|\mainmatter|.

\section{More font (output) encodings}
The font encoding name \texttt{T3} has been allocated to the encoding
used in the new 256-character \textsc{IPA} fonts (for the phonetic
alphabet) produced by Rei Fukui. His package, \textsf{tipa},
gives access to these fonts and should soon be available.  (The
encoding named \texttt{OT3} is the 128-character encoding used in the
\textsc{IPA} fonts produced by Washington State University.)



\section{More input encodings supported}
The \textsf{inputenc} package now supports the IBM codepage~852 used
in Eastern Europe, with the option~\texttt{[cp852]} contributed by
Petr~Sojka.

Also, the \textsf{inputenc} package now activates most `control codes'
with \textsc{ascii} values below 32.
Currently none of the encodings in the standard distribution makes use
of these positions.

\section{Fixes and improvements}
The \LaTeX\ kernel has only had minor changes, apart from
\verb|\@startsection| mentioned above.
However, some small fixes have been incorporated removing the
following problems:

\begin{itemize}
\item
  In tabular and array, previous versions of \LaTeX\ `lost' the
  inter-column space from an `\texttt{l}'-column, when that column
  was completely empty.

\item
  Previously, the use of the \verb|\nofiles| command could change
  the \emph{vertical spacing} in a document.\\ A side effect of fixing
  this is that when \verb|\nofiles| is used, \verb|\label| puts a
  blank line in the log file.

\item
  \LaTeX~often loads fonts `on demand'. Previously, this could
  happen inside the argument of an accent command and this would
  cause the accent to appear in the wrong place.

\end{itemize}

\section{Changes to the `tools' packages}

\begin{itemize}
\item
  The \textsf{longtable} package now uses a modified algorithm,
  contributed by David Kastrup, to align the `chunks' of a table.
  It is now unnecessary to edit the document to add
  \verb|\setlongtables| before the final run of \LaTeX.
  In certain cases of overlapping \verb|\multicolumn| entries, the new
  algorithm will produce better column widths than the old (at the
  price of extra passes through \LaTeX).

\item
  The \textsf{dcolumn} package now has the extra possibility of
  specifying the number of digits both \emph{before} and after the
  `decimal point'. This makes it easy to centre the column of numbers
  under a wide heading.
\end{itemize}

\section{New copy of the \LaTeX\ bug database}
\verb|http://www.tex.ac.uk/ctan/latex/bugs.html| will soon have links
to a copy of the searchable \LaTeX\ bugs database at Mainz (Germany)
as well as the original copy at Sussex (England).

\end{document}
