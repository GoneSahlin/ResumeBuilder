% $Id: pdftex-dvi.tex 1513 2006-02-13 00:22:35Z karl $
% Karl Berry and others, 2004. Public domain.
% 
% This file is not used in TeX Live 2005.
% 
% For TeX Live 2004, we started building the "latex" format with the
% pdfetex engine.  This is nice because it gives users access to various
% advanced typesetting features.  But it's problematic because user
% documents have used, ever since the inception of pdftex, the test
%   \ifx\pdfoutput\undefined
% to determine whether they are generating pdf or dvi, whether to use
% the [pdftex] option to graphicx and hyperref, etc.
% 
% Also, some packages (geometry) test \pdfpageheight and the like,
% similarly.
% 
% Although the generic ifpdf.sty style file tests this more robustly, it
% is asking a lot for everyone to switch.
% 
% So this file, read at format-creation time, is our attempt at making
% the change compatible: we *un*define various pdf-related primitives,
% so that tests like the above will continue to work.  At the same time,
% we make the undefined primitives available as \normalPRIMITIVE, so
% anyone who really needs to can still get them, despite everything.

\let\normalpdfannot\pdfannot
\let\pdfannnot\undefined

\let\normalpdfhorigin\pdfhorigin
\let\pdfhorigin\undefined

\let\normalpdfinfo\pdfinfo
\let\pdfinfo\undefined

\let\normalpdfliteral\pdfliteral
\let\pdfliteral\undefined

\let\normalpdfmapfile\pdfmapfile
\let\pdfmapfile\undefined

\let\normalpdfoutput\pdfoutput
\let\pdfoutput\undefined

\let\normalpdfpageheight\pdfpageheight
\let\pdfpageheight\undefined

\let\normalpdfpagewidth\pdfpagewidth
\let\pdfpagewidth\undefined

\let\normalpdftexversion\pdftexversion
\let\pdftexversion\undefined

\let\normalpdfvorigin\pdfvorigin
\let\pdfvorigin\undefined

\let\normalpdftexrevision\pdftexrevision
\let\pdftexrevision\undefined

\let\normalpdfxform\pdfxform
\let\pdfxform\undefined

\let\normalpdfximage\pdfximage
\let\pdfximage\undefined

% these public primitives remain usable in dvi mode:
% \pdfadjustspacing
% \pdfavoidoverfull
% \pdffontattr
% \pdffontexpand
% \pdffontname
% \pdffontsize
% \*code

\endinput
